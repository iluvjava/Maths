\documentclass[11pt]{beamer}
\usetheme{Madrid}
\usepackage[utf8]{inputenc}

\usepackage{hyperref}
\usepackage{amsmath}
\usepackage{amsfonts}
\usepackage{amssymb}
\usepackage{graphicx}
\DeclareMathOperator{\argmin}{argmin}
\usepackage{algorithmic}
\usepackage{algorithm}
\usepackage{wrapfig}
\usepackage{subcaption}
\graphicspath{{.}}

\author{Generic Name}
\title{Title for Super Fancy Stuff}
% Informe o seu email de contato no comando a seguir
% Por exemplo, alcebiades.col@ufes.br
\newcommand{\email}{lalala@lala.la}
\setbeamercovered{transparent}
\setbeamertemplate{navigation symbols}{}
%\logo{}
\institute[]{Some Super Fancy Institution}
\date{\today}
\subject{Subject Title }

% ---------------------------------------------------------
% Selecione um estilo de referência
\bibliographystyle{IEEEtran}

%\bibliographystyle{abbrv}
%\setbeamertemplate{bibliography item}{\insertbiblabel}
% ---------------------------------------------------------

% ---------------------------------------------------------
\newtheorem{remark}{Remark}
\newtheorem{assumption}{Assumption}

\begin{document}

\begin{frame}
    \titlepage
\end{frame}

\begin{frame}{ToC}
    \tableofcontents
\end{frame}

\section{This is the First Section}
    \subsection{Taxonomy of Proximal type of Methods}
        \begin{frame}{Frame Title}
    
            \begin{block}{Formula Presented in Block}
                \begin{align}
                    \min_{x} g(x) + h(x)
                \end{align}    
            \end{block}
            
            \begin{itemize}
                \item [1.]Throughout this presentation, we assume the objective of a function $f$ is the sum of 2 functions.
                \item [2.]We are interested in the paper: FISTA (Fast Iterative-Shrinkage Algorithm) by Beck and Teboulle \cite{paper:FISTA}. 
                \pause 
                \item [1.] When $h = \delta_Q$ with $Q$ closed and convex with $Q\subseteq \text{ri}\circ \text{dom}(g)$, we use projected subgradient. 
                \item [2.] When $g$ is \textbf{\emph{strongly smooth}} and $h$ is \textbf{closed convex proper} whose proximal oracle is easy to compute, we consider the use of FISTA. 
            \end{itemize}
                
        \end{frame}
        
    \subsection{The Proximal Operator}
        \begin{frame}{Frame Title}
            \begin{definition}[Definition of Something]
                Let $f$ be convex closed and proper, then the proximal operator parameterized by $\alpha > 0$ is a non-expansive mapping defined as: 
                \begin{align*}
                    \text{prox}_{f, \alpha}(x) := 
                    \arg\min_{y}\left\lbrace
                        f(y) + \frac{1}{2\alpha} \Vert y - x\Vert^2
                    \right\rbrace. 
                \end{align*}
            \end{definition}  
            \begin{remark}
                When $f$ is convex, closed, and proper, 
            \end{remark}
        \end{frame}

        \begin{frame}{Prox is the Resolvant of Subgradient}
            \begin{lemma}[The Lemma]\label{lemma:prox_alternative_form}
                When the function $f$ is convex closed and proper, the $\text{prox}_{\alpha, f}$ can be viewed as the following operator $(I + \alpha \partial f)^{-1}$. 
            \end{lemma}
            \begin{proof}
                Minimizer satisfies zero subgradient condition, 
                {\scriptsize
                \begin{align*}
                    \mathbf 0 &\in \partial
                    \left[
                        \left.
                            f(y) + \frac{1}{2\alpha} \Vert y - x\Vert^2 
                        \right| y
                    \right](y^+)
                    \\
                    \mathbf 0 &\in \partial f(y^+) + \frac{1}{\alpha}(y^+ - x)
                    \\
                    \frac{x}{\alpha} &\in 
                    (\partial f + \alpha^{-1}I)(y^+)
                    \\
                    x &\in 
                    (\alpha \partial f + I)(y^+)
                    \\
                    y &\in (\alpha\partial f+ I)^{-1}(x).
                \end{align*}
                }
            \end{proof}
                
        \end{frame}
        
        
    \subsection{Strong Smoothness}
        \begin{frame}{Equivalence of Strong Smoothness and Lipschitz Gradient}
            \begin{theorem}[Lipschitz Gradient Equivalence under Convexity]
                Suppose $g$ is differentiable on the entire of $\mathbb E$. It is closed convex proper. It is strongly smooth with parameter $\alpha$ if and only if the gradient $\nabla g$ is globally Lipschitz continuous with a parameter of $\alpha$ and $g$ is closed and convex. 
                \begin{align*}
                    \Vert \nabla g(x) -\nabla g(y)\Vert \le 
                    \alpha 
                    \Vert x - y \Vert\quad \forall x, y\in \mathbb E
                \end{align*}
            \end{theorem}
            \begin{proof}
                Using line integral, we can prove Lipschitz gradient implies strong smoothness without convexity. The converse requires convexity and applying generalized Cauchy Inequality to (iv) in Theorem 5.8 for Beck's textbook \cite{book:first_order_opt}. 
            \end{proof}
            
        \end{frame}
    \subsection{A Major Assumption}    
        \begin{frame}{A Major Assumption}
            \begin{assumption}[Convex Smooth Nonsmooth with Bounded Minimizers]\label{assumption:1}
                We will assume that $g:\mathbb E\mapsto \mathbb R$ is \textbf{strongly smooth} with constant $L_g$ and $h:\mathbb E \mapsto \bar{\mathbb R}$ \textbf{is closed convex and proper}. We define $f := g + h$ to be the summed function and $\text{ri}\circ \text{dom}(g) \cap \text{ri}\circ \text{dom}(h) \neq \emptyset$. We also assume that a set of minimizers exists for the function $f$ and that the set is bounded. Denote the minimizer using $\bar x$. 
            \end{assumption}
        \end{frame}
        
    
\section{A New Fancy Section}
    \subsection{A Fancy Subsetction for Algorithm}
        \begin{frame}{The Accelerated Proximal Gradient Method}
            \begin{block}{Momentum Template Method}
                \begin{algorithm}[H]
                    \begin{algorithmic}[1]
                        \STATE{\textbf{Input:} $x^{(0)}, x^{(-1)}, L, h, g$; 2 initial guesses and stepsize L}
                        \STATE{$y^{(0)} = x^{(0)} + \theta_k (x^{(0)} - x^{(-1)})$}
                        \FOR{$k = 1, \cdots, N$}
                            \STATE{$x^{(k)} = \text{prox}_{h, L^{-1}}(y^{(k)} + L^{-1}\nabla g(y^{(k)})) =: \mathcal P_{L^{-1}}^{g, h}(y^{(k)})$}
                            \STATE{$y^{(k + 1)} = x^{(k)} + \theta_k(x^{(k)} - x^{(k - 1)})$}
                        \ENDFOR
                    \end{algorithmic}
                    \caption{Template Proximal Gradient Method With Momentum}\label{alg:fista_template}
                \end{algorithm}
            \end{block}
        \end{frame}


    
        
    
\section{References}
    \begin{frame}{References}
        
        \bibliography{refs.bib}
    \end{frame}

\end{document}