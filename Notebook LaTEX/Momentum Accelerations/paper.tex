\documentclass[12pt]{article}

% 
%\usepackage{showkeys}
%\usepackage{drftcite}
\usepackage{exscale,relsize}
\usepackage{amsmath}
\usepackage{amsfonts}
\usepackage[hidelinks]{hyperref}
\usepackage{amssymb}
\usepackage{calc}
\usepackage{theorem}
\usepackage{pifont}      % needed by dingautolist
\usepackage{array}
\usepackage{color}
\usepackage{enumerate}
\usepackage{bbm}
\usepackage{graphicx}
\usepackage{float}
\usepackage{subfigure}




% FORMATTING ===================================================================
\oddsidemargin -0.1cm
\textwidth  16.5cm
\topmargin  0.0cm
\headheight 0.0cm
\textheight 21.0cm
\parindent  4mm
\parskip    10pt
%\parskip    8pt
\tolerance  3000

% MATH SYMBOLS =================================================================


% These are Heniz's notations. 
\newcommand{\To}{\ensuremath{\rightrightarrows}}
\newcommand{\GX}{\ensuremath{\Gamma}}
\newcommand{\mal}{\ensuremath{\mathfrak{m}}}
\newcommand{\mumu}{\ensuremath{{\mu\mu}}}
\newcommand{\paver}{\ensuremath{\mathcal{P}}}
\newcommand{\ZZZ}{\ensuremath{{X \times X^*}}}
\newcommand{\RRR}{\ensuremath{{\RR \times \RR}}}
\newcommand{\todo}{\hookrightarrow\textsf{TO DO:}}

\newcommand{\emp}{\ensuremath{\varnothing}}
%\newcommand{\la}{\ensuremath{\langle}}
%\newcommand{\ra}{\ensuremath{\rangle}}
\newcommand{\infconv}{\ensuremath{\mbox{\small$\,\square\,$}}}
\newcommand{\pscal}{\ensuremath{\scal{\cdot}{\cdot}}}
\newcommand{\Tt}{\ensuremath{\mathfrak{T}}}
\newcommand{\YY}{\ensuremath{\mathcal Y}}
\newcommand{\XX}{\ensuremath{\mathcal X}}
\newcommand{\HH}{\ensuremath{\mathcal H}}
\newcommand{\XP}{\ensuremath{\mathcal X}^*}
\newcommand{\st}{\ensuremath{\;|\;}}
\newcommand{\zeroun}{\ensuremath{\left]0,1\right[}}

\newcommand{\lev}[1]{\ensuremath{\mathrm{lev}_{\leq #1}\:}}
\newcommand{\moyo}[2]{\ensuremath{\sideset{_{#2}}{}{\operatorname{}}\!#1}}
\newcommand{\pair}[2]{\left\langle{{#1},{#2}}\right\rangle}
%\newcommand{\scal}[2]{\left.\left\langle{#1}\:\right| {#2}  \right\rangle}
\newcommand{\scal}[2]{\langle{{#1},{#2}}\rangle}
\newcommand{\Scal}[2]{\left\langle{{#1},{#2}}\right\rangle}
%\newcommand{\scal}[2]{\braket{ {#1},{#2}}}

\newcommand{\yosida}{\ensuremath{ \; {}^}}
\newcommand{\exi}{\ensuremath{\exists\,}}
\newcommand{\GG}{\ensuremath{\mathcal G}}
\newcommand{\RR}{\ensuremath{\mathbb R}}
\newcommand{\SSS}{\ensuremath{\mathbb S}}
\newcommand{\CC}{\ensuremath{\mathbb C}}
\newcommand{\Real}{\ensuremath{\mathrm{Re}\,}}
\newcommand{\ii}{\ensuremath{\mathrm i}}
\newcommand{\RP}{\ensuremath{\left[0,+\infty\right[}}
\newcommand{\RPX}{\ensuremath{\left[0,+\infty\right]}}
\newcommand{\RPP}{\ensuremath{\,\left]0,+\infty\right[}}
\newcommand{\RX}{\ensuremath{\,\left]-\infty,+\infty\right]}}
\newcommand{\RXX}{\ensuremath{\,\left[-\infty,+\infty\right]}}
\newcommand{\KK}{\ensuremath{\mathbb K}}
\newcommand{\NN}{\ensuremath{\mathbb N}}
\newcommand{\nnn}{\ensuremath{{n \in \NN}}}
\newcommand{\thalb}{\ensuremath{\tfrac{1}{2}}}
\newcommand{\pfrac}[2]{\ensuremath{\mathlarger{\tfrac{#1}{#2}}}}
\newcommand{\zo}{\ensuremath{{\left]0,1\right]}}}
\newcommand{\lzo}{\ensuremath{{\lambda \in \left]0,1\right]}}}
%\newcommand{\toppsepp}{\setlength{\partopsep}{-5pt}}
\newcommand{\menge}[2]{\big\{{#1} \mid {#2}\big\}}


% MATH OPERATORS ===============================================================
% \newcommand{\monos}{\ensuremath{\mathcal M}}
\newcommand{\DD}{\operatorname{dom}f}
\newcommand{\IDD}{\ensuremath{\operatorname{int}\operatorname{dom}f}}
\newcommand{\CDD}{\ensuremath{\overline{\operatorname{dom}}\,f}}
\newcommand{\clspan}{\ensuremath{\overline{\operatorname{span}}}}
\newcommand{\cone}{\ensuremath{\operatorname{cone}}}
\newcommand{\dom}{\ensuremath{\operatorname{dom}}}
\newcommand{\closu}{\ensuremath{\operatorname{cl}}}
\newcommand{\cont}{\ensuremath{\operatorname{cont}}}
\newcommand{\mons}{\ensuremath{\mathcal{A}}}
\newcommand{\gra}{\ensuremath{\operatorname{gra}}}
\newcommand{\epi}{\ensuremath{\operatorname{epi}}}
\newcommand{\prox}{\ensuremath{\operatorname{Prox}_{\mu}}}
\newcommand{\hprox}{\ensuremath{\operatorname{prox}}}
\newcommand{\intdom}{\ensuremath{\operatorname{int}\operatorname{dom}}\,}
\newcommand{\inte}{\ensuremath{\operatorname{int}}}
\newcommand{\sri}{\ensuremath{\operatorname{sri}}}
\newcommand{\reli}{\ensuremath{\operatorname{ri}}}
\newcommand{\cart}{\ensuremath{\mbox{\LARGE{$\times$}}}}


\newcommand{\average}{\ensuremath{\mathcal{R}_{\mu}({\bf A},{\boldsymbol \lambda})}}
\newcommand{\averagebar}{\ensuremath{\mathcal{R}_{1}({\bf A},\bar{\lambda})}}
\newcommand{\averageonelambda}{\ensuremath{\mathcal{R}({\bf A},{\boldsymbol \lambda})}}
\newcommand{\averageonehalf}{\ensuremath{\mathcal{R}_{1}(A,1/2)}}
\newcommand{\averageinverse}{\ensuremath{\mathcal{R}_{\mu^{-1}}({\bf A}^{-1},{\boldsymbol \lambda})}}
\newcommand{\averageoneinverse}{\ensuremath{\mathcal{R}({\bf A}^{-1},{\boldsymbol \lambda})}}
\newcommand{\averagef}{\ensuremath{\mathcal{P}_{\mu}(f,\lambda)}}
\newcommand{\averagefone}{\ensuremath{\mathcal{P}_{1}(f,\lambda)}}
\newcommand{\averagefd}{\ensuremath{\mathcal{P}_{\mu}((f_{1},\ldots, f_{n}),(\lambda_{1},\ldots, \lambda_{n}))}}
\newcommand{\averagefik}{\ensuremath{\mathcal{P}_{\mu_{k}}((f_{1,k},\ldots,f_{n,k}),
(\lambda_{1,k},\ldots,\lambda_{n,k}))}}
\newcommand{\averagesub}{\ensuremath{\mathcal{R}_{\mu}(\partial f,\lambda)}}
\newcommand{\res}{\ensuremath{\mathcal{R}_{\mu}}}
\newcommand{\resmuk}{\ensuremath{\mathcal{R}_{\mu_{k}}}}
\newcommand{\newres}{\ensuremath{\mathcal{R}}}
\newcommand{\resmualpha}{\ensuremath{\mathcal{R}_{\alpha\mu}}}
\newcommand{\averageone}{\ensuremath{\mathcal{R}_{1}}}
\newcommand{\harm}{\ensuremath{\mathcal{H}(A,\lambda)}}
\newcommand{\arithmetic}{\ensuremath{\mathcal{A}(A,\lambda)}}

\newcommand{\WC}{\ensuremath{{\mathfrak W}}}
\newcommand{\SC}{\ensuremath{{\mathfrak S}}}
\newcommand{\card}{\ensuremath{\operatorname{card}}}
\newcommand{\bd}{\ensuremath{\operatorname{bdry}}}
\newcommand{\ran}{\ensuremath{\operatorname{ran}}}
\newcommand{\rec}{\ensuremath{\operatorname{rec}}}
\newcommand{\rank}{\ensuremath{\operatorname{rank}}}
\newcommand{\kernel}{\ensuremath{\operatorname{ker}}}
\newcommand{\conv}{\ensuremath{\operatorname{conv}}}
\newcommand{\segh}{\ensuremath{\operatorname{seg}}}
\newcommand{\boxx}{\ensuremath{\operatorname{box}}}
\newcommand{\clconv}{\ensuremath{\overline{\operatorname{conv}}\,}}
\newcommand{\cldom}{\ensuremath{\overline{\operatorname{dom}}\,}}
\newcommand{\clran}{\ensuremath{\overline{\operatorname{ran}}\,}}
\newcommand{\Nf}{\ensuremath{\nabla f}}
\newcommand{\NNf}{\ensuremath{\nabla^2f}}
\newcommand{\Fix}{\ensuremath{\operatorname{Fix}}}
\newcommand{\FFix}{\ensuremath{\overline{\operatorname{Fix}}\,}}
\newcommand{\aFix}{\ensuremath{\widetilde{\operatorname{Fix}\,}}}
\newcommand{\Id}{\ensuremath{\operatorname{Id}}}
\newcommand{\Max}{\ensuremath{\operatorname{max}}}
\newcommand{\Bb}{\ensuremath{\mathfrak{B}}}
\newcommand{\BB}{\ensuremath{\mathbb{B}}}
\newcommand{\Fb}{\ensuremath{\overrightarrow{\mathfrak{B}}}}
\newcommand{\Fprox}{\ensuremath{\overrightarrow{\operatorname{prox}}}}
\newcommand{\Bprox}{\ensuremath{\overleftarrow{\operatorname{prox}}}}
\newcommand{\Bproj}{\ensuremath{\overleftarrow{\operatorname{P}}}}
\newcommand{\Ri}{\ensuremath{\mathfrak{R}_i}}
\newcommand{\Dn}{\ensuremath{\,\overset{D}{\rightarrow}\,}}
\newcommand{\nDn}{\ensuremath{\,\overset{D}{\not\rightarrow}\,}}
\newcommand{\weakly}{\ensuremath{\,\rightharpoonup}\,}
\newcommand{\weaklys}{\ensuremath{\,\overset{*}{\rightharpoonup}}\,}
\newcommand{\gr}{\ensuremath{\operatorname{gra}}}
\newcommand{\g}{\ensuremath{\,\overset{g}{\rightarrow}}\,}
\newcommand{\p}{\ensuremath{\,\overset{p}{\rightarrow}}\,}
\newcommand{\e}{\ensuremath{\,\overset{e}{\rightarrow}}\,}
\newcommand{\Tbar}{\ensuremath{\overline{T}}}
\newcommand{\n}{\ensuremath{\,\overset{n}{\rightarrow}}\,}

\newcommand{\minf}{\ensuremath{-\infty}}
\newcommand{\pinf}{\ensuremath{+\infty}}
\renewcommand{\iff}{\ensuremath{\Leftrightarrow}}
\renewcommand{\phi}{\ensuremath{\varphi}}
%\newcommand{\Real}{\ensuremath{\mathrm{Re}\,}}
\newcommand{\negent}{\ensuremath{\operatorname{negent}}}
\newcommand{\neglog}{\ensuremath{\operatorname{neglog}}}
\newcommand{\halb}{\ensuremath{\tfrac{1}{2}}}
\newcommand{\bT}{\ensuremath{\mathbf{T}}}
\newcommand{\bX}{\ensuremath{\mathbf{X}}}
\newcommand{\bL}{\ensuremath{\mathbf{L}}}
\newcommand{\bD}{\ensuremath{\boldsymbol{\Delta}}}
\newcommand{\bc}{\ensuremath{\mathbf{c}}}
\newcommand{\by}{\ensuremath{\mathbf{y}}}
\newcommand{\bx}{\ensuremath{\mathbf{x}}}
\newcommand{\bA}{{\bf A}}
\newcommand{\Other}{Indeterminate }
\newcommand{\other}{indeterminate }


%%% Raf's stuff  ===============================================================
\newcommand{\al}{\alpha}
\newcommand{\la}{\lambda}
\newcommand{\La}{\Lambda}
\newcommand{\pluss}{{\hskip1pt \raise1pt\vbox{\hrule width6pt \vskip1pt
\hrule width6pt}\kern-4pt{\lower1pt\hbox{\vrule height6pt \kern1pt\vrule
height6pt}}\hskip5pt}}
\newcommand{\timess}{\star}
\newcommand{\argmax}{\mathop{\rm argmax}\limits}
\newcommand{\argmin}{\mathop{\rm argmin}\limits}
\newcommand{\product}{\langle\cdot,\cdot\rangle}
\newcommand{\im}{\mathrm{Im}}
\newcommand{\multival}{\ensuremath{X\to 2^{X^*}}}
\newcommand{\SX}{\ensuremath{2^{X^*}}}



% THEOREM AND ENVIRONMENTS.  ===================================================

%\newenvironment{deflist}[1][\quad]%
%{\begin{list}{}{\renewcommand{\makelabel}[1]{\textrm{##1~}\hfil}%
%\settowidth{\labelwidth}{\textrm{#1~}}%
%\setlength{\leftmargin}{\labelwidth+\labelsep}}}%requires macro calc.sty
%{\end{list}}
%\newtheorem{theorem}{Theorem}%[section]
\newtheorem{theorem}{Theorem}[section]
\newtheorem{lemma}[theorem]{Lemma}
\newtheorem{fact}[theorem]{Fact}
\newtheorem{corollary}[theorem]{Corollary}
\newtheorem{proposition}[theorem]{Proposition}
\newtheorem{definition}[theorem]{Definition}
\newtheorem{conjecture}[theorem]{Conjecture}
\newtheorem{observation}[theorem]{Observation}
\newtheorem{openprob}[theorem]{Open Problem}
\theoremstyle{plain}{\theorembodyfont{\rmfamily}
\newtheorem{assumption}[theorem]{Assumption}}
\theoremstyle{plain}{\theorembodyfont{\rmfamily}
\newtheorem{condition}[theorem]{Condition}}
\theoremstyle{plain}{\theorembodyfont{\rmfamily}
\newtheorem{algorithm}[theorem]{Algorithm}}
\theoremstyle{plain}{\theorembodyfont{\rmfamily}
\newtheorem{example}[theorem]{Example}}
\theoremstyle{plain}{\theorembodyfont{\rmfamily}
\newtheorem{remark}[theorem]{Remark}}
\theoremstyle{plain}{\theorembodyfont{\rmfamily}
\newtheorem{application}[theorem]{Application}}
\def\proof{\noindent{\it Proof}. \ignorespaces}
%\def\endproof{\vbox{\hrule height0.6pt\hbox{\vrule height1.3ex%
%width0.6pt\hskip0.8ex\vrule width0.6pt}\hrule height0.6pt}}
%\numberwithin{equation}{section}
\def\endproof{\ensuremath{\quad \hfill \blacksquare}}

\renewcommand\theenumi{(\roman{enumi})}
\renewcommand\theenumii{(\alph{enumii})}
\renewcommand{\labelenumi}{\rm (\roman{enumi})}
\renewcommand{\labelenumii}{\rm (\alph{enumii})}

\newcommand{\boxedeqn}[1]{%
    \[\fbox{%
        \addtolength{\linewidth}{-2\fboxsep}%
        \addtolength{\linewidth}{-2\fboxrule}%
        \begin{minipage}{\linewidth}%
        \begin{equation}#1\end{equation}%
        \end{minipage}%
      }\]%
  }


\newcommand{\hilight}[1]{\colorbox{yellow}{#1}}


%\usepackage{showkeys}
%\usepackage{drftcite}
\usepackage{exscale,relsize}
\usepackage{amsmath}
\usepackage{amsfonts}
\usepackage[colorlinks=true, linkcolor=blue]{hyperref}
\usepackage{amssymb}
\usepackage{calc}
\usepackage{theorem}
\usepackage{pifont}      % needed by dingautolist
\usepackage{array}
\usepackage{color}
\usepackage{enumerate}
\usepackage{bbm}
\usepackage{graphicx}
\usepackage{subcaption}
\usepackage{caption}

% \usepackage{amsthm}


% Hongda's packages
\usepackage{algpseudocode, algorithm}
\usepackage{mathtools}


% IF use the below packge, use `\printbibliography' to print out the bibliograph 
% For this one 
% 
% \usepackage[
%     backend=biber,
%     style=numeric,
%     sorting=nyt
% ]{biblatex}
% \addbibresource{references/PPM.bib}
% \addbibresource{references/NesterovMomentum.bib}
% \addbibresource{references/Books.bib}
% \addbibresource{references/BregmanDiv.bib}

% FORMATTING ===================================================================
\oddsidemargin -0.1cm
\textwidth  16.5cm
\topmargin  0.0cm
\headheight 0.0cm
\textheight 21.0cm
\parindent  4mm
\parskip    10pt
%\parskip    8pt
\tolerance  3000

% DRAFT FORMATTING =============================================================
% These are for todo notes, advise taken from 
% https://tex.stackexchange.com/questions/81666/extend-page-width-or-margin-for-todonotes-comments-or-other-package-comments
% \oddsidemargin=\dimexpr\oddsidemargin + 3cm\relax % DON'T USE

\ifthenelse{\boolean{draftmode}}{
    \evensidemargin=\dimexpr\evensidemargin  + 6cm\relax 
    \oddsidemargin=\dimexpr\oddsidemargin + 6cm\relax
    \paperwidth=\dimexpr \paperwidth + 12cm\relax 
    \marginparwidth=\dimexpr \marginparwidth  + 6cm\relax
    \paperheight=\dimexpr \paperheight + 6cm\relax
}{
    
}


% THEOREM AND ENVIRONMENTS.  ===================================================

%\newenvironment{deflist}[1][\quad]%
%{\begin{list}{}{\renewcommand{\makelabel}[1]{\textrm{##1~}\hfil}%
%\settowidth{\labelwidth}{\textrm{#1~}}%
%\setlength{\leftmargin}{\labelwidth+\labelsep}}}%requires macro calc.sty
%{\end{list}}
%\newtheorem{theorem}{Theorem}%[section]
\newtheorem{theorem}{Theorem}[section]
\newtheorem{lemma}[theorem]{Lemma}
\newtheorem{fact}[theorem]{Fact}
\newtheorem{corollary}[theorem]{Corollary}
\newtheorem{proposition}[theorem]{Proposition}
\newtheorem{definition}[theorem]{Definition}
\newtheorem{conjecture}[theorem]{Conjecture}
\newtheorem{observation}[theorem]{Observation}
\newtheorem{openprob}[theorem]{Open Problem}
\theoremstyle{plain}{\theorembodyfont{\rmfamily}
\newtheorem{assumption}[theorem]{Assumption}}
\theoremstyle{plain}{\theorembodyfont{\rmfamily}
\newtheorem{condition}[theorem]{Condition}}
\theoremstyle{plain}{\theorembodyfont{\rmfamily}}

% Removed due conflict with the algorithm environment. 
% \newtheorem{algorithm}[theorem]{Algorithm}}

\theoremstyle{plain}{\theorembodyfont{\rmfamily}
\newtheorem{example}[theorem]{Example}}
\theoremstyle{plain}{\theorembodyfont{\rmfamily}
\newtheorem{remark}[theorem]{Remark}}
\theoremstyle{plain}{\theorembodyfont{\rmfamily}
\newtheorem{application}[theorem]{Application}}

\def\proof{\noindent{\it Proof}. \ignorespaces}
%\def\endproof{\vbox{\hrule height0.6pt\hbox{\vrule height1.3ex%
%width0.6pt\hskip0.8ex\vrule width0.6pt}\hrule height0.6pt}}
%\numberwithin{equation}{section}
\def\endproof{\ensuremath{\quad \hfill \blacksquare}}

\renewcommand\theenumi{(\roman{enumi})}
\renewcommand\theenumii{(\alph{enumii})}
\renewcommand{\labelenumi}{\rm (\roman{enumi})}
\renewcommand{\labelenumii}{\rm (\alph{enumii})}

\numberwithin{equation}{section}



% These are Heniz's notations. 
\newcommand{\To}{\ensuremath{\rightrightarrows}}
\newcommand{\GX}{\ensuremath{\Gamma}}
\newcommand{\mal}{\ensuremath{\mathfrak{m}}}
\newcommand{\mumu}{\ensuremath{{\mu\mu}}}
\newcommand{\paver}{\ensuremath{\mathcal{P}}}
\newcommand{\ZZZ}{\ensuremath{{X \times X^*}}}
\newcommand{\RRR}{\ensuremath{{\RR \times \RR}}}
\newcommand{\todo}{\hookrightarrow\textsf{TO DO:}}

\newcommand{\emp}{\ensuremath{\varnothing}}
%\newcommand{\la}{\ensuremath{\langle}}
%\newcommand{\ra}{\ensuremath{\rangle}}
\newcommand{\infconv}{\ensuremath{\mbox{\small$\,\square\,$}}}
\newcommand{\pscal}{\ensuremath{\scal{\cdot}{\cdot}}}
\newcommand{\Tt}{\ensuremath{\mathfrak{T}}}
\newcommand{\YY}{\ensuremath{\mathcal Y}}
\newcommand{\XX}{\ensuremath{\mathcal X}}
\newcommand{\HH}{\ensuremath{\mathcal H}}
\newcommand{\XP}{\ensuremath{\mathcal X}^*}
\newcommand{\st}{\ensuremath{\;|\;}}
\newcommand{\zeroun}{\ensuremath{\left]0,1\right[}}

\newcommand{\lev}[1]{\ensuremath{\mathrm{lev}_{\leq #1}\:}}
\newcommand{\moyo}[2]{\ensuremath{\sideset{_{#2}}{}{\operatorname{}}\!#1}}
\newcommand{\pair}[2]{\left\langle{{#1},{#2}}\right\rangle}
%\newcommand{\scal}[2]{\left.\left\langle{#1}\:\right| {#2}  \right\rangle}
\newcommand{\scal}[2]{\langle{{#1},{#2}}\rangle}
\newcommand{\Scal}[2]{\left\langle{{#1},{#2}}\right\rangle}
%\newcommand{\scal}[2]{\braket{ {#1},{#2}}}

\newcommand{\yosida}{\ensuremath{ \; {}^}}
\newcommand{\exi}{\ensuremath{\exists\,}}
\newcommand{\GG}{\ensuremath{\mathcal G}}
\newcommand{\RR}{\ensuremath{\mathbb R}}
\newcommand{\SSS}{\ensuremath{\mathbb S}}
\newcommand{\CC}{\ensuremath{\mathbb C}}
\newcommand{\Real}{\ensuremath{\mathrm{Re}\,}}
\newcommand{\ii}{\ensuremath{\mathrm i}}
\newcommand{\RP}{\ensuremath{\left[0,+\infty\right[}}
\newcommand{\RPX}{\ensuremath{\left[0,+\infty\right]}}
\newcommand{\RPP}{\ensuremath{\,\left]0,+\infty\right[}}
\newcommand{\RX}{\ensuremath{\,\left]-\infty,+\infty\right]}}
\newcommand{\RXX}{\ensuremath{\,\left[-\infty,+\infty\right]}}
\newcommand{\KK}{\ensuremath{\mathbb K}}
\newcommand{\NN}{\ensuremath{\mathbb N}}
\newcommand{\nnn}{\ensuremath{{n \in \NN}}}
\newcommand{\thalb}{\ensuremath{\tfrac{1}{2}}}
\newcommand{\zo}{\ensuremath{{\left]0,1\right]}}}
\newcommand{\lzo}{\ensuremath{{\lambda \in \left]0,1\right]}}}
%\newcommand{\toppsepp}{\setlength{\partopsep}{-5pt}}
\newcommand{\menge}[2]{\big\{{#1} \mid {#2}\big\}}
\newcommand{\pfrac}[2]{\ensuremath{\mathlarger{\tfrac{#1}{#2}}}}


% MATH OPERATORS ===============================================================
% \newcommand{\monos}{\ensuremath{\mathcal M}}
\newcommand{\DD}{\operatorname{dom}f}
\newcommand{\IDD}{\ensuremath{\operatorname{int}\operatorname{dom}f}}
\newcommand{\CDD}{\ensuremath{\overline{\operatorname{dom}}\,f}}
\newcommand{\clspan}{\ensuremath{\overline{\operatorname{span}}}}
\newcommand{\cone}{\ensuremath{\operatorname{cone}}}
\newcommand{\dom}{\ensuremath{\operatorname{dom}}}
\newcommand{\closu}{\ensuremath{\operatorname{cl}}}
\newcommand{\cont}{\ensuremath{\operatorname{cont}}}
\newcommand{\mons}{\ensuremath{\mathcal{A}}}
\newcommand{\gra}{\ensuremath{\operatorname{gra}}}
\newcommand{\epi}{\ensuremath{\operatorname{epi}}}
\newcommand{\prox}{\ensuremath{\operatorname{Prox}_{\mu}}}
\newcommand{\hprox}{\ensuremath{\operatorname{prox}}}
\newcommand{\intdom}{\ensuremath{\operatorname{int}\operatorname{dom}}\,}
\newcommand{\inte}{\ensuremath{\operatorname{int}}}
\newcommand{\sri}{\ensuremath{\operatorname{sri}}}
\newcommand{\reli}{\ensuremath{\operatorname{ri}}}
\newcommand{\cart}{\ensuremath{\mbox{\LARGE{$\times$}}}}


\newcommand{\average}{\ensuremath{\mathcal{R}_{\mu}({\bf A},{\boldsymbol \lambda})}}
\newcommand{\averagebar}{\ensuremath{\mathcal{R}_{1}({\bf A},\bar{\lambda})}}
\newcommand{\averageonelambda}{\ensuremath{\mathcal{R}({\bf A},{\boldsymbol \lambda})}}
\newcommand{\averageonehalf}{\ensuremath{\mathcal{R}_{1}(A,1/2)}}
\newcommand{\averageinverse}{\ensuremath{\mathcal{R}_{\mu^{-1}}({\bf A}^{-1},{\boldsymbol \lambda})}}
\newcommand{\averageoneinverse}{\ensuremath{\mathcal{R}({\bf A}^{-1},{\boldsymbol \lambda})}}
\newcommand{\averagef}{\ensuremath{\mathcal{P}_{\mu}(f,\lambda)}}
\newcommand{\averagefone}{\ensuremath{\mathcal{P}_{1}(f,\lambda)}}
\newcommand{\averagefd}{\ensuremath{\mathcal{P}_{\mu}((f_{1},\ldots, f_{n}),(\lambda_{1},\ldots, \lambda_{n}))}}
\newcommand{\averagefik}{\ensuremath{\mathcal{P}_{\mu_{k}}((f_{1,k},\ldots,f_{n,k}),
(\lambda_{1,k},\ldots,\lambda_{n,k}))}}
\newcommand{\averagesub}{\ensuremath{\mathcal{R}_{\mu}(\partial f,\lambda)}}
\newcommand{\res}{\ensuremath{\mathcal{R}_{\mu}}}
\newcommand{\resmuk}{\ensuremath{\mathcal{R}_{\mu_{k}}}}
\newcommand{\newres}{\ensuremath{\mathcal{R}}}
\newcommand{\resmualpha}{\ensuremath{\mathcal{R}_{\alpha\mu}}}
\newcommand{\averageone}{\ensuremath{\mathcal{R}_{1}}}
\newcommand{\harm}{\ensuremath{\mathcal{H}(A,\lambda)}}
\newcommand{\arithmetic}{\ensuremath{\mathcal{A}(A,\lambda)}}

\newcommand{\WC}{\ensuremath{{\mathfrak W}}}
\newcommand{\SC}{\ensuremath{{\mathfrak S}}}
\newcommand{\card}{\ensuremath{\operatorname{card}}}
\newcommand{\bd}{\ensuremath{\operatorname{bdry}}}
\newcommand{\ran}{\ensuremath{\operatorname{ran}}}
\newcommand{\rec}{\ensuremath{\operatorname{rec}}}
\newcommand{\rank}{\ensuremath{\operatorname{rank}}}
\newcommand{\kernel}{\ensuremath{\operatorname{ker}}}
\newcommand{\conv}{\ensuremath{\operatorname{conv}}}
\newcommand{\segh}{\ensuremath{\operatorname{seg}}}
\newcommand{\boxx}{\ensuremath{\operatorname{box}}}
\newcommand{\clconv}{\ensuremath{\overline{\operatorname{conv}}\,}}
\newcommand{\cldom}{\ensuremath{\overline{\operatorname{dom}}\,}}
\newcommand{\clran}{\ensuremath{\overline{\operatorname{ran}}\,}}
\newcommand{\Nf}{\ensuremath{\nabla f}}
\newcommand{\NNf}{\ensuremath{\nabla^2f}}
\newcommand{\Fix}{\ensuremath{\operatorname{Fix}}}
\newcommand{\FFix}{\ensuremath{\overline{\operatorname{Fix}}\,}}
\newcommand{\aFix}{\ensuremath{\widetilde{\operatorname{Fix}\,}}}
\newcommand{\Id}{\ensuremath{\operatorname{Id}}}
\newcommand{\Max}{\ensuremath{\operatorname{max}}}
\newcommand{\Bb}{\ensuremath{\mathfrak{B}}}
\newcommand{\BB}{\ensuremath{\mathbb{B}}}
\newcommand{\Fb}{\ensuremath{\overrightarrow{\mathfrak{B}}}}
\newcommand{\Fprox}{\ensuremath{\overrightarrow{\operatorname{prox}}}}
\newcommand{\Bprox}{\ensuremath{\overleftarrow{\operatorname{prox}}}}
\newcommand{\Bproj}{\ensuremath{\overleftarrow{\operatorname{P}}}}
\newcommand{\Ri}{\ensuremath{\mathfrak{R}_i}}
\newcommand{\Dn}{\ensuremath{\,\overset{D}{\rightarrow}\,}}
\newcommand{\nDn}{\ensuremath{\,\overset{D}{\not\rightarrow}\,}}
\newcommand{\weakly}{\ensuremath{\,\rightharpoonup}\,}
\newcommand{\weaklys}{\ensuremath{\,\overset{*}{\rightharpoonup}}\,}
\newcommand{\gr}{\ensuremath{\operatorname{gra}}}
\newcommand{\g}{\ensuremath{\,\overset{g}{\rightarrow}}\,}
\newcommand{\p}{\ensuremath{\,\overset{p}{\rightarrow}}\,}
\newcommand{\e}{\ensuremath{\,\overset{e}{\rightarrow}}\,}
\newcommand{\Tbar}{\ensuremath{\overline{T}}}
\newcommand{\n}{\ensuremath{\,\overset{n}{\rightarrow}}\,}

\newcommand{\minf}{\ensuremath{-\infty}}
\newcommand{\pinf}{\ensuremath{+\infty}}
\renewcommand{\iff}{\ensuremath{\Leftrightarrow}}
% \renewcommand{\phi}{\ensuremath{\varphi}}
%\newcommand{\Real}{\ensuremath{\mathrm{Re}\,}}
\newcommand{\negent}{\ensuremath{\operatorname{negent}}}
\newcommand{\neglog}{\ensuremath{\operatorname{neglog}}}
\newcommand{\halb}{\ensuremath{\tfrac{1}{2}}}
\newcommand{\bT}{\ensuremath{\mathbf{T}}}
\newcommand{\bX}{\ensuremath{\mathbf{X}}}
\newcommand{\bL}{\ensuremath{\mathbf{L}}}
\newcommand{\bD}{\ensuremath{\boldsymbol{\Delta}}}
\newcommand{\bc}{\ensuremath{\mathbf{c}}}
\newcommand{\by}{\ensuremath{\mathbf{y}}}
\newcommand{\bx}{\ensuremath{\mathbf{x}}}
\newcommand{\bA}{{\bf A}}
\newcommand{\Other}{Indeterminate }
\newcommand{\other}{indeterminate }


%%% Raf's stuff  ===============================================================
\newcommand{\al}{\alpha}
\newcommand{\la}{\lambda}
\newcommand{\La}{\Lambda}
\newcommand{\pluss}{{\hskip1pt \raise1pt\vbox{\hrule width6pt \vskip1pt
\hrule width6pt}\kern-4pt{\lower1pt\hbox{\vrule height6pt \kern1pt\vrule
height6pt}}\hskip5pt}}
\newcommand{\timess}{\star}
\newcommand{\argmax}{\mathop{\rm argmax}\limits}
\newcommand{\argmin}{\mathop{\rm argmin}\limits}
\newcommand{\product}{\langle\cdot,\cdot\rangle}
\newcommand{\im}{\mathrm{Im}}
\newcommand{\multival}{\ensuremath{X\to 2^{X^*}}}
\newcommand{\SX}{\ensuremath{2^{X^*}}}

\newcommand{\inlinecode}[1]{\texttt{\footnotesize #1}}

\usepackage{listings} \lstset{basicstyle=\footnotesize\ttfamily,breaklines=true}
\usepackage{xcolor}
\lstdefinelanguage{Julia}%
  {morekeywords={abstract,break,case,catch,const,continue,do, else, elseif,%
      end, export, false, for, function, immutable, import, importall, if, in,%
      macro, module, otherwise, quote, return, switch, true, try, type, typealias,%
      using, while},%
   sensitive=true,%
   alsoother={$},%
   morecomment=[l]\#,%
   morecomment=[n]{\#=}{=\#},%
   morestring=[s]{"}{"},%
   morestring=[m]{'}{'},%
}[keywords,comments,strings]%
\lstset{%
    language         = Julia,
    basicstyle       = \ttfamily,
    keywordstyle     = \bfseries\color{blue},
    stringstyle      = \color{magenta},
    commentstyle     = \color{ForestGreen},
    showstringspaces = false,
}

\begin{document}

\title{{\fontfamily{ptm}\selectfont The Proximal Point interpretation of Nesterov accelerated proximal gradient}}

\author{
    Author 1 Name, Author 2 Name
    \thanks{
        Subject type, Some Department of Some University, Location of the University,
        Country. E-mail: \texttt{alto@mail.ubc.ca}.
    }
}

\date{\today}

\maketitle

% \vskip 8mm

\begin{abstract} 
    \noindent
    Nesterov accelreated gradient method has been in the spotlight for the past decades due its wide spread applications and theories of optimal convergence. 
    Decades later it still opens up new interpretations. 
    Our work suggests a proximal point interpretation of accelerated gradient method for the method of accelerated proximal gradient method as a major extension to the interpretation proposed by Ahn and Sra \cite{ahn_understanding_2022}. 
    The proofs had been streamlined, extended and new error terms are added to allow a larger set of stepsize sequence for the PPM. 
    Additionally, we proposed a line search method to dynamically adjust the strong convexity index $\mu$ and Lipschitz constant of the gradient in algorithm implementations based on the PPM understanding, with numerical experiments. 
    
\end{abstract}

\noindent{\bfseries 2010 Mathematics Subject Classification:}
Primary 47H05, 52A41, 90C25; Secondary 15A09, 26A51, 26B25, 26E60, 47H09, 47A63.
\noindent{\bfseries Keywords:}

\section{Introduction}
    Recent works from Ahn and Sra \cite{ahn_understanding_2022} and Nesterov \cite{nesterov_lectures_2018} inspired content in this section.
    They explored the interpretation of Nesterov acceleration as a proximal of an upper surrogate function, and then a lower surrogate function. 
    Inspired by such an interpretation, we generalize the framework to the case of $h = f + g$ with $f$ Lipschitz smooth and $g$ convex and friendly to a proximal operator. 
    \par
    Classical analysis and extension of Nesterov accelerated gradient existed. 
    See \cite{guler_new_1992} for an extension of the Nesterov accelerated gradient method to the proximal point method for convex programming. 
    However the classical analysis found in \cite[chapter 2]{nesterov_lectures_2018} involves the assumption of a specific kind of Lypunov function and cherry picked Nesterov's estimating sequence to assist the proof for the parameters in the algorithm. 
    In Ahn's work however, the complexities are packaged into the PPM interpretation of accelerated gradient. 
    It uses a lemma from Moreau envelope to derive the Lypunov analysis, the sizes, and parameters for several variants of the algorithm. 
    \par
    Numerous notable variations of Nesterov accelerated gradient exists. \cite[(6.1.19)]{nesterov_lectures_2018} described a variant of accelerated gradient restricted to a convex domain $Q$. 
    Beck and Toubolle \cite{beck_fast_2009} introduced a variant the problem type of smooth plus non-smooth, known as FISTA. 
    For a variant of accelerated gradient where the iterates converge, see \cite{chambolle_convergence_2015}. 
    Extension such as the Harpen acceleration for the resolvent operator in general is outside of the scope since doesn't have a Moreau envelope. 
    \par
    A wide varieties of interpretation for the Nesterov accelerated gradient exist in the literatures. 
    Consult \cite{su_differential_2015} for a dynamical system interpretation of Nesterov acceleration. 
    The dynamical system interpretation of the algorithm however, lead to the valuable insights that restarting the accerlated gradient algorithm would lead to faster convergence rate for the class of strongly convex function. 
    \par
    % The paper is organized as follow: 
    % \begin{enumerate}
    %     \item Section 
    % \end{enumerate}
    
\section{Preliminaries}\label{sec:preliminaries}
    In this section we introduce the a descent lemma for Proximal Point Method (PPM) in the convex case. 
    We define the proximal gradient mapping $\mathcal T_L$, and gradient mapping $\mathcal G_L$ for function satisfying assumption 
    \ref*{assumption:smooth-nonsmooth-sum}. 
    A lower bound function is identified using the gradient mapping operator and proved in lemma 
    \ref*{lemma:grad_map_linearization}, 
    this is a key component of the proximal point interpretation of the accelerated gradient method. 
    \begin{assumption}\label{assumption:smooth-nonsmooth-sum}
        Let $h = f + g$ where $f, g$ are convex and $f$ is Lipschitz-Smooth. 
    \end{assumption}
    
    \begin{theorem}[Proximal Descent Inequality]\label{thm:ppm_descent_ineq}
        Let $f: \RR^n \mapsto \overline \RR^n$ $\beta$ be strongly convex with $\beta \ge 0$, fix any $x \in \RR^n$, define $p = \hprox_f(x)$.
        For all $y \in \RR$ it verifies
        $$
            \left(f(p) + \frac{1}{2}\Vert x - p\Vert^2\right)
            - 
            \left(
                f(y) + \frac{1}{2}\Vert x - y\Vert^2 
            \right)
            \le 
            - \frac{(1 + \beta)}{2}\Vert y - p\Vert^2. 
        $$
        Recall: $\hprox_f(x) = \argmin_{u}\left\lbrace f(u) + \frac{1}{2}\Vert u - x\Vert^2 \right\rbrace$. 
    \end{theorem}
    \begin{remark}
        We use this theorem to prove the convergence of the proximal point method. 
        See the proof (\cite{bauschke_convex_2017}, theorem 12.26). 
    \end{remark}
 
    \begin{definition}[The Gradient Mapping]
        \label{def:gradient_mapping}
        Suppose $h = f + g$ satisfies 
        \hyperref[assumption:smooth-nonsmooth-sum]{assumption \ref*{assumption:smooth-nonsmooth-sum}}. 
        Define the proximal gradient operator
        $$
            \mathcal T_L(x) := \hprox_{L^{-1}g}(x - L^{-1}\nabla f(x)),
        $$
        and the gradient mapping operator
        $$
            \mathcal G_L(x) = L(x - \mathcal T_L(x)). 
        $$
    \end{definition}
    \begin{remark}
        The name ``gradient mapping" comes from \cite[(2.2.54)]{nesterov_lectures_2018}, however, Nesterov was referring to only the case when $g$ is an indicator function of a convex set in his writing. 
        Of course, in Amir Beck \cite[10.3.2]{beck_first-order_nodate}, it has the exact same definition for gradient mapping as the above. 
    \end{remark}

    \begin{lemma}[Gradient Mapping Approximates Subgradient]
        Suppose $h = f + g$ satisfies 
        \hyperref[assumption:smooth-nonsmooth-sum]{assumption \ref{assumption:smooth-nonsmooth-sum}}, 
        let $\mathcal T_L, \mathcal G_L$ be given by 
        \hyperref[def:gradient_mapping]{definition \ref*{def:gradient_mapping}}.
        Then for all $x$, the gradient mapping verifies
        \begin{align*}
            x^+ &= \mathcal T_L(x), 
            \\
            \mathcal G_L(x) = L(x - x^+) &\in  \nabla f(x) + \partial g(x^+). 
        \end{align*}
    \end{lemma}
    \begin{proof}
        Using the resolvent definition of the proximal gradient operator and the fact that the single-valuedness in the convex settings, $x^+$ has relations: 
        \begin{align*}
            x^+ &= [I + L^{-1}\partial g]^{-1}\circ [I - L^{-1}\nabla f](x)
            \\
            [I + L^{-1}\partial g](x^+) &\ni [I - L^{-1}\nabla f](x)
            \\
            x^+ + L^{-1}\partial g(x^+) &\ni x - L^{-1}\nabla f(x)
            \\
            x^+ - x + L^{-1}\partial g(x^+) &\ni L^{-1}\nabla f(x)
            \\
            L(x^+ - x) + \partial g(x^+) &\ni - \nabla f(x)
            \\
            L(x - x^+) &\in \nabla f(x) + \partial g(x^+)
            \\
            \mathcal G_L(x) &\in \nabla f(x) + \partial g(x^+). 
        \end{align*}
    \end{proof}

    \begin{lemma}[Linearized Gradient Mapping Lower Bound]
    \label{lemma:grad_map_linearization}
        Suppose that $h = f + g$ satisfies 
        \hyperref[assumption:smooth-nonsmooth-sum]{assumption \ref*{assumption:smooth-nonsmooth-sum}}, 
        further assume that $f$ is strongly convex with index $\mu \ge 0$. 
        Let $x^+ = \mathcal T_L(x)$ as given in 
        \hyperref[def:gradient_mapping]{definition \ref*{def:gradient_mapping}}. 
        Then for all $z \in \RR$, it satisfies
        \begin{align*}
            h(z) &\ge 
            h(x^+) + \langle \mathcal G_L (x), z - x\rangle + 
            \frac{L}{2}\Vert x - x^+\Vert^2 + \frac{\mu}{2}
            \Vert z - x\Vert^2. 
        \end{align*}
    \end{lemma}
    \begin{proof}
        Using the $L$-smoothness of $f$ and convexity of $g, f$, it has inequalities
        \begin{align*}
            &f(x^+) \le 
            f(x) + \langle \nabla f(x), x^+ - x\rangle
            + \frac{L}{2}\Vert x - x^+\Vert^2, 
            \\
            &
            \frac{\mu}{2}\Vert z - x\Vert^2+ 
            f(x) + \langle \nabla f(x), z - x\rangle 
            \le f(z), 
            \\
            &g(x^+) \le 
            g(z) + \langle \partial g(x^+), x^+ - z\rangle. 
        \end{align*}
        Apply the above by considering the following sequence of relations
        \begin{align*}
            h(x^+) &= f(x^+) + g(x^+)
            \\&
            \begin{aligned}
                &\le 
                \left(
                    f(x) + \langle \nabla f(x), x^+ - x\rangle
                    + \frac{L}{2}\Vert x - x^+\Vert^2
                \right)
                \\
                &\qquad  
                + (g(z) + \langle \partial g(x^+), x^+ - z\rangle)
            \end{aligned}
            \\&
            \begin{aligned}
                &\le 
                \left(
                    f(z) - \langle \nabla f(x), z - x\rangle
                    - \frac{\mu}{2}\Vert z - x\Vert^2
                    + \langle \nabla f(x), x^+ - x\rangle
                    + 
                    \frac{L}{2}\Vert x - x^+\Vert^2
                \right)
                \\
                &\qquad 
                +
                (g(z) + \langle \partial g(x^+), x^+ - z\rangle)
            \end{aligned}
            \\&
            \begin{aligned}
                &= 
                (f(z) + h(z)) 
                \\
                &\qquad 
                + \left(
                    \langle \nabla f(x), x - z\rangle + 
                    \langle \nabla f(x), x^+ - x\rangle + 
                    \langle \partial g(x^+), x^+ - z\rangle
                \right) 
                \\ 
                &\qquad 
                - \frac{\mu}{2}\Vert z - x\Vert^2
                + \frac{L}{2}\Vert x - x^+\Vert^2
            \end{aligned}
            \\& 
            \begin{aligned}
                &= h(z) + 
                \left(
                    \langle \nabla f(x), x - x^+ + x^+ - z\rangle + 
                    \langle \nabla f(x), x^+ - x\rangle + 
                    \langle \partial g(x^+), x^+ - z\rangle
                \right)
                \\
                &\qquad 
                - \frac{\mu}{2}\Vert z - x\Vert^2
                + \frac{L}{2}\Vert x - x^+\Vert^2
            \end{aligned}
            \\& 
            \begin{aligned}
                &= h(z) + 
                \langle \nabla f(x) + \partial g(x^+), x^+ - z\rangle 
                - \frac{\mu}{2}\Vert z - x\Vert^2
                + \frac{L}{2}\Vert x - x^+\Vert^2
            \end{aligned}
            \\& 
            \begin{aligned}
                &= h(z) + 
                    \langle \mathcal G_L(x), x^+ - z\rangle 
                - \frac{\mu}{2}\Vert z - x\Vert^2
                + \frac{L}{2}\Vert x - x^+\Vert^2
            \end{aligned}
            \\& 
            \begin{aligned}
                &= h(z) + \langle L(x - x^+), x^+ - x + x - z\rangle 
                - \frac{\mu}{2}\Vert z - x\Vert^2
                + \frac{L}{2}\Vert x - x^+\Vert^2
            \end{aligned}
            \\&
            \begin{aligned}
                &= h(z) + 
                \underbrace{\langle L(x - x^+), x - z\rangle}_{
                    = - \langle \mathcal G_L (x), z - x\rangle
                }
                - \frac{\mu}{2}\Vert z - x\Vert^2
                - \frac{L}{2}\Vert x - x^+\Vert^2
            \end{aligned}. 
        \end{align*}
        Moving everything except $h(z)$ from the RHS to the LHS yield the desired inequality. 
    \end{proof}
    \begin{remark}
        Observe that the linearization $h(x^+) + \langle \mathcal G_L(x), z - x\rangle$ is anchored at $x^+$, instead of $x$. 
        Geometrically, it's tilted and it ``prefers" the sharp corners of a convex function, if, $x$ is close to a sharp corner. 
        The inequality is analogous to \cite[(2.2.57)]{nesterov_lectures_2018}. 
    \end{remark}

\section{The PPM interpreation of accelerated gradient}\label{sec:ppm_interp_of_ag}
    In this section, we present the PPM formulation of the accelerated proximal gradient method. 
    These interpretations of \cite{ahn_understanding_2022} are extension in the non-smooth context using the proximal gradient mapping operator. 
    The next definition starts the discussion. 
    \begin{definition}[Linear Lower Bounding Function]\label{def:gradmap-linear-lowerbnd-fxn}
        Let $h = f+ g$ satisfies
        \hyperref[assumption:smooth-nonsmooth-sum]{definition \ref*{assumption:smooth-nonsmooth-sum}}, 
        $\mathcal G_L$ given by 
        \hyperref[def:gradient_mapping]{definition \ref*{def:gradient_mapping}}. 
        Define for all $y$ the function
        \begin{align*}
            l_h(x; y) = h(\mathcal T_L y) + \langle \mathcal G_L(y), x - y \rangle 
            + \frac{L}{2}\Vert y - \mathcal T_L y\Vert^2. 
        \end{align*}
    \end{definition} 
    \begin{remark}
        The function satisfies $l_h(x; y) \le h(x)$ for all $x \in \RR, y \in \RR$ by 
        \hyperref[lemma:grad_map_linearization]
        {lemma \ref*{lemma:grad_map_linearization}}. 
    \end{remark}
    \par
    With the above definition, it's possible to formulate a generic form of accelerated gradient method using $l_h(x,y)$ as two proximal point methods anchored at some iterates $x_t, x_{t + 1}$.
    The formulation is generic because of undetermined stepsize parameter $\eta_t, \tilde \eta_t$ from the two proximal point. 
    \par
    We state the PPM interpretation of accelerated gradient in 
    \hyperref[def:ag_prox_grad_ppm]{definition \ref*{def:ag_prox_grad_ppm}}, 
    which has the equivalent form as presented in
    \hyperref[def:ag_prox_grad_generic]
    {definition \ref*{def:ag_prox_grad_generic}}. 
    \hyperref[prop:derive_ag_prox_grad_tript]
    {Proposition \ref*{prop:derive_ag_prox_grad_tript}} 
    shows their equivalence. 
    \par
    Throughout this section, we assume 
    \begin{enumerate}
        \item $h=f + g$ satisfies 
            \hyperref[assumption:smooth-nonsmooth-sum]
            {assumption \ref*{assumption:smooth-nonsmooth-sum}}, 
        \item Using $h$ as given from above, let $\mathcal T_L, \mathcal G_L$ be given by 
            \hyperref[def:gradient_mapping]
            {definition \ref*{def:gradient_mapping}}. 
        \item Using all above, let $l_h$ be given by 
            \hyperref[def:gradmap-linear-lowerbnd-fxn]
            {definition \ref*{def:gradmap-linear-lowerbnd-fxn}}. 
    \end{enumerate}

    \begin{definition}[AG Proximal Gradient PPM Generic Form]
    \label{def:ag_prox_grad_ppm}
        Define $\eta_t, \tilde \eta_t$ to be $> 0$ for all $t \in \N$. 
        With initial iterate $x_0, y_0$, 
        The generic form has iterates $x_t, y_t$ for all $t \in \N$ that satisfy: 
        $$
        \begin{aligned}
            x_{t + 1} &= \argmin_{x} \left\lbrace
                l_h(x; y_t) + \frac{1}{2\tilde \eta_{t + 1}} 
                \Vert x - x_t\Vert^2
            \right\rbrace,
            \\
            y_{t + 1}&= 
            \argmin_{x}
            \left\lbrace
                l_h(x; y_t) + \frac{L}{2}\Vert x - y_t\Vert^2 + 
                \frac{1}{2\eta_{t + 1}} \Vert x - x_{t + 1}\Vert^2
            \right\rbrace.
        \end{aligned}
        $$
        
    \end{definition}

    \begin{definition}[AG Proximal Gradient Generic Form]
    \label{def:ag_prox_grad_generic}
        Define $\eta_t, \tilde \eta_t$ to be $> 0$ for all $t \in \N$. 
        With initial iterate $x_0, y_0$.
        The generic form has iterates $(y_t, x_{t + 1}, z_{t + 1})$ such that 
        $$
        \begin{aligned}
            y_t^+ &= \mathcal T_L(y_t)
            \\
            y_t &= (1 + L\eta_t)^{-1}(x_t + L\eta_t z_t)
            \\
            x_{t + 1} &= x_t - \tilde \eta_{t + 1} \mathcal G_L(y_t)
            \\
            z_{t + 1} &= y_t - L^{-1}\mathcal G_L(y_t)
        \end{aligned}
        $$
        for all $t\in \mathbb N$. 
    \end{definition}
    \begin{remark}
        Observe that $z_{t + 1} = y_t^+$. 
    \end{remark}
    
    \begin{proposition}
    \label{prop:derive_ag_prox_grad_tript}
       We have equalities
        \begin{align*}
            x_{t + 1} &= \argmin_{x}
            \left\lbrace
                l_h(x, y_t) + \frac{1}{2\tilde \eta_{t + 1}} \Vert x - x_t\Vert^2
            \right\rbrace
            \\
            &= x_t - \tilde\eta_{t + 1} \mathcal G_L(y_t), 
            \\
            y_{t + 1} &= \argmin_{x}
            \left\lbrace
                    h(y_t^+) + \langle \mathcal G_L(y_t), x - y_t\rangle + \frac{L}{2}\Vert x -y_t\Vert^2 + \frac{1}{2\eta_{t + 1}}\Vert x - x_{t + 1}\Vert^2
            \right\rbrace
            \\
            &= (1 + L\eta_{t + 1})^{-1}
            (x_{t + 1} + L\eta_{t + 1}(y_t - L^{-1}\mathcal  G_L(y_t))). 
        \end{align*}
        This thows
        \hyperref[def:ag_prox_grad_ppm]{definition \ref*{def:ag_prox_grad_ppm}}
        and 
        \hyperref[def:ag_prox_grad_generic]{definition \ref*{def:ag_prox_grad_generic}} 
        are equivalent. 
    \end{proposition}
    \begin{proof}
        Let $y_t^+ = \mathcal T_L(y_t)$, recall that $l_h(x; y_t) = h(y_t^+) + \langle \mathcal G_L(y_t), x -y_t\rangle \le f(x)$ by 
        \hyperref[lemma:grad_map_linearization]
        {lemma \ref*{lemma:grad_map_linearization}}
        Since $l_h(x; y_t)$ is a simple linear function wrt $x$, minimizing the quandratic where to get $x_{t + 1} = x_t - \tilde\eta_{t + 1} \mathcal G_L(y_t)$; for $y_{t + 1}$, complete the square on the second and the third terms: 
        \begin{align*}
            & \frac{L}{2}\left(
                2\langle L^{-1}\mathcal G_L(y_t), x - y_t\rangle + 
                \Vert x - y_t\Vert^2
            \right)
            \\
            &= 
            \frac{L}{2}
            \left(
                - \Vert L^{-1} \mathcal G_L(y_t)\Vert^2  
                + \Vert L^{-1} \mathcal G_L(y_t)\Vert^2 
                + 
                2\langle L^{-1} \mathcal G_L(y_t), x - y_t\rangle + 
                \Vert x - y_t\Vert^2
            \right)
            \\
            &= \frac{L}{2}\left(
                - \Vert L^{-1}\mathcal G_L(y_t)\Vert^2  
                + \Vert x - (y_t - L^{-1}\mathcal G_L(y_t))
                \Vert^2
            \right), 
        \end{align*}
        therefore it transforms into 
        \begin{align*}
            y_{t + 1} &=\argmin_{x} \left\lbrace
                \frac{L}{2}\left\Vert 
                    x - (y_t - L^{-1}\mathcal G_L(y_t))
                \right\Vert^2
                + \frac{1}{2\eta_{t + 1}}\Vert x - x_{t + 1}\Vert^2
            \right\rbrace
            \\
            &=
            \frac{
                \left(y_t - L^{-1}\mathcal G_L(y_t)\right) + x_{t + 1}
            }{L + \eta_{t + 1}^{-1}}.
        \end{align*}
        Define $z_{t + 1} = y_t - \mathcal G_L(y_t) = \mathcal T_L(y_t)$, which is the proximal gradient set, then the above expression simplifies to 
        $$
        y_{t + 1} = (1 + L\eta_{t +1})^{-1}(x_{t + 1}+ L\eta_{t + 1}z_{t + 1}). 
        $$
    \end{proof}
    \begin{remark}
        $y_{t + 1}$ is the minimizer of a simple quadratic. 
        Given that the original function $h$ is potentially non-smooth, therefore it's not always an upper bound of $h(x)$. 
        The upper bound interpretation of the smooth case as proposed by Ahn \cite{ahn_understanding_2022}, Sra for the update of $y_{t + 1}$ fails when $h$ is non-smooth! 
    \end{remark}

\section{Generic Lyapunov analysis for Accelerated gradient via PPM}
\label{sec:generic_ag_ppm_lyapunov_analysis}
    In this section we derive the generic convergence rate formuated by $\eta_i, \tilde \eta_i$. 
    \par
    The Lypunov function in 
    \hyperref[thm:generic_ag_convergence]{\ref*{thm:generic_ag_convergence}}
    is identical to Ahn and Sra 
    \cite[section 4.2]{ahn_understanding_2022}, 
    except for the involvement of the gradient mapping. 
    We adapated it into the context of proximal gradient by 
    \hyperref[lemma:nsmooth_agg_lyapunov_upper_bound]
    {lemma \ref*{lemma:nsmooth_agg_lyapunov_upper_bound}}
    \par
    We present the convergence rate by compatifying results into 
    \hyperref[lemma:nsmooth_agg_lyapunov_upper_bound]
    {lemma \ref*{lemma:nsmooth_agg_lyapunov_upper_bound}} 
    to assist the proof of  
    \hyperref[thm:generic_ag_convergence]
    {Theorem \ref*{thm:generic_ag_convergence}} 
    which states new results that are extension to Ahn and Sra's works by introducing an error term to the Lypunov analysis, allowing for a weaker constraints for the stepsize paramters $\tilde \eta_i, \eta_i$ from 
    \hyperref[def:ag_prox_grad_generic]{definition \ref*{def:ag_prox_grad_generic}}. 
    \hyperref[thm:ag_generic_stepsize_constraints]{Theorem \ref*{thm:ag_generic_stepsize_constraints}}
    states the weakened constraints on the stepsize parameters $\tilde \eta_i, \eta_i$. 
    \par
    Through out this section, we continue the assumption for $h = f + g$ and gradient mapping $\mathcal G_L$ the same as previous section. 

    \begin{lemma}[Lyapunov Inequality]. 
    \label{lemma:nsmooth_agg_lyapunov_upper_bound}\;\\
        Fix any $\bar x, x, \tilde\eta > 0$, define
        \begin{align*}
            \phi(u) &:= \tilde\eta
            \left(
                h( \mathcal T_L\bar x) + \langle \mathcal G_L\bar x, u - \bar x\rangle
                + \frac{L}{2}\Vert \bar x - \mathcal T_L \bar x\Vert^2 
            \right), 
            \\
            x^+ &:= 
            \hprox_\phi(x) = x - \tilde \eta\mathcal G_L \bar x. 
        \end{align*}
        with $x_*$ being the minimizer of $h$, it has 
        \begin{align*}
            \Upsilon_{1}  &:= 
            \tilde \eta(h(\mathcal T_L \bar x) - h(x_*))
            + 
            \frac{1}{2}(\Vert x^+ - x_*\Vert^2 - \Vert x - x_*\Vert^2)
            \\
            &\quad \le 
            - \tilde \eta \langle \mathcal G_L\bar x, x^+ - \mathcal T_L \bar x\rangle
            + 
            \frac{\tilde \eta L}{2} \Vert \bar x - \mathcal T_L \bar x\Vert^2
            - 
            \frac{1}{2}\Vert x^+ - x\Vert^2, 
            \\
            \forall z':
            \Upsilon_2 &:= 
            h(\mathcal T_L \bar x) - h(z') 
            \le 
            \langle \mathcal G_L\bar x, \mathcal T_L \bar x - z'\rangle + 
            \frac{L}{2}\Vert \mathcal T_L \bar x - \bar x\Vert^2. 
        \end{align*}
    \end{lemma}
    \begin{observation}
        Function $\phi$ is a linear function that qualifies as a lower bound of $\tilde \eta h$, anchored at $\bar x$ by 
        \hyperref[lemma:grad_map_linearization]{lemma \ref*{lemma:grad_map_linearization}}.
    \end{observation}
    \begin{proof}
        Directly observe that we have 
        \begin{align*}
            \quad &
            h(\mathcal T_L\bar x) + 
            \langle \mathcal G_L \bar x, u - \bar x\rangle + 
            \frac{L}{2}\Vert \bar x - \mathcal T_L \bar x\Vert^2
            \\
            &= 
            h(\mathcal T_L \bar x) + 
            \langle \mathcal G_L \bar x, u - \mathcal T_L \bar x\rangle
            + 
            \langle \mathcal G_L \bar x, \mathcal T_L \bar x - \bar x\rangle
            + 
            \frac{L}{2}\Vert \bar x - \mathcal T_L \bar x \Vert^2
            \\
            &= 
            h(\mathcal T_L \bar x) + 
            \langle \mathcal G_L \bar x, u - \mathcal T_L \bar x\rangle
            + 
            \langle 
                L (\bar x - \mathcal T_L \bar x)
                , 
                \mathcal T_L \bar x - \bar x
            \rangle
            + 
            \frac{L}{2}\Vert \bar x - \mathcal T_L \bar x\Vert^2
            \\
            &= 
            h(\mathcal T_L \bar x) + 
            \langle \mathcal G_L \bar x, u - \mathcal T_L \bar x\rangle 
            - \frac{L}{2} \Vert \bar x - \mathcal T_L \bar x\Vert^2. 
        \end{align*}
        By PPM descent inequality
        \hyperref[thm:ppm_descent_ineq]{theorem \ref*{thm:ppm_descent_ineq}}
        , let $x_*$ be a minimizer, it claims that 
        {\small
        \begin{align*}
            & 
            \phi(x^+) - \phi (x^*) + 
            \frac{1}{2}\left(
                \Vert x^+ - x_*\Vert^2 - 
                \Vert x - x_*\Vert^2
            \right)
            \le 
            - \frac{1}{2}\Vert x^+ - x\Vert^2
            \\
            \implies &
            \tilde 
            \eta 
            \left(   
                h(\mathcal T_L \bar x) + 
                \langle \mathcal G_L \bar x, u - \mathcal T_L \bar x\rangle 
                - \frac{L}{2} \Vert \bar x - \mathcal T_L \bar x\Vert^2
            \right)
            - \tilde \eta h (x_*)
            +
            \frac{1}{2}\left(
                \Vert x^+ - x_*\Vert^2 - 
                \Vert x - x_*\Vert^2
            \right)
            \\
            &\le 
            - \frac{1}{2}\Vert x^+ - x\Vert^2
            \\
            \iff &
            \tilde \eta (h(\mathcal T_L\bar x) - h(x_*))  
            + 
            \frac{1}{2}\left(
                \Vert x^+ - x_*\Vert^2 - 
                \Vert x - x_*\Vert^2
            \right)
            \\
            & \le 
            - \tilde \eta \langle \mathcal G_L\bar x, x^+ - \mathcal T_L \bar x\rangle
            + 
            \frac{\tilde \eta L}{2} \Vert \bar x - \mathcal T_L \bar x\Vert^2
            - 
            \frac{1}{2}\Vert x^+ - x\Vert^2. 
        \end{align*}
        }
        Next, for all $z, z'$, it would have by smoothness of $f$: 
        \begin{align*}
            f(z) - f(z') &= 
            f(z) - f(\bar x) + f(\bar x) - f(z')
            \\
            &\le 
            \langle 
                \nabla f(\bar x), z - \bar x
            \rangle + 
            \frac{L}{2}\Vert z - \bar x\Vert^2 
            + 
            \langle 
                \nabla f(\bar x), 
                \bar x - z'
            \rangle
            \\
            &= 
            \langle \nabla f(\bar x), z - z'\rangle
            + 
            \frac{L}{2}\Vert z - \bar x\Vert^2. 
        \end{align*}
        The convexity of $g$ yields: 
        \begin{align*}
            g(z) + 
            \langle 
                \partial g(z), z' - z
            \rangle 
            &\le g(z')
            \\
            g(z) - g(z') 
            &\le 
            \langle \partial g(z), z - z'\rangle. 
        \end{align*}
        Adding them yield 
        \begin{align*}
            h(z) - h(z') &\le 
            \langle \nabla f(\bar x) + \partial g(z), z - z'\rangle + 
            \frac{L}{2}\Vert z - \bar x\Vert^2. 
        \end{align*}
        Setting $z = \mathcal T_L \bar x$, we have the desired results. 
    \end{proof}
    \begin{remark}
    \label{remark:upsilon-upperbound-for-iterates}
        Let $y_t, x_t$ be given by
        \hyperref[def:ag_prox_grad_generic]
        {definition \ref*{def:ag_prox_grad_generic}}.
        Set $\bar x = y_t, x = x_t$, then 
        \begin{align*}
            x^+ &= x_t - \mathcal G_L y_t = x_{t + 1},
            \\
            z_{t + 1} &= \mathcal T_L y_t = \mathcal T_L \bar x.
        \end{align*}
        Set $z' = z_t, \tilde \eta = \tilde \eta_{t + 1}$, 
        the results can be written as 
        \begin{align*}
            \Upsilon_{1, t + 1}^{\text{AG}}
            &:= 
            \tilde \eta_{t + 1}(h(z_{t + 1}) - h(x_*))
            + 
            \frac{1}{2}(\Vert x_{t + 1} - x_*\Vert^2 + \Vert x_t - x_*\Vert^2)
            \\
            &\quad \le 
            - \tilde \eta_{t + 1}\langle \mathcal G_L y_t, x_{t + 1} - z_{t + 1}\rangle
            + 
            \frac{\tilde \eta_{t + 1}L}{2}
            \Vert  
                y_t - z_{t + 1}
            \Vert^2
            - 
            \frac{1}{2}
            \Vert x_{t + 1} - x_t\Vert^2,  
            \\ 
            \Upsilon_{2, t + 1}^{\text{AG}}
            &:= 
            h(z_{t + 1}) - h(z_t) \le 
            \langle \mathcal G_L \bar x, z_{t + 1} - z_t \rangle + 
            \frac{L}{2}\Vert z_{t + 1} - y_t\Vert^2. 
        \end{align*}
        It will be used next. 
    \end{remark}

    \begin{theorem}[Generic AG Convergence]
    \label{thm:generic_ag_convergence}
        Let $\Upsilon_{1, t + 1}^\text{AG}, \Upsilon_{2, t + 1}^\text{AG}$ be given by 
        \hyperref[remark:upsilon-upperbound-for-iterates]
        {remark \ref*{remark:upsilon-upperbound-for-iterates}}. 
        Let $z_t$ be given by 
        \hyperref[def:ag_prox_grad_generic]
        {definition \ref*{def:ag_prox_grad_generic}}
        Define the Lyapunov function $\Phi_t\; \forall t \in \{0\}\cup \N$, $S_t$ and $\sigma_t$: 
        \begin{align*}
            \Phi_t &:= \left(
                \sum_{i = 1}^{t} \tilde\eta_{i}
            \right) (h(z_t) - h(x_*)) + \frac{1}{2}\Vert x_t - x_*\Vert^2 \quad \forall t \in \N
            \\
            \Phi_0 &:= \frac{1}{2}\Vert x_0 - x_*\Vert^2, 
            \\
            S_t &:= \sum_{i = 1}^{t} \delta_i \quad \forall t \in \N,
            \\
            \sigma_t &:= \sum_{i = 1}^{t}\tilde \eta_i \quad \forall t \in \N. 
        \end{align*}
        If there exists a sequence of $\eta_i, \tilde \eta_i$ as defined in
        \hyperref[def:ag_prox_grad_ppm]
        {definition \ref*{def:ag_prox_grad_ppm}} 
        such that
        \begin{align*}
            & \Phi_{t + 1} - \Phi_{t} =
            \left(
                \sum_{i = 1}^{t} \tilde \eta_i
            \right)\Upsilon_{2, t + 1}^{\text{AG}} 
            + 
            \Upsilon_{1, t + 1}^{\text{AG}} 
            \le \delta_{t + 1} \quad 
            \forall t \in \mathbb N, 
            \\
            & \Upsilon_{1, 1}^{\text{AG}} \le \delta_1. 
        \end{align*}
        for some $\delta_i, i \in \N$. 
        Then 
        \begin{align*}
            h(z_T) - h(x_*) &\le 
            \sigma_T^{-1}\left(
                S_{T} + \frac{1}{2}\Vert x_0 - x_*\Vert^2
            \right). 
        \end{align*}
        Where $x_*$ is an minimizer of $h$. 
        So $h(z_T) - h(x_*)$ has convergence rate $\mathcal O(\sigma_T^{-1})$ when $S_T \le 0$, and $\mathcal O(\sigma_T^{-1}S_T)$ when $S_T > 0$. 
    \end{theorem}
    \begin{proof}
        By definition we have
        {\footnotesize
        \begin{align*}
            \Phi_{t + 1} - \Phi_t 
            &= 
            \left(
                \sum_{i = 1}^{t+1} \tilde\eta_{i}
            \right) (h(z_{t + 1}) - h(x_*)) 
            - 
            \left(
                \sum_{i = 1}^{t} \tilde\eta_{i}
            \right) (h(z_{t}) - h(x_*)) 
            + \frac{1}{2}\Vert x_t - x_*\Vert^2
            - \frac{1}{2}\Vert x_{t + 1} - x_*\Vert^2
            \\
            &= 
            \tilde \eta_{t + 1} (h(z_{t + 1}) - h(z_*))
            +
            \left(
                \sum_{i = 1}^{t} \tilde \eta_i
            \right)(h(z_{t + 1}) - h(z_t))
            + \frac{1}{2}\Vert x_t - x_*\Vert^2
            - \frac{1}{2}\Vert x_{t + 1} - x_*\Vert^2
            \\
            &= \left(
                \sum_{i = 1}^{t} \tilde \eta_i
            \right)\Upsilon_{2, t + 1}^{\text{AG}} + \Upsilon_{1, t + 1}^{\text{AG}} \le \delta_{t + 1}. 
        \end{align*}
        }
        Telescoping for $t = 0, \cdots, T- 1$
        \begin{align*}
            \Phi_T - \Phi_0 = 
            \sum_{i = 0}^{T - 1}\Phi_{i + 1} - \Phi_i 
            &\le 
            \sum_{i = 0}^{T - 1}\delta_{i + 1}
            = S_{T}. 
        \end{align*}
        So then $\Phi_T - \Phi_0$ yields: 
        $$
        \begin{aligned}
            \sigma_T (h(z_T) - h(x_*)) 
            + \frac{1}{2}\Vert x_t - x_*\Vert^2 
            - \frac{1}{2}\Vert x_0 - x_*\Vert^2 
            &\le S_{T}
            \\
            \implies 
            \sigma_T(h(z_T) - h(x_*))
            &\le 
            S_T + \frac{1}{2}\Vert x_0 - x_*\Vert^2
            \\
            h(z_T) - h(x_*) &\le 
            \sigma_T^{-1}\left(
                S_{T} + \frac{1}{2}\Vert x_0 - x_*\Vert^2
            \right),
        \end{aligned}
        $$
        which yields a convergence rate $\mathcal O(\sigma_T^{-1}S_{T})$. 
        When $S_T = 0$, the convergence rate is $O(\sigma_T^{-1})$ instead. 
    \end{proof}

    \begin{theorem}[Constraints of PPM stepsize sequence]
    \label{thm:ag_generic_stepsize_constraints}\;\\
        Let iterates: $z_t, x_t, y_t$. 
        \hyperref[def:ag_prox_grad_generic]
        {definition \ref*{def:ag_prox_grad_generic}}. 
        If the stepsize $\eta_i, \tilde \eta_i$ satisfies relations 
        \begin{align*}
            \begin{cases}
                \tilde \eta_{t + 1} (\tilde \eta_{t + 1} - L^{-1})
                - L^{-1} \sum_{i= 1}^{t}\tilde \eta_i 
                = 
                \epsilon_{t + 1} \tilde \eta_{t + 1}
                & \forall t \in \mathbb N, 
                \\
                L \eta_t \tilde \eta_{t + 1} = \sum_{i=1}^{t}\tilde \eta_i 
                & \forall t \in \mathbb N. 
            \end{cases}
        \end{align*}
        Then for all $t \in \N$
        $$
            \epsilon_{t + 1} = \tilde \eta_{t + 1} - \eta_t - L^{-1}, 
        $$
        and 
        \begin{align*}
            \Phi_{t + 1} - \Phi_t =
            \Upsilon_{1, t + 1}^\text{AG} + 
            \sigma_t\Upsilon_{1, t + 1}^{\text{AG}} 
            &\le \epsilon_{t + 1}\tilde\eta_{t + 1} \Vert \mathcal G_L(y_t)\Vert^2 \le \delta_{t + 1}.
        \end{align*}
    \end{theorem}
    \begin{proof}
        With $t \in \mathbb N \cup \{0\}$ fixed, 
        recall that for the proximal gradient PPM generic form 
        (as formulated in 
        \hyperref[def:ag_prox_grad_generic]{definition \ref*{def:ag_prox_grad_generic}}
        ) for $t\in \mathbb N$ it has: 
        \begin{align*}
            y_t &= (1 + L\eta_t)^{-1}(x_t + L\eta_t z_t)
            \\
            x_{t + 1} &= x_t - \tilde \eta_{t + 1} \mathcal G_L(y_t)
            \\
            z_{t + 1} &= y_t - L^{-1}\mathcal G_L(y_t). 
        \end{align*}
        Recall the upper bounds from 
        \ref*{thm:generic_ag_convergence}, 
        it has 
        \begin{align*}
            \Upsilon_{1, t + 1}^\text{AG}
            &= 
            \tilde\eta_{t + 1} (h(z_{t + 1}) - h(x_*)) + 
            \frac{1}{2} (
                \Vert x_{t + 1} - x_*\Vert^2
                - 
                \Vert x_t - x_*\Vert^2
            )
            \\
            &\le 
            - \frac{1}{2}\Vert x_{t + 1} - x_t\Vert^2 
            + \frac{\tilde\eta_{t + 1}L}{2}\Vert z_{t + 1} - y_t\Vert^2
            - \langle 
                \tilde\eta_{t + 1} \mathcal G_L(y_t), 
                x_{t + 1} - z_{t + 1}
            \rangle
            \\
            \Upsilon_{2, t + 1}^\text{AG}
            &= 
            h(z_{t + 1}) - h(z_t) 
            \le 
            \langle \mathcal G_L(y_t), z_{t + 1} - z_t\rangle + 
            \frac{L}{2}\Vert z_{t + 1} - y_t\Vert^2. 
        \end{align*}
        By the updates, vector $x_{t + 1} - x_t$ and $z_{t + 1} - y_t$ are parallel by observations: 
        \begin{align*}
            x_{t + 1} - x_t &= -\tilde\eta_{t + 1}\mathcal G_L(y_t), 
            \\
            z_{t + 1} - y_t &= -L^{-1}\mathcal G_L(y_t). 
        \end{align*}
        This allows for 
        \begin{align*}
            \Upsilon_{1, t + 1}^{\text{AG}} 
            &\le 
            - \frac{1}{2}\Vert x_{t + 1} - x_t\Vert^2 + 
            \frac{\tilde\eta_{t + 1}L}{2}\Vert z_{t + 1} - y_t\Vert^2 
            - 
            \langle \tilde\eta_{t + 1}\mathcal G_L (y_t), x_{t + 1} - z_{t + 1} \rangle
            \\
            &= 
            - \frac{1}{2}\Vert \tilde\eta_{t + 1} \mathcal G_L(y_t)\Vert^2 + 
            \frac{\tilde\eta_{t + 1}L}{2}\Vert L^{-1} \mathcal G_L(y_t)\Vert^2
            - 
            \langle \tilde\eta_{t + 1} \mathcal G_L(y_t), x_{t + 1} - z_{t + 1} \rangle
            \\
            &= 
            \frac{1}{2}\left(
                - \tilde\eta_{t + 1}^2 + 
                L^{-1}\tilde\eta_{t + 1}
            \right)\Vert \mathcal G_L(y_t)\Vert^2
            - 
            \langle 
                \tilde\eta_{t + 1} \mathcal G_L(y_t), 
                (x_{t + 1} - x_{t}) + x_t
                + (y_t - z_{t + 1}) - y_t
            \rangle
            \\
            &= 
            \frac{1}{2}\left(
                L^{-1}\tilde\eta_{t + 1}
                - \tilde\eta_{t + 1}^2
            \right)\Vert \mathcal G_L(y_t)\Vert^2
            - 
            \langle 
                \tilde\eta_{t + 1} \mathcal G_L(y_t), 
                -\tilde\eta_{t + 1}\mathcal G_L(y_t) + x_t 
                + L^{-1}\mathcal G_L(y_t) - y_t
            \rangle
            \\
            &= 
            \frac{1}{2}\left(
                L^{-1}\tilde\eta_{t + 1}
                - \tilde\eta_{t + 1}^2
            \right)\Vert \mathcal G_L(y_t)\Vert^2
            - \langle 
                \tilde\eta_{t +1}\mathcal G_L(y_t), 
                (L^{-1} - \tilde\eta_{t + 1})\mathcal G_L(y_t) + x_t - y_t
            \rangle
            \\
            &= \frac{1}{2}\left(
                L^{-1}\tilde\eta_{t + 1} - \tilde\eta_{t + 1}^2 
                + 2 \tilde\eta_{t + 1}^2 - 2\tilde\eta_{t + 1}L^{-1}
            \right)\Vert \mathcal G_L(y_t)\Vert^2
            - 
            \langle 
                \tilde\eta_{t + 1} \mathcal G_L(y_t), 
                x_t - y_t
            \rangle
            \\
            &= 
            \frac{1}{2}\left(
                \tilde\eta_{t + 1}^2 - \tilde\eta_{t + 1}L^{-1}
            \right)\Vert \mathcal G_L(y_t)\Vert^2 
            + \langle \tilde\eta_{t + 1} \mathcal G_L(y_t), y_t - x_t\rangle.
        \end{align*}
        Similarly 
        \begin{align*}
            \Upsilon_{2, t + 1}^{\text{AG}} 
            &= 
            \langle \mathcal G_L(y_t), z_{t + 1} - z_t\rangle + 
            \frac{L}{2}\Vert z_{t + 1} - y_t\Vert^2
            \\
            &= 
            \langle \mathcal G_L(y_t), z_{t + 1} - y_t + y_t - z_t\rangle
            + \frac{L}{2}\Vert z_{t + 1} - y_t\Vert^2
            \\
            &= 
            \langle \mathcal G_L(y_t), - L^{-1} \mathcal G_L(y_t) + y_t - z_t\rangle
            + 
            \frac{L}{2}\Vert L^{-1}\mathcal G_L(y_t)\Vert^2
            \\
            &= 
            -L^{-1}\Vert \mathcal G_L(y_t)\Vert^2 
            + 
            (1/2)L^{-1}\Vert \mathcal G_L(y_t)\Vert^2 
            + 
            \langle \mathcal G_L(y_t), y_t - z_t\rangle
            \\
            &= 
            -(1/2)L^{-1}\Vert \mathcal G_L(y_t)\Vert^2
            + 
            \langle \mathcal G_L(y_t), y_t - z_t\rangle. 
        \end{align*}
        Observe that the cross product term for $\Upsilon_{1, t + 1}^\text{AG}, \Upsilon_{2, t + 1}^\text{AG}$ doesn't match. 
        Hence let's consider the update for $y_t$, which can be written as $y_t - x_t = L \eta_t (z_t - y_t)$. We make the choice to do surgery on upper bound of $\Upsilon_{2, t + 1}^\text{AG}$, so $\langle \mathcal G_L(y_t), y_t - x_t\rangle = \langle \mathcal G_L(y_t), L \eta_t (z_t - y_t)\rangle$. 
        With this in mind, RHS of $\phi_{t + 1} - \phi_t$ yields: 
        {\footnotesize
        \begin{align*}
            &\Upsilon_{1, t + 1}^\text{AG} + 
            \left(
                \sum_{i = 1}^{t}\tilde\eta_i 
            \right)\Upsilon_{1, t + 1}^{\text{AG}}
            \\
            &\le 
            \frac{1}{2}\left(
                \tilde\eta_{t + 1}^2 - \tilde\eta_{t + 1}L^{-1}
            \right)\Vert \mathcal G_L(y_t)\Vert^2 
            + 
            \langle \tilde\eta_{t + 1} \mathcal G_L(y_t), L\eta_t(z_t - y_t)\rangle
            \\ 
            &\quad 
            + 
            \left(
                \sum_{i = 1}^{t}\tilde\eta_i 
            \right)\left(
                -(1/2)L^{-1}\Vert \mathcal G_L(y_t)\Vert^2
                + 
                \langle \mathcal G_L(y_t), y_t - z_t\rangle
            \right)
            \\
            &= 
            \left(
                \frac{1}{2}\tilde\eta_{t + 1}\left(
                    \tilde \eta_{t +1} - L^{-1}
                \right)
                - 
                \frac{1}{2L}\sum_{i = 1}^{t}\tilde \eta_i
            \right)\Vert \mathcal G_L(y_t)\Vert^2 + 
            \left(
                L\eta_t \tilde \eta_{t + 1} - \sum_{i = 1}^{t}\tilde \eta_i
            \right)\langle \mathcal G_L(y_t), z_t - y_t\rangle. 
        \end{align*}
        }
        The non-negativity of $\Vert \mathcal G_L(y_t) \Vert^2$ characterize the culmulative error $\delta_{t + 1}$ in the Lypunov analysis through sequence $\epsilon_i$. 
        Setting the coefficient of $\Vert \mathcal G_L(y_t) \Vert^2$ to be $\epsilon_{t + 1}\tilde \eta_{t + 1}$, it yields a system of inequality: 
        \begin{align*}
            \begin{cases}
                \tilde \eta_{t + 1} (\tilde \eta_{t + 1} - L^{-1})
                - L^{-1} \sum_{i= 1}^{t}\tilde \eta_i 
                = 
                \epsilon_{t + 1} \tilde \eta_{t + 1}
                & \forall t \in \N, 
                \\
                L \eta_t \tilde \eta_{t + 1} = \sum_{i=1}^{t}\tilde \eta_i 
                & \forall t \in \N. 
            \end{cases}
        \end{align*}
        It requires base case $L\eta_0\tilde\eta_{1} = 0$, assume $\sigma_0 = 0$. 
        We use $\sum_{i = 1}^t \tilde \eta_i = \sigma_t$, simplifying the first equation we have 
        \begin{align*}
            \tilde \eta_{t + 1} (\tilde \eta_{t + 1} - L^{-1})
            - L^{-1} \sigma_t
            &= 
            \epsilon_{t + 1} \tilde \eta_{t + 1}
            \\
            \iff 
            \tilde \eta_{t + 1} ^2 - L \tilde \eta_{t + 1} 
            &= 
            \epsilon_{t + 1} \tilde \eta_{t + 1} + L^{-1} \sigma_t
            \\
            &= 
            \epsilon_{t + 1} \tilde \eta_{t + 1} 
            + L^{-1}(L \eta_t \tilde \eta_{t + 1})
            \\
            \iff 
            \tilde \eta_{t + 1} &= \epsilon_{t + 1} + \eta_t + L^{-1}. 
        \end{align*}
        At the last step, we divided both side of the equation by $\tilde \eta_{t + 1} > 0$.
        The parameter gives relation $\epsilon_{t+1} = \tilde \eta_{t + 1} - \eta_t - L^{-1}$. 
        Hence, it gives us the following system of equality to work with 
        \begin{align*}
            \forall t \in \N: 
            \begin{cases}
                \tilde \eta_{t + 1} = \epsilon_{t + 1} + \eta_t + L^{-1}, 
                \\
                L \eta_t \tilde \eta_{t + 1} = \sigma_t.     
            \end{cases}
        \end{align*}
        With that, we can solve a relation between $\eta_{t + 1}$ in terms of the sequence $\epsilon$ and $\eta_t$.
        Consider the equality 
        \begin{align*}
            L \sigma_{t + 1} &= L \tilde \eta_{t + 1} + L \sigma_t
            \\
            &=
            L \tilde \eta_{t + 1}  + L (L \eta_t \tilde \eta_{t + 1})
            \\
            &= L \tilde \eta_{t+ 1} + L \eta_t (L \tilde \eta_{t + 1})
            \\
            &=  L \tilde \eta_{t+ 1} + L \eta_t (L \epsilon_{t + 1} + L \eta_t + 1)
            \\
            &=  L \tilde \eta_{t+ 1} + L \eta_t (L \eta_t + 1) + L^2 \eta_t \epsilon_{t + 1}
            \\
            &= L (\epsilon_{t +1} + \eta_t + L^{-1}) + L\eta_t(L \eta_t + 1) + L^2\eta_t \epsilon_{t + 1}
            \\
            &= L \epsilon_{t + 1} + (L\eta_t + 1)^2 + L^2\eta_t \epsilon_{t + 1}
            \\
            &= 
            L \epsilon_{t + 1}(1 + L \eta_t) + (1 + L \eta_t)^2. 
        \end{align*}
        At the same time we have 
        \begin{align*}
            L \sigma_{t + 1} &= L^2 \eta_{t + 1}\tilde \eta_{t + 1} 
            \\
            &= L\eta_{t + 1}(1 + L \eta_{t + 1} + \epsilon_{t + 2})
            \\
            &= L\eta_{t + 1}(1 + L \eta_{t + 1}) + \epsilon_{t + 2}L\eta_{t + 1}. 
        \end{align*}
        Therefore, it generates the following equation: 
        \begin{align*}
            L\eta_{t + 1} (1 + L \eta_{t + 1}) 
            + 
            \epsilon_{t + 2} L \eta_{t + 1} 
            &= 
            L\epsilon_{t + 1}(1 + L \eta_t)  + (1 + L\eta_t)^2
            \\
            (L\eta_{t + 1} + L^2\eta_{t + 1}^2)
            + 
            \epsilon_{t + 2} L \eta_{t + 1} 
            + 
            \frac{1}{4}
            &= 
            L\epsilon_{t + 1}(1 + L \eta_t)  + (1 + L\eta_t)^2
            + 
            \frac{1}{4}
            \\
            (L\eta_{t + 1} + L^2\eta_{t + 1}^2 + 1/4)
            + 
            \epsilon_{t + 2} L \eta_{t + 1} 
            + \epsilon_{t + 2}
            &= 
            L\epsilon_{t + 1}(1 + L \eta_t)  + (1 + L\eta_t)^2
            + \frac{1}{4}
            + \epsilon_{t + 2}
            \\
            (L\eta_{t + 1} + 1/2)^2 + \epsilon_{t + 2}(L \eta_{t + 1} + 1)
            &= 
            L \epsilon_{t + 1}(1 + L \eta_t) + (1 + L\eta_t)^2
            + \frac{1}{4} + \epsilon_{t + 2}
            \\
            \text{ with: } & a_t = 1 + L \eta_t = \tilde \eta_{t + 1} - \epsilon_{t + 1}
            \\
            (a_{t + 1} - 1/2)^2 + \epsilon_{t + 2}a_{t + 1}
            &= 
            L \epsilon_{t + 1}a_t + a_t^2 + 1/4 + \epsilon_{t + 1}
            \\
            a_{t + 1}^2 + 1/4 - a_{t + 1} + \epsilon_{t + 2}a_{t + 1}
            &= 
            L \epsilon_{t + 1}a_t + a_t^2 + 1/4 + \epsilon_{t + 1}
            \\
            a_{t + 1}^2 + a_{t + 1}(\epsilon_{t + 2} - 1)
            &= 
            \underbrace{
                a_t(L \epsilon_{t + 1} + a_t) + \epsilon_{t + 1}
            }_{c_{t + 1}}. 
        \end{align*}
        Solving reveals the relations: 
        \begin{align*}
            \begin{cases}
                a_{t + 1} = (1/2)\left(
                1 - \epsilon_{t + 2} + \sqrt{(1 - \epsilon_{t + 2}) + 4 c_{t + 1}}
                \right), 
                \\
                c_{t + 1} = a_t (L \epsilon_{t + 1} + a_t) + \epsilon_{t + 1}. 
            \end{cases}
        \end{align*}
        Observe that in the case where we choose $\epsilon_t = 0\; \forall t \in \N$, the above relation simplifies to 
        \begin{align*}
            a_{t + 1} &= (1/2)\left(
                1 + \sqrt{1 + 4 c_{t + 1}}
            \right), 
            \\
            c_{t + 1} &= a_t^2. 
        \end{align*}
        This relation is the Famous Nesterov momentum sequence. 
        At the same time, we can analyize the convergence rate of the algorithm by the abstract convergence lemma, producing: 
        \begin{align*}
            \Phi_{t + 1} - \Phi_t =
            \Upsilon_{1, t + 1}^\text{AG} + 
            \sigma_t\Upsilon_{1, t + 1}^{\text{AG}} 
            &\le \epsilon_{t + 1}\eta_{t + 1} \Vert \mathcal G_L(y_t)\Vert^2 \le \delta_{t + 1}
        \end{align*}
        Telescoping yields: 
        \begin{align*}
            S_{T} = 
            \sum_{i = 0}^{T- 1} \delta_i 
            &= 
            \sum_{i = 0}^{T - 1} \epsilon_{i + 1}\tilde\eta_{i + 1}\Vert \mathcal G_L(y_i)\Vert^2
            \\
            &\le \sum_{i = 0}^{T - 1}\max(\epsilon_{i + 1} \tilde\eta_{i + 1}\Vert \mathcal G_L(y_i)\Vert^2, 0). 
        \end{align*}
        Under an ideal case where we wish to attain accelerations, we want $\lim_{T \rightarrow \infty} S_T < \infty$. 
        One way to accomplish is choose the error sequence $\epsilon_i, i \in \N$ to be bounded by for all $i \in \N$, $\epsilon_i$ should satisfy: 
        \begin{align*}
            \epsilon_{i + 1}\tilde \eta_{i + 1}
            \Vert \mathcal G_L(y_i)\Vert^2 
            &\le \delta_{i + 1}
            \\
            \iff 
            \epsilon_{i + 1}
            &\le 
            \frac{\delta_{i + 1} }{\tilde\eta_{t + 1}\Vert \mathcal G_L(y_i)\Vert^2}. 
        \end{align*}
        for any $\sum_{i = 0}^{T - 1}\delta_{i + 1}$ converges to a limit as $T \rightarrow \infty$. 

    \end{proof}

\section{Recovering existing variants of Nesterov accelerated gradient}\label{sec:recovery}
    \subsection{Recovering FISTA}
        From 
        \hyperref[thm:ag_generic_stepsize_constraints]
        {theorem \ref*{thm:ag_generic_stepsize_constraints}}, 
        setting $\epsilon_i = 0$ for all $i$ yields $a_t = 1 + L\eta_t = \tilde \eta_{t + 1}$, for all $t \in \N$ where $a_t$ is the Nesterov sequence.
        In this case, there exists a momentum form of the generic form and it reduces to FISTA from Beck and Toubolle, or Similar Triangle Method as referred to by Ahn and Sra. 

        \begin{lemma}[Deriving Similar Triangle Form I]
            \quad \\
            With the choice of stepszie $\tilde \eta_{t + 1} = \eta_t + L^{-1}$ 
            in 
            \hyperref[def:ag_prox_grad_generic]{definition \ref*{def:ag_prox_grad_generic}}
            \begin{align*}
                z_{t + 1} &= y_t - L^{-1} \mathcal G_L(y_t)
                \\
                x_{t + 1} &= z_{t + 1} + L\eta_t (z_{t + 1} - z_t)
                \\
                y_{t + 1} &= 
                (1 + L\eta_{t + 1})^{-1}
                (
                x_{t + 1} + L\eta_{t + 1}z_{t + 1}
                ). 
            \end{align*}
            It also has an equivalent momentum form 
            \begin{align*}
                z_{t + 1} &= y_t - L^{-1}\mathcal G_L(y_t)
                \\
                y_{t + 1} &= z_{t + 1} + (1 + L\eta_{t + 1})^{-1}L\eta_t (z_{t + 1} - z_t). 
            \end{align*}
        \end{lemma}
        \begin{proof}
            To do, we show that updates sequence $x_{t + 1} = z_{t + 1} + L\eta_t (z_{t + 1} - z_t)$ is equivalent to $x_{t + 1} = x_t + \tilde\eta_{t + 1}\nabla f(y_t)$. 
            Starting with the former, susbtitute definition of $z_{t + 1}$, $z_{t + 1} = z_t + L^{-1}\nabla f(y_t)$ expanding: 
            \begin{align*}
                x_{t + 1} &= y_t - L^{-1}\mathcal G_L(y_t) 
                + L \eta_t y_t - \eta_t \mathcal G_L(y_t) - L\eta_t z_t
                \\
                &= 
                (1 + L\eta_t)y_t - (\eta_t + L^{-1})\mathcal G_L(y_t) - L\eta_t z_t
                \\
                &= \eta_t Lz_t + x_t -(\eta_t + L^{-1}) \mathcal G_L(y_t)  - L\eta_t z_t
                \\
                &= x_t - (\eta_t + L^{-1})\mathcal G_L(y_t). 
            \end{align*}
            So $x_{t + 1} = x_t + \tilde \eta_{t + 1}\mathcal G_L(y_t)$, by assumption $\tilde \eta_{t + 1} = \eta_t + L^{-1}$
            Reducing it to the classic momentum form starts by 
            \begin{align*}
                y_{t + 1} &= (1 + L\eta_{t + 1})^{-1} (x_{t + 1} + L\eta_{t + 1}z_{t + 1})
                \\
                &= (1 + L\eta_{t + 1})^{-1} (
                    z_{t + 1} + L\eta_t (z_{t + 1} - z_t) + L\eta_{t + 1} z_{t + 1}
                )
                \\
                &= 
                (1 + L\eta_{t + 1})^{-1} (
                    (1 + L\eta_{t + 1})z_{t + 1} + L\eta_t(z_{t + 1} - z_t)
                )
                \\
                &= z_{t + 1} + (1 + L\eta_{t + 1})^{-1}L\eta_t (z_{t + 1} - z_t), 
            \end{align*}
            it negates the $x_t$ variables, therefore we have 
            \begin{align*}
                z_{t + 1} &= y_t - L^{-1} \mathcal G_L (y_t)
                \\
                y_{t + 1} &= z_{t + 1} + (1 + L\eta_{t + 1})^{-1}L\eta_t (z_{t + 1} - z_t).
            \end{align*}
            This is FISTA by the relation $a_t = 1 + L\eta_t$, hence $(1 + L\eta_{t + 1})^{-1}L\eta_t$ is $(a_t - 1)/a_{t + 1}$.
            Where we proved in 
            \hyperref[thm:ag_generic_stepsize_constraints]
                {theorem \ref*{thm:ag_generic_stepsize_constraints}}
            that $a_t$ is the Nesterov momentum sequence. 
        \end{proof}
        \begin{remark}
            In this remark we clarify the name ``similar triangle" as given in the literatures. 
            We think it is a fitting name, becaues it has a similar triangle in it. 
            We list the following observations
            \begin{enumerate}
                \item 
                The updates for $y_{t}$ from the algorithm has 
                $$
                    y_t = (1 + L\eta_t)^{-1} x_t + L\eta_t(1 + L\eta_t)^{-1} z_t, 
                $$
                therefore, $y_t$ is a convex combinations of $x_t, z_t$, so $z_t, y_t, x_t$ are three collinear points. 
                We have the ratio $\Vert y_t - x_t\Vert/\Vert z_t - y_t\Vert = L\eta_t$. 
                \item 
                The updates for $z_{t + 1}, x_{t + 1}$ are based on $y_t, x_t$ displaced by $L^{-1} \mathcal G_L(y_t), \tilde\eta_{t +1} \mathcal G_L(y_t)$, therefore vector $z_{t + 1} - y_t$ parallels to $x_{t + 1} - x_t$. 
                \item The updates for $x_{t + 1}$ has $x_{t + 1} - z_{t + 1} = L\eta_t \left(z_{t + 1} - z_t\right)$, therefore, the three points $z_t, x_{t + 1}, z_{t + 1}$ are collinear. 
                The ratio between line segment has $\Vert x_{t + 1} - z_{t + 1}\Vert/\Vert z_{t + 1} - z_t\Vert = L\eta_t$. 
            \end{enumerate}
            By these tree observations, the triangle $z_{t}, z_{t + 1}, y_t$ similar to triangle $z_t, x_{t + 1}, x_t$. 
            Finally, similar remarks about the similar triangle form can be found in Ahn, Sra's paper \cite{ahn_understanding_2022} as well. 
            
        \end{remark} 
        
    \subsection{Recovering the accelerated gradient variant appeared in Ryu's Book}
    \subsection{Recovering Chambolle, Dossal 2015}

\section{Algorithmic Improvements}\label{sec:algorithm_improved}
    In this section, we state a new variant of the accelerated gradient algorithm using the PPM interpretation to algorithmically conduct a line search routine for $L, \mu \ge 0$ the Lipschitz constant and strong convexity index of the smooth part of the objective function. 
    This variant will be parameter free while still retaining the optimal convergence rate for all convex functions. 

\section{Numerical Experiments}\label{sec:numerical_experiments}

\appendix


% \printbibliography

\bibliographystyle{siam}
\bibliography{references/refs}

\end{document}
