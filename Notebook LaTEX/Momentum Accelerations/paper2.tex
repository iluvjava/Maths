\documentclass[12pt]{article}

% 
%\usepackage{showkeys}
%\usepackage{drftcite}
\usepackage{exscale,relsize}
\usepackage{amsmath}
\usepackage{amsfonts}
\usepackage[hidelinks]{hyperref}
\usepackage{amssymb}
\usepackage{calc}
\usepackage{theorem}
\usepackage{pifont}      % needed by dingautolist
\usepackage{array}
\usepackage{color}
\usepackage{enumerate}
\usepackage{bbm}
\usepackage{graphicx}
\usepackage{float}
\usepackage{subfigure}




% FORMATTING ===================================================================
\oddsidemargin -0.1cm
\textwidth  16.5cm
\topmargin  0.0cm
\headheight 0.0cm
\textheight 21.0cm
\parindent  4mm
\parskip    10pt
%\parskip    8pt
\tolerance  3000

% MATH SYMBOLS =================================================================


% These are Heniz's notations. 
\newcommand{\To}{\ensuremath{\rightrightarrows}}
\newcommand{\GX}{\ensuremath{\Gamma}}
\newcommand{\mal}{\ensuremath{\mathfrak{m}}}
\newcommand{\mumu}{\ensuremath{{\mu\mu}}}
\newcommand{\paver}{\ensuremath{\mathcal{P}}}
\newcommand{\ZZZ}{\ensuremath{{X \times X^*}}}
\newcommand{\RRR}{\ensuremath{{\RR \times \RR}}}
\newcommand{\todo}{\hookrightarrow\textsf{TO DO:}}

\newcommand{\emp}{\ensuremath{\varnothing}}
%\newcommand{\la}{\ensuremath{\langle}}
%\newcommand{\ra}{\ensuremath{\rangle}}
\newcommand{\infconv}{\ensuremath{\mbox{\small$\,\square\,$}}}
\newcommand{\pscal}{\ensuremath{\scal{\cdot}{\cdot}}}
\newcommand{\Tt}{\ensuremath{\mathfrak{T}}}
\newcommand{\YY}{\ensuremath{\mathcal Y}}
\newcommand{\XX}{\ensuremath{\mathcal X}}
\newcommand{\HH}{\ensuremath{\mathcal H}}
\newcommand{\XP}{\ensuremath{\mathcal X}^*}
\newcommand{\st}{\ensuremath{\;|\;}}
\newcommand{\zeroun}{\ensuremath{\left]0,1\right[}}

\newcommand{\lev}[1]{\ensuremath{\mathrm{lev}_{\leq #1}\:}}
\newcommand{\moyo}[2]{\ensuremath{\sideset{_{#2}}{}{\operatorname{}}\!#1}}
\newcommand{\pair}[2]{\left\langle{{#1},{#2}}\right\rangle}
%\newcommand{\scal}[2]{\left.\left\langle{#1}\:\right| {#2}  \right\rangle}
\newcommand{\scal}[2]{\langle{{#1},{#2}}\rangle}
\newcommand{\Scal}[2]{\left\langle{{#1},{#2}}\right\rangle}
%\newcommand{\scal}[2]{\braket{ {#1},{#2}}}

\newcommand{\yosida}{\ensuremath{ \; {}^}}
\newcommand{\exi}{\ensuremath{\exists\,}}
\newcommand{\GG}{\ensuremath{\mathcal G}}
\newcommand{\RR}{\ensuremath{\mathbb R}}
\newcommand{\SSS}{\ensuremath{\mathbb S}}
\newcommand{\CC}{\ensuremath{\mathbb C}}
\newcommand{\Real}{\ensuremath{\mathrm{Re}\,}}
\newcommand{\ii}{\ensuremath{\mathrm i}}
\newcommand{\RP}{\ensuremath{\left[0,+\infty\right[}}
\newcommand{\RPX}{\ensuremath{\left[0,+\infty\right]}}
\newcommand{\RPP}{\ensuremath{\,\left]0,+\infty\right[}}
\newcommand{\RX}{\ensuremath{\,\left]-\infty,+\infty\right]}}
\newcommand{\RXX}{\ensuremath{\,\left[-\infty,+\infty\right]}}
\newcommand{\KK}{\ensuremath{\mathbb K}}
\newcommand{\NN}{\ensuremath{\mathbb N}}
\newcommand{\nnn}{\ensuremath{{n \in \NN}}}
\newcommand{\thalb}{\ensuremath{\tfrac{1}{2}}}
\newcommand{\pfrac}[2]{\ensuremath{\mathlarger{\tfrac{#1}{#2}}}}
\newcommand{\zo}{\ensuremath{{\left]0,1\right]}}}
\newcommand{\lzo}{\ensuremath{{\lambda \in \left]0,1\right]}}}
%\newcommand{\toppsepp}{\setlength{\partopsep}{-5pt}}
\newcommand{\menge}[2]{\big\{{#1} \mid {#2}\big\}}


% MATH OPERATORS ===============================================================
% \newcommand{\monos}{\ensuremath{\mathcal M}}
\newcommand{\DD}{\operatorname{dom}f}
\newcommand{\IDD}{\ensuremath{\operatorname{int}\operatorname{dom}f}}
\newcommand{\CDD}{\ensuremath{\overline{\operatorname{dom}}\,f}}
\newcommand{\clspan}{\ensuremath{\overline{\operatorname{span}}}}
\newcommand{\cone}{\ensuremath{\operatorname{cone}}}
\newcommand{\dom}{\ensuremath{\operatorname{dom}}}
\newcommand{\closu}{\ensuremath{\operatorname{cl}}}
\newcommand{\cont}{\ensuremath{\operatorname{cont}}}
\newcommand{\mons}{\ensuremath{\mathcal{A}}}
\newcommand{\gra}{\ensuremath{\operatorname{gra}}}
\newcommand{\epi}{\ensuremath{\operatorname{epi}}}
\newcommand{\prox}{\ensuremath{\operatorname{Prox}_{\mu}}}
\newcommand{\hprox}{\ensuremath{\operatorname{prox}}}
\newcommand{\intdom}{\ensuremath{\operatorname{int}\operatorname{dom}}\,}
\newcommand{\inte}{\ensuremath{\operatorname{int}}}
\newcommand{\sri}{\ensuremath{\operatorname{sri}}}
\newcommand{\reli}{\ensuremath{\operatorname{ri}}}
\newcommand{\cart}{\ensuremath{\mbox{\LARGE{$\times$}}}}


\newcommand{\average}{\ensuremath{\mathcal{R}_{\mu}({\bf A},{\boldsymbol \lambda})}}
\newcommand{\averagebar}{\ensuremath{\mathcal{R}_{1}({\bf A},\bar{\lambda})}}
\newcommand{\averageonelambda}{\ensuremath{\mathcal{R}({\bf A},{\boldsymbol \lambda})}}
\newcommand{\averageonehalf}{\ensuremath{\mathcal{R}_{1}(A,1/2)}}
\newcommand{\averageinverse}{\ensuremath{\mathcal{R}_{\mu^{-1}}({\bf A}^{-1},{\boldsymbol \lambda})}}
\newcommand{\averageoneinverse}{\ensuremath{\mathcal{R}({\bf A}^{-1},{\boldsymbol \lambda})}}
\newcommand{\averagef}{\ensuremath{\mathcal{P}_{\mu}(f,\lambda)}}
\newcommand{\averagefone}{\ensuremath{\mathcal{P}_{1}(f,\lambda)}}
\newcommand{\averagefd}{\ensuremath{\mathcal{P}_{\mu}((f_{1},\ldots, f_{n}),(\lambda_{1},\ldots, \lambda_{n}))}}
\newcommand{\averagefik}{\ensuremath{\mathcal{P}_{\mu_{k}}((f_{1,k},\ldots,f_{n,k}),
(\lambda_{1,k},\ldots,\lambda_{n,k}))}}
\newcommand{\averagesub}{\ensuremath{\mathcal{R}_{\mu}(\partial f,\lambda)}}
\newcommand{\res}{\ensuremath{\mathcal{R}_{\mu}}}
\newcommand{\resmuk}{\ensuremath{\mathcal{R}_{\mu_{k}}}}
\newcommand{\newres}{\ensuremath{\mathcal{R}}}
\newcommand{\resmualpha}{\ensuremath{\mathcal{R}_{\alpha\mu}}}
\newcommand{\averageone}{\ensuremath{\mathcal{R}_{1}}}
\newcommand{\harm}{\ensuremath{\mathcal{H}(A,\lambda)}}
\newcommand{\arithmetic}{\ensuremath{\mathcal{A}(A,\lambda)}}

\newcommand{\WC}{\ensuremath{{\mathfrak W}}}
\newcommand{\SC}{\ensuremath{{\mathfrak S}}}
\newcommand{\card}{\ensuremath{\operatorname{card}}}
\newcommand{\bd}{\ensuremath{\operatorname{bdry}}}
\newcommand{\ran}{\ensuremath{\operatorname{ran}}}
\newcommand{\rec}{\ensuremath{\operatorname{rec}}}
\newcommand{\rank}{\ensuremath{\operatorname{rank}}}
\newcommand{\kernel}{\ensuremath{\operatorname{ker}}}
\newcommand{\conv}{\ensuremath{\operatorname{conv}}}
\newcommand{\segh}{\ensuremath{\operatorname{seg}}}
\newcommand{\boxx}{\ensuremath{\operatorname{box}}}
\newcommand{\clconv}{\ensuremath{\overline{\operatorname{conv}}\,}}
\newcommand{\cldom}{\ensuremath{\overline{\operatorname{dom}}\,}}
\newcommand{\clran}{\ensuremath{\overline{\operatorname{ran}}\,}}
\newcommand{\Nf}{\ensuremath{\nabla f}}
\newcommand{\NNf}{\ensuremath{\nabla^2f}}
\newcommand{\Fix}{\ensuremath{\operatorname{Fix}}}
\newcommand{\FFix}{\ensuremath{\overline{\operatorname{Fix}}\,}}
\newcommand{\aFix}{\ensuremath{\widetilde{\operatorname{Fix}\,}}}
\newcommand{\Id}{\ensuremath{\operatorname{Id}}}
\newcommand{\Max}{\ensuremath{\operatorname{max}}}
\newcommand{\Bb}{\ensuremath{\mathfrak{B}}}
\newcommand{\BB}{\ensuremath{\mathbb{B}}}
\newcommand{\Fb}{\ensuremath{\overrightarrow{\mathfrak{B}}}}
\newcommand{\Fprox}{\ensuremath{\overrightarrow{\operatorname{prox}}}}
\newcommand{\Bprox}{\ensuremath{\overleftarrow{\operatorname{prox}}}}
\newcommand{\Bproj}{\ensuremath{\overleftarrow{\operatorname{P}}}}
\newcommand{\Ri}{\ensuremath{\mathfrak{R}_i}}
\newcommand{\Dn}{\ensuremath{\,\overset{D}{\rightarrow}\,}}
\newcommand{\nDn}{\ensuremath{\,\overset{D}{\not\rightarrow}\,}}
\newcommand{\weakly}{\ensuremath{\,\rightharpoonup}\,}
\newcommand{\weaklys}{\ensuremath{\,\overset{*}{\rightharpoonup}}\,}
\newcommand{\gr}{\ensuremath{\operatorname{gra}}}
\newcommand{\g}{\ensuremath{\,\overset{g}{\rightarrow}}\,}
\newcommand{\p}{\ensuremath{\,\overset{p}{\rightarrow}}\,}
\newcommand{\e}{\ensuremath{\,\overset{e}{\rightarrow}}\,}
\newcommand{\Tbar}{\ensuremath{\overline{T}}}
\newcommand{\n}{\ensuremath{\,\overset{n}{\rightarrow}}\,}

\newcommand{\minf}{\ensuremath{-\infty}}
\newcommand{\pinf}{\ensuremath{+\infty}}
\renewcommand{\iff}{\ensuremath{\Leftrightarrow}}
\renewcommand{\phi}{\ensuremath{\varphi}}
%\newcommand{\Real}{\ensuremath{\mathrm{Re}\,}}
\newcommand{\negent}{\ensuremath{\operatorname{negent}}}
\newcommand{\neglog}{\ensuremath{\operatorname{neglog}}}
\newcommand{\halb}{\ensuremath{\tfrac{1}{2}}}
\newcommand{\bT}{\ensuremath{\mathbf{T}}}
\newcommand{\bX}{\ensuremath{\mathbf{X}}}
\newcommand{\bL}{\ensuremath{\mathbf{L}}}
\newcommand{\bD}{\ensuremath{\boldsymbol{\Delta}}}
\newcommand{\bc}{\ensuremath{\mathbf{c}}}
\newcommand{\by}{\ensuremath{\mathbf{y}}}
\newcommand{\bx}{\ensuremath{\mathbf{x}}}
\newcommand{\bA}{{\bf A}}
\newcommand{\Other}{Indeterminate }
\newcommand{\other}{indeterminate }


%%% Raf's stuff  ===============================================================
\newcommand{\al}{\alpha}
\newcommand{\la}{\lambda}
\newcommand{\La}{\Lambda}
\newcommand{\pluss}{{\hskip1pt \raise1pt\vbox{\hrule width6pt \vskip1pt
\hrule width6pt}\kern-4pt{\lower1pt\hbox{\vrule height6pt \kern1pt\vrule
height6pt}}\hskip5pt}}
\newcommand{\timess}{\star}
\newcommand{\argmax}{\mathop{\rm argmax}\limits}
\newcommand{\argmin}{\mathop{\rm argmin}\limits}
\newcommand{\product}{\langle\cdot,\cdot\rangle}
\newcommand{\im}{\mathrm{Im}}
\newcommand{\multival}{\ensuremath{X\to 2^{X^*}}}
\newcommand{\SX}{\ensuremath{2^{X^*}}}



% THEOREM AND ENVIRONMENTS.  ===================================================

%\newenvironment{deflist}[1][\quad]%
%{\begin{list}{}{\renewcommand{\makelabel}[1]{\textrm{##1~}\hfil}%
%\settowidth{\labelwidth}{\textrm{#1~}}%
%\setlength{\leftmargin}{\labelwidth+\labelsep}}}%requires macro calc.sty
%{\end{list}}
%\newtheorem{theorem}{Theorem}%[section]
\newtheorem{theorem}{Theorem}[section]
\newtheorem{lemma}[theorem]{Lemma}
\newtheorem{fact}[theorem]{Fact}
\newtheorem{corollary}[theorem]{Corollary}
\newtheorem{proposition}[theorem]{Proposition}
\newtheorem{definition}[theorem]{Definition}
\newtheorem{conjecture}[theorem]{Conjecture}
\newtheorem{observation}[theorem]{Observation}
\newtheorem{openprob}[theorem]{Open Problem}
\theoremstyle{plain}{\theorembodyfont{\rmfamily}
\newtheorem{assumption}[theorem]{Assumption}}
\theoremstyle{plain}{\theorembodyfont{\rmfamily}
\newtheorem{condition}[theorem]{Condition}}
\theoremstyle{plain}{\theorembodyfont{\rmfamily}
\newtheorem{algorithm}[theorem]{Algorithm}}
\theoremstyle{plain}{\theorembodyfont{\rmfamily}
\newtheorem{example}[theorem]{Example}}
\theoremstyle{plain}{\theorembodyfont{\rmfamily}
\newtheorem{remark}[theorem]{Remark}}
\theoremstyle{plain}{\theorembodyfont{\rmfamily}
\newtheorem{application}[theorem]{Application}}
\def\proof{\noindent{\it Proof}. \ignorespaces}
%\def\endproof{\vbox{\hrule height0.6pt\hbox{\vrule height1.3ex%
%width0.6pt\hskip0.8ex\vrule width0.6pt}\hrule height0.6pt}}
%\numberwithin{equation}{section}
\def\endproof{\ensuremath{\quad \hfill \blacksquare}}

\renewcommand\theenumi{(\roman{enumi})}
\renewcommand\theenumii{(\alph{enumii})}
\renewcommand{\labelenumi}{\rm (\roman{enumi})}
\renewcommand{\labelenumii}{\rm (\alph{enumii})}

\newcommand{\boxedeqn}[1]{%
    \[\fbox{%
        \addtolength{\linewidth}{-2\fboxsep}%
        \addtolength{\linewidth}{-2\fboxrule}%
        \begin{minipage}{\linewidth}%
        \begin{equation}#1\end{equation}%
        \end{minipage}%
      }\]%
  }


\newcommand{\hilight}[1]{\colorbox{yellow}{#1}}


%\usepackage{showkeys}
%\usepackage{drftcite}
\usepackage{exscale,relsize}
\usepackage{amsmath}
\usepackage{amsfonts}
\usepackage[colorlinks=true, linkcolor=blue]{hyperref}
\usepackage{amssymb}
\usepackage{calc}
\usepackage{theorem}
\usepackage{pifont}      % needed by dingautolist
\usepackage{array}
\usepackage{color}
\usepackage{enumerate}
\usepackage{bbm}
\usepackage{graphicx}
\usepackage{subcaption}
\usepackage{caption}

% \usepackage{amsthm}


% Hongda's packages
\usepackage{algpseudocode, algorithm}
\usepackage{mathtools}


% IF use the below packge, use `\printbibliography' to print out the bibliograph 
% For this one 
% 
% \usepackage[
%     backend=biber,
%     style=numeric,
%     sorting=nyt
% ]{biblatex}
% \addbibresource{references/PPM.bib}
% \addbibresource{references/NesterovMomentum.bib}
% \addbibresource{references/Books.bib}
% \addbibresource{references/BregmanDiv.bib}

% FORMATTING ===================================================================
\oddsidemargin -0.1cm
\textwidth  16.5cm
\topmargin  0.0cm
\headheight 0.0cm
\textheight 21.0cm
\parindent  4mm
\parskip    10pt
%\parskip    8pt
\tolerance  3000

% DRAFT FORMATTING =============================================================
% These are for todo notes, advise taken from 
% https://tex.stackexchange.com/questions/81666/extend-page-width-or-margin-for-todonotes-comments-or-other-package-comments
% \oddsidemargin=\dimexpr\oddsidemargin + 3cm\relax % DON'T USE

\ifthenelse{\boolean{draftmode}}{
    \evensidemargin=\dimexpr\evensidemargin  + 6cm\relax 
    \oddsidemargin=\dimexpr\oddsidemargin + 6cm\relax
    \paperwidth=\dimexpr \paperwidth + 12cm\relax 
    \marginparwidth=\dimexpr \marginparwidth  + 6cm\relax
    \paperheight=\dimexpr \paperheight + 6cm\relax
}{
    
}


% THEOREM AND ENVIRONMENTS.  ===================================================

%\newenvironment{deflist}[1][\quad]%
%{\begin{list}{}{\renewcommand{\makelabel}[1]{\textrm{##1~}\hfil}%
%\settowidth{\labelwidth}{\textrm{#1~}}%
%\setlength{\leftmargin}{\labelwidth+\labelsep}}}%requires macro calc.sty
%{\end{list}}
%\newtheorem{theorem}{Theorem}%[section]
\newtheorem{theorem}{Theorem}[section]
\newtheorem{lemma}[theorem]{Lemma}
\newtheorem{fact}[theorem]{Fact}
\newtheorem{corollary}[theorem]{Corollary}
\newtheorem{proposition}[theorem]{Proposition}
\newtheorem{definition}[theorem]{Definition}
\newtheorem{conjecture}[theorem]{Conjecture}
\newtheorem{observation}[theorem]{Observation}
\newtheorem{openprob}[theorem]{Open Problem}
\theoremstyle{plain}{\theorembodyfont{\rmfamily}
\newtheorem{assumption}[theorem]{Assumption}}
\theoremstyle{plain}{\theorembodyfont{\rmfamily}
\newtheorem{condition}[theorem]{Condition}}
\theoremstyle{plain}{\theorembodyfont{\rmfamily}}

% Removed due conflict with the algorithm environment. 
% \newtheorem{algorithm}[theorem]{Algorithm}}

\theoremstyle{plain}{\theorembodyfont{\rmfamily}
\newtheorem{example}[theorem]{Example}}
\theoremstyle{plain}{\theorembodyfont{\rmfamily}
\newtheorem{remark}[theorem]{Remark}}
\theoremstyle{plain}{\theorembodyfont{\rmfamily}
\newtheorem{application}[theorem]{Application}}

\def\proof{\noindent{\it Proof}. \ignorespaces}
%\def\endproof{\vbox{\hrule height0.6pt\hbox{\vrule height1.3ex%
%width0.6pt\hskip0.8ex\vrule width0.6pt}\hrule height0.6pt}}
%\numberwithin{equation}{section}
\def\endproof{\ensuremath{\quad \hfill \blacksquare}}

\renewcommand\theenumi{(\roman{enumi})}
\renewcommand\theenumii{(\alph{enumii})}
\renewcommand{\labelenumi}{\rm (\roman{enumi})}
\renewcommand{\labelenumii}{\rm (\alph{enumii})}

\numberwithin{equation}{section}



% These are Heniz's notations. 
\newcommand{\To}{\ensuremath{\rightrightarrows}}
\newcommand{\GX}{\ensuremath{\Gamma}}
\newcommand{\mal}{\ensuremath{\mathfrak{m}}}
\newcommand{\mumu}{\ensuremath{{\mu\mu}}}
\newcommand{\paver}{\ensuremath{\mathcal{P}}}
\newcommand{\ZZZ}{\ensuremath{{X \times X^*}}}
\newcommand{\RRR}{\ensuremath{{\RR \times \RR}}}
\newcommand{\todo}{\hookrightarrow\textsf{TO DO:}}

\newcommand{\emp}{\ensuremath{\varnothing}}
%\newcommand{\la}{\ensuremath{\langle}}
%\newcommand{\ra}{\ensuremath{\rangle}}
\newcommand{\infconv}{\ensuremath{\mbox{\small$\,\square\,$}}}
\newcommand{\pscal}{\ensuremath{\scal{\cdot}{\cdot}}}
\newcommand{\Tt}{\ensuremath{\mathfrak{T}}}
\newcommand{\YY}{\ensuremath{\mathcal Y}}
\newcommand{\XX}{\ensuremath{\mathcal X}}
\newcommand{\HH}{\ensuremath{\mathcal H}}
\newcommand{\XP}{\ensuremath{\mathcal X}^*}
\newcommand{\st}{\ensuremath{\;|\;}}
\newcommand{\zeroun}{\ensuremath{\left]0,1\right[}}

\newcommand{\lev}[1]{\ensuremath{\mathrm{lev}_{\leq #1}\:}}
\newcommand{\moyo}[2]{\ensuremath{\sideset{_{#2}}{}{\operatorname{}}\!#1}}
\newcommand{\pair}[2]{\left\langle{{#1},{#2}}\right\rangle}
%\newcommand{\scal}[2]{\left.\left\langle{#1}\:\right| {#2}  \right\rangle}
\newcommand{\scal}[2]{\langle{{#1},{#2}}\rangle}
\newcommand{\Scal}[2]{\left\langle{{#1},{#2}}\right\rangle}
%\newcommand{\scal}[2]{\braket{ {#1},{#2}}}

\newcommand{\yosida}{\ensuremath{ \; {}^}}
\newcommand{\exi}{\ensuremath{\exists\,}}
\newcommand{\GG}{\ensuremath{\mathcal G}}
\newcommand{\RR}{\ensuremath{\mathbb R}}
\newcommand{\SSS}{\ensuremath{\mathbb S}}
\newcommand{\CC}{\ensuremath{\mathbb C}}
\newcommand{\Real}{\ensuremath{\mathrm{Re}\,}}
\newcommand{\ii}{\ensuremath{\mathrm i}}
\newcommand{\RP}{\ensuremath{\left[0,+\infty\right[}}
\newcommand{\RPX}{\ensuremath{\left[0,+\infty\right]}}
\newcommand{\RPP}{\ensuremath{\,\left]0,+\infty\right[}}
\newcommand{\RX}{\ensuremath{\,\left]-\infty,+\infty\right]}}
\newcommand{\RXX}{\ensuremath{\,\left[-\infty,+\infty\right]}}
\newcommand{\KK}{\ensuremath{\mathbb K}}
\newcommand{\NN}{\ensuremath{\mathbb N}}
\newcommand{\nnn}{\ensuremath{{n \in \NN}}}
\newcommand{\thalb}{\ensuremath{\tfrac{1}{2}}}
\newcommand{\zo}{\ensuremath{{\left]0,1\right]}}}
\newcommand{\lzo}{\ensuremath{{\lambda \in \left]0,1\right]}}}
%\newcommand{\toppsepp}{\setlength{\partopsep}{-5pt}}
\newcommand{\menge}[2]{\big\{{#1} \mid {#2}\big\}}
\newcommand{\pfrac}[2]{\ensuremath{\mathlarger{\tfrac{#1}{#2}}}}


% MATH OPERATORS ===============================================================
% \newcommand{\monos}{\ensuremath{\mathcal M}}
\newcommand{\DD}{\operatorname{dom}f}
\newcommand{\IDD}{\ensuremath{\operatorname{int}\operatorname{dom}f}}
\newcommand{\CDD}{\ensuremath{\overline{\operatorname{dom}}\,f}}
\newcommand{\clspan}{\ensuremath{\overline{\operatorname{span}}}}
\newcommand{\cone}{\ensuremath{\operatorname{cone}}}
\newcommand{\dom}{\ensuremath{\operatorname{dom}}}
\newcommand{\closu}{\ensuremath{\operatorname{cl}}}
\newcommand{\cont}{\ensuremath{\operatorname{cont}}}
\newcommand{\mons}{\ensuremath{\mathcal{A}}}
\newcommand{\gra}{\ensuremath{\operatorname{gra}}}
\newcommand{\epi}{\ensuremath{\operatorname{epi}}}
\newcommand{\prox}{\ensuremath{\operatorname{Prox}_{\mu}}}
\newcommand{\hprox}{\ensuremath{\operatorname{prox}}}
\newcommand{\intdom}{\ensuremath{\operatorname{int}\operatorname{dom}}\,}
\newcommand{\inte}{\ensuremath{\operatorname{int}}}
\newcommand{\sri}{\ensuremath{\operatorname{sri}}}
\newcommand{\reli}{\ensuremath{\operatorname{ri}}}
\newcommand{\cart}{\ensuremath{\mbox{\LARGE{$\times$}}}}


\newcommand{\average}{\ensuremath{\mathcal{R}_{\mu}({\bf A},{\boldsymbol \lambda})}}
\newcommand{\averagebar}{\ensuremath{\mathcal{R}_{1}({\bf A},\bar{\lambda})}}
\newcommand{\averageonelambda}{\ensuremath{\mathcal{R}({\bf A},{\boldsymbol \lambda})}}
\newcommand{\averageonehalf}{\ensuremath{\mathcal{R}_{1}(A,1/2)}}
\newcommand{\averageinverse}{\ensuremath{\mathcal{R}_{\mu^{-1}}({\bf A}^{-1},{\boldsymbol \lambda})}}
\newcommand{\averageoneinverse}{\ensuremath{\mathcal{R}({\bf A}^{-1},{\boldsymbol \lambda})}}
\newcommand{\averagef}{\ensuremath{\mathcal{P}_{\mu}(f,\lambda)}}
\newcommand{\averagefone}{\ensuremath{\mathcal{P}_{1}(f,\lambda)}}
\newcommand{\averagefd}{\ensuremath{\mathcal{P}_{\mu}((f_{1},\ldots, f_{n}),(\lambda_{1},\ldots, \lambda_{n}))}}
\newcommand{\averagefik}{\ensuremath{\mathcal{P}_{\mu_{k}}((f_{1,k},\ldots,f_{n,k}),
(\lambda_{1,k},\ldots,\lambda_{n,k}))}}
\newcommand{\averagesub}{\ensuremath{\mathcal{R}_{\mu}(\partial f,\lambda)}}
\newcommand{\res}{\ensuremath{\mathcal{R}_{\mu}}}
\newcommand{\resmuk}{\ensuremath{\mathcal{R}_{\mu_{k}}}}
\newcommand{\newres}{\ensuremath{\mathcal{R}}}
\newcommand{\resmualpha}{\ensuremath{\mathcal{R}_{\alpha\mu}}}
\newcommand{\averageone}{\ensuremath{\mathcal{R}_{1}}}
\newcommand{\harm}{\ensuremath{\mathcal{H}(A,\lambda)}}
\newcommand{\arithmetic}{\ensuremath{\mathcal{A}(A,\lambda)}}

\newcommand{\WC}{\ensuremath{{\mathfrak W}}}
\newcommand{\SC}{\ensuremath{{\mathfrak S}}}
\newcommand{\card}{\ensuremath{\operatorname{card}}}
\newcommand{\bd}{\ensuremath{\operatorname{bdry}}}
\newcommand{\ran}{\ensuremath{\operatorname{ran}}}
\newcommand{\rec}{\ensuremath{\operatorname{rec}}}
\newcommand{\rank}{\ensuremath{\operatorname{rank}}}
\newcommand{\kernel}{\ensuremath{\operatorname{ker}}}
\newcommand{\conv}{\ensuremath{\operatorname{conv}}}
\newcommand{\segh}{\ensuremath{\operatorname{seg}}}
\newcommand{\boxx}{\ensuremath{\operatorname{box}}}
\newcommand{\clconv}{\ensuremath{\overline{\operatorname{conv}}\,}}
\newcommand{\cldom}{\ensuremath{\overline{\operatorname{dom}}\,}}
\newcommand{\clran}{\ensuremath{\overline{\operatorname{ran}}\,}}
\newcommand{\Nf}{\ensuremath{\nabla f}}
\newcommand{\NNf}{\ensuremath{\nabla^2f}}
\newcommand{\Fix}{\ensuremath{\operatorname{Fix}}}
\newcommand{\FFix}{\ensuremath{\overline{\operatorname{Fix}}\,}}
\newcommand{\aFix}{\ensuremath{\widetilde{\operatorname{Fix}\,}}}
\newcommand{\Id}{\ensuremath{\operatorname{Id}}}
\newcommand{\Max}{\ensuremath{\operatorname{max}}}
\newcommand{\Bb}{\ensuremath{\mathfrak{B}}}
\newcommand{\BB}{\ensuremath{\mathbb{B}}}
\newcommand{\Fb}{\ensuremath{\overrightarrow{\mathfrak{B}}}}
\newcommand{\Fprox}{\ensuremath{\overrightarrow{\operatorname{prox}}}}
\newcommand{\Bprox}{\ensuremath{\overleftarrow{\operatorname{prox}}}}
\newcommand{\Bproj}{\ensuremath{\overleftarrow{\operatorname{P}}}}
\newcommand{\Ri}{\ensuremath{\mathfrak{R}_i}}
\newcommand{\Dn}{\ensuremath{\,\overset{D}{\rightarrow}\,}}
\newcommand{\nDn}{\ensuremath{\,\overset{D}{\not\rightarrow}\,}}
\newcommand{\weakly}{\ensuremath{\,\rightharpoonup}\,}
\newcommand{\weaklys}{\ensuremath{\,\overset{*}{\rightharpoonup}}\,}
\newcommand{\gr}{\ensuremath{\operatorname{gra}}}
\newcommand{\g}{\ensuremath{\,\overset{g}{\rightarrow}}\,}
\newcommand{\p}{\ensuremath{\,\overset{p}{\rightarrow}}\,}
\newcommand{\e}{\ensuremath{\,\overset{e}{\rightarrow}}\,}
\newcommand{\Tbar}{\ensuremath{\overline{T}}}
\newcommand{\n}{\ensuremath{\,\overset{n}{\rightarrow}}\,}

\newcommand{\minf}{\ensuremath{-\infty}}
\newcommand{\pinf}{\ensuremath{+\infty}}
\renewcommand{\iff}{\ensuremath{\Leftrightarrow}}
% \renewcommand{\phi}{\ensuremath{\varphi}}
%\newcommand{\Real}{\ensuremath{\mathrm{Re}\,}}
\newcommand{\negent}{\ensuremath{\operatorname{negent}}}
\newcommand{\neglog}{\ensuremath{\operatorname{neglog}}}
\newcommand{\halb}{\ensuremath{\tfrac{1}{2}}}
\newcommand{\bT}{\ensuremath{\mathbf{T}}}
\newcommand{\bX}{\ensuremath{\mathbf{X}}}
\newcommand{\bL}{\ensuremath{\mathbf{L}}}
\newcommand{\bD}{\ensuremath{\boldsymbol{\Delta}}}
\newcommand{\bc}{\ensuremath{\mathbf{c}}}
\newcommand{\by}{\ensuremath{\mathbf{y}}}
\newcommand{\bx}{\ensuremath{\mathbf{x}}}
\newcommand{\bA}{{\bf A}}
\newcommand{\Other}{Indeterminate }
\newcommand{\other}{indeterminate }


%%% Raf's stuff  ===============================================================
\newcommand{\al}{\alpha}
\newcommand{\la}{\lambda}
\newcommand{\La}{\Lambda}
\newcommand{\pluss}{{\hskip1pt \raise1pt\vbox{\hrule width6pt \vskip1pt
\hrule width6pt}\kern-4pt{\lower1pt\hbox{\vrule height6pt \kern1pt\vrule
height6pt}}\hskip5pt}}
\newcommand{\timess}{\star}
\newcommand{\argmax}{\mathop{\rm argmax}\limits}
\newcommand{\argmin}{\mathop{\rm argmin}\limits}
\newcommand{\product}{\langle\cdot,\cdot\rangle}
\newcommand{\im}{\mathrm{Im}}
\newcommand{\multival}{\ensuremath{X\to 2^{X^*}}}
\newcommand{\SX}{\ensuremath{2^{X^*}}}

\newcommand{\inlinecode}[1]{\texttt{\footnotesize #1}}

\usepackage{listings} \lstset{basicstyle=\footnotesize\ttfamily,breaklines=true}
\usepackage{xcolor}
\lstdefinelanguage{Julia}%
  {morekeywords={abstract,break,case,catch,const,continue,do, else, elseif,%
      end, export, false, for, function, immutable, import, importall, if, in,%
      macro, module, otherwise, quote, return, switch, true, try, type, typealias,%
      using, while},%
   sensitive=true,%
   alsoother={$},%
   morecomment=[l]\#,%
   morecomment=[n]{\#=}{=\#},%
   morestring=[s]{"}{"},%
   morestring=[m]{'}{'},%
}[keywords,comments,strings]%
\lstset{%
    language         = Julia,
    basicstyle       = \ttfamily,
    keywordstyle     = \bfseries\color{blue},
    stringstyle      = \color{magenta},
    commentstyle     = \color{ForestGreen},
    showstringspaces = false,
}

\begin{document}

\title{{\fontfamily{ptm}\selectfont The Proximal Point interpretation of Nesterov accelerated proximal gradient}}

\author{
    Hongda Li
    \thanks{Department of Mathematics, I.K. Barber Faculty of Science,
    The University of British Columbia, Kelowna, BC Canada V1V 1V7. E-mail:  \texttt{alto@mail.ubc.ca}.}~ and~Xianfu Wang
    \thanks{Department of Mathematics, I.K. Barber Faculty of Science,
    The University of British Columbia, Kelowna, BC Canada V1V 1V7. E-mail:  \texttt{shawn.wang@ubc.ca}.}
}

\date{\today}

\maketitle

% \vskip 8mm

\begin{abstract} 
    \noindent
    Nesterov accelreated gradient method has been in the spotlight for the past decades due its wide spread applications and theories of optimal convergence. 
    Decades later it still opens up new interpretations. 
    Our work suggests a proximal point interpretation of accelerated gradient method for the method of accelerated proximal gradient method as a major extension to the interpretation proposed by Ahn and Sra \cite{ahn_understanding_2022}. 
    The proofs had been streamlined, extended and new error terms are added to allow a larger set of stepsize sequence for the proximal point interpreation. 
    Additionally, we conduct numerical experiment for a line search method that dynamically adjust the strong convexity index $\mu$ and Lipschitz constant of the gradient in for algorithm  using the proximal point understanding of accelerated gradient. 
    
\end{abstract}

\noindent{\bfseries 2010 Mathematics Subject Classification:}
Primary 47H05, 52A41, 90C25; Secondary 15A09, 26A51, 26B25, 26E60, 47H09, 47A63.
\noindent{\bfseries Keywords:}

\section{Introduction}
    
    Nesterov \cite{nesterov_method_1983} proposed an algorithm which achieves an optimal convergence rate for a convex, Lipschitz smooth function. 
    Here we refer Nesterov accelerated gradient algorithm simply as accelerated gradient, and the ideas behind as Nesterov acceleration. 
    Recent works from Ahn and Sra \cite{ahn_understanding_2022} and Nesterov \cite{nesterov_lectures_2018} inspired content in this paper.
    Ahn and Sra explored the interpretation of Nesterov acceleration as the proximal point method (PPM) applied to an upper surrogate function and lower surrogate function. 
    We generalized the framework to support composite additive objective form $h = f + g$ where $f$ is $L$-Lipschitz smooth and $g$ is convex lower semi-continuous whose proximal mapping can be evaluated exactly for algorithm implementations. 
    We survey the literatures before stating our contributions. 
    
    % As a major extension, our work describes the method of proximal gradient using the PPM interpretations. 
    % We also compactify the proof using Lypunov analysis, which allows us to show the convergence still holds even if there is error in the evaluations of the proximal gradient operator. 
    % We derive a major variant of the Nesterov acceleation algorithm
    % \cite{chambolle_convergence_2015}, \cite{beck_fast_2009-1}, \cite[Chapter 12]{ryu_large-scale_2022}
    % of the accelerated gradient (AG) and show that they all falls within the same framework. 

    \subsection{Background}
        In the literatures, extensions of the accelerated gradient uses Nesterov's estimating sequence for convergence analysis. 
        Guller \cite{guler_new_1992} applied the techniques to proximal point method and obtained accelerated convergence rates. 
        Necoara \cite{necoara_linear_2019} obtained convergence rate of accelerated gradient for non-strongly convex functions. 
        Nesterov's book \cite[chapter 2]{nesterov_lectures_2018} introduced the technique. 
        It involves constructing a sequence of functions: $\phi_k:\RR^n \mapsto \RR^n$ and assume there exists a sequence $(x_k)_{k\in \N \cup \{0\}}$ a priori such that $f(x_k) \le \phi_k^* := \min_x \phi_k(x)$. 
        The representation for $\phi_k$ involves many parameters. 
        Disregarding its complicated nature, the technique is powerful. 
        
        \par
        In Ahn's works, they showed an alternative proof using the proximal point interpretation of the accelerated gradient. 
        It used Moreau envelope for the Lyapunov analysis and parameters from the proximal point method interprations.  
        It has fewer parameters while still encompasses several variants of the accelerated gradient algorithm under the same framework. 
        Their approach is inspired by works from Defazio \cite{defazio_curved_2019}, Zeyuan and Lorenzo \cite{allen-zhu_linear_2016}. 
        
        \par
        Numerous variations of accelerated gradient exists in the literatures. \cite[(6.1.19)]{nesterov_lectures_2018} described a variant of accelerated gradient restricted to a convex domain $Q$ using Bregman Divergence. 
        Beck and Toubolle \cite{beck_fast_2009} introduced a variant the problem type of smooth plus non-smooth, known as FISTA. 
        Beck's Book \cite[(10.7.7)]{beck_first-order_2017} states a variant of FISTA called V-FISTA which achieves an optimal convergence rate for strongly convex. 
        See Chambolle and Dossal \cite{chambolle_convergence_2015} for a variant of accelerated gradient where the iterates converge (weakly in Hilbert space). 
        
        \par
        Various interpretations of the accelerated gradient exist in literatures. 
        Consult \cite{su_differential_2016} for a dynamical system interpretation of Nesterov acceleration. 
        The interpretation leads the valuable insight that restarting the accerlated gradient algorithm periodically leads to faster convergence rate for the class of strongly convex function. 
        Early attempt such as Monotone FISTA and their convergence rate analysis can be found in Beck and Touboulle \cite{beck_fast_2009}. 
        Aujol \cite{aujol_optimal_2019} showed that the convergence rate of original FISTA proposed by Beck and Touboulle \cite{beck_fast_2009} is sublinear hence suboptimal. 
        Prompted the two discoveries, a copious amount of ideas surrounding restart heuristics appeared in the literatures. 
        See Aujol \cite{aujol_parameter-free_2024}, Calatroni and Chambolle \cite{calatroni_backtracking_2019} and references within for algorithms that has optimal convergence rate for $L$ smooth and $\mu \ge 0$ strongly convex function, without needing the values of $\mu, L$ run the algorithm. 
        In general, proving the convergence for algorithm heuristics is difficult and requires sometimes requires a profound understanding. 
        See Moursi et al. \cite{moursi_accelerated_2023} preprint and references within for a proof of a specific type of parameter free restarting heuristics without function values for one dimensional functions and its extension and literature reviews on various type of heuristic employed to achieve fast linear convergence on accelerated gradient for strongly convex objective. 

        % \par
        % The paper is organized as follow: 
        % \begin{enumerate}
        %     \item Section 
        % \end{enumerate}
    \subsection{Contributions}
        \par
        In addition to the results in Ahn and Sra \cite{ahn_understanding_2022}, 
        \hyperref[sec:different-forms-acc-grad]{Section \ref*{sec:different-forms-acc-grad}}
        extends the formualtion of the PPM interpretation of accelerated gradient for composite additive convex objective using the proximal gradient operator. 
        A sleeker argument using Lyapunov analysis and the proximal gradient lemma 
        (\hyperref[lemma:grad_map_linearization]{Lemma \ref*{lemma:grad_map_linearization}}) achieves a weaker condition on the stepsize sequences needed for optimal convergence. 
        It covers more variants of proximal accelerated gradient type algorithm in the literatures. 
        \par
        This paper is organized as follows. 
        \hyperref[sec:preliminaries]{Section \ref*{sec:preliminaries}} introduces the proximal gradient operator, the gradient mapping, and the proximal gradient lemma. 
        They are pivotal derive the convergence rate and the generic formulations of accelerated proximal gradient via PPM interpretations. 
        \hyperref[sec:different-forms-acc-grad]{Section \ref*{sec:different-forms-acc-grad}}
        identifies precisely the relations between the parameters in PPM and variants of Nesterov accelerations found in the literatures. 
        \hyperref[sec:lyapunov-analysis]{Section \ref*{sec:lyapunov-analysis}} gives a sleek Lypapunov argument for the convergence of the PPM form of Nesterov acceleration algorithm, it generalizes the stepsize conditions further from previous works by Ahn and Sra. 
        \hyperref[sec:numerical-experiments]{Section \ref*{sec:numerical-experiments}}
        suggests new line search heuristic ideas that determines the parameters $\mu, L$ during the execution of the algorithm. 
        The design attempts to retain fast linear convergence without parameters $\mu, L$ prior.         
        Numerical experiments are conducted with applications to demonstrate the potential of the idea. 

    
\section{Preliminaries}\label{sec:preliminaries}

    \textcolor{red}{STATE THE OBJECTIVES OF THIS SECTION CLEARER!}

    Let the ambient space be $\RR^n$. 
    In this section, we define the proximal gradient mapping $\mathcal T_L$, and gradient mapping $\mathcal G_L$ for function satisfying 
    \hyperref[ass:smooth-nonsmooth-sum-lipschitz-grad]
    {Assumption \ref*{ass:smooth-nonsmooth-sum-lipschitz-grad}}. 
    A lower bound function is identified using the gradient mapping operator and proved in
    \hyperref[lemma:grad_map_linearization]{Lemma \ref*{lemma:grad_map_linearization}}, 
    this is a key component of the proximal point interpretation of the accelerated gradient method. 
    \begin{assumption}\label{ass:smooth-nonsmooth-sum-lipschitz-grad}
        Let $h = f + g$ where $f, g$ are convex, lower semi-continuous and $f$ is Lipschitz-Smooth with constant $L$. 
    \end{assumption}
    \begin{definition}[Strongly convex functions]
        A function $f: X \mapsto \RR$ is $\beta$-strongly convex
        with $\beta\geq 0$ if $f - \beta \frac{\Vert \cdot\Vert^2}{2}$ is convex.
    \end{definition}

    \begin{definition}[The gradient mapping]
        \label{def:gradient_mapping}
        Suppose $h = f + g$ satisfies 
        \hyperref[ass:smooth-nonsmooth-sum-lipschitz-grad]
        {Assumption \ref*{ass:smooth-nonsmooth-sum-lipschitz-grad}}. 
        Define the proximal gradient operator
        $$
            \mathcal T_L(x) := \hprox_{L^{-1}g}(x - L^{-1}\nabla f(x)),
        $$
        and the gradient mapping operator
        $$
            \mathcal G_L(x) = L(x - \mathcal T_L(x)). 
        $$
    \end{definition}
    \begin{remark}
        The name ``gradient mapping" comes from \cite[(2.2.54)]{nesterov_lectures_2018}, however, Nesterov was referring to only the case when $g$ is an indicator function of a convex set in his writing. 
        Of course, in Amir Beck \cite[10.3.2]{beck_first-order_2017}, it has the exact same definition for gradient mapping as the above. 
    \end{remark}

    \begin{lemma}[Gradient mapping approximates subgradient]
    \label{lemma:grad-map-approx-subgrad}\; \\
        Suppose $h = f + g$ satisfies 
        \hyperref[ass:smooth-nonsmooth-sum-lipschitz-grad]
        {Assumption \ref{ass:smooth-nonsmooth-sum-lipschitz-grad}}, 
        let $\mathcal T_L, \mathcal G_L$ be given by 
        \hyperref[def:gradient_mapping]{Definition \ref*{def:gradient_mapping}}.
        Then for all $x$, the gradient mapping verifies
        \begin{align*}
            x^+ &= \mathcal T_L(x), 
            \\
            \mathcal G_L(x) := L(x - x^+) &\in  \nabla f(x) + \partial g(x^+). 
        \end{align*}
        Equivalently, $\exists v \in \partial g(x^+)$ such that $\mathcal G_L(x) = \nabla f(x) + v = L(x - x^+)$. 
    \end{lemma}
    \begin{proof}
        Using the resolvent definition of the proximal gradient operator and the fact that the single-valuedness in the convex settings, $x^+$ has relations: 
        \begin{align*}
            x^+ &= [I + L^{-1}\partial g]^{-1}\circ [I - L^{-1}\nabla f](x)
            \\
            [I + L^{-1}\partial g](x^+) &\ni [I - L^{-1}\nabla f](x)
            \\
            x^+ + L^{-1}\partial g(x^+) &\ni x - L^{-1}\nabla f(x)
            \\
            x^+ - x + L^{-1}\partial g(x^+) &\ni L^{-1}\nabla f(x)
            \\
            L(x^+ - x) + \partial g(x^+) &\ni - \nabla f(x)
            \\
            L(x - x^+) &\in \nabla f(x) + \partial g(x^+)
            \\
            \mathcal G_L(x) &\in \nabla f(x) + \partial g(x^+). 
        \end{align*}
    \end{proof}

    \begin{lemma}[Linearized gradient mapping lower bound]
    \label{lemma:grad_map_linearization}\; \\
        Suppose that $h = f + g$ satisfies 
        \hyperref[ass:smooth-nonsmooth-sum-lipschitz-grad]
        {Assumption \ref*{ass:smooth-nonsmooth-sum-lipschitz-grad}}, 
        further assume that $f$ is strongly convex with index $\mu \ge 0$. 
        Let $x^+ = \mathcal T_L(x)$ as given in 
        \hyperref[def:gradient_mapping]{Definition \ref*{def:gradient_mapping}}. 
        Then for all $z \in \RR$, it satisfies
        \begin{align*}
            h(z) &\ge 
            h(x^+) + 
            \langle \mathcal G_L (x), z - x\rangle 
            + 
            \frac{L}{2}\Vert x - x^+\Vert^2 + \frac{\mu}{2}
            \Vert z - x\Vert^2. 
        \end{align*}
    \end{lemma}
    \begin{proof}
        Using the $L$-smoothness of $f$ and convexity of $g, f$, we obtained inequalities
        \begin{align*}
            &f(x^+) \le 
            f(x) + \langle \nabla f(x), x^+ - x\rangle
            + \frac{L}{2}\Vert x - x^+\Vert^2, 
            \\
            &
            \frac{\mu}{2}\Vert z - x\Vert^2+ 
            f(x) + \langle \nabla f(x), z - x\rangle 
            \le f(z), 
            \\
            &g(x^+) \le 
            g(z) + \langle v, x^+ - z\rangle\quad 
            \forall v \in \partial g(x^+). 
        \end{align*}
        For all $v \in \partial g (x^+)$, apply the above by considering the following sequence of relations
        \begin{align*}
            h(x^+) &= f(x^+) + g(x^+)
            \\&
            \begin{aligned}
                &\le 
                \left(
                    f(x) + \langle \nabla f(x), x^+ - x\rangle
                    + \frac{L}{2}\Vert x - x^+\Vert^2
                \right)
                \\
                &\qquad  
                + (g(z) + \langle v, x^+ - z\rangle)
            \end{aligned}
            \\&
            \begin{aligned}
                &\le 
                \left(
                    f(z) - \langle \nabla f(x), z - x\rangle
                    - \frac{\mu}{2}\Vert z - x\Vert^2
                    + \langle \nabla f(x), x^+ - x\rangle
                    + 
                    \frac{L}{2}\Vert x - x^+\Vert^2
                \right)
                \\
                &\qquad 
                +
                (g(z) + \langle v, x^+ - z\rangle)
            \end{aligned}
            \\&
            \begin{aligned}
                &= 
                (f(z) + h(z)) 
                \\
                &\qquad 
                + \left(
                    \langle \nabla f(x), x - z\rangle + 
                    \langle \nabla f(x), x^+ - x\rangle + 
                    \langle v, x^+ - z\rangle
                \right) 
                \\ 
                &\qquad 
                - \frac{\mu}{2}\Vert z - x\Vert^2
                + \frac{L}{2}\Vert x - x^+\Vert^2
            \end{aligned}
            \\& 
            \begin{aligned}
                &= h(z) + 
                \left(
                    \langle \nabla f(x), x - x^+ + x^+ - z\rangle + 
                    \langle \nabla f(x), x^+ - x\rangle + 
                    \langle v, x^+ - z\rangle
                \right)
                \\
                &\qquad 
                - \frac{\mu}{2}\Vert z - x\Vert^2
                + \frac{L}{2}\Vert x - x^+\Vert^2
            \end{aligned}
            \\& 
            \begin{aligned}
                &= h(z) + 
                \langle \nabla f(x) + v, x^+ - z\rangle 
                - \frac{\mu}{2}\Vert z - x\Vert^2
                + \frac{L}{2}\Vert x - x^+\Vert^2.
            \end{aligned}
        \end{align*}
        As showed in  
        \hyperref[lemma:grad-map-approx-subgrad]{Lemma \ref*{lemma:grad-map-approx-subgrad}}, 
        we have $\mathcal G_L(x) \in \nabla f(x) + \partial g(x^+)$.
        Choose $v \in \partial g(x^+)$ such that $\mathcal G_L(x) = \nabla f(x) + v = L(x - x^+)$ then
        \begin{align*}
            h(x^+) & 
            \le  
            h(z) + \langle L(x - x^+), x^+ - x + x - z\rangle 
            - \frac{\mu}{2}\Vert z - x\Vert^2
            + \frac{L}{2}\Vert x - x^+\Vert^2
            \\
            &= h(z) + 
            \underbrace{\langle L(x - x^+), x - z\rangle}_{
                = - \langle \mathcal G_L (x), z - x\rangle
            }
            - \frac{\mu}{2}\Vert z - x\Vert^2
            - \frac{L}{2}\Vert x - x^+\Vert^2.
        \end{align*}
        Moving everything except $h(z)$ from the RHS to the LHS yields the desired inequality. 
    \end{proof}
    \begin{remark}
        The inequality is analogous to \cite[(2.2.57)]{nesterov_lectures_2018}, however Nesterov stated it only for the case when $g$ is an indicator function of some convex set $Q$. 
        Additionally, the inequality has a bizzare resemblence to the proximal inequality. 
        Let $h = g + h$ be convex, $f \equiv 0$, consider $x^+ = \hprox_{\eta g}(x)$ for some $\eta > 0$, then by proximal operator being a resolvent it has $\eta^{-1}(x - x^+)\in \partial g(x^+)$, so by convexity it produces inequality for all $z\in \RR^n$: 
        \begin{align*}
            g(z) - g(x^+) 
            &\ge \langle \eta^{-1}(x - x^+), z - x^+\rangle
            \\
            &= \langle \eta^{-1}(x - x^+), z - x + x - x^+\rangle 
            \\
            &= \langle \eta^{-1}(x - x^+), z - x \rangle + 
            \eta\Vert \eta^{-1}(x - x^+)\Vert^2. 
        \end{align*}
    \end{remark}

\section{Different forms of Nesterov's accelerated gradient}\label{sec:different-forms-acc-grad}
    \begin{assumption}\label{ass:smooth-non-smooth-lip-scvx}
        Let $h=f + g$ satisfies 
        \hyperref[ass:smooth-nonsmooth-sum-lipschitz-grad]
        {Assumption \ref*{ass:smooth-nonsmooth-sum-lipschitz-grad}}, 
        Let $\mathcal G_L, \mathcal T_L$ be the gradient mapping and proximal gradient operator as given by 
        \hyperref[def:gradient_mapping]
        {Definition \ref*{def:gradient_mapping}}. 
        In addition, we also assume that $f$ is strongly convex with constant $L \ge \mu \ge 0$. 
    \end{assumption}
        In this section, ``SC'' means strongly convex, ``PPM'' means proximal point method and ``APG'' means accelerated proximal gradient. 
        
        \textcolor{red}{STATE THE OBJECTIVES OF THIS SECTION CLEARER!}
        
        Definitions listed below are generic algorithms with underdetermined parameters that can define to accomodate accelerated proximal gradient algorithms appeared in the literatures. 
        We also state the results on the conditions for algebraic equivalency between different forms of accelrated proximal gradient. 
    \subsection{Generic forms of Nesterov accelerated gradient}
        \newcommand{\SCPPMAPG}{SC PPM APG}
        \begin{definition}[\SCPPMAPG]\label{def:SC-PPM-APG}
            
            Let $h = f + g, \mathcal G_L, \mathcal T_L$ and parameter $L \ge \mu \ge 0$ be given by
            \hyperref[ass:smooth-non-smooth-lip-scvx]
            {Assumption \ref*{ass:smooth-non-smooth-lip-scvx}}. 
            Define the lower bouding function $l(\cdot, x)$ for any fixed $x \in \RR^n$: 
            \begin{align*}
                (\forall z \in \RR^n) \; l_h(z; x) = h(\mathcal T_L x) + \langle \mathcal G_L (x), z - x\rangle
                + 
                \frac{L}{2}\Vert x - \mathcal T_L (x)\Vert^2 + \frac{\mu}{2}\Vert z - x\Vert^2. 
            \end{align*}
            For any positive sequence of stepsizes $(\tilde \eta_t)_{t \in \N}, (\eta_t)_{t \in \N}$, and any initial iterates $x_0 = y_0$, let parameter $L \ge \mu \ge 0$, we define an algorithm such that iterates $(x_t, y_t)_{t \in \N}$ is given by: 
            \begin{align*}
                x_{t + 1} &= \argmin_{x} \left\lbrace
                l_h(x; y_t) + \frac{1}{2\tilde \eta_{t}} 
                \Vert x - x_t\Vert^2
                \right\rbrace
                \\
                &= (\mu\tilde \eta_{t} + 1)^{-1} 
                (\mu\tilde \eta_{t}y_t + x_t - \tilde \eta_{t}\mathcal G_L(y_t))
                \\
                y_{t + 1}&= 
                \argmin_{x}
                \left\lbrace
                    h(\mathcal T_L y_t) + \langle \mathcal G_L(y_t), x - y_t\rangle + \frac{L}{2}\Vert x -y_t\Vert^2
                    + \frac{1}{2\eta_{t + 1}}\Vert x - x_{t + 1}\Vert^2
                \right\rbrace
                \\
                &= (1 + L \eta_{t +1})^{-1}(L\eta_{t + 1}(y_t - L^{-1}\mathcal G_L(y_t)) + x_{t + 1})
            \end{align*}
        \end{definition}
        \begin{remark}
            For a derivation of the iterates see 
            \hyperref[proof:derivations-SC-PPM-AGP]
            {Proof \ref*{proof:derivations-SC-PPM-AGP}}. 
            The update sequence can be equivalently written as: 
            \begin{align*}
                w_{t} &= (\mu\tilde \eta_{t} + 1)^{-1}(\mu \tilde \eta_{t} y_t + x_t) 
                \\
                x_{t + 1}&= w_t - \tilde \eta_{t}(\mu\tilde \eta_{t} + 1)^{-1} \mathcal G_L(y_t)
                \\
                z_{t + 1}&= y_t - L^{-1}\mathcal G_L(y_t)
                \\
                y_{t + 1} &= (1 + L\eta_{t + 1})^{-1}(L\eta_{t + 1}z_{t + 1} + x_{t + 1}). 
            \end{align*}
            If $h = f$, so $g \equiv 0$ then $\mathcal G_L = \nabla f$, then this is equivalent to \cite[(6.24)]{ahn_understanding_2022}. 
        \end{remark}

        \begin{definition}[Generic APG]\label{def:generic-apg}
            Let $h = f + g, \mathcal G_L, \mathcal T_L, L$ be given by 
            \hyperref[ass:smooth-non-smooth-lip-scvx]{Assumption \ref*{ass:smooth-non-smooth-lip-scvx}}.
            This algorithm is a special case of
            \hyperref[def:SC-PPM-APG]{Definition \ref*{def:SC-PPM-APG}}
            and it's obtained by setting $\mu = 0$: 
            \begin{align*}
                x_{t + 1} &= x_t - \tilde \eta_{t} \mathcal G_L (y_t)
                \\
                z_{t + 1} &= y_t - L^{-1}\mathcal G_L (y_t)
                \\
                y_{t + 1} &= (1 + L \eta_{t + 1})^{-1}(L\eta_{t + 1}z_{t + 1} + x_{t + 1}). 
            \end{align*}

        \end{definition}

        \begin{definition}[SC Generic Similar Triangle]\label{def:SC-similar-triangle}
            Let $h = f + g, \mathcal G_L, \mathcal T_L, L$ be given by 
            \hyperref[ass:smooth-non-smooth-lip-scvx]
            {Assumption \ref*{ass:smooth-non-smooth-lip-scvx}}. 
            Let iterates $(x_t, y_t, z_t)$ be given by 
            \hyperref[def:SC-PPM-APG]{Definition \ref*{def:SC-PPM-APG}}. 
            If in addition, the stepsizes $\eta_t, \tilde \eta_t$ satisfies equality 
            $$
                \tilde\eta_{t} = \eta_t + L^{-1} + L^{-1} \mu \tilde\eta_{t},
            $$
            then 
            $$
            x_{t + 1} = z_{t + 1} + 
            \frac{L\eta_t}{1 + \mu \tilde \eta_{t}} 
            (z_{t + 1} - z_t). 
            $$ 
        \end{definition}

        \begin{definition}[Generic Nesterov's Momentum]\label{def:generic-nesterov-momentum-form}
            Let $h = f + g, \mathcal G_L, \mathcal T_L$ and parameters $L$ be given by 
            \hyperref[ass:smooth-non-smooth-lip-scvx]
            {Assumption \ref*{ass:smooth-non-smooth-lip-scvx}}. 
            Let sequence $\theta_{t + 1} \ge 0$ be a generic momentum sequence. 
            Then the momentum form of the algorithm has iterates satisfying the recurrences: 
            $$
            \begin{aligned}
                z_{t + 1} &= y_t - L^{-1}\mathcal G_L(y_t)
                \\
                y_{t + 1 } &= z_{t + 1} + \theta_{t + 1}(z_{t + 1} - z_t). 
            \end{aligned}
            $$
        \end{definition}
        \begin{remark}
            It's called Nesterov's momentum because there is also inertia method which is a different class of methods because the point where the gradient is evaluated is different. 
        \end{remark}

    \subsection{Nesterov type acceleration in the literatures}
        \begin{definition}[Chambolle, Dossal 2015]\label{def:chambolle-dossal-2015}
            Let $h = f + g, \mathcal G_L, \mathcal T_L, L$ be given by 
            \hyperref[ass:smooth-non-smooth-lip-scvx]
            {Assumption \ref*{ass:smooth-non-smooth-lip-scvx}}. 
            The algorithm generates iterates $(y_n, x_n, z_n)$ and the sequence $(t_n)_{n \in \N}$ satisfying the recurrences for all $n \in \N \cup \{0\}$: 
            \begin{align}
                z_{n + 1} &= y_n - L^{-1}\mathcal G_L(y_n)
                \\
                x_{n + 1} &= z_n + t_{n} (z_{n + 1} - z_n), 
                \\
                y_{n + 1} &= \left(
                    1 - \frac{1}{t_{n + 1}}
                \right)z_{n + 1} + \left(
                    \frac{1}{t_{n + 1}}
                \right)x_{n + 1}. 
            \end{align}
            Where the sequence $t_n$ satisfies $t_{n + 1}^2 - t_n^2 \le t_{n + 1}$ for all $n \in \N\cup \{0\}$. 
        \end{definition}
        \begin{remark}
            This is proposed in Chambolle and Dossal \cite{chambolle_convergence_2015}'s paper. 
        \end{remark}

        \begin{definition}[V-FISTA]\label{def:v-fista}
            With $h = f + g, \mathcal G_L, \mathcal T_L$ and parameters $L \ge \mu > 0$  be given by Assumption Set 2. 
            Then V-FISTA is presented as: 
            \begin{align*}
                z_{t + 1} 
                &= y_t - L^{-1}\mathcal G_L(y_t)
                \\
                y_{t + 1} &= z_{t + 1} + 
                \frac{\sqrt{\kappa} - 1}{\sqrt{\kappa} + 1}
                (z_{t +1} - z_t). 
            \end{align*}
            Where the condition number $\kappa = L/\mu$. 
        \end{definition}
        \begin{remark}
            This is presented in Beck's Book chapter 10 \cite[(10.7.7)]{beck_first-order_2017}. 
        \end{remark}

    \subsection{The relations between these forms}\label{sec:relations-between-forms}
        One pivotal intermadiate form is 
        \hyperref[def:SC-similar-triangle]
        { Definition \ref*{def:SC-similar-triangle}} 
        (SC Similar Triangle Form), it is an instance of
        \hyperref[def:SC-PPM-APG]
        {Definition \ref*{def:SC-PPM-APG}}
        where algorithms in the literatures are derived from. 
        \hyperref[prop:s-cvx-generic-sim-triangle-from-s-cvx-ppm-apg]
        {Proposition \ref*{prop:s-cvx-generic-sim-triangle-from-s-cvx-ppm-apg}}
        Shows that SC Similar Triangle Form is an instance of \SCPPMAPG where the step size sequences $\tilde \eta_t, \eta_t$ satisfies equality: 
        $$
            \tilde \eta_t =\eta_t + L^{-1} + L^{-1}\mu \tilde \eta_t. 
        $$
        \hyperref[prop:momentum-equiv-to-similar-triangle]
        {Proposition \ref*{prop:momentum-equiv-to-similar-triangle}}
        shows that SC Similar Triangle Form is algebraically equivalent to 
        \hyperref[def:generic-nesterov-momentum-form]{Definition \ref*{def:generic-nesterov-momentum-form}} (Generic Nesterov's Momentum form).
        The momentum parameter is linked to the step size parameters by the equality: 
        \begin{align*}
            \theta_{t + 1} = \frac{L\eta_t}{(1 + \mu \tilde\eta_{t})(1 + L\eta_{t + 1})}.
        \end{align*} 
        \hyperref[prop:cham-dossal-2015-is-similar-triangle]
        {Proposition \ref{prop:cham-dossal-2015-is-similar-triangle}}
        shows that Chambolle, Dossal 2015 (\hyperref[def:chambolle-dossal-2015]
            {Definition \ref*{def:chambolle-dossal-2015}})
        is an instance of SC Similar Triangle form if and only if when $\mu = 0$. 
        The parameters $t_n$ is the literatures is linked to the step size parameters $\tilde \eta_n, \eta_n$ by the equality: 
        \begin{align*}
            t_n = L \tilde \eta_n. 
        \end{align*}

        
\section{Lyapunov analysis}\label{sec:lyapunov-analysis}
    
    \textcolor{red}{STATE THE OBJECTIVES OF THIS SECTION CLEARER!RESTATE CONTRIBUTIONS IN MORE DETAILS}

    In this section we prove the convergence rate of Similar triangle form when $\mu = 0$. 
    For that we make the following necessary assumption: 
    \begin{assumption}\label{ass:comp-with-minimizers}
        Let $h = f + g$ satisfies 
        \hyperref[ass:smooth-non-smooth-lip-scvx]
        {Assumption \ref*{ass:smooth-non-smooth-lip-scvx}}, 
        in addition assume that $x_+ \in \argmin_x \{h(x)\}$. 
    \end{assumption}
    Next we state the generic algorithm. 
    \begin{definition}[Similar Triangle with initial conditions]
        \label{def:similar-triangle-to-prove}
        Assume
        \hyperref[ass:comp-with-minimizers]
        {Assumption \ref*{ass:comp-with-minimizers}}
        for all $k\in \N \cup \{0\}$, let the iterates $(y_k, x_{k + 1}, z_{k +1})$ be generated by the following relations: 
        \begin{align}
            y_k &= (1 - (L\tilde \eta_k)^{-1}) z_k + (L\tilde \eta_k)^{-1} x_k, 
            \\
            x_{k + 1} &= x_k - \tilde \eta_k \mathcal G_L(y_k), 
            \\
            z_{k + 1} &= y_k - L^{-1} \mathcal G_L(y_k). 
            \label{eqn:similar-triangle-to-prove}
        \end{align}
        This is obtained by setting $\tilde \eta_k = L^{-1} + \eta_k$ for all $k \in \N\cup \{0\}$, hence making it an instance of the Similar Triangle form 
        (\hyperref[def:SC-similar-triangle]
            {Definition \ref*{def:SC-similar-triangle}}) 
        where $\mu =0$. 
        Let $\tilde \eta_0 = L^{-1}$, $\sigma_0 = L^{-1}$, so $y_0 = x_0$ is the initial conditions. 
        Define the notations 
        $$
            (\forall k \in \N \cup \{0\})\; \Delta_k = f(z_k) - f(x_+). 
        $$
    \end{definition}
    
    \begin{proposition}[Convergence of Similar Triangle]
        Let iterates $(y_k, x_{k + 1}, z_{k + 1})_{k \in \N\cup \{0\}}$ be given by 
        \hyperref[def:similar-triangle-to-prove]
        {Definition \ref*{def:similar-triangle-to-prove}}. 
        If there exists a sequence $\sigma_k$ such that $\forall k \in \N$ satisfies
        \begin{align*}
            \sigma_k - \sigma_{k - 1}&\le \tilde \eta_k 
            \\
            \sigma_k &= L\tilde \eta_k^2. 
        \end{align*}
        Then the quantity: 
        \begin{align*}
            (\forall k \in \N)\; 
            \Phi_k &= \sigma_k \Delta_k + (1/2) \Vert x_{k + 1} - x_+\Vert^2, 
        \end{align*}
        has $\Phi_{i} - \Phi_{i - 1} \le 0$ for all $i\in \N$.
        Hence $\Delta_N$ converges at a rate of $\mathcal O(\sigma_N^{-1})$
    \end{proposition}
    \begin{proof}
        We label several important expression used for stating the convergecne results, and then we establish the proof for them at the end of the proof. 
        Firstly we have for all $t \in \N\cup \{0\}$: 
        \begin{align}
            \Delta_k &\le
            - \langle \mathcal G_L y_k, x_+ - y_k \rangle - \frac{1}{2L} \Vert \mathcal G_L y_k\Vert^2
            \notag
            \\
            \iff 
            0& \ge 
            \Delta_k + \langle \mathcal G_L y_k, x_+ - y_k\rangle + \frac{1}{2L}\Vert \mathcal G_L y_k\Vert^2. 
            \label{eqn:st2p-1}
        \end{align}
        \begin{align}
            \Delta_k - \Delta_{k -1} 
            &\le 
            - \langle \mathcal G_L y_k, z_k - y_k\rangle - \frac{1}{2L}\Vert \mathcal G_L y_k\Vert^2
            \notag
            \\
            \iff 
            0 &\ge
            \Delta_k - \Delta_{k - 1}
            + \langle \mathcal G_L y_k, z_k - y_k\rangle
            + \frac{1}{2L} \Vert \mathcal G_L y_k \Vert^2. 
            \label{eqn:st2p-2}
        \end{align}
        If we assume that $\sigma_k - \sigma_{k - 1} \le \epsilon_k, \epsilon_k \ge 0$ for all $k \in \N$ it has: 
        \begin{align}
            \sigma_k\Delta_k - \sigma_{k -1}\Delta_{k -1}
            &\le 
            (\sigma_k - \epsilon_k)(\Delta_k - \Delta_{k -1}) + \epsilon_k \Delta_k. 
            \label{eqn:st2p-3}
        \end{align}
        Finally, for all $k \in \N \cup \{0\}$ we have the equality: 
        \begin{align}
            \frac{1}{2}
            (
                \Vert x_{k + 1} - x_+\Vert^2 - \Vert x_k - x_+\Vert^2
            )
            &= 
            - \tilde \eta_k \langle \mathcal G_L, x_k - x_+\rangle
            + \frac{\tilde \eta_k^2}{2}\Vert \mathcal G_L(y_k)\Vert^2. 
            \label{eqn:st2p-4}
        \end{align}
        \par
        Next, we show that 
        Equation (\ref*{eqn:st2p-1}), 
        (\ref*{eqn:st2p-2}), 
        (\ref*{eqn:st2p-3}), 
        (\ref*{eqn:st2p-4}) 
        can construct the convergence rate without too many details. 
        \\
        Consider $(\sigma_k - \epsilon_k)(\ref*{eqn:st2p-2}) + \epsilon_k(\ref*{eqn:st2p-1}) \le 0$ for any $k \in \N$, it simplifies to (it's proved later):
        {\small
        \begin{align}
            (\sigma_k - \epsilon_k)(\Delta_k - \Delta_{k - 1}) + \epsilon_k \Delta_k 
            + 
            \left(
                \langle 
                    \mathcal G_L y_k, (\sigma_k - \epsilon_k)z_k - \sigma_k y_k + \epsilon_k x_+
                \rangle
                + 
                \frac{\sigma_k}{2L}\Vert \mathcal G_L y_k\Vert^2
            \right) &\le 0. 
            \label{eqn:st2p-5}
        \end{align}
        }
        Using the algorithm update, recall that the iterates satisfy the relations, for all $k \in \N$: 
        \begin{align*}
            y_k &= (1 - (L \tilde \eta_k)^{-1})z_k + (L\tilde \eta_k)^{-1}x_k
            \\
            L\tilde \eta_k y_k &= 
            (L \tilde \eta_k - 1)z_k + x_k
            \\
            -x_k &= 
            (L\tilde \eta_k - 1)z_k - L \tilde \eta_k y_k, 
        \end{align*}
        if we add the equality $L\tilde \eta_k = \epsilon_k^{-1}\sigma_k$, it transform the above into:
        \begin{align*}
            -x_k &= (\epsilon_k^{-1}\sigma_k - 1)z_k - \epsilon_k^{-1}\sigma_k y_k,
        \end{align*}
        factor out $\epsilon_k$ in the cross product term in Equation (\ref*{eqn:st2p-5}) , it allows us to substitute the cross product in $(\ref*{eqn:st2p-5})$: 
        \begin{align}
            (\sigma_k - \epsilon_k)(\Delta_k - \Delta_{k - 1}) + \epsilon_k \Delta_k 
            + 
            \left(
                \epsilon_k 
                \langle \mathcal G_L y_k, x_+ - x_k\rangle
                + 
                \frac{\sigma_k}{2L}\Vert \mathcal G_L y_k\Vert^2
            \right) &\le 0. 
            \label{eqn:st2p-5.1}
        \end{align}
        Comparing 
        Equation (\ref*{eqn:st2p-4}) 
        with the above, if we let $\epsilon_k = \tilde \eta_k$ it makes $L \tilde \eta_k^2 = \sigma_t$ because $L\tilde \eta_k = \epsilon_k^{-1} \sigma_k$, giving us equality: 
        \begin{align*}
            \epsilon_k \langle  \mathcal G_Ly_k, x_+ - x_k\rangle + 
            \frac{\sigma_k}{2L}\Vert \mathcal G_L y_k\Vert^2
            = 
            \tilde \eta_k
            \langle  \mathcal G_Ly_k, x_+ - x_k\rangle
            + 
            \frac{\tilde \eta_k^2}{2}\Vert \mathcal G_L y_k\Vert^2. 
        \end{align*}
        Substituting (\ref*{eqn:st2p-4}) into (\ref*{eqn:st2p-5.1}) yields for all $k \in \N$ inequality: 
        \begin{align*}
            (\sigma_k - \epsilon_k)(\Delta_k - \Delta_{k - 1}) + \epsilon_k \Delta_k 
            + 
            \frac{1}{2}
            (
                \Vert x_{k + 1} - x_+\Vert^2 - \Vert x_k - x_+\Vert^2
            )
            &\le 0
            \\
            \underset{(\ref*{eqn:st2p-3})}{\implies}
            \sigma_k \Delta_k - \sigma_{k - 1}\Delta_{k - 1}
            + 
            \frac{1}{2}
            (
                \Vert x_{k + 1} - x_+\Vert^2 - \Vert x_k - x_+\Vert^2
            )
            &\le 
            0
            \\
            \iff 
            \sigma_k \Delta_k + \frac{1}{2}\Vert x_{k + 1} - x_+\Vert^2
            - 
            \left(
                \sigma_{k - 1}\Delta_{k - 1}
                + 
                \frac{1}{2}\Vert x_k - x_+\Vert^2
            \right)
            &\le 
            0.
        \end{align*}
        Let's makes Lyapunov quantity $\Phi_k = \sigma_k \Delta_k + (1/2)\Vert x_{k + 1} - x_+\Vert^2$. 
        Hence the above inequality is $\Phi_k - \Phi_{k - 1}$, telescoping it then yields the inequality: 
        \begin{align*}
            \sum_{i = 1}^{N}\Phi_i - \Phi_{i - 1} = \Phi_N - \Phi_0 
            &\le 0
            \\
            \sigma_N \Delta_N - \sigma_0 \Delta_0 &\le 
            \frac{1}{2}
            \left(
                \Vert x_1 - x_+\Vert^2 - \Vert x_{N + 1} - x_+\Vert^2
            \right)
            \\
            \sigma_N \Delta_N 
            &\le 
            L^{-1}(f(z_1) - f(x_+))
            +
            \frac{1}{2}\left(
                \Vert x_1 - x_+\Vert^2 - \Vert x_{N +1} - x_+\Vert^2
            \right)
            \\
            \implies 
            \sigma_N (f(z_{N + 1}) - f(x_+)) &\le 
            L^{-1}(f(z_1) - f(x_+)) + 
            \frac{1}{2}\Vert x_1 - x_+\Vert^2.
        \end{align*}
        hence it derives the convergence rate of $\mathcal O(\sigma_{N}^{-1})$. 
        \\
        The proof for Equation 
        (\ref*{eqn:st2p-1}, \ref*{eqn:st2p-2}, \ref*{eqn:st2p-3}, \ref*{eqn:st2p-4}, \ref*{eqn:st2p-5}) and Equation (\ref*{eqn:st2p-5.1}) are continued in 
        \hyperref[sec:st2p_proof_continued]{Appendix \ref*{sec:st2p_proof_continued}}. 
    \end{proof}
    \begin{observation}
        Substitute $\sigma_k = L \tilde \eta_k^2$ into $\sigma_k - \sigma_{k - 1} \le \tilde \eta_k$ yields inequality:
        \begin{align*}
            & L \tilde \eta_k^2 - L \tilde \eta_{k - 1}^2 
            \le 
            \tilde \eta_k
            \\
            & \text{If }  \tilde \eta_k := L^{-1} t_k, \text{ then equivalently: }
            \\
            & L^{-1}t_k^2 - L^{-1}t_{k - 1}^2
            \le L^{-1}t_k 
            \\
            & 
            t_k^2 - t_{k - 1}^2
            \le t_k, 
        \end{align*}
        The sequence $\sigma_k$ is defined to represent the convergence rate. 
        It's made with the intention to separate the parameter that relates to the rate of the convergence analysis of the algorithm (which is $\sigma_k$), and the step size used in the algorithm (which is $\tilde \eta_t$) because they are fundamentally different concepts. 
    \end{observation}

\section{Numerical experiments}\label{sec:numerical-experiments}


% \printbibliography

\bibliographystyle{siam}
\bibliography{references/refs}

\appendix
\section{Postoned proofs}
    \subsection{Proof for \SCPPMAPG}\label{proof:derivations-SC-PPM-AGP}
        \begin{proof}
            (Proof of 
            \hyperref[def:SC-PPM-APG]
            {Definition \ref*{def:SC-PPM-APG}})
            \\
            The functions inside of ``argmin" is easy to solve because they are just quadratic functions. 
            We write it here for future verifications and a peace of the mind. 
            \begin{align*}
                x_{t + 1} &= \argmin_{x}\left\lbrace
                    \langle \mathcal G_L(y_t), x - y_t\rangle 
                    + 
                    \frac{\mu}{2}\Vert x - y_t\Vert^2 +  
                    \frac{1}{2\tilde \eta_{t}}\Vert x - x_t\Vert^2
                \right\rbrace
                \\
                \iff 
                \mathbf 0 & = 
                \mathcal G_L(y_t) + \mu(x - y_t) + \tilde \eta_{t}^{-1}(x - x_t)
                \\
                &= 
                \mathcal G_L(y_t) + (\mu + \tilde \eta_{t}^{-1}) x - \mu y_t - \tilde \eta_{t}^{-1} x_t
                \\
                \iff 
                (\mu + \tilde \eta_{t}^{-1})x 
                &= 
                \mu y_t + \tilde \eta_{t}^{-1} x_t - \mathcal G_L(y_t)
                \\
                \implies 
                x &= (\mu + \tilde \eta_{t}^{-1})^{-1 }
                (\mu y_t + \tilde \eta_{t}^{-1} x_t - \mathcal G_L(y_t)). 
            \end{align*}
            We can make the assumption that $\mu + \eta_{t}^{-1} > 0$ because $\tilde\eta_t > 0$. 
            Similarly for $y_{t + 1}$, it's solving a simple quadratic minimization problem, yielding: 
            \begin{align*}
                \mathbf 0 &= \mathcal G_L(y_t) + L(x - y_t) + \eta_{t + 1}^{-1}(x - x_{t + 1})
                \\
                &= (L + \eta_{t + 1}^{-1})x - L y_t - \eta_{t + 1}^{-1}x_{t + 1} + \mathcal G_L(y_t) 
                \\
                (L + \eta_{t + 1}^{-1})x &= 
                Ly_t + \eta_{t + 1}^{-1} x_{t + 1} - \mathcal G_L(y_t)
                \\
                \implies 
                x &= 
                (L\eta_{t + 1} + 1)^{-1}(L\eta_{t + 1}(y_t - L^{-1}\mathcal G_L(y_t)) + x_{t + 1}). 
            \end{align*}
            We had verified the results for the peace of the mind. 
        \end{proof}

    \subsection{Analysis of the forms}
        \begin{proposition}[SC Generic similar triangle]
            \label{prop:s-cvx-generic-sim-triangle-from-s-cvx-ppm-apg}
            \hyperref[def:SC-similar-triangle]
            {Definition \ref*{def:SC-similar-triangle}}
            is a special case of the 
            \hyperref[def:SC-PPM-APG]{Definition \ref*{def:SC-PPM-APG}}. 
            (\SCPPMAPG). 
            Suppose that $(x_t, y_t, z_t), \eta_t, \tilde \eta_t$ be the iterates and the stepsize sequences be given by the 
            \SCPPMAPG.
            If in addition, the sequence $\tilde \eta_t, \eta$ satisfies the conditions for all $t \in \N$
            \begin{align*}
                \tilde\eta_{t} &= \eta_t + L^{-1} + L^{-1} \mu \tilde\eta_{t}, 
            \end{align*}
            then $x_{t +1}$ in the the AG Generic Form has alternative representation: 
            \begin{align*}
                x_{t + 1} &= 
                z_{t + 1} + 
                \frac{L\eta_t}{1 + \mu \tilde \eta_{t}} 
                (z_{t + 1} - z_t). 
            \end{align*}
            This would at the end, produce the following relations for all $t \in \N$: 
            \begin{align*}
                z_{t + 1} &= 
                y_t - L^{-1}\mathcal G_L(y_t), 
                \\
                x_{t + 1}&= 
                z_{t + 1} + \frac{L\eta_t}{1 + \mu\tilde \eta_{t}}(z_{t + 1} - z_t), 
                \\
                y_{t + 1}&= 
                (1 + L\eta_{t + 1})^{-1} (L\eta_{t + 1}z_{t + 1} + x_{t + 1}). 
            \end{align*}
        \end{proposition}
        \begin{proof}
            We start by showing that there exists a constant $\alpha \in \RR$ such that $z_{t + 1} - z_t = \alpha (x_{t + 1} - z_{t + 1})$ by $\tilde \eta_{t} = \eta_t + L^{-1} + L^{-1} \mu \tilde \eta_{t}$. 
            Firstly, we have the equality which it's proved at the end. 
            \begin{align*}
                z_{t + 1} - z_t
                &= 
                - (L\eta_t)^{-1} y_t 
                - L^{-1}\mathcal G_L(y_t) + (L \eta_t)^{-1} x_t. 
                \tag{1}
            \end{align*}
            Next, we have this equality which we proved later at the end: 
            \begin{align*}
                x_{t + 1} - z_{t + 1}
                &= 
                (1 + \mu\tilde \eta_{t})^{-1}
                \left(
                    x_t - y_t +     
                    \left(
                        - \tilde \eta_{t} + L^{-1}
                        + \mu \tilde \eta_{t}L^{-1}
                    \right)
                    \mathcal G_L(y_t)
                \right).\tag{2} 
            \end{align*}
            Since 
            \begin{align*}
                \tilde\eta_{t} &= \eta_t + L^{-1} + L^{-1} \mu \tilde\eta_{t}
                \\
                (1 - L^{-1}\mu)\tilde \eta_{t}
                &= L^{-1} + \eta_t 
                \\
                - \tilde \eta_{t} + L^{-1}\mu \tilde \eta_{t}
                + L^{-1}
                &= - \eta_t, 
            \end{align*}
            so substituting 
            \begin{align*}
                x_{t + 1} - z_{t + 1}
                &= 
                (1 + \mu \tilde \eta_{t})^{-1}
                (x_t - y_t - \eta_t \mathcal G_L(y_t))
                \\
                &= (1 + \mu \tilde \eta_{t})^{-1}
                \eta_t(\eta_{t}^{-1}(x_t - y_t) - \mathcal G_L(y_t))
                \\
                &= (1 + \mu \tilde \eta_{t})^{-1}
                \eta_t L(z_{t + 1} - z_t)
                \\
                x_{t + 1} &= 
                z_{t + 1} + 
                \frac{L \eta_t}{1 + \mu \tilde \eta_t}(z_{t + 1} - z_t). 
            \end{align*}
            To show (1): 
            \begin{align*}
                y_{t} &= (1 + L\eta_{t})^{-1}(L\eta_{t}z_{t} + x_{t})
                \\
                (1 + L\eta_t)y_t - x_t &= L\eta_t z_t
                \\
                z_t & = (L\eta_t)^{-1}((1 + L\eta_t)y_t - x_t), 
                \\[1em]
                z_{t + 1} - z_t 
                &= \underbrace{ y_t - L^{-1}\mathcal G_L(y_t)}_{=z_{t + 1}}
                - \underbrace{(L\eta_t)^{-1}((1 + L\eta_t)y_t - x_t)}_{=z_t}
                \\
                &= 
                y_t - L^{-1} \mathcal G_L(y_t) - (L\eta_t)^{-1}y_t - y_t + (L\eta_t)^{-1} x_t
                \\
                &= 
                -L^{-1}\mathcal G_L(y_t) + (L\eta_t)^{-1}(x_t - y_t)
                \\
                &= 
                L^{-1}(\eta_t^{-1}(x_t - y_t) -\mathcal G_L(y_t)). 
            \end{align*}
            To show (2): 
            \begin{align*}
                x_{t + 1} - z_{t + 1}&= 
                \left(
                    (1 + \mu \tilde \eta_{t})^{-1} (\mu \tilde \eta_{t }y_t + x_t)
                    - \frac{\tilde \eta_{t }}{1 + \mu\tilde \eta_{t }}
                    \mathcal G_L(y_t)
                \right) - \left(
                    y_t - L^{-1}\mathcal G_L(y_t)
                \right)
                \\
                &= 
                (1 + \mu \tilde \eta_{t})^{-1}
                \left(
                    x_t + \mu \tilde \eta_{t} y_t
                    - \tilde \eta_{t} \mathcal G_L(y_t)
                    - (1 + \mu \tilde \eta_{t})
                    (y_t - L^{-1}\mathcal G_L(y_t))
                \right)
                \\
                &= 
                (1 + \mu\tilde \eta_{t})^{-1}
                \left(
                    x_t - y_t + 
                    (
                        -\tilde \eta_{t} + 
                        (1 + \mu\tilde \eta_{t})L^{-1}
                    )
                    \mathcal G_L(y_t)
                \right)
                \\
                &= 
                (1 + \mu\tilde \eta_{t})^{-1}
                \left(
                    x_t - y_t +     
                    (
                        - \tilde \eta_{t} + L^{-1}
                        + \mu \tilde \eta_{t}L^{-1}
                    )
                    \mathcal G_L(y_t)
                \right). 
            \end{align*}
        \end{proof}

        \begin{proposition}[Momentum form is equivalent to Similar Triangle form]
            \label{prop:momentum-equiv-to-similar-triangle}
            \;\\
            Let iterates $(x_t, y_t, z_t), \eta_t, \tilde \eta_t$ be given by the relations in 
            \hyperref[def:SC-similar-triangle]
            {Definition \ref*{def:SC-similar-triangle}}
            then it's algebraically equivalent to: 
            \begin{align*}
                y_{t + 1} &= z_{t + 1} + 
                \frac{L\eta_t}{(1 + \mu \tilde\eta_{t})(1 + L\eta_{t + 1})}(z_{t + 1} - z_t)
                \\
                z_{t + 1} &= y_t - L^{-1}\mathcal G_L(y_t). 
            \end{align*}
            Which fits
            \hyperref[def:generic-nesterov-momentum-form]
            {Definition \ref*{def:generic-nesterov-momentum-form}}
            with 
            $$
                \theta_{t + 1} = \frac{L\eta_t}{(1 + \mu \tilde\eta_{t})(1 + L\eta_{t + 1})}.
            $$
        \end{proposition}
        \begin{proof}
            From the similar triangle form, directly substituting the value of $x_{t + 1}$ into the expression for $y_{t + 1}$ it yields: 
            \begin{align*}
                y_{t + 1} &= 
                (1 + L\eta_{t + 1})^{-1} (L\eta_{t + 1}z_{t + 1} + x_{t + 1})
                \\
                &= 
                (1 + L\eta_{t + 1})^{-1}
                \left(
                    L\eta_{t + 1}z_{t + 1} + z_{t + 1} + \frac{L\eta_t}{1 + \mu\tilde \eta_t}(z_{t + 1} - z_t)
                \right)
                \\
                &= 
                (1 + L\eta_{t + 1})^{-1}
                \left(
                    (1 + L\eta_{t + 1})z_{t +1} + 
                    \frac{L\eta_t}{1 + \mu\tilde \eta_t}(z_{t + 1} - z_t)
                \right)
                \\
                &= 
                z_{t + 1} + 
                \frac{L\eta_t}{(1 + L\eta_{t + 1})(1 + \mu\tilde \eta_t)}
                (z_{t + 1} - z_t)
            \end{align*}
        \end{proof}
        
        \begin{proposition}[Chambolle Dossal as an instance of Similar Triangle Form]
            \label{prop:cham-dossal-2015-is-similar-triangle}
            \;\\
            Let iterates $(x_t, y_t, z_t)$ and stepszie sequence $\tilde \eta_t, \eta_t$ be generated by SC Generic Similar Triangle form
            (\hyperref[def:SC-similar-triangle]
            {Definition \ref*{def:SC-similar-triangle}}). 
            Then the algorithm are equivalent to Chambolle and Dossal 2015 algorithms 
            (\hyperref[def:chambolle-dossal-2015]
            {Definition \ref*{def:chambolle-dossal-2015}})
            if and only if we have sequence $t_n = L \tilde \eta_{n}$ and $\mu = 0$. 
        \end{proposition}
        \begin{proof}
            First, we present $\eta_t$ using $\tilde \eta_t$. 
            The Generic Similar Triangle Form admits has $x_{t + 1}$ to be: 
            \begin{align*}
                x_{t + 1} 
                &= z_{t + 1} + (1 + \mu\tilde \eta_{t})^{-1}L\eta_t (z_{t + 1} - z_t)
                \\
                &=  
                (1 + L\eta_t(1 + \mu \tilde \eta_{t})^{-1})z_{t + 1}
                - L\eta_t(1 + \mu\tilde \eta_{t})^{-1} z_t
                \\
                &= 
                \left(
                    \frac{1 + \mu\tilde \eta_{t} + L \eta_t}{1 + \mu \tilde \eta_{t}}
                \right)z_{t + 1}
                + 
                z_t - 
                \left(
                    1 + \frac{ L\eta_t}{\mu \tilde \eta_t + 1}
                \right)z_t
                \\
                & 
                \textcolor{gray}{
                    \text{use: } \tilde \eta_t = \eta_t + L^{-1} + L^{-1}\mu\tilde \eta_t
                }
                \\
                &= 
                \left(
                    \frac{L\tilde \eta_{t}}{1 + \mu\tilde \eta_{t}}
                \right)z_{t + 1}
                + 
                z_t 
                - 
                \left(
                    \frac{L\tilde \eta_{t}}{1 + \mu \tilde \eta_{t}}
                \right)z_t 
                \\
                &= z_t + 
                \left(
                    \frac{L\tilde \eta_{t}}{1 + \mu \tilde \eta_{t}}
                \right)
                (z_{t + 1} - z_t). 
            \end{align*}
            Recall that Chmabolle Dossal 2015 has $x_{n + 1} = z_t + t_n(z_{t + 1} - z_t)$, 
            therefore above suggests that $t_n = L \tilde \eta_{n}(1 + \mu \tilde \eta_{n})^{-1}$. 
            From the second equality to the third equality, we use the relations $L\tilde \eta_{t} = \eta_t + 1 + \mu\tilde \eta_{t}$.
            At the same time, we check that: 
            \begin{align*}
                y_{t + 1} &= \left(
                    \frac{L\eta_{t + 1}}{1 + L \eta_{t + 1}}
                \right)z_{t + 1} + 
                \left(
                    \frac{1}{1 + L \eta_{t + 1}}
                \right) x_{t + 1}
                \\
                &= 
                \left(
                    1 - \frac{1}{1 + L\eta_{t + 1}}
                \right)z_{t + 1} + 
                \left(
                    \frac{1}{1 + L \eta_{t + 1}}
                \right)x_{t + 1}. 
            \end{align*}
            Recall that in Chambolle Dossal 2015, itertes $y_{n + 1} = (1 - t_{n + 1}^{-1})z_{n + 1} + t_{n + 1}^{-1} x_{n + 1}$. 
            Therefore, the above results suggests that $t_{n + 1} = 1 + L\eta_{n + 1}$. 
            However, by the relations between $\tilde \eta_n, \eta_n$ as asserted via the similar triangle form, we have 
            \begin{align*}
                \tilde \eta_t &= \eta_t + L^{-1} + L^{-1}\mu \tilde \eta_t
                \\
                \iff 
                L\tilde \eta_t - \mu \tilde \eta_t 
                &= 
                L \eta_t + 1
                \\
                \iff 
                (L - \mu)\tilde \eta_t &= L\eta_t + 1. 
            \end{align*}
            For $t_n$ to be the same, these two have to be equivalent, but for all $\mu > 0$
            \begin{align*}
                L\tilde \eta_n (1 + \mu \tilde \eta_n)^{-1} \not = 
                (L - \mu )\tilde \eta_n. 
            \end{align*}
            Therefore, they are only equivalent when $\mu = 0$, then the generic similar triangle algorithm would be equivalently represented as the algrithm in Chambolle Dossal 2015's paper. 
        \end{proof}
        \begin{proposition}[V-FISTA is Similar Triangle Method]
            V-FISTA 
            (\hyperref[def:v-fista]{Definition \ref*{def:v-fista}}
            ) is a form of Generic Similar Triangle Form, where the stepsizes $\tilde \eta_t, \eta_t$ are constants: 
            \begin{align*}
                \tilde \eta_t 
                &= \frac{1}{\mu(\sqrt{\kappa} - 1)}
                \quad \forall t \in \N, 
                \\
                \eta_t
                &= 
                \frac{1}{\mu\sqrt{\kappa}}
                \quad \forall t \in \N. 
            \end{align*}
            This choice of step size sequences simplifies it to:
            \begin{align*}
                y_{t + 1} &= z_{t + 1} + 
                \frac{\sqrt{\kappa} - 1}{\sqrt{\kappa} + 1}
                (z_{t +1} - z_t), 
                \\
                z_{t + 1} 
                &= y_t - L^{-1}\mathcal G_L(y_t). 
            \end{align*}
        \end{proposition}
        \begin{proof}
            Removing the subscript for the index because the stepsizes are constant in this case, observe that we have 
            $$
            \begin{aligned}
                L\eta &= \frac{L}{\mu \sqrt{\kappa}} = \frac{\kappa}{\sqrt{\kappa}} = \sqrt{\kappa}, 
                \\
                \mu \tilde \eta &= 
                \frac{1}{\sqrt{\kappa} - 1}, 
                \\
                L\tilde \eta &= 
                \frac{\kappa}{\sqrt{\kappa} - 1}. 
            \end{aligned}
            $$
            With that it establishes relations
            $$
            \begin{aligned}
                \frac{L\eta }{(1 + \mu \tilde \eta)(1 + L\eta)}
                &= 
                \frac{\sqrt{\kappa}}{
                    \left(
                        1 + \frac{1}{\sqrt{\kappa} - 1}
                    \right)
                    \left(
                        1 + \sqrt{\kappa}
                    \right)
                }
                \\
                &= \frac{\sqrt{\kappa}}{
                    \left(
                        \frac{\sqrt{\kappa}}{\sqrt{\kappa} - 1}
                    \right)(1 + \sqrt{\kappa})
                }
                \\
                &=
                \frac{\sqrt{\kappa}}{1 + \sqrt{\kappa}}\left(
                    \frac{\sqrt{\kappa} - 1}{\sqrt{\kappa}}
                \right)
                \\
                &= 
                \frac{\sqrt{\kappa} - 1}{\sqrt{\kappa} + 1}. 
            \end{aligned}
            $$
            The sequence is valid because 
            $$
            \begin{aligned}
                \tilde \eta_{t} 
                &= \eta_t + L^{-1} + L^{-1} \mu \tilde \eta_{t}
                \\
                \iff
                (L - \mu)\tilde \eta_{t}
                &= 
                1 + L \eta_t 
                \\
                \iff
                L \tilde \eta_{t} - 
                \mu \tilde \eta_{t}
                &= 1 + L \eta_t. 
            \end{aligned}
            $$
            Starting from the LHS it yields: 
            $$
            \begin{aligned}
                L\tilde \eta - \mu \tilde \eta 
                &= \frac{\kappa}{\sqrt{\kappa} - 1} - 
                \frac{1}{\sqrt{\kappa} - 1}
                \\
                &= 
                \frac{\kappa - 1}{\sqrt{\kappa} - 1}
                \\
                &= 
                \frac{(\sqrt{\kappa} + 1)(\sqrt{\kappa} - 1)}{\sqrt{\kappa} - 1}
                \\
                &= 1 + \sqrt{\kappa} = 1 + L \eta. 
            \end{aligned}
            $$
        \end{proof}
    \subsection{Lyapunov analysis proof continued}\label{sec:st2p_proof_continued}
        \textbf{Proof for Equation (\ref*{eqn:st2p-1}, \ref*{eqn:st2p-2})}:
        \\
        Fix any $k \in \N\cup \{0\}$, apply
        \hyperref[lemma:grad_map_linearization]
        {Lemma \ref*{lemma:grad_map_linearization}}
        with $z = z_{k + 1}, x = x_+$: 
        \begin{align*}
            \Delta_k &= h(z_{k + 1}) - h(x_+) = h(\mathcal T_L y_k) - h(x_+) 
            \\
            &\le 
            - \langle \mathcal G_l y_k, x_+ - y_k \rangle 
            - \frac{L}{2}\Vert L^{-1} \mathcal G_Ly_k\Vert^2
            \\
            &= 
            - \langle \mathcal G_l y_k, x_+ - y_k \rangle 
            - \frac{1}{2L}\Vert \mathcal G_L y_k\Vert^2
            \\
            \iff 
            0& \ge \Delta_k + \langle \mathcal G_L y_k, x_+ - y_k\rangle + \frac{1}{2L}\Vert \mathcal G_L y_k\Vert^2. 
        \end{align*}
        Similarly, by setting $z = z_{k + 1}, x = z_k$, it gives 
        \begin{align*}
            \Delta_k - \Delta_{k - 1} &= 
            h(z_{k + 1}) - h(x_+) - (h(z_k) - h(x_+)) 
            \\
            &= h(z_{k + 1}) - h(z_k)
            \\
            &= h( \mathcal G_L y_k) - h(z_k)
            \\
            &\le 
            - \langle \mathcal G_L y_k, z_k - y_k\rangle - 
            \frac{1}{2L} \Vert \mathcal G_L y_k \Vert^2
            \\
            \iff 
            0 &\ge
            \Delta_k - \Delta_{k - 1}
            + \langle \mathcal G_L y_k, z_k - y_k\rangle
            + \frac{1}{2L} \Vert \mathcal G_L y_k \Vert^2.
        \end{align*}
        \\
        \textbf{Proof for (\ref*{eqn:st2p-3})}: 
        \\
        Assume that $\sigma_k - \sigma_{k -1} \le \epsilon_k$ where $\epsilon_k \ge 0$ then for all $k \in \N$: 
        \begin{align*}
            &
            \sigma_k \Delta_k - \sigma_{k - 1} \Delta_{k - 1}
            \\
            & 
            \textcolor{gray}{
                \text{by: } \sigma_k - \sigma_{k - 1} \le \epsilon_k, \epsilon_k \ge 0
            }
            \\
            &\le 
            \sigma_k \Delta_k - (\sigma_k - \epsilon_k)\Delta_{k - 1}
            \\
            &= 
            (\sigma_k - \epsilon_k)\Delta_k - (\sigma_k - \epsilon_k)\Delta_{k - 1}
            + \epsilon \Delta_k 
            \\
            &= (\sigma_k - \epsilon_k)(\Delta_k - \Delta_{k - 1}) + \epsilon_k \Delta_k. 
        \end{align*}
        \\
        \textbf{Proof for (\ref*{eqn:st2p-4})}:
        \\
        Using the relations of the iterates as defined by the algorithm, it has $\forall k \in \N\cup \{0\}$
        \begin{align*}
            & \frac{1}{2} 
            \left(
                \Vert x_{k + 1} - x_+\Vert^2
                - 
                \Vert x_k - x_+\Vert^2 
            \right)
            \\
            &=
            \frac{1}{2} 
            \left(
                \Vert x_k - \tilde \eta_k\mathcal G_L (y_k) - x_+ \Vert^2 
                - 
                \Vert x_k - x_+\Vert^2
            \right)
            \\
            &= 
            \frac{1}{2}\left(
                -2\langle \tilde \eta_k \mathcal G_L(y_k), x_k - x_+\rangle
                + 
                \tilde \eta_k^2 \Vert \mathcal G_L(y_k)\Vert^2
            \right)
            \\
            &= 
            \langle \tilde \eta_k \mathcal G_L(y_k), x_k - x_+\rangle
                + 
            \frac{\tilde \eta_k^2}{2} \Vert \mathcal G_L(y_k)\Vert^2. 
        \end{align*}
        \\
        \textbf{Proof for (\ref*{eqn:st2p-5})}:
        \\
        To show $(\sigma_k - \epsilon_k)(\ref*{eqn:st2p-2}) + \epsilon_k(\ref*{eqn:st2p-1}) \le 0$, assume $\sigma_k - \epsilon_k \ge 0$, $\epsilon_k \ge 0$, so: 
        {\footnotesize
        \begin{align*}
            (\sigma_k - \epsilon_k)
            \left(
                \Delta_k - \Delta_{k - 1}
                + \langle \mathcal G_L y_k, z_k - y_k\rangle
                + \frac{1}{2L} \Vert \mathcal G_L y_k \Vert^2
            \right)
            + 
            \epsilon_k 
            \left(
                \Delta_k + \langle \mathcal G_L y_k, x_+ - y_k\rangle + \frac{1}{2L}\Vert \mathcal G_L y_k\Vert^2
            \right)
            &\le 
            0
            \\
            \iff
            (\sigma_k - \epsilon_k)(\Delta_k - \Delta_{k - 1}) 
            + \epsilon_k \Delta_k 
            + 
            \left(
                \langle \mathcal G_L y_k, 
                    (\sigma_k - \epsilon_k)(z_k - y_k)
                    + 
                    \epsilon_k (x_+ - y_k)
                \rangle
                + 
                \frac{\sigma_k}{2L}\Vert \mathcal G_Ly_k\Vert^2
            \right) 
            &\le 0
            \\
            \iff
            (\sigma_k - \epsilon_k)(\Delta_k - \Delta_{k - 1}) 
            + \epsilon_k \Delta_k 
            + 
            \left(
                \langle \mathcal 
                    G_L y_k, 
                    (\sigma_k - \epsilon_k)z_k + 
                    (\epsilon_k - \sigma_k - \epsilon_k)y_k
                    + 
                    \epsilon_k x_+
                \rangle
                + 
                \frac{\sigma_k}{2L}\Vert \mathcal G_Ly_k\Vert^2
            \right) 
            &\le 0
            \\
            \iff
            (\sigma_k - \epsilon_k)(\Delta_k - \Delta_{k - 1}) 
            + \epsilon_k \Delta_k 
            + 
            \left(
                \langle \mathcal 
                    G_L y_k, 
                    (\sigma_k - \epsilon_k)z_k - \sigma_k y_k + \epsilon_k x_+
                \rangle
                + 
                \frac{\sigma_k}{2L}\Vert \mathcal G_Ly_k\Vert^2
            \right) 
            &\le 0. 
        \end{align*}
        }

    

\end{document}
