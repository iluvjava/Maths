\documentclass[12pt]{article}

% 
%\usepackage{showkeys}
%\usepackage{drftcite}
\usepackage{exscale,relsize}
\usepackage{amsmath}
\usepackage{amsfonts}
\usepackage[hidelinks]{hyperref}
\usepackage{amssymb}
\usepackage{calc}
\usepackage{theorem}
\usepackage{pifont}      % needed by dingautolist
\usepackage{array}
\usepackage{color}
\usepackage{enumerate}
\usepackage{bbm}
\usepackage{graphicx}
\usepackage{float}
\usepackage{subfigure}




% FORMATTING ===================================================================
\oddsidemargin -0.1cm
\textwidth  16.5cm
\topmargin  0.0cm
\headheight 0.0cm
\textheight 21.0cm
\parindent  4mm
\parskip    10pt
%\parskip    8pt
\tolerance  3000

% MATH SYMBOLS =================================================================


% These are Heniz's notations. 
\newcommand{\To}{\ensuremath{\rightrightarrows}}
\newcommand{\GX}{\ensuremath{\Gamma}}
\newcommand{\mal}{\ensuremath{\mathfrak{m}}}
\newcommand{\mumu}{\ensuremath{{\mu\mu}}}
\newcommand{\paver}{\ensuremath{\mathcal{P}}}
\newcommand{\ZZZ}{\ensuremath{{X \times X^*}}}
\newcommand{\RRR}{\ensuremath{{\RR \times \RR}}}
\newcommand{\todo}{\hookrightarrow\textsf{TO DO:}}

\newcommand{\emp}{\ensuremath{\varnothing}}
%\newcommand{\la}{\ensuremath{\langle}}
%\newcommand{\ra}{\ensuremath{\rangle}}
\newcommand{\infconv}{\ensuremath{\mbox{\small$\,\square\,$}}}
\newcommand{\pscal}{\ensuremath{\scal{\cdot}{\cdot}}}
\newcommand{\Tt}{\ensuremath{\mathfrak{T}}}
\newcommand{\YY}{\ensuremath{\mathcal Y}}
\newcommand{\XX}{\ensuremath{\mathcal X}}
\newcommand{\HH}{\ensuremath{\mathcal H}}
\newcommand{\XP}{\ensuremath{\mathcal X}^*}
\newcommand{\st}{\ensuremath{\;|\;}}
\newcommand{\zeroun}{\ensuremath{\left]0,1\right[}}

\newcommand{\lev}[1]{\ensuremath{\mathrm{lev}_{\leq #1}\:}}
\newcommand{\moyo}[2]{\ensuremath{\sideset{_{#2}}{}{\operatorname{}}\!#1}}
\newcommand{\pair}[2]{\left\langle{{#1},{#2}}\right\rangle}
%\newcommand{\scal}[2]{\left.\left\langle{#1}\:\right| {#2}  \right\rangle}
\newcommand{\scal}[2]{\langle{{#1},{#2}}\rangle}
\newcommand{\Scal}[2]{\left\langle{{#1},{#2}}\right\rangle}
%\newcommand{\scal}[2]{\braket{ {#1},{#2}}}

\newcommand{\yosida}{\ensuremath{ \; {}^}}
\newcommand{\exi}{\ensuremath{\exists\,}}
\newcommand{\GG}{\ensuremath{\mathcal G}}
\newcommand{\RR}{\ensuremath{\mathbb R}}
\newcommand{\SSS}{\ensuremath{\mathbb S}}
\newcommand{\CC}{\ensuremath{\mathbb C}}
\newcommand{\Real}{\ensuremath{\mathrm{Re}\,}}
\newcommand{\ii}{\ensuremath{\mathrm i}}
\newcommand{\RP}{\ensuremath{\left[0,+\infty\right[}}
\newcommand{\RPX}{\ensuremath{\left[0,+\infty\right]}}
\newcommand{\RPP}{\ensuremath{\,\left]0,+\infty\right[}}
\newcommand{\RX}{\ensuremath{\,\left]-\infty,+\infty\right]}}
\newcommand{\RXX}{\ensuremath{\,\left[-\infty,+\infty\right]}}
\newcommand{\KK}{\ensuremath{\mathbb K}}
\newcommand{\NN}{\ensuremath{\mathbb N}}
\newcommand{\nnn}{\ensuremath{{n \in \NN}}}
\newcommand{\thalb}{\ensuremath{\tfrac{1}{2}}}
\newcommand{\pfrac}[2]{\ensuremath{\mathlarger{\tfrac{#1}{#2}}}}
\newcommand{\zo}{\ensuremath{{\left]0,1\right]}}}
\newcommand{\lzo}{\ensuremath{{\lambda \in \left]0,1\right]}}}
%\newcommand{\toppsepp}{\setlength{\partopsep}{-5pt}}
\newcommand{\menge}[2]{\big\{{#1} \mid {#2}\big\}}


% MATH OPERATORS ===============================================================
% \newcommand{\monos}{\ensuremath{\mathcal M}}
\newcommand{\DD}{\operatorname{dom}f}
\newcommand{\IDD}{\ensuremath{\operatorname{int}\operatorname{dom}f}}
\newcommand{\CDD}{\ensuremath{\overline{\operatorname{dom}}\,f}}
\newcommand{\clspan}{\ensuremath{\overline{\operatorname{span}}}}
\newcommand{\cone}{\ensuremath{\operatorname{cone}}}
\newcommand{\dom}{\ensuremath{\operatorname{dom}}}
\newcommand{\closu}{\ensuremath{\operatorname{cl}}}
\newcommand{\cont}{\ensuremath{\operatorname{cont}}}
\newcommand{\mons}{\ensuremath{\mathcal{A}}}
\newcommand{\gra}{\ensuremath{\operatorname{gra}}}
\newcommand{\epi}{\ensuremath{\operatorname{epi}}}
\newcommand{\prox}{\ensuremath{\operatorname{Prox}_{\mu}}}
\newcommand{\hprox}{\ensuremath{\operatorname{prox}}}
\newcommand{\intdom}{\ensuremath{\operatorname{int}\operatorname{dom}}\,}
\newcommand{\inte}{\ensuremath{\operatorname{int}}}
\newcommand{\sri}{\ensuremath{\operatorname{sri}}}
\newcommand{\reli}{\ensuremath{\operatorname{ri}}}
\newcommand{\cart}{\ensuremath{\mbox{\LARGE{$\times$}}}}


\newcommand{\average}{\ensuremath{\mathcal{R}_{\mu}({\bf A},{\boldsymbol \lambda})}}
\newcommand{\averagebar}{\ensuremath{\mathcal{R}_{1}({\bf A},\bar{\lambda})}}
\newcommand{\averageonelambda}{\ensuremath{\mathcal{R}({\bf A},{\boldsymbol \lambda})}}
\newcommand{\averageonehalf}{\ensuremath{\mathcal{R}_{1}(A,1/2)}}
\newcommand{\averageinverse}{\ensuremath{\mathcal{R}_{\mu^{-1}}({\bf A}^{-1},{\boldsymbol \lambda})}}
\newcommand{\averageoneinverse}{\ensuremath{\mathcal{R}({\bf A}^{-1},{\boldsymbol \lambda})}}
\newcommand{\averagef}{\ensuremath{\mathcal{P}_{\mu}(f,\lambda)}}
\newcommand{\averagefone}{\ensuremath{\mathcal{P}_{1}(f,\lambda)}}
\newcommand{\averagefd}{\ensuremath{\mathcal{P}_{\mu}((f_{1},\ldots, f_{n}),(\lambda_{1},\ldots, \lambda_{n}))}}
\newcommand{\averagefik}{\ensuremath{\mathcal{P}_{\mu_{k}}((f_{1,k},\ldots,f_{n,k}),
(\lambda_{1,k},\ldots,\lambda_{n,k}))}}
\newcommand{\averagesub}{\ensuremath{\mathcal{R}_{\mu}(\partial f,\lambda)}}
\newcommand{\res}{\ensuremath{\mathcal{R}_{\mu}}}
\newcommand{\resmuk}{\ensuremath{\mathcal{R}_{\mu_{k}}}}
\newcommand{\newres}{\ensuremath{\mathcal{R}}}
\newcommand{\resmualpha}{\ensuremath{\mathcal{R}_{\alpha\mu}}}
\newcommand{\averageone}{\ensuremath{\mathcal{R}_{1}}}
\newcommand{\harm}{\ensuremath{\mathcal{H}(A,\lambda)}}
\newcommand{\arithmetic}{\ensuremath{\mathcal{A}(A,\lambda)}}

\newcommand{\WC}{\ensuremath{{\mathfrak W}}}
\newcommand{\SC}{\ensuremath{{\mathfrak S}}}
\newcommand{\card}{\ensuremath{\operatorname{card}}}
\newcommand{\bd}{\ensuremath{\operatorname{bdry}}}
\newcommand{\ran}{\ensuremath{\operatorname{ran}}}
\newcommand{\rec}{\ensuremath{\operatorname{rec}}}
\newcommand{\rank}{\ensuremath{\operatorname{rank}}}
\newcommand{\kernel}{\ensuremath{\operatorname{ker}}}
\newcommand{\conv}{\ensuremath{\operatorname{conv}}}
\newcommand{\segh}{\ensuremath{\operatorname{seg}}}
\newcommand{\boxx}{\ensuremath{\operatorname{box}}}
\newcommand{\clconv}{\ensuremath{\overline{\operatorname{conv}}\,}}
\newcommand{\cldom}{\ensuremath{\overline{\operatorname{dom}}\,}}
\newcommand{\clran}{\ensuremath{\overline{\operatorname{ran}}\,}}
\newcommand{\Nf}{\ensuremath{\nabla f}}
\newcommand{\NNf}{\ensuremath{\nabla^2f}}
\newcommand{\Fix}{\ensuremath{\operatorname{Fix}}}
\newcommand{\FFix}{\ensuremath{\overline{\operatorname{Fix}}\,}}
\newcommand{\aFix}{\ensuremath{\widetilde{\operatorname{Fix}\,}}}
\newcommand{\Id}{\ensuremath{\operatorname{Id}}}
\newcommand{\Max}{\ensuremath{\operatorname{max}}}
\newcommand{\Bb}{\ensuremath{\mathfrak{B}}}
\newcommand{\BB}{\ensuremath{\mathbb{B}}}
\newcommand{\Fb}{\ensuremath{\overrightarrow{\mathfrak{B}}}}
\newcommand{\Fprox}{\ensuremath{\overrightarrow{\operatorname{prox}}}}
\newcommand{\Bprox}{\ensuremath{\overleftarrow{\operatorname{prox}}}}
\newcommand{\Bproj}{\ensuremath{\overleftarrow{\operatorname{P}}}}
\newcommand{\Ri}{\ensuremath{\mathfrak{R}_i}}
\newcommand{\Dn}{\ensuremath{\,\overset{D}{\rightarrow}\,}}
\newcommand{\nDn}{\ensuremath{\,\overset{D}{\not\rightarrow}\,}}
\newcommand{\weakly}{\ensuremath{\,\rightharpoonup}\,}
\newcommand{\weaklys}{\ensuremath{\,\overset{*}{\rightharpoonup}}\,}
\newcommand{\gr}{\ensuremath{\operatorname{gra}}}
\newcommand{\g}{\ensuremath{\,\overset{g}{\rightarrow}}\,}
\newcommand{\p}{\ensuremath{\,\overset{p}{\rightarrow}}\,}
\newcommand{\e}{\ensuremath{\,\overset{e}{\rightarrow}}\,}
\newcommand{\Tbar}{\ensuremath{\overline{T}}}
\newcommand{\n}{\ensuremath{\,\overset{n}{\rightarrow}}\,}

\newcommand{\minf}{\ensuremath{-\infty}}
\newcommand{\pinf}{\ensuremath{+\infty}}
\renewcommand{\iff}{\ensuremath{\Leftrightarrow}}
\renewcommand{\phi}{\ensuremath{\varphi}}
%\newcommand{\Real}{\ensuremath{\mathrm{Re}\,}}
\newcommand{\negent}{\ensuremath{\operatorname{negent}}}
\newcommand{\neglog}{\ensuremath{\operatorname{neglog}}}
\newcommand{\halb}{\ensuremath{\tfrac{1}{2}}}
\newcommand{\bT}{\ensuremath{\mathbf{T}}}
\newcommand{\bX}{\ensuremath{\mathbf{X}}}
\newcommand{\bL}{\ensuremath{\mathbf{L}}}
\newcommand{\bD}{\ensuremath{\boldsymbol{\Delta}}}
\newcommand{\bc}{\ensuremath{\mathbf{c}}}
\newcommand{\by}{\ensuremath{\mathbf{y}}}
\newcommand{\bx}{\ensuremath{\mathbf{x}}}
\newcommand{\bA}{{\bf A}}
\newcommand{\Other}{Indeterminate }
\newcommand{\other}{indeterminate }


%%% Raf's stuff  ===============================================================
\newcommand{\al}{\alpha}
\newcommand{\la}{\lambda}
\newcommand{\La}{\Lambda}
\newcommand{\pluss}{{\hskip1pt \raise1pt\vbox{\hrule width6pt \vskip1pt
\hrule width6pt}\kern-4pt{\lower1pt\hbox{\vrule height6pt \kern1pt\vrule
height6pt}}\hskip5pt}}
\newcommand{\timess}{\star}
\newcommand{\argmax}{\mathop{\rm argmax}\limits}
\newcommand{\argmin}{\mathop{\rm argmin}\limits}
\newcommand{\product}{\langle\cdot,\cdot\rangle}
\newcommand{\im}{\mathrm{Im}}
\newcommand{\multival}{\ensuremath{X\to 2^{X^*}}}
\newcommand{\SX}{\ensuremath{2^{X^*}}}



% THEOREM AND ENVIRONMENTS.  ===================================================

%\newenvironment{deflist}[1][\quad]%
%{\begin{list}{}{\renewcommand{\makelabel}[1]{\textrm{##1~}\hfil}%
%\settowidth{\labelwidth}{\textrm{#1~}}%
%\setlength{\leftmargin}{\labelwidth+\labelsep}}}%requires macro calc.sty
%{\end{list}}
%\newtheorem{theorem}{Theorem}%[section]
\newtheorem{theorem}{Theorem}[section]
\newtheorem{lemma}[theorem]{Lemma}
\newtheorem{fact}[theorem]{Fact}
\newtheorem{corollary}[theorem]{Corollary}
\newtheorem{proposition}[theorem]{Proposition}
\newtheorem{definition}[theorem]{Definition}
\newtheorem{conjecture}[theorem]{Conjecture}
\newtheorem{observation}[theorem]{Observation}
\newtheorem{openprob}[theorem]{Open Problem}
\theoremstyle{plain}{\theorembodyfont{\rmfamily}
\newtheorem{assumption}[theorem]{Assumption}}
\theoremstyle{plain}{\theorembodyfont{\rmfamily}
\newtheorem{condition}[theorem]{Condition}}
\theoremstyle{plain}{\theorembodyfont{\rmfamily}
\newtheorem{algorithm}[theorem]{Algorithm}}
\theoremstyle{plain}{\theorembodyfont{\rmfamily}
\newtheorem{example}[theorem]{Example}}
\theoremstyle{plain}{\theorembodyfont{\rmfamily}
\newtheorem{remark}[theorem]{Remark}}
\theoremstyle{plain}{\theorembodyfont{\rmfamily}
\newtheorem{application}[theorem]{Application}}
\def\proof{\noindent{\it Proof}. \ignorespaces}
%\def\endproof{\vbox{\hrule height0.6pt\hbox{\vrule height1.3ex%
%width0.6pt\hskip0.8ex\vrule width0.6pt}\hrule height0.6pt}}
%\numberwithin{equation}{section}
\def\endproof{\ensuremath{\quad \hfill \blacksquare}}

\renewcommand\theenumi{(\roman{enumi})}
\renewcommand\theenumii{(\alph{enumii})}
\renewcommand{\labelenumi}{\rm (\roman{enumi})}
\renewcommand{\labelenumii}{\rm (\alph{enumii})}

\newcommand{\boxedeqn}[1]{%
    \[\fbox{%
        \addtolength{\linewidth}{-2\fboxsep}%
        \addtolength{\linewidth}{-2\fboxrule}%
        \begin{minipage}{\linewidth}%
        \begin{equation}#1\end{equation}%
        \end{minipage}%
      }\]%
  }


\newcommand{\hilight}[1]{\colorbox{yellow}{#1}}
\usepackage{ifthen}\newboolean{draftmode}\setboolean{draftmode}{true}


%\usepackage{showkeys}
%\usepackage{drftcite}
\usepackage{exscale,relsize}
\usepackage{amsmath}
\usepackage{amsfonts}
\usepackage[colorlinks=true, linkcolor=blue]{hyperref}
\usepackage{amssymb}
\usepackage{calc}
\usepackage{theorem}
\usepackage{pifont}      % needed by dingautolist
\usepackage{array}
\usepackage{color}
\usepackage{enumerate}
\usepackage{bbm}
\usepackage{graphicx}
\usepackage{subcaption}
\usepackage{caption}

% \usepackage{amsthm}


% Hongda's packages
\usepackage{algpseudocode, algorithm}
\usepackage{mathtools}


% IF use the below packge, use `\printbibliography' to print out the bibliograph 
% For this one 
% 
% \usepackage[
%     backend=biber,
%     style=numeric,
%     sorting=nyt
% ]{biblatex}
% \addbibresource{references/PPM.bib}
% \addbibresource{references/NesterovMomentum.bib}
% \addbibresource{references/Books.bib}
% \addbibresource{references/BregmanDiv.bib}

% FORMATTING ===================================================================
\oddsidemargin -0.1cm
\textwidth  16.5cm
\topmargin  0.0cm
\headheight 0.0cm
\textheight 21.0cm
\parindent  4mm
\parskip    10pt
%\parskip    8pt
\tolerance  3000

% DRAFT FORMATTING =============================================================
% These are for todo notes, advise taken from 
% https://tex.stackexchange.com/questions/81666/extend-page-width-or-margin-for-todonotes-comments-or-other-package-comments
% \oddsidemargin=\dimexpr\oddsidemargin + 3cm\relax % DON'T USE

\ifthenelse{\boolean{draftmode}}{
    \evensidemargin=\dimexpr\evensidemargin  + 6cm\relax 
    \oddsidemargin=\dimexpr\oddsidemargin + 6cm\relax
    \paperwidth=\dimexpr \paperwidth + 12cm\relax 
    \marginparwidth=\dimexpr \marginparwidth  + 6cm\relax
    \paperheight=\dimexpr \paperheight + 6cm\relax
}{
    
}


% THEOREM AND ENVIRONMENTS.  ===================================================

%\newenvironment{deflist}[1][\quad]%
%{\begin{list}{}{\renewcommand{\makelabel}[1]{\textrm{##1~}\hfil}%
%\settowidth{\labelwidth}{\textrm{#1~}}%
%\setlength{\leftmargin}{\labelwidth+\labelsep}}}%requires macro calc.sty
%{\end{list}}
%\newtheorem{theorem}{Theorem}%[section]
\newtheorem{theorem}{Theorem}[section]
\newtheorem{lemma}[theorem]{Lemma}
\newtheorem{fact}[theorem]{Fact}
\newtheorem{corollary}[theorem]{Corollary}
\newtheorem{proposition}[theorem]{Proposition}
\newtheorem{definition}[theorem]{Definition}
\newtheorem{conjecture}[theorem]{Conjecture}
\newtheorem{observation}[theorem]{Observation}
\newtheorem{openprob}[theorem]{Open Problem}
\theoremstyle{plain}{\theorembodyfont{\rmfamily}
\newtheorem{assumption}[theorem]{Assumption}}
\theoremstyle{plain}{\theorembodyfont{\rmfamily}
\newtheorem{condition}[theorem]{Condition}}
\theoremstyle{plain}{\theorembodyfont{\rmfamily}}

% Removed due conflict with the algorithm environment. 
% \newtheorem{algorithm}[theorem]{Algorithm}}

\theoremstyle{plain}{\theorembodyfont{\rmfamily}
\newtheorem{example}[theorem]{Example}}
\theoremstyle{plain}{\theorembodyfont{\rmfamily}
\newtheorem{remark}[theorem]{Remark}}
\theoremstyle{plain}{\theorembodyfont{\rmfamily}
\newtheorem{application}[theorem]{Application}}

\def\proof{\noindent{\it Proof}. \ignorespaces}
%\def\endproof{\vbox{\hrule height0.6pt\hbox{\vrule height1.3ex%
%width0.6pt\hskip0.8ex\vrule width0.6pt}\hrule height0.6pt}}
%\numberwithin{equation}{section}
\def\endproof{\ensuremath{\quad \hfill \blacksquare}}

\renewcommand\theenumi{(\roman{enumi})}
\renewcommand\theenumii{(\alph{enumii})}
\renewcommand{\labelenumi}{\rm (\roman{enumi})}
\renewcommand{\labelenumii}{\rm (\alph{enumii})}

\numberwithin{equation}{section}



% These are Heniz's notations. 
\newcommand{\To}{\ensuremath{\rightrightarrows}}
\newcommand{\GX}{\ensuremath{\Gamma}}
\newcommand{\mal}{\ensuremath{\mathfrak{m}}}
\newcommand{\mumu}{\ensuremath{{\mu\mu}}}
\newcommand{\paver}{\ensuremath{\mathcal{P}}}
\newcommand{\ZZZ}{\ensuremath{{X \times X^*}}}
\newcommand{\RRR}{\ensuremath{{\RR \times \RR}}}
\newcommand{\todo}{\hookrightarrow\textsf{TO DO:}}

\newcommand{\emp}{\ensuremath{\varnothing}}
%\newcommand{\la}{\ensuremath{\langle}}
%\newcommand{\ra}{\ensuremath{\rangle}}
\newcommand{\infconv}{\ensuremath{\mbox{\small$\,\square\,$}}}
\newcommand{\pscal}{\ensuremath{\scal{\cdot}{\cdot}}}
\newcommand{\Tt}{\ensuremath{\mathfrak{T}}}
\newcommand{\YY}{\ensuremath{\mathcal Y}}
\newcommand{\XX}{\ensuremath{\mathcal X}}
\newcommand{\HH}{\ensuremath{\mathcal H}}
\newcommand{\XP}{\ensuremath{\mathcal X}^*}
\newcommand{\st}{\ensuremath{\;|\;}}
\newcommand{\zeroun}{\ensuremath{\left]0,1\right[}}

\newcommand{\lev}[1]{\ensuremath{\mathrm{lev}_{\leq #1}\:}}
\newcommand{\moyo}[2]{\ensuremath{\sideset{_{#2}}{}{\operatorname{}}\!#1}}
\newcommand{\pair}[2]{\left\langle{{#1},{#2}}\right\rangle}
%\newcommand{\scal}[2]{\left.\left\langle{#1}\:\right| {#2}  \right\rangle}
\newcommand{\scal}[2]{\langle{{#1},{#2}}\rangle}
\newcommand{\Scal}[2]{\left\langle{{#1},{#2}}\right\rangle}
%\newcommand{\scal}[2]{\braket{ {#1},{#2}}}

\newcommand{\yosida}{\ensuremath{ \; {}^}}
\newcommand{\exi}{\ensuremath{\exists\,}}
\newcommand{\GG}{\ensuremath{\mathcal G}}
\newcommand{\RR}{\ensuremath{\mathbb R}}
\newcommand{\SSS}{\ensuremath{\mathbb S}}
\newcommand{\CC}{\ensuremath{\mathbb C}}
\newcommand{\Real}{\ensuremath{\mathrm{Re}\,}}
\newcommand{\ii}{\ensuremath{\mathrm i}}
\newcommand{\RP}{\ensuremath{\left[0,+\infty\right[}}
\newcommand{\RPX}{\ensuremath{\left[0,+\infty\right]}}
\newcommand{\RPP}{\ensuremath{\,\left]0,+\infty\right[}}
\newcommand{\RX}{\ensuremath{\,\left]-\infty,+\infty\right]}}
\newcommand{\RXX}{\ensuremath{\,\left[-\infty,+\infty\right]}}
\newcommand{\KK}{\ensuremath{\mathbb K}}
\newcommand{\NN}{\ensuremath{\mathbb N}}
\newcommand{\nnn}{\ensuremath{{n \in \NN}}}
\newcommand{\thalb}{\ensuremath{\tfrac{1}{2}}}
\newcommand{\zo}{\ensuremath{{\left]0,1\right]}}}
\newcommand{\lzo}{\ensuremath{{\lambda \in \left]0,1\right]}}}
%\newcommand{\toppsepp}{\setlength{\partopsep}{-5pt}}
\newcommand{\menge}[2]{\big\{{#1} \mid {#2}\big\}}
\newcommand{\pfrac}[2]{\ensuremath{\mathlarger{\tfrac{#1}{#2}}}}


% MATH OPERATORS ===============================================================
% \newcommand{\monos}{\ensuremath{\mathcal M}}
\newcommand{\DD}{\operatorname{dom}f}
\newcommand{\IDD}{\ensuremath{\operatorname{int}\operatorname{dom}f}}
\newcommand{\CDD}{\ensuremath{\overline{\operatorname{dom}}\,f}}
\newcommand{\clspan}{\ensuremath{\overline{\operatorname{span}}}}
\newcommand{\cone}{\ensuremath{\operatorname{cone}}}
\newcommand{\dom}{\ensuremath{\operatorname{dom}}}
\newcommand{\closu}{\ensuremath{\operatorname{cl}}}
\newcommand{\cont}{\ensuremath{\operatorname{cont}}}
\newcommand{\mons}{\ensuremath{\mathcal{A}}}
\newcommand{\gra}{\ensuremath{\operatorname{gra}}}
\newcommand{\epi}{\ensuremath{\operatorname{epi}}}
\newcommand{\prox}{\ensuremath{\operatorname{Prox}_{\mu}}}
\newcommand{\hprox}{\ensuremath{\operatorname{prox}}}
\newcommand{\intdom}{\ensuremath{\operatorname{int}\operatorname{dom}}\,}
\newcommand{\inte}{\ensuremath{\operatorname{int}}}
\newcommand{\sri}{\ensuremath{\operatorname{sri}}}
\newcommand{\reli}{\ensuremath{\operatorname{ri}}}
\newcommand{\cart}{\ensuremath{\mbox{\LARGE{$\times$}}}}


\newcommand{\average}{\ensuremath{\mathcal{R}_{\mu}({\bf A},{\boldsymbol \lambda})}}
\newcommand{\averagebar}{\ensuremath{\mathcal{R}_{1}({\bf A},\bar{\lambda})}}
\newcommand{\averageonelambda}{\ensuremath{\mathcal{R}({\bf A},{\boldsymbol \lambda})}}
\newcommand{\averageonehalf}{\ensuremath{\mathcal{R}_{1}(A,1/2)}}
\newcommand{\averageinverse}{\ensuremath{\mathcal{R}_{\mu^{-1}}({\bf A}^{-1},{\boldsymbol \lambda})}}
\newcommand{\averageoneinverse}{\ensuremath{\mathcal{R}({\bf A}^{-1},{\boldsymbol \lambda})}}
\newcommand{\averagef}{\ensuremath{\mathcal{P}_{\mu}(f,\lambda)}}
\newcommand{\averagefone}{\ensuremath{\mathcal{P}_{1}(f,\lambda)}}
\newcommand{\averagefd}{\ensuremath{\mathcal{P}_{\mu}((f_{1},\ldots, f_{n}),(\lambda_{1},\ldots, \lambda_{n}))}}
\newcommand{\averagefik}{\ensuremath{\mathcal{P}_{\mu_{k}}((f_{1,k},\ldots,f_{n,k}),
(\lambda_{1,k},\ldots,\lambda_{n,k}))}}
\newcommand{\averagesub}{\ensuremath{\mathcal{R}_{\mu}(\partial f,\lambda)}}
\newcommand{\res}{\ensuremath{\mathcal{R}_{\mu}}}
\newcommand{\resmuk}{\ensuremath{\mathcal{R}_{\mu_{k}}}}
\newcommand{\newres}{\ensuremath{\mathcal{R}}}
\newcommand{\resmualpha}{\ensuremath{\mathcal{R}_{\alpha\mu}}}
\newcommand{\averageone}{\ensuremath{\mathcal{R}_{1}}}
\newcommand{\harm}{\ensuremath{\mathcal{H}(A,\lambda)}}
\newcommand{\arithmetic}{\ensuremath{\mathcal{A}(A,\lambda)}}

\newcommand{\WC}{\ensuremath{{\mathfrak W}}}
\newcommand{\SC}{\ensuremath{{\mathfrak S}}}
\newcommand{\card}{\ensuremath{\operatorname{card}}}
\newcommand{\bd}{\ensuremath{\operatorname{bdry}}}
\newcommand{\ran}{\ensuremath{\operatorname{ran}}}
\newcommand{\rec}{\ensuremath{\operatorname{rec}}}
\newcommand{\rank}{\ensuremath{\operatorname{rank}}}
\newcommand{\kernel}{\ensuremath{\operatorname{ker}}}
\newcommand{\conv}{\ensuremath{\operatorname{conv}}}
\newcommand{\segh}{\ensuremath{\operatorname{seg}}}
\newcommand{\boxx}{\ensuremath{\operatorname{box}}}
\newcommand{\clconv}{\ensuremath{\overline{\operatorname{conv}}\,}}
\newcommand{\cldom}{\ensuremath{\overline{\operatorname{dom}}\,}}
\newcommand{\clran}{\ensuremath{\overline{\operatorname{ran}}\,}}
\newcommand{\Nf}{\ensuremath{\nabla f}}
\newcommand{\NNf}{\ensuremath{\nabla^2f}}
\newcommand{\Fix}{\ensuremath{\operatorname{Fix}}}
\newcommand{\FFix}{\ensuremath{\overline{\operatorname{Fix}}\,}}
\newcommand{\aFix}{\ensuremath{\widetilde{\operatorname{Fix}\,}}}
\newcommand{\Id}{\ensuremath{\operatorname{Id}}}
\newcommand{\Max}{\ensuremath{\operatorname{max}}}
\newcommand{\Bb}{\ensuremath{\mathfrak{B}}}
\newcommand{\BB}{\ensuremath{\mathbb{B}}}
\newcommand{\Fb}{\ensuremath{\overrightarrow{\mathfrak{B}}}}
\newcommand{\Fprox}{\ensuremath{\overrightarrow{\operatorname{prox}}}}
\newcommand{\Bprox}{\ensuremath{\overleftarrow{\operatorname{prox}}}}
\newcommand{\Bproj}{\ensuremath{\overleftarrow{\operatorname{P}}}}
\newcommand{\Ri}{\ensuremath{\mathfrak{R}_i}}
\newcommand{\Dn}{\ensuremath{\,\overset{D}{\rightarrow}\,}}
\newcommand{\nDn}{\ensuremath{\,\overset{D}{\not\rightarrow}\,}}
\newcommand{\weakly}{\ensuremath{\,\rightharpoonup}\,}
\newcommand{\weaklys}{\ensuremath{\,\overset{*}{\rightharpoonup}}\,}
\newcommand{\gr}{\ensuremath{\operatorname{gra}}}
\newcommand{\g}{\ensuremath{\,\overset{g}{\rightarrow}}\,}
\newcommand{\p}{\ensuremath{\,\overset{p}{\rightarrow}}\,}
\newcommand{\e}{\ensuremath{\,\overset{e}{\rightarrow}}\,}
\newcommand{\Tbar}{\ensuremath{\overline{T}}}
\newcommand{\n}{\ensuremath{\,\overset{n}{\rightarrow}}\,}

\newcommand{\minf}{\ensuremath{-\infty}}
\newcommand{\pinf}{\ensuremath{+\infty}}
\renewcommand{\iff}{\ensuremath{\Leftrightarrow}}
% \renewcommand{\phi}{\ensuremath{\varphi}}
%\newcommand{\Real}{\ensuremath{\mathrm{Re}\,}}
\newcommand{\negent}{\ensuremath{\operatorname{negent}}}
\newcommand{\neglog}{\ensuremath{\operatorname{neglog}}}
\newcommand{\halb}{\ensuremath{\tfrac{1}{2}}}
\newcommand{\bT}{\ensuremath{\mathbf{T}}}
\newcommand{\bX}{\ensuremath{\mathbf{X}}}
\newcommand{\bL}{\ensuremath{\mathbf{L}}}
\newcommand{\bD}{\ensuremath{\boldsymbol{\Delta}}}
\newcommand{\bc}{\ensuremath{\mathbf{c}}}
\newcommand{\by}{\ensuremath{\mathbf{y}}}
\newcommand{\bx}{\ensuremath{\mathbf{x}}}
\newcommand{\bA}{{\bf A}}
\newcommand{\Other}{Indeterminate }
\newcommand{\other}{indeterminate }


%%% Raf's stuff  ===============================================================
\newcommand{\al}{\alpha}
\newcommand{\la}{\lambda}
\newcommand{\La}{\Lambda}
\newcommand{\pluss}{{\hskip1pt \raise1pt\vbox{\hrule width6pt \vskip1pt
\hrule width6pt}\kern-4pt{\lower1pt\hbox{\vrule height6pt \kern1pt\vrule
height6pt}}\hskip5pt}}
\newcommand{\timess}{\star}
\newcommand{\argmax}{\mathop{\rm argmax}\limits}
\newcommand{\argmin}{\mathop{\rm argmin}\limits}
\newcommand{\product}{\langle\cdot,\cdot\rangle}
\newcommand{\im}{\mathrm{Im}}
\newcommand{\multival}{\ensuremath{X\to 2^{X^*}}}
\newcommand{\SX}{\ensuremath{2^{X^*}}}

\newcommand{\inlinecode}[1]{\texttt{\footnotesize #1}}
% 
\usepackage{listings} \lstset{basicstyle=\footnotesize\ttfamily,breaklines=true}
\usepackage{xcolor}
\lstdefinelanguage{Julia}%
  {morekeywords={abstract,break,case,catch,const,continue,do, else, elseif,%
      end, export, false, for, function, immutable, import, importall, if, in,%
      macro, module, otherwise, quote, return, switch, true, try, type, typealias,%
      using, while},%
   sensitive=true,%
   alsoother={$},%
   morecomment=[l]\#,%
   morecomment=[n]{\#=}{=\#},%
   morestring=[s]{"}{"},%
   morestring=[m]{'}{'},%
}[keywords,comments,strings]%
\lstset{%
    language         = Julia,
    basicstyle       = \ttfamily,
    keywordstyle     = \bfseries\color{blue},
    stringstyle      = \color{magenta},
    commentstyle     = \color{ForestGreen},
    showstringspaces = false,
}

\begin{document}

\newcommand{\dist}{\ensuremath{\operatorname{dist}}}

\title{{\fontfamily{ptm}\selectfont Extending the Convergence Results of Nesterov's Acceleration using the Proximal Gradient Gap}}

\author{
    Author 1 Name, Author 2 Name
    \thanks{
        Subject type, Some Department of Some University, Location of the University,
        Country. E-mail: \texttt{author.name@university.edu}.
    }
}

\date{March 2, 2020}

\maketitle
\todoinline{This paper is currently in draft mode. Check source to change options. \cite{bauschke_convex_2017}}
% \vskip 8mm

\begin{abstract} 
    This paper gives definitive answers to open problems in Necoara et al. \cite{necoara_linear_2019}. 
    Linear convergence of Nesterov's accelerated gradient method is possible in a boarder context of functions. 
\end{abstract}

\noindent{\bfseries 2010 Mathematics Subject Classification:}
Primary 47H05, 52A41, 90C25; Secondary 15A09, 26A51, 26B25, 26E60, 47H09, 47A63.
\noindent{\bfseries Keywords:}


\section{Introduction}
    \textbf{Notations.}
    Unless specified, our ambient space is $\RR^n$ with Euclidean norm $\Vert \cdot\Vert$.
    Let $C\subseteq \RR^n$, $\Pi_C(\cdot)$ denotes the projection onto the set $C$, i.e: the closest point in $C$ to another point in $\RR^n$. 
    We denote $\delta_C$ to be the indicator function for the set $C$. 
    For a function of $F = f + g$, and a $B\ge 0$ where $f$ is $\mathcal C^1$ differentiable, and $g$ is l.s.c, we consider the proximal gradient operator: 
    \begin{align*}
        T_B(x) &= \argmin_{z} \left\lbrace
            g(z) + f(x) + \langle \nabla f(x), z - x\rangle + \frac{B}{2}\Vert x - z\Vert^2
        \right\rbrace
        \\
        &= \hprox_{B^{-1}g}(x - B^{-1}\nabla f(x)). 
    \end{align*}
    We also define the gradient mapping operator $\mathcal G_B(x) = B^{-1}(x - T_B(x))$. 
    \par
    \textbf{Major results}. 
    

\section{Precursors materials for our proofs of linear convergence}
    The following two definitions defines the accelerated proximal gradient algorithm. 
    % DEFINITION =======================================================================================================
    \begin{definition}[similar triangle form of accelerated proximal gradient]\;\label{def:st-apg}\\
        The definition is about $((\alpha_k)_{k \ge 0}, (q_k)_{k \ge 0}, (B_k)_{k \ge 0}, (y_k)_{k \ge 0}, (x_k)_{k \ge -1}, (v_k)_{k \ge -1})$. 
        These sequences satisfy:
        \begin{enumerate}[nosep]
            \item $x_{-1}, y_{- 1}\in \RR^n$ are arbitrary initial condition of the algorithm;
            \item $(q_k)_{k \ge 1}$ be a sequence such that $q_k \in [0, 1)$ for all $k \ge 1$;
            \item $(\alpha_k)_{k \ge 1}$ be a sequence such that $\alpha_0 \in (0, 1]$, and for all $k \ge 1$ it has $\alpha_k \in (q_k, 1)$;
            \item $(B_k)_{k \ge 0}$ has $B_k \ge 0$, it's a nonnegative sequence. 
        \end{enumerate}
        Then an algorithm satisfies the similar triangle form of Nesterov's accelerated gradient if it generates iterates $(y_k, x_k, v_k)_{k \ge 1}$ such that for all $k\ge 0$: 
        \begin{align*}
            y_k &= \left(\frac{\alpha_k - q_k}{1 - q_k}\right)v_{k - 1} 
            + \left(\frac{1 - \alpha_k}{1 - q_k}\right) x_{k - 1},
            \\
            x_k &= T_{B_k}(y_k), D_f(x_k, y_k) \le \frac{B_k}{2}\Vert x_k - y_k\Vert^2, 
            \\
            v_k &= x_{k - 1} + \alpha_k^{-1}(x_k - x_{k - 1}). 
        \end{align*}
    \end{definition}
    \begin{remark}
        If we choose $a_k = 1$ then the definition is equivalent to proximal gradient algorithm, with $v_k = x_k$ for all $k \ge 0$. 
        
    \end{remark}
    % DEFINITION =======================================================================================================
    \begin{definition}[relaxed momentum sequence]\label{def:rlx-momentum-seq}
        The following definition is about sequences $((\alpha_k)_{k \ge 0}, (q_k)_{k \ge 0}$. 
        They satisfy:
        \begin{enumerate}[nosep]
            \item $(q_k)_{k \ge 0}$ is a sequence such that $q_k \in [0, 1)$ for all $k \ge 0$;
            \item $(\alpha_k)_{k \ge 0}$ has $\alpha_k \in (0, 1]$, and for all $k \ge 0$. 
        \end{enumerate}
    \end{definition}
    % DEFINITION =======================================================================================================
    \begin{definition}[proximal gradient gap]\label{def:pg-gap}
        Let $F = f + g$ where $f$ is $L$ Lipschitz smooth and $g$ is convex. 
        Then the proximal gradient mapping $T_B(x) = \hprox_{B^{-1}g}(x -  B^{-1}\nabla f(x))$ is a singleton, and $\dom T_B = \RR^n$. 
        Let $\mu, B$ be parameters such that $B > \mu \ge 0$. 
        We define the proximal gradient gap $\mathcal E(z, y, \mu, B)$ is the mapping: 
        \begin{align*}
            \mathcal E(z, y, \mu, B) &:= 
            F(z) - F(T_B(y)) 
            - \left\langle B(y - T_B(y)), z - y\right\rangle
            - \frac{\mu}{2}\Vert z - y\Vert^2
            - \frac{B}{2}\Vert y - T_B(y)\Vert^2. 
        \end{align*}
    \end{definition}
    \begin{remark}
        This expression is the same as the proximal gradient inequality up to a negative sign, after moving everything to one side. 
    \end{remark}
    For the sum of smooth and nonsmooth objective, a lot can be said about the proximal gradient gap of the function. 
    \begin{lemma}[proximal gradient gap under convexity]\label{lemma:pg-under-cnvx}

    \end{lemma}
    
\section{Deriving the proximal gap inequality}
    To derive the convergence rate of algorithm satisfying Definition \ref{def:st-apg}, \ref{def:rlx-momentum-seq}, we leverage Definition \ref{def:pg-gap}. 
    The first two subsections prepare for the results and the convergence results are derived by the end. 
    \par
    The following assumption is about the Proximal Gradient gap. 
    % ASSUMPTION =======================================================================================================
    \begin{assumption}[generic assumptions for proximal gradient gap inequality]\;\label{ass:pggap-dscnt}\\
        The following assumption is about $(F, f, g, \mathcal E, \mu, L)$, it is the configuration needed to derive the convergence rate of algorithms that satisfy Definition \ref{def:st-apg}. 
        We assume that there exists $\mu \ge 0$ such that the followings are true: 
        \begin{enumerate}[nosep]
            \item\label{ass:pggap-dscnt-itm1} Let $F = f + g$ where $f$ is $L$ Lipschitz smooth and, $g$ is closed convex and proper.
            \item\label{ass:pggap-dscnt-itm2} $\forall y \in \RR^n, \exists B \ge 0\; \exists \bar y$ such that $\mathcal E(\bar y, y, \mu, B) \ge 0$. 
            % \item For all $z, y \in \RR^n$, there exists $B > \mu$ such that $\mathcal E(z, y, \mu, B) + \frac{\mu}{2}\Vert z - y\Vert^2 \ge 0$.   
        \end{enumerate}
    \end{assumption}
    \begin{remark}
        Note that:
        \begin{enumerate}
            \item If the function $f$ is convex and smooth, $g$ is convex then, all conditions are satisfied for $\mu = 0$, and for all $\bar y \in \RR^n$. 
            \item If $\mu \ge 0$ satisfies item (ii), (iii), then it also satisfies for all $\tilde \mu$ such that $0 \le \tilde \mu \le \mu$. For the best convergence rate, the largest such $\mu$ is of our interest. 
        \end{enumerate}
    \end{remark}
    Using the above assumption, in this section, we can build up to Lemma \ref{lemma:cnvg-prep-part3}. 
    The lemma is crucially important and, it gives us the inequality on what happen between iterates $x_{k}, v_{k}$ and $v_{k - 1}, x_{k - 1}$ generated by algorithm satisfying Definition \ref{def:st-apg}. 
    % LEMMA ========================================================================================================
    \begin{lemma}[equivalent representations of the iterates part I]\label{lemma:st-iterates-alt-form-part1}
        Suppose that the sequences of $\alpha_k, q_k, y_k, v_k, x_k, B_k$ satisfy the similar triangle form, then for all $k \ge 0$ the iterates $v_k$ admits the following equivalent representations: 
        \begin{align*}
            v_k
            &= v_{k - 1} + \alpha_k^{-1}q_k(y_k - v_{k - 1}) - \alpha_{k}^{-1}B_k^{-1}\mathcal G_{B_k}(y_k). 
        \end{align*}
    \end{lemma}
    \begin{proof}
        Consider all $k \ge 1$. 
        The relations are direct, immediately from the update rule in Definition \ref{def:st-apg} of $y_k$ we have
        \begin{enumerate}[nosep]
            \item[(a)] $(\alpha_k - 1)x_{k - 1} = (\alpha_k - q_k)v_{k - 1} - (1 - q_k)y_k$. 
            \item[(b)] $x_k = y_k - B_k^{-1}\mathcal G_{B_k}(y_k)$. 
        \end{enumerate}
        Using the above and the update rule for $v_k$ in Definition \ref{def:st-apg}. 
        \begin{align*}
            v_k &= x_{k - 1} + \alpha_k^{-1}(x_k - x_{k - 1})
            \\
            &= (1 - \alpha_k^{-1})x_{k - 1} + \alpha_k^{-1}x_k
            \\
            &= \alpha_k^{-1}(\alpha_k - 1)x_{k - 1} + \alpha_k^{-1}x_k
            \\
            &\underset{\text{(a)}}{=} \alpha_k^{-1}(\alpha_k - q_k)v_{k - 1} - \alpha_{k}^{-1}(1 - q_k)y_k 
            + \alpha_k^{-1}x_k
            \\
            &\underset{\text{(b)}}{=} (1 - \alpha_k^{-1}q_k) v_{k - 1} - (\alpha_k^{-1} - \alpha_k^{-1}q_k)y_k
            + \alpha_k^{-1}(y_k - B_k^{-1}\mathcal G_{B_k}(y_k)). 
            \\
            &= 
            (1 - \alpha_k^{-1}q_k) v_{k - 1} + \alpha_k^{-1}q_ky_k
            - \alpha_k^{-1}B_k^{-1}\mathcal G_{B_k}(y_k)
            \\
            &= v_{k - 1} + \alpha_k^{-1}q_k(y_k - v_{k - 1}) - \alpha_{k}^{-1}B_k^{-1}\mathcal G_{B_k}(y_k). 
        \end{align*}
    \end{proof}
    \begin{lemma}[equivalent representations of the iterates part II]\;\label{lemma:st-iterates-alt-form-part2}\\
        Suppose the sequences of $\alpha_k, q_k, y_k, v_k, x_k, B_k$ satisfy the similar triangle form, then for all $k \ge 0$ the iterates $v_k$ admits the following equivalent representations: 
        \begin{align*}
            y_k &= 
            x_{k -1} + (1 - q_k)^{-1}(\alpha_{k - 1}^{-1} - 1)(\alpha_k - q_k)(x_{k - 1} - x_{k - 2}). 
        \end{align*}
    \end{lemma}
    \begin{proof}
        For all $k\ge 1$, from the update rules in Definition \ref{def:st-apg}: 
        \begin{align*}
            (1 - q_k)y_k &= 
            (\alpha_k - q_k) v_{k - 1} 
            + (1 - \alpha_k)x_{k - 1}
            \\
            &= 
            (\alpha_k - q_k)\left(
                x_{k - 2} + \alpha_{k - 1}^{-1}(x_{k - 1} - x_{k - 2})
            \right) 
            + (1 - \alpha_k)x_{k - 1}
            \\
            &= 
            (\alpha_k - q_k)((1 - \alpha_{k - 1}^{-1})x_{k -2} + \alpha_{k - 1}^{-1}x_{k - 1}) 
            + (1 - \alpha_k)x_{k - 1}
            \\
            &= (\alpha_k - q_k)(1 - \alpha_{k - 1}^{-1})x_{k - 2}
            + \left(
                \frac{\alpha_k - q_k}{\alpha_{k - 1}} + 1 - \alpha_k
            \right)x_{k - 1}. 
        \end{align*}
        Divide by $(1 - q_k)$ on both sides yield: 
        \begin{align*}
            y_k &= 
            \frac{(\alpha_k - q_k)(1 - \alpha_{k - 1}^{-1})}{1 - q_k}x_{k - 2} 
            + \left(
                \frac{\alpha_k - q_k}{\alpha_{k - 1}(1 - q_k)} + \frac{1 - \alpha_k}{1 - q_k}
            \right)x_{k - 1}
            \\
            &= 
            \frac{(\alpha_k - q_k)(1 - \alpha_{k - 1}^{-1})}{1 - q_k}x_{k - 2} 
            + \left(
                \frac{
                    (\alpha_{k - 1}^{-1} -1)(\alpha_k - q_k)
                    + \alpha_k - q_k + 1 - \alpha_k
                }{1 - q_k}
            \right)x_{k - 1}
            \\
            &= \frac{(\alpha_k - q_k)(1 - \alpha_{k - 1}^{-1})}{1 - q_k}x_{k - 2} 
            + \left(
                \frac{
                    (\alpha_{k - 1}^{-1} -1)(\alpha_k - q_k)
                }{1 - q_k} + 1
            \right)x_{k - 1}
            \\
            &= x_{k -1} + (1 - q_k)^{-1}\left(\alpha_{k - 1}^{-1} - 1\right)(\alpha_k - q_k)(x_{k - 1} - x_{k - 2}). 
        \end{align*}
    \end{proof}
    
    \subsection{Preparations for the convergence rate proof}
        The following lemma summarize important results that give a swift exposition for the proofs show up at the end for the convergence rate. 
        % LEMMA ========================================================================================================
        \begin{lemma}[convergence preparations part I]\label{lemma:cnvg-prep-part1}
            Let $(F, f, g,  \mathcal E, \mu, L)$ satisfies Assumption \ref{ass:pggap-dscnt}. 
            Suppose that 
            \begin{enumerate}[nosep]
                \item The iterates $(y_k, v_k, x_k)_{k \ge 0}$ satisfies Definition \ref{def:st-apg} where $T_B$ defined using $F = f + g$. 
                \item The sequences $(\alpha_k)_{k\ge 0}, (q_k)_{k \ge 0}$ satisfy Definition \ref{def:rlx-momentum-seq}. 
                \item We choose the parameters $q_k$ has $q_k = \mu/B_k$, with $B_k > \mu$, for all $k \ge 0$. 
            \end{enumerate}
            Then, for all $\bar x \in \RR^n$, $k \ge 0$: 
            \begin{align*}
                & \frac{B_k\alpha_k^2}{2}\Vert \bar x - v_k\Vert^2 
                - \frac{B_k\alpha_k(\alpha_k - q_k)}{2}\Vert \bar x - v_{k - 1}\Vert^2
                \\
                &=
                \frac{\alpha_k \mu}{2}
                \Vert \bar x - v_{k - 1}\Vert^2
                + \frac{q_k\mu}{2}\Vert y_k - v_{k - 1} \Vert
                + \frac{q_k}{2\mu}\Vert \mathcal G_{B_k}(y_k)\Vert^2
                \\ &\quad 
                - \langle
                    q_k(y_k - v_{k - 1}) + \alpha_k(v_{k - 1} - \bar x), \mathcal G_{B_k}(y_k)
                \rangle
                    + \alpha_k \mu
                    \left\langle 
                        v_{k - 1} - \bar x, 
                        y_k - v_{k - 1}
                    \right\rangle. 
            \end{align*}
        \end{lemma}
        \begin{proof}
            Consider any $\bar x \in \RR^n$. 
            \begin{align*}
                & \frac{B_k\alpha_k^2}{2}\Vert \bar x - v_k\Vert^2 
                \\
                &\underset{(1)}{=} \frac{B_k\alpha_k^2}{2}\left\Vert
                    \bar x - 
                    \left(
                        v_{k - 1} 
                        + \alpha_k^{-1}q_k(y_k - v_{k - 1})
                        - \alpha_k^{-1} B_k^{-1} \mathcal G_{B_k}(y_k)
                    \right)
                \right\Vert^2
                \\
                &= \frac{B_k\alpha_k^2}{2}\left\Vert
                    (\bar x - v_{k - 1})
                    - \alpha_k^{-1}\left(
                        q_k(y_k - v_{k - 1})
                        -  B_k^{-1} \mathcal G_{B_k}(y_k)
                    \right)
                \right\Vert^2
                \\
                &= \frac{B_k\alpha_k^2}{2}
                \Vert \bar x - v_{k - 1}\Vert^2
                + \frac{B_k}{2}\left\Vert 
                    q_k(y_k - v_{k - 1})
                    -  B_k^{-1} \mathcal G_{B_k}(y_k)
                \right\Vert^2
                    \\ &\quad 
                    - \alpha_k B_k
                    \left\langle 
                        \bar x - v_{k - 1}, 
                        q_k(y_k - v_{k - 1}) - B_k^{-1}\mathcal G_{B_k}(y_k)
                    \right\rangle
                \\
                &= 
                \frac{B_k\alpha_k^2}{2}
                \Vert \bar x - v_{k - 1}\Vert^2
                + \frac{B_kq_k^2}{2}\Vert y_k - v_{k - 1} \Vert
                + \frac{1}{2B_k}\Vert \mathcal G_{B_k}(y_k)\Vert^2
                - q_k\langle 
                    y_k - v_{k - 1}, \mathcal G_{B_k}(y_k)
                \rangle
                    \\ &\quad 
                    + B_k\alpha_k
                    \left\langle 
                        v_{k - 1} - \bar x, 
                        q_k(y_k - v_{k - 1}) - B_k^{-1}\mathcal G_{B_k}(y_k)
                    \right\rangle
                \\
                &= 
                \frac{B_k\alpha_k^2}{2}
                \Vert \bar x - v_{k - 1}\Vert^2
                + \frac{B_kq_k^2}{2}\Vert y_k - v_{k - 1} \Vert
                + \frac{1}{2B_k}\Vert \mathcal G_{B_k}(y_k)\Vert^2
                - q_k\langle 
                    y_k - v_{k - 1}, \mathcal G_{B_k}(y_k)
                \rangle
                    \\ &\quad 
                    + 
                    B_k\alpha_k\left\langle 
                        v_{k - 1} - \bar x, 
                        q_k(y_k - v_{k - 1}) 
                    \right\rangle
                    - \alpha_k \left\langle 
                        v_{k - 1} - \bar x,
                        \mathcal G_{B_k}(y_k)
                    \right\rangle
                \\
                &= 
                \frac{\alpha_k^2B_k}{2}
                \Vert \bar x - v_{k - 1}\Vert^2
                + \frac{q_k^2B_k}{2}\Vert y_k - v_{k - 1} \Vert
                + \frac{1}{2B_k}\Vert \mathcal G_{B_k}(y_k)\Vert^2
                    \\ &\quad 
                    - \langle
                        q_k(y_k - v_{k - 1}) + \alpha_k(v_{k - 1} - \bar x), \mathcal G_{B_k}(y_k)
                    \rangle
                    \\ &\quad 
                    + \alpha_k q_k B_k
                    \left\langle 
                        v_{k - 1} - \bar x, 
                        y_k - v_{k - 1}
                    \right\rangle. 
            \end{align*}
            At (1) we used Lemma \ref{lemma:st-iterates-alt-form-part1}. 
            Subtracting $\frac{B_k\alpha_k(\alpha_k - q_k)}{2}\Vert \bar x - v_{k - 1}\Vert^2$ from both sides, the coefficient of $\Vert \bar x - v_{k - 1}\Vert^2$ comes out to be: 
            \begin{align*}
                & \frac{\alpha_k^2B_k}{2}
                - \frac{B_k\alpha_k(\alpha_k - q_k)}{2}
                = \frac{B_k}{2}(
                    \alpha_k^2 - \alpha_k(\alpha_k - q_k)
                )
                \underset{(1)}{=} \frac{B_k\alpha_kq_k}{2}. 
            \end{align*}
            At (1), we used the relation $(1 - \alpha_k)\rho_{k - 1}\alpha_{k - 1}^2 = \alpha_k(\alpha_k - q_k)$ as in Definition \ref{def:rlx-momentum-seq}, for all $k \ge 1$. 
            Therefore, we have the equality: 
            \begin{align*}
                &
                \frac{B_k\alpha_k^2}{2}\Vert \bar x - v_k\Vert^2 
                - \frac{B_k\alpha_k(\alpha_k - q_k)}{2}\Vert \bar x - v_{k - 1}\Vert^2
                \\
                &= 
                \frac{\alpha_k q_k B_k}{2}\Vert \bar x - v_{k - 1}\Vert^2
                + \frac{q_k^2B_k}{2}\Vert y_k - v_{k - 1} \Vert
                + \frac{1}{2B_k}\Vert \mathcal G_{B_k}(y_k)\Vert^2
                    \\ &\quad 
                    - \langle
                        q_k(y_k - v_{k - 1}) + \alpha_k(v_{k - 1} - \bar x), \mathcal G_{B_k}(y_k)
                    \rangle
                    \\ &\quad 
                    + \alpha_k q_k B_k
                    \left\langle 
                        v_{k - 1} - \bar x, 
                        y_k - v_{k - 1}
                    \right\rangle. 
                \\
                &\underset{(1)}{=} 
                \frac{\alpha_k \mu}{2}
                \Vert \bar x - v_{k - 1}\Vert^2
                + \frac{q_k\mu}{2}\Vert y_k - v_{k - 1} \Vert
                + \frac{q_k}{2\mu}\Vert \mathcal G_{B_k}(y_k)\Vert^2
                    \\ &\quad 
                - \langle
                        q_k(y_k - v_{k - 1}) + \alpha_k(v_{k - 1} - \bar x), \mathcal G_{B_k}(y_k)
                    \rangle
                    + \alpha_k \mu
                    \left\langle 
                        v_{k - 1} - \bar x, 
                        y_k - v_{k - 1}
                    \right\rangle. 
            \end{align*}
            At (1), we used the relation that $B_k = \mu/q_k$, for all $k \ge 0$. 
        \end{proof}
        % LEMMA ========================================================================================================
        \begin{lemma}[convergence preparations part II]\label{lemma:cnvg-prep-part2}
            The iterates $(y_k, x_k, v_k)_{k \ge 0}$ satisfies Definition \ref{def:st-apg} then, for all $k \ge 0, \bar x \in \RR^n$ the following identities: 
            \begin{align*}
                \alpha_k(v_{k - 1} - \bar x) 
                + q_k(y_k - v_{k - 1})
                + x_{k - 1} - y_k
                &= 
                \alpha_k(x_{k - 1} - \bar x). 
            \end{align*}
        \end{lemma}
        \begin{proof}
            We first establish two intermediate results. 
            From Definition \ref{def:st-apg}, it has for all $k \ge 0$: 
            \begin{align*}
                y_k &= 
                \left(\frac{\alpha_k - q_k}{1 - q_k}\right)v_{k - 1} 
                + \left(\frac{1 - \alpha_k}{1 - q_k}\right) x_{k - 1}
                \\
                &= 
                \left(
                    1 - \frac{1 - \alpha_k}{1 - q_k}
                \right)v_{k - 1} 
                + \left(\frac{1 - \alpha_k}{1 - q_k}\right) x_{k - 1}
                \\
                \iff 
                y_k - v_{k - 1}
                &= \left(
                    \frac{1 - \alpha_k}{1 - q_k}
                \right)(x_{k - 1} - v_{k - 1}). 
            \end{align*}
            Similarly: 
            \begin{align*}
                y_k &= 
                \left(\frac{\alpha_k - q_k}{1 - q_k}\right)v_{k - 1} 
                + \left(\frac{1 - \alpha_k}{1 - q_k}\right) x_{k - 1}
                \\
                &= 
                \left(\frac{\alpha_k - q_k}{1 - q_k}\right)v_{k - 1} 
                + \left(1 - \frac{\alpha_k - q_k}{1 - q_k}\right) x_{k - 1}
                \\
                \iff 
                y_k - x_{k - 1} &= 
                \left(
                    \frac{\alpha_k - q_k}{1 - q_k}
                \right)(v_{k - 1} - x_{k - 1}). 
            \end{align*}
            Now, we use the above two results and, it derives 
            \begin{align*}
                & \alpha_k(v_{k - 1} - \bar x) 
                + q_k(y_k - v_{k - 1})
                + x_{k - 1} - y_k
                \\
                &=
                \alpha_k(v_{k - 1} - \bar x)
                + q_k\left(
                    \frac{1 - \alpha_k}{1 - q_k}
                \right)(x_{k - 1} - v_{k - 1})
                - \left(
                    \frac{\alpha_k - q_k}{1 - q_k}    
                \right)(v_{k - 1} - x_{k - 1}). 
                \\
                &= 
                \alpha_k(v_{k - 1} - \bar x)
                + (1 - q_k)^{-1}\left(
                    q_k - q_k\alpha_k +(\alpha_k - q_k)
                \right)
                (x_{k - 1} - v_{k - 1})
                \\
                &= 
                \alpha_k(v_{k - 1} - \bar x)
                + \alpha_k(x_{k - 1} - v_{k - 1})
                \\
                &= \alpha_k(x_{k - 1} - \bar x). 
            \end{align*}

        \end{proof}
        % LEMMA ========================================================================================================
        \begin{lemma}[convergence preparations part III]\;\label{lemma:cnvg-prep-part3}\\
            Suppose that all the following are satisfied
            \begin{enumerate}[nosep]
                \item $(F, f, g, \mathcal E, \mu, L)$ satisfies Assumption \ref{ass:pggap-dscnt}. 
                \item The sequences $(\alpha_k)_{k \ge 0}, (q_k)_{k \ge 0}$ satisfies Definition \ref{def:rlx-momentum-seq}. 
                \item We choose $(q_k)_{k \ge 0}$ is given by $q_k = \mu/B_k$ for all $k \ge 0$. 
                \item The sequence $(y_k, v_k, x_k)_{k \ge 0}$ satisfies Definition \ref{def:st-apg}. 
            \end{enumerate}
            Then, $\forall k \ge 1$, there exists $\bar x_k \in \RR^n$, such that: 
            \begin{align*}
                & F(x_k) - F(\bar x_k)
                + \frac{B_k\alpha_k^2}{2}\Vert \bar x - v_k\Vert^2 
                \\
                &\le 
                (1 - \alpha_k)(F(x_{k - 1}) - F(\bar x_k))
                + \frac{B_k\alpha_k(\alpha_k - q_k)}{2}\Vert \bar x_k - v_{k - 1}\Vert^2
                \\ &\quad 
                    -(1 - \alpha_k)\left(
                        \mathcal E(x_{k - 1}, y_k, \mu, B_k)
                        + \frac{\mu}{2}\Vert x_{k - 1} - y_k\Vert^2
                    \right). 
            \end{align*}
        \end{lemma}
        \begin{proof}
            Recall Definition \ref{def:pg-gap}, consider: 
            \begin{align*}
                &\mathcal E(x_{k - 1}, y_k, \mu, B_k) 
                \\
                &=
                F(x_{k - 1}) - F(T_{B_k}(y_k))
                - \langle 
                    B_k(y_k - T_{B_k}(y_k)),
                    x_{k - 1} - y_k
                \rangle
                - \frac{\mu}{2}\Vert x_{k - 1} - y_k\Vert^2
                - \frac{B_k}{2}\Vert y_k - x_k\Vert^2
                \\
                &= 
                F(x_{k - 1}) - F(x_k)
                - \langle 
                    \mathcal G_{B_k}(y_k), 
                    x_{k - 1} - y_k
                \rangle
                - \frac{\mu}{2}\Vert x_{k - 1} - y_k\Vert^2
                - \frac{q_k}{2\mu}\Vert \mathcal G_{B_k}(y_k)\Vert^2. 
            \end{align*}
            On the second equality above, we used $B_k(y_k - x_k)= \mathcal G_{B_k}(y_k)$, and $B_k = \mu/q_k$.
            For all $k \ge 0$, we define $\Xi_k$ and simplify using the above result: 
            \begin{align*}
                \Xi_k &\hspace{-0.3em}\defeq 
                \mathcal E(x_{k - 1}, y_k, \mu, B_k)
                + F(x_k) - F(\bar x_k)
                - (1 - \alpha_k)(F(x_{k - 1}) - F(\bar x_k))
                \\
                &= 
                \mathcal E(x_{k - 1}, y_k, \mu, B_k)
                + F(x_k) - \alpha_kF(\bar x_k) - (1 - \alpha_k)F(x_{k - 1})
                \\
                &=
                \alpha_kF(x_{k - 1}) - \alpha_k F(\bar x_k)
                - \langle \mathcal G_{B_k}(y_k), x_{k - 1} - y_k\rangle
                - \frac{\mu}{2}\Vert x_{k - 1} - y_k\Vert^2 
                - \frac{q_k}{2\mu}\Vert \mathcal G_{B_k}(y_k)\Vert^2. 
            \end{align*}
            Now consider the new term $\Xi_k'$ which we defined and simplify below: 
            {\allowdisplaybreaks
            \begin{align*}
                \Xi_k'
                &\hspace{-0.3em}:=
                \Xi_k + \frac{B_k\alpha_k^2}{2}\Vert \bar x_k - v_k\Vert^2 
                - \frac{B_k\alpha_k(\alpha_k - q_k)}{2}\Vert \bar x_k - v_{k - 1}\Vert^2
                \\
                &\underset{(1)}{=} 
                \alpha_kF(x_{k - 1}) - \alpha_k F(\bar x_k)
                - \langle \mathcal G_{B_k}(y_k), x_{k - 1} - y_k\rangle
                - \frac{\mu}{2}\Vert x_{k - 1} - y_k\Vert^2 
                - \frac{q_k}{2\mu}\Vert \mathcal G_{B_k}(y_k)\Vert^2
                    \\&\quad 
                    + \frac{\alpha_k \mu}{2}
                    \Vert \bar x_k - v_{k - 1}\Vert^2
                    + \frac{q_k\mu}{2}\Vert y_k - v_{k - 1} \Vert^2
                    + \frac{q_k}{2\mu}\Vert \mathcal G_{B_k}(y_k)\Vert^2
                    \\ &\quad 
                    - \langle
                        q_k(y_k - v_{k - 1}) + \alpha_k(v_{k - 1} - \bar x_k), \mathcal G_{B_k}(y_k)
                    \rangle
                    + \alpha_k \mu
                    \left\langle 
                        v_{k - 1} - \bar x_k, 
                        y_k - v_{k - 1}
                    \right\rangle
                \\
                &= 
                \alpha_kF(x_{k - 1}) - \alpha_k F(\bar x_k)
                - \langle \mathcal G_{B_k}(y_k), x_{k - 1} - y_k\rangle
                - \frac{\mu}{2}\Vert x_{k - 1} - y_k\Vert^2 
                    \\&\quad 
                    + \frac{\alpha_k \mu}{2}
                    \Vert \bar x_k - v_{k - 1}\Vert^2
                    + \frac{q_k\mu}{2}\Vert y_k - v_{k - 1} \Vert^2
                    \\ &\quad 
                    - \langle
                        q_k(y_k - v_{k - 1}) + \alpha_k(v_{k - 1} - \bar x_k), \mathcal G_{B_k}(y_k)
                    \rangle
                    + \alpha_k \mu
                    \left\langle 
                        v_{k - 1} - \bar x_k, 
                        y_k - v_{k - 1}
                    \right\rangle
                \\
                &= 
                \alpha_kF(x_{k - 1}) - \alpha_k F(\bar x_k)
                - \frac{\mu}{2}\Vert x_{k - 1} - y_k\Vert^2 
                + \frac{\alpha_k \mu}{2}
                \Vert \bar x_k - v_{k - 1}\Vert^2
                + \frac{q_k\mu}{2}\Vert y_k - v_{k - 1} \Vert^2
                    \\ &\quad 
                    - \langle
                        x_{k - 1} - y_k + q_k(y_k - v_{k - 1}) + \alpha_k(v_{k - 1} - \bar x_k), 
                        \mathcal G_{B_k}(y_k)
                    \rangle
                    \\&\quad 
                    + \alpha_k \mu
                    \left\langle 
                        v_{k - 1} - \bar x_k, 
                        q_k(y_k - v_{k - 1}) 
                    \right\rangle
                \\
                &\underset{(2)}{=} 
                \alpha_kF(x_{k - 1}) - \alpha_k F(\bar x_k)
                - \frac{\mu}{2}\Vert x_{k - 1} - y_k\Vert^2 
                    \\&\quad 
                    + \frac{\alpha_k \mu}{2}
                    \Vert \bar x_k - v_{k - 1}\Vert^2
                    + \frac{q_k\mu}{2}\Vert y_k - v_{k - 1} \Vert^2
                    \\ &\quad 
                    - \alpha_k\langle
                        x_{k - 1} - \bar x_k, 
                        \mathcal G_{B_k}(y_k)
                    \rangle
                    + \alpha_k \mu
                    \left\langle 
                        v_{k - 1} - \bar x_k, 
                        q_k(y_k - v_{k - 1}) 
                    \right\rangle
                \\
                &=
                \alpha_kF(x_{k - 1}) - \alpha_k F(\bar x_k)
                - \frac{\mu}{2}\Vert x_{k - 1} - y_k\Vert^2 
                    \\&\quad 
                    + \frac{\alpha_k \mu}{2}
                    \Vert \bar x_k - v_{k - 1}\Vert^2
                    + \frac{\alpha_k\mu}{2}\Vert y_k - v_{k - 1}\Vert^2
                    + \alpha_k \mu
                    \left\langle 
                        v_{k - 1} - \bar x_k, 
                        q_k(y_k - v_{k - 1}) 
                    \right\rangle
                    \\ &\quad 
                    - \alpha_k\langle
                        x_{k - 1} - \bar x_k, 
                        \mathcal G_{B_k}(y_k)
                    \rangle
                    + \frac{q_k\mu - \mu\alpha_k}{2}\Vert y_k - v_{k - 1} \Vert^2
                \\
                &= 
                \alpha_kF(x_{k - 1}) - \alpha_k F(\bar x_k)
                - \frac{\mu}{2}\Vert x_{k - 1} - y_k\Vert^2 
                    \\&\quad 
                    + \frac{\alpha_k\mu}{2}\Vert y_k - \bar x_k\Vert^2
                    - \alpha_k\langle
                        x_{k - 1} - \bar x_k, 
                        \mathcal G_{B_k}(y_k)
                    \rangle
                    + \frac{q_k\mu - \mu\alpha_k}{2}\Vert y_k - v_{k - 1} \Vert^2
                \\
                &\underset{(3)}{\le} 
                \alpha_kF(x_{k - 1}) - \alpha_k F(\bar x_k)
                - \frac{\mu}{2}\Vert x_{k - 1} - y_k\Vert^2 
                    \\&\quad 
                    + \frac{\alpha_k\mu}{2}\Vert y_k - \bar x_k\Vert^2
                    - \alpha_k\langle
                        x_{k - 1} - \bar x_k, 
                        \mathcal G_{B_k}(y_k)
                    \rangle. 
                \\
                &= \alpha_k\left(
                    F(x_{k  - 1}) - F(\bar x_k) 
                    - \langle x_{k - 1} - \bar x_k, \mathcal G_{B_k}(y_k)\rangle
                    +  \frac{\mu}{2}\Vert y_k - \bar x_k\Vert^2
                \right)
                - \frac{\mu}{2}\Vert x_{k - 1} - y_k\Vert^2. 
            \end{align*}
            }
            At (1), we used Lemma \ref{lemma:cnvg-prep-part1}, and substituted $\Xi_k$. 
            At (2), we used Lemma \ref{lemma:cnvg-prep-part2} to simplify the inner product. 
            At (3), we used the $\alpha_k > q_k$ as in Definition \ref{def:rlx-momentum-seq}, hence it makes the coefficient $q_k \mu - \mu\alpha_k \le 0$, which gives us the inequality. 
            Now, subtracting $\mathcal E(x_{k - 1}, y_k, \mu, B_k)$ from both sides of the inequality will yield: 
            {\allowdisplaybreaks\small
            \begin{align*}
                & \Xi_k' - \mathcal E(x_{k - 1}, y_k, \mu, B_k)
                \\
                &\le 
                \alpha_k\left(
                    F(x_{k  - 1}) - F(\bar x_k) 
                    - \langle x_{k - 1} - \bar x_k, \mathcal G_{B_k}(y_k)\rangle
                    +  \frac{\mu}{2}\Vert y_k - \bar x_k\Vert^2
                    - \mathcal E(x_{k - 1}, y_k, \mu, B_k)
                \right)
                    \\ &\quad 
                    - \frac{\mu}{2}\Vert x_{k - 1} - y_k\Vert^2
                    - (1 - \alpha_k)\mathcal E(x_{k - 1}, y_k, \mu, B_k)
                \\
                &\underset{(1)}{=} 
                \alpha_k\left(
                    F(x_k) - F(\bar x_k) 
                    - \langle \mathcal G_{B_k}(y_k), x_{k - 1} - \bar x_k - (x_{k - 1} - y_k)\rangle
                \right)
                    \\&\quad 
                    + \alpha_k\left(
                        \frac{\mu}{2}\Vert y_k - \bar x_k\Vert^2
                        + \frac{\mu}{2}\Vert x_{k - 1} - y_k\Vert^2
                        + \frac{B_k}{2} \Vert y_k - x_k\Vert^2
                    \right)
                    \\ &\quad 
                    - \frac{\mu}{2}\Vert x_{k - 1} - y_k\Vert^2
                    - (1 - \alpha_k)\mathcal E(x_{k - 1}, y_k, \mu, B_k)
                \\
                &= \alpha_k\left(
                    F(x_k) - F(\bar x_k) 
                    - \langle \mathcal G_{B_k}(y_k), y_k - \bar x_k\rangle
                    + \frac{\mu}{2}\Vert y_k - \bar x_k\Vert^2
                    + \frac{B_k}{2} \Vert y_k - x_k\Vert^2
                    + \frac{\mu}{2}\Vert x_{k - 1} - y_k\Vert^2
                \right)
                    \\ &\quad 
                    - \frac{\mu}{2}\Vert x_{k - 1} - y_k\Vert^2
                    - (1 - \alpha_k)\mathcal E(x_{k - 1}, y_k, \mu, B_k)
                \\
                &=
                \alpha_k\left(
                    - \mathcal E(\bar x_k, y_k, \mu, B_k)
                    + \frac{\mu}{2}\Vert x_{k - 1} - y_k\Vert^2
                \right)
                    \\ &\quad 
                    - \frac{\mu}{2}\Vert x_{k - 1} - y_k\Vert^2
                    - (1 - \alpha_k)\mathcal E(x_{k - 1}, y_k, \mu, B_k)
                \\
                &\underset{(2)}{\le} 
                \frac{\alpha_k \mu}{2}\Vert x_{k - 1} - y_k\Vert^2
                - \frac{\mu}{2}\Vert x_{k - 1} - y_k\Vert^2
                - (1 - \alpha_k)\mathcal E(x_{k - 1}, y_k, \mu, B_k)
                \\
                &= -(1 - \alpha_k)\left(
                    \mathcal E(x_{k - 1}, y_k, \mu, B_k)
                    + \frac{\mu}{2}\Vert x_{k - 1} - y_k\Vert^2
                \right). 
            \end{align*}
            }
            At (1), we substituted $\mathcal E(x_{k - 1}, y_k, \mu, B_k)$. 
            At (2), we used the $\mathcal E(\bar x_k, y_k, \mu, B_k)\le 0$ by choosing $\bar x_k = \bar y$ in Assumption \ref{ass:pggap-dscnt}\ref{ass:pggap-dscnt-itm2}  to make the inequality. 
            Now, recall the definitions of $\Xi_k, \Xi_k'$ which means we proved the following: 
            \begin{align*}
                & \Xi_k' - \mathcal E(x_{k - 1}, y_k, \mu, B_k) 
                \\
                &= 
                \Xi_k 
                + \frac{B_k\alpha_k^2}{2}\Vert \bar x_k - v_k\Vert^2 
                - \frac{B_k\alpha_k(\alpha_k - q_k)}{2}\Vert \bar x_k - v_{k - 1}\Vert^2
                - \mathcal E(x_{k - 1}, y_k, \mu, B_k) 
                \\
                &= \mathcal E(x_{k - 1}, y_k, \mu, B_k)
                + F(x_k) - F(\bar x_k)
                - (1 - \alpha_k)(F(x_{k - 1}) - F(\bar x_k))
                    \\ &\quad 
                    + \frac{B_k\alpha_k^2}{2}\Vert \bar x_k - v_k\Vert^2 
                    - \frac{B_k\alpha_k(\alpha_k - q_k)}{2}\Vert \bar x_k - v_{k - 1}\Vert^2
                    - \mathcal E(x_{k - 1}, y_k, \mu, B_k)
                \\
                &=  
                F(x_k) - F(\bar x_k)
                - (1 - \alpha_k)(F(x_{k - 1}) - F(\bar x_k))
                    \\ &\quad 
                    + \frac{B_k\alpha_k^2}{2}\Vert \bar x_k - v_k\Vert^2 
                    - \frac{B_k\alpha_k(\alpha_k - q_k)}{2}\Vert \bar x_k - v_{k - 1}\Vert^2
                \\
                &\le -(1 - \alpha_k)\left(
                    \mathcal E(x_{k - 1}, y_k, \mu, B_k)
                    + \frac{\mu}{2}\Vert x_{k - 1} - y_k\Vert^2
                \right). 
            \end{align*}
        \end{proof}
    \subsection{Proving the convergence rate}
        To finally find the convergence rate, we will strengthen Assumption \ref{ass:pggap-dscnt}. 
        Our convergence rate is expressed for sequences $(\alpha_k)_{k \ge 0}, (q_k)_{k \ ge 0}$ if they satisfy Definition \ref{def:rlx-momentum-seq}. 
        This means that any sequence with $\alpha_k \in (q_k, 1)$ would work. 
        \par
        The following assumptions describe the behaviors of an algorithm satisfying Definition \ref{def:st-apg}, its parameters, and the properties of the objective function. 
        \begin{assumption}[Assumptions for linear convergence rate]\;\label{ass:lin-cnvg}\\
            Let $(F, f, g, \mathcal E, \mu, L)$ satisfies Assumption \ref{ass:pggap-dscnt}. 
            In addition, we strengthened the prior assumptions:
            \begin{enumerate}[nosep]
                \item Define the set of minimizers $X^+ = \argmin_{z\in \RR^n}\{ f(z) + g(z)\}$, it has $X^+ \neq \emptyset$. 
                \item Strenghening Assumption \ref{ass:pggap-dscnt}\ref{ass:pggap-dscnt-itm2}, assume $\exists \mu > 0$ such that $\forall y \in \RR^n$, it has $\mathcal E \left(\Pi_{X^+}y, y, \mu, B\right) \ge 0$. 
                \item For all $y, z \in \RR^n$, it satisfies that $\mathcal E(z, y, \mu, B) + \mu/2 \Vert z - y\Vert^2 \ge 0$. 
            \end{enumerate}
            Now, suppose an algorithm which optimizes a $F = f + g$ satisfying all previous assumptions, and it generates iterates $(y_k, x_k, v_k)_{k\ge 0}$ that satisfies Definition \ref{def:pg-gap}. 
            In addition, we assume that the parameter sequences $(\alpha_k)_{k \ge 0},(q_k)_{k \ge 0}, (B_k)_{k \ge 0}$ satisfy the following: 
            \begin{enumerate}[nosep]\setcounter{enumi}{3}
                \item The sequences $\alpha_k, q_k$ are given by Definition \ref{def:rlx-momentum-seq}. 
                \item The sequence $q_k = \mu/B_k$, with $B_k > \mu$. 
                \item For all $k \ge 0$, $\Pi_{X^+}y_k$ is unique. 
            \end{enumerate}
        \end{assumption}
        \begin{assumption}[forever sublinear convergence rate in the convex case]\;\label{ass:forever-cnvx-cnvg}\\
            Let $(F, f, g, \mathcal E, \mu, L)$ satisfies Assumption \ref{ass:pggap-dscnt}. 
            In addition, we strengthened the prior assumptions:
            \begin{enumerate}[nosep]
                \item Define the set of minimizers $X^+ = \argmin_{z\in \RR^n}\{ f(z) + g(z)\}$, it has $X^+ \neq \emptyset$. 
                \item The function $f$ is convex. Hence assumption Assumtion \ref{ass:pggap-dscnt} \ref{ass:pggap-dscnt-itm2} satisfied for all $\bar y$, $\mu = 0$. 
            \end{enumerate}
            Now, suppose an algorithm which optimizes a $F = f + g$ satisfying all previous assumptions, and it generates iterates $(y_k, x_k, v_k)_{k\ge 0}$ that satisfies Definition \ref{def:pg-gap}. 
            In addition, we assume that the parameter sequences $(\alpha_k)_{k \ge 0},(q_k)_{k \ge 0}, (B_k)_{k \ge 0}$ satisfy the following: 
            \begin{enumerate}[nosep]
                \item The sequence $B_k \ge B_{k - 1}$, it's a non-decreasing sequence. 
                \item $\alpha_k \ge \alpha_{k - 1}$, it's a non-increasing sequence. 
            \end{enumerate}
        \end{assumption}
        \begin{assumption}[optimal convergence rate in the convex case]
            
        \end{assumption}

    \subsection{Convergence results based on these assumptions}
            % THEOREM ======================================================================================================
            \begin{theorem}[convergence with generic sequence]\;\label{thm:cnvg-generic-seq}\\
                Let $(F, f, g, \mathcal E, \mu, L)$, and sequences $(\alpha_k)_{k \ge 0}, (q_k)_{k \ge 0}, (B_k)_{k \ge 0}$ 
                satisfy Assumption \ref{ass:lin-cnvg}. 
                Denote 
                $$
                    \beta_k = \prod_{i = 1}^k\max\left(
                        1 - \alpha_i, 
                        \frac{B_i\alpha_i(\alpha_i - q_i)}{\alpha_{i - 1}^2B_{i - 1}}
                    \right). 
                $$
                Then, there exists a unique $\bar x \in \RR^n$ such that for all $k \ge 1$, $\bar x = \Pi_{X^+}y_k$, and it satisfies
                \begin{align*}
                    F(x_k) - F(\bar x) + \frac{B_k}{\alpha_k^2}\Vert \bar x - v_k\Vert^2 
                    \le 
                    \beta_k\left(
                        F(x_0) - F(\bar x) + \frac{\alpha_0B_0}{2}\Vert \bar x - v_0\Vert^2
                    \right). 
                \end{align*}
                If in addition, we assume $x_{-1} = v_{-1}, \alpha_0 = 1$, then the above inequality simplifies: 
                \begin{align*}
                    & F(x_k) - F(\bar x) + \frac{B_k}{\alpha_k^2}\Vert \bar x - v_k\Vert^2 \le 
                    \frac{\beta_kB_0}{2}\Vert \bar x - x_{-1}\Vert^2.
                \end{align*}
            \end{theorem}
            \begin{proof}
                Set $\bar x_k = \bar x$ in Lemma \ref{lemma:cnvg-prep-part3} then for all $k\ge 1$ it has 
                \begin{align*}
                    & F(x_k) - F(\bar x) + \frac{B_k\alpha_k^2}{2}\Vert \bar x- v_k\Vert^2 
                    \\
                    &\le \left(
                        1 - \alpha_k
                    \right)(F(x_{k- 1}) - F(\bar x))
                    + \frac{(1 - \alpha_k)B_k\alpha_{k -1}^2 \rho_{k - 1}}{2}\Vert \bar x - v_{k - 1}\Vert^2
                    \\
                    &\underset{\text{(1)}}{=} 
                    \left(
                        1 - \alpha_k
                    \right)(F(x_{k- 1}) - F(\bar x))
                    + \frac{\alpha_k(\alpha_k - q_k)B_k\alpha_{k - 1}^2}{\alpha_{k - 1}^2}\Vert \bar x - v_{k - 1}\Vert^2
                    \\
                    &= \max\left(
                        1 - \alpha_k, 
                        \frac{B_i\alpha_k(\alpha_k - q_k)}{\alpha_{k - 1}^2B_{k - 1}}
                    \right)
                    \left(
                        F(x_{k - 1}) - F(\bar x)
                        + \frac{B_{k - 1}\alpha_{k - 1}^2}{2}\Vert \bar x - v_{k - 1}\Vert^2
                    \right). 
                \end{align*}
                From the above, a recurrence relation is formed for $k \ge 1$, unrolling the recurrence relation it has 
                \begin{align*}
                    & F(x_k) - F(\bar x) + \frac{B_k\alpha_k^2}{2}\Vert \bar x- v_k\Vert^2 
                    \\&\le 
                    \prod_{i = 1}^k\max\left(
                        1 - \alpha_i, 
                        \frac{B_i\alpha_i(\alpha_i - q_i)}{\alpha_{i - 1}^2B_{i - 1}}
                    \right)
                    \left(
                        F(x_0) - F(\bar x)
                        + \frac{B_0\alpha_0^2}{2}\Vert \bar x - v_0\Vert^2
                    \right). 
                \end{align*}
                When $x_{-1} = v_{-1}$, from Definition \ref{def:st-apg}, when $k = 0$ it has $y_0 = v_{-1} = x_{-1}$, so $x_0 = T_{B_0}(y_0)$. 
                Because $\alpha_0 = 1$, it also has $v_0 = x_0$. 
                Choose $z = \bar x, y = x_{-1}$ we have from Assumption \ref{ass:pggap-dscnt} (iii) that
                \begin{align*}
                    0 &\le \mathcal E(\bar x, x_{-1}, \mu, B_0) - \frac{\mu}{2}\Vert \bar x - x_{-1}\Vert^2
                    \\
                    &=
                    F(\bar x) - F(x_{-1}) 
                    - B_0\left\langle x_{-1} - T_{B_0}(x_{-1}), \bar x - x_{-1} \right\rangle
                    - \frac{B_0}{2}\left\Vert x_{-1} - T_{B_0}x_{-1}\right\Vert^2
                    \\
                    &= 
                    F(\bar x) - F(x_{-1}) 
                    - \frac{B_0}{2}\Vert \bar x - T_{B_0}x_{-1}\Vert^2 
                    + \frac{B_0}{2}\Vert \bar x - x_{-1}\Vert^2. 
                    \\
                    &= F(\bar x) - F(x_{-1}) 
                    - \frac{B_0}{2}\Vert \bar x - v_0\Vert^2 
                    + \frac{B_0}{2}\Vert \bar x - x_{-1}\Vert^2. 
                \end{align*}
                Substitute the above into the RHS of the inequality of previous results to complete the proof. 
            \end{proof}
            \par
            \textcolor{red}{
                SOMETHING HAS LIMITATIONS, LET US EXPLAIN.
                The parameter $q_k = \mu/B_k$ cannot be eliminated from Definition \ref{def:st-apg}. 
                If one were to implement any algorithm without the exact knowledge of $\mu$, then the convergence results we just derived are not applicable. 
                Implementations of the algorithm, and the convergence results requires the presence of both parameters $q_k, \alpha_k$. 
                Doomed. 
                Because in the applications of our interests, the exact knowledge of $\mu$ is impossible to know. 
            }
            \par
            The theorem below will definitively close the case to show that there exists choices for the sequence such that a linear convergence exists, meaning that $\beta_k$ will decrease at a linear rate. 
            \begin{lemma}[beta sequence linear bound]\;\label{lemma:beta-seq}\\
                Let sequences $(\alpha_k)_{k \ge 0}, (q_k)_{k \ge 0}, (B_k)_{k \ ge 0}$ satisfy Assumption \ref{ass:lin-cnvg}. 
                Define 
                $$
                \beta_k := 
                \prod_{i = 1}^k\max\left(
                    1 - \alpha_i, 
                    \frac{B_i\alpha_i(\alpha_i - q_i)}{\alpha_{i - 1}^2B_{i - 1}}
                \right).
                $$
                Assume in addition $\forall k \ge 0$, $\exists B : B_k = B_{k - 1} = B$ hence there exists $q$ such that $(\forall k \ge 0)\; q_k = q$. 
                Then, linear convergence of $\beta_k$ is possible under the following scenarios: 
                \begin{enumerate}
                    \item If for all $k \ge 1$, $\sqrt{q} \le \alpha_k \le \alpha_{k - 1}$ so it's non-increasing, then $\beta_k = \Pi_{i = 1}^k(1 - q/\alpha_{i - 1})$.
                    Since $\alpha_k \in (0, 1)$ and it's monotone, it has upper bound $\beta_k \le (1 - q)^k$. 
                    \item If for all $k \ge 1$, $\alpha_k = \alpha_{k - 1}$, so there exists $\alpha = \alpha_k$, making the sequence a constant, then $\beta_k = \max(1 - q/\alpha, 1 - \alpha)^k$. And, it's lowest when $\alpha_k = \sqrt{q}$. 
                \end{enumerate}
                By the assumption that $q \in (0, 1)$ hence we had a linear convergence rate in each of the above scenarios. 
            \end{lemma}
            \begin{proof}
                Since $B_k$ is a constant, and by Definition \ref{def:st-apg} it has $B_k = \frac{\mu}{q_k}$ it would make $q_k$ to be a constant for all $k \ge 0$, which we denote $q := q_k$. 
                \par
                \textbf{Proof of (i)}. 
                Using mononicity of sequence $\alpha_k$, it has $\alpha_i / \alpha_{i - 1} \le 1$ hence for all $i \ge 1$ it has 
                \begin{align*}
                    & \max\left(
                        1 - \alpha_i, 
                        \frac{\alpha_i(\alpha_i - q)}{\alpha_{i - 1}^2}
                    \right)
                    \\
                    &\le 
                    \max\left(
                        1 - \alpha_i, \frac{\alpha_i - q}{\alpha_{i - 1}}
                    \right)
                    \\
                    &\le \max\left(
                        1 - \alpha_i, 1 - \frac{q}{\alpha_{i - 1}}
                    \right)
                \end{align*}
                Then, observe that when $\alpha_i \ge \sqrt{q}$ it has: 
                \begin{align*}
                    1 - \frac{q}{\alpha_{i - 1}} &> 1 - \frac{q}{\sqrt{q}}
                    \\
                    &= 1 - \sqrt{q} \ge 1 - \alpha_k. 
                \end{align*}
                Hence
                \begin{align*}
                    \max\left(
                        1 - \alpha_i, 
                        \frac{\alpha_i(\alpha_i - q)}{\alpha_{i - 1}^2}
                    \right)\le 
                    \left(1 - \frac{q}{\alpha_{k - 1}}\right).
                \end{align*}
                The result now follows by the definition of $\beta_k$. 
                \par 
                \textbf{Proof of (ii)}. 
                The proof is obvious by setting both $q_i, \alpha_i$ to be a constant. 
            \end{proof}
        
        
\section{Examples for our assumptions}
    We will check Assumption \ref{ass:lin-cnvg} and, propose examples for it.
    \subsection{quasi strongly convex function}
        In Necoara et al.'s setting, they have the following assumptions for their objective function. 
        \begin{assumption}[The settings for Necoara's et al]\;\label{ass:necoara}\\
            The assumption is about $(f, X, X^+, f^+)$ where 
            \begin{enumerate}[nosep]
                \item $X \subseteq \RR^n$ is a closed convex set. 
                \item $f: X\rightarrow \RR$ has $L$ Lipschitz continuous gradient, and it's convex. 
                \item $X^+ = \argmin_{x \in X} f(x)\neq \emptyset$, and we denote $f^+$ to be the minimum value. 
            \end{enumerate}
        \end{assumption}
        The following definition on Quasi Strongly Convex function (Q-SCNVX) is taken from \cite[Definition 1]{necoara_linear_2019}. 
        \begin{definition}[Q-SCNVX]\label{def:Q-SCNVX}
            Let $(f, X, X^+, f^+)$ be given by Assumption \ref{ass:necoara}. 
            We define $f$ to be Quasi Strong Convex on $X$ if there exists $\kappa_f  > 0$ such that $\forall x \in \RR^n$ , with $\bar x = \Pi_{X^+}x$ it has:
            \begin{align*}
                f^+ - f(x) - \langle \nabla f(x), \bar x - x\rangle - \frac{\kappa_f}{2}\Vert \bar x - x\Vert^2 
                &\ge 0. 
            \end{align*}
        \end{definition}
        
        \begin{proposition}[Q-SCNVX is an example of our assumptions]\label{prop:qscnvx-ass-ok}
            Let $(f, X, X^+, f^+)$ satisfies Assumption \ref{ass:necoara}.
            Let $\mathcal E$ be given by Definition \ref{def:pg-gap} with $g = \delta_X$. 
            Then for all $x \in \RR^n$, let $\bar x = \Pi_{X^+}x, x^+ = T_B(x)$, there exists $B \ge 0$, such that 
            \begin{align*}
                0 &\le 
                \mathcal E(\bar x, x, B, \kappa_f)
                = 
                f(\bar x) - f(x^+) 
                - \frac{B}{2}\Vert x - x^+\Vert^2 
                - B\langle \bar x - x, x - x^+\rangle
                - \frac{\kappa_f}{2}\Vert \bar x - x\Vert^2. 
            \end{align*}
            Therefore, it satisfies Assumption \ref{ass:pggap-dscnt}. 
        \end{proposition}
        \begin{proof}
            Let $h = z \mapsto \delta_X(z) + \langle \nabla f(x), z\rangle + \frac{B}{2}\Vert z - x\Vert^2$.
            So it has $T_B(x) = \argmin_{z\in \RR^n} h(z)$. 
            Furthermore, $h(z)$ is a $B$ strongly convex function by convexity of $X$ and, the fact that other parts are just the sum of a linear and quadratic function. 
            Denote $x^+ = T_B(x)$, since $x^+$ is a minimizer therefore it has $(\forall x \in X)\; h(z) - h(x^+) \ge \frac{B}{2}\Vert z - x^+\Vert^2$. 
            That was the quadratic growth condition of $h$. 
            Next, denote $\bar x  = \Pi_{X^+}x$, let $z = \bar x$ then the condition becomes: 
            \begin{align*}
                & \frac{B}{2}\Vert \bar x - x^+\Vert^2 
                \\
                &\le 
                \langle \nabla f(x), \bar x\rangle + \frac{B}{2}\Vert \bar x - x\Vert^2 - 
                \langle \nabla f(x), x^+\rangle + \frac{B}{2}\Vert x^+ - x\Vert^2
                \\
                &=
                \langle \nabla f(x), \bar x - x\rangle - \langle \nabla f(x), x^+ - x\rangle
                + \frac{B}{2}\Vert \bar x - x\Vert^2 
                - \frac{B}{2}\Vert x^+ - x\Vert^2
                \\
                &= - D_f(\bar x, x) - f(x) + f(\bar x) - f(x^+) + f(x) + D_f(x^+, x)
                + \frac{B}{2}\Vert \bar x - x\Vert^2 
                - \frac{B}{2}\Vert x^+ - x\Vert^2
                \\
                &= 
                - D_f(\bar x, x) + f(\bar x) - f(x^+) 
                + \frac{B}{2}\Vert \bar x - x\Vert^2 
                + D_f(x^+, x)
                - \frac{B}{2}\Vert x^+ - x\Vert^2
                \\
                &\underset{\text{(1)}}{\le} 
                - \frac{\kappa_f}{2}\Vert \bar x - x\Vert^2
                + f(\bar x) - f(x^+)
                + \frac{B}{2}\Vert \bar x - x\Vert^2 
                + 0
            \end{align*}
            At (1), we used the fact that $f$ is assumed satisfy \ref{def:Q-SCNVX} because $\bar x = \Pi_{X^+}x$, in addition the inequality $D_f(x^+, x) \le \frac{B}{2}\Vert x^+ - x\Vert^2$ is true because of Lipschitz gradient Assumption in \ref{ass:necoara} hence for all $B\ge L$, it's true for all $x$. 
            Rearranging it has 
            \begin{align*}
                0 &\le
                f(\bar x) - f(x^+) 
                + \frac{B}{2}\Vert \bar x - x\Vert^2 
                - \frac{B}{2}\Vert \bar x - x^+\Vert^2
                - \frac{\kappa_f}{2}\Vert \bar x - x\Vert^2
                \\
                &= 
                f(\bar x) - f(x^+) 
                - \frac{B}{2}\Vert x - x^+\Vert^2 
                - B\langle \bar x - x, x - x^+\rangle
                - \frac{\kappa_f}{2}\Vert \bar x - x\Vert^2
                \\
                &= \mathcal E(\bar x, x, B, \kappa_f). 
            \end{align*}
            The above results showed that Assumption \ref{ass:pggap-dscnt} (ii) has been satisfied, with $\bar y = \Pi_{X^+}y$. 
            To see (iii) in Assumption \ref{ass:pggap-dscnt}, observe that Assumption \ref{ass:necoara} has $f$ is convex. 
        \end{proof}
    
    \subsection{A nonconvex example}

    \subsection{Examples of Q-SCNVX}



\bibliographystyle{siam}

\bibliography{references/refs.bib}


\end{document}
