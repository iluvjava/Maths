\documentclass[11pt]{beamer}
\usetheme{Madrid}
\usepackage[utf8]{inputenc}
\usepackage{hyperref}
\usepackage{amsmath}
\usepackage{amsfonts}
\usepackage{amssymb}
\usepackage{graphicx}
% \DeclareMathOperator{\argmin}{argmin}
\usepackage{algorithmic}
\usepackage{algorithm}
\usepackage{wrapfig}
\usepackage{subcaption}
\usepackage{tcolorbox}
\usepackage{mathtools}


% THEMES AND BEAMER SETTINGS ===================================================
% \usetheme{Madrid}
\graphicspath{{.}}


% These are Heniz's notations. 
\newcommand{\To}{\ensuremath{\rightrightarrows}}
\newcommand{\GX}{\ensuremath{\Gamma}}
\newcommand{\mal}{\ensuremath{\mathfrak{m}}}
\newcommand{\mumu}{\ensuremath{{\mu\mu}}}
\newcommand{\paver}{\ensuremath{\mathcal{P}}}
\newcommand{\ZZZ}{\ensuremath{{X \times X^*}}}
\newcommand{\RRR}{\ensuremath{{\RR \times \RR}}}
% \newcommand{\todo}{\hookrightarrow\textsf{TO DO:}}

\newcommand{\emp}{\ensuremath{\varnothing}}
%\newcommand{\la}{\ensuremath{\langle}}
%\newcommand{\ra}{\ensuremath{\rangle}}
\newcommand{\infconv}{\ensuremath{\mbox{\small$\,\square\,$}}}
\newcommand{\pscal}{\ensuremath{\scal{\cdot}{\cdot}}}
\newcommand{\Tt}{\ensuremath{\mathfrak{T}}}
\newcommand{\YY}{\ensuremath{\mathcal Y}}
\newcommand{\XX}{\ensuremath{\mathcal X}}
\newcommand{\HH}{\ensuremath{\mathcal H}}
\newcommand{\XP}{\ensuremath{\mathcal X}^*}
% \newcommand{\st}{\ensuremath{\;|\;}} CONFLICT WITH `soul' PACKAGE.
\newcommand{\zeroun}{\ensuremath{\left]0,1\right[}}

\newcommand{\lev}[1]{\ensuremath{\mathrm{lev}_{\leq #1}\:}}
\newcommand{\moyo}[2]{\ensuremath{\sideset{_{#2}}{}{\operatorname{}}\!#1}}
\newcommand{\pair}[2]{\left\langle{{#1},{#2}}\right\rangle}
%\newcommand{\scal}[2]{\left.\left\langle{#1}\:\right| {#2}  \right\rangle}
\newcommand{\scal}[2]{\langle{{#1},{#2}}\rangle}
\newcommand{\Scal}[2]{\left\langle{{#1},{#2}}\right\rangle}
%\newcommand{\scal}[2]{\braket{ {#1},{#2}}}

\newcommand{\yosida}{\ensuremath{ \; {}^}}
\newcommand{\exi}{\ensuremath{\exists\,}}
\newcommand{\GG}{\ensuremath{\mathcal G}}
\newcommand{\RR}{\ensuremath{\mathbb R}}
\newcommand{\SSS}{\ensuremath{\mathbb S}}
\newcommand{\CC}{\ensuremath{\mathbb C}}
\newcommand{\Real}{\ensuremath{\mathrm{Re}\,}}
\newcommand{\ii}{\ensuremath{\mathrm i}}
\newcommand{\RP}{\ensuremath{\left[0,+\infty\right[}}
\newcommand{\RPX}{\ensuremath{\left[0,+\infty\right]}}
\newcommand{\RPP}{\ensuremath{\,\left]0,+\infty\right[}}
\newcommand{\RX}{\ensuremath{\,\left]-\infty,+\infty\right]}}
\newcommand{\RXX}{\ensuremath{\,\left[-\infty,+\infty\right]}}
\newcommand{\KK}{\ensuremath{\mathbb K}}
\newcommand{\NN}{\ensuremath{\mathbb N}}
\newcommand{\nnn}{\ensuremath{{n \in \NN}}}
\newcommand{\thalb}{\ensuremath{\tfrac{1}{2}}}
\newcommand{\zo}{\ensuremath{{\left]0,1\right]}}}
\newcommand{\lzo}{\ensuremath{{\lambda \in \left]0,1\right]}}}
%\newcommand{\toppsepp}{\setlength{\partopsep}{-5pt}}
\newcommand{\menge}[2]{\big\{{#1} \mid {#2}\big\}}
\newcommand{\pfrac}[2]{\ensuremath{\mathlarger{\tfrac{#1}{#2}}}}


% MATH OPERATORS ===============================================================
% \newcommand{\monos}{\ensuremath{\mathcal M}}
\newcommand{\DD}{\operatorname{dom}f}
\newcommand{\IDD}{\ensuremath{\operatorname{int}\operatorname{dom}f}}
\newcommand{\CDD}{\ensuremath{\overline{\operatorname{dom}}\,f}}
\newcommand{\clspan}{\ensuremath{\overline{\operatorname{span}}}}
\newcommand{\cone}{\ensuremath{\operatorname{cone}}}
\newcommand{\dom}{\ensuremath{\operatorname{dom}}}
\newcommand{\closu}{\ensuremath{\operatorname{cl}}}
\newcommand{\cont}{\ensuremath{\operatorname{cont}}}
\newcommand{\mons}{\ensuremath{\mathcal{A}}}
\newcommand{\gra}{\ensuremath{\operatorname{gra}}}
\newcommand{\epi}{\ensuremath{\operatorname{epi}}}
\newcommand{\prox}{\ensuremath{\operatorname{Prox}_{\mu}}}
\newcommand{\hprox}{\ensuremath{\operatorname{prox}}}
\newcommand{\intdom}{\ensuremath{\operatorname{int}\operatorname{dom}}\,}
\newcommand{\inte}{\ensuremath{\operatorname{int}}}
\newcommand{\sri}{\ensuremath{\operatorname{sri}}}
\newcommand{\reli}{\ensuremath{\operatorname{ri}}}
\newcommand{\cart}{\ensuremath{\mbox{\LARGE{$\times$}}}}


\newcommand{\average}{\ensuremath{\mathcal{R}_{\mu}({\bf A},{\boldsymbol \lambda})}}
\newcommand{\averagebar}{\ensuremath{\mathcal{R}_{1}({\bf A},\bar{\lambda})}}
\newcommand{\averageonelambda}{\ensuremath{\mathcal{R}({\bf A},{\boldsymbol \lambda})}}
\newcommand{\averageonehalf}{\ensuremath{\mathcal{R}_{1}(A,1/2)}}
\newcommand{\averageinverse}{\ensuremath{\mathcal{R}_{\mu^{-1}}({\bf A}^{-1},{\boldsymbol \lambda})}}
\newcommand{\averageoneinverse}{\ensuremath{\mathcal{R}({\bf A}^{-1},{\boldsymbol \lambda})}}
\newcommand{\averagef}{\ensuremath{\mathcal{P}_{\mu}(f,\lambda)}}
\newcommand{\averagefone}{\ensuremath{\mathcal{P}_{1}(f,\lambda)}}
\newcommand{\averagefd}{\ensuremath{\mathcal{P}_{\mu}((f_{1},\ldots, f_{n}),(\lambda_{1},\ldots, \lambda_{n}))}}
\newcommand{\averagefik}{\ensuremath{\mathcal{P}_{\mu_{k}}((f_{1,k},\ldots,f_{n,k}),
(\lambda_{1,k},\ldots,\lambda_{n,k}))}}
\newcommand{\averagesub}{\ensuremath{\mathcal{R}_{\mu}(\partial f,\lambda)}}
\newcommand{\res}{\ensuremath{\mathcal{R}_{\mu}}}
\newcommand{\resmuk}{\ensuremath{\mathcal{R}_{\mu_{k}}}}
\newcommand{\newres}{\ensuremath{\mathcal{R}}}
\newcommand{\resmualpha}{\ensuremath{\mathcal{R}_{\alpha\mu}}}
\newcommand{\averageone}{\ensuremath{\mathcal{R}_{1}}}
\newcommand{\harm}{\ensuremath{\mathcal{H}(A,\lambda)}}
\newcommand{\arithmetic}{\ensuremath{\mathcal{A}(A,\lambda)}}

\newcommand{\WC}{\ensuremath{{\mathfrak W}}}
\newcommand{\SC}{\ensuremath{{\mathfrak S}}}
\newcommand{\card}{\ensuremath{\operatorname{card}}}
\newcommand{\bd}{\ensuremath{\operatorname{bdry}}}
\newcommand{\ran}{\ensuremath{\operatorname{ran}}}
\newcommand{\rec}{\ensuremath{\operatorname{rec}}}
\newcommand{\rank}{\ensuremath{\operatorname{rank}}}
\newcommand{\kernel}{\ensuremath{\operatorname{ker}}}
\newcommand{\conv}{\ensuremath{\operatorname{conv}}}
\newcommand{\segh}{\ensuremath{\operatorname{seg}}}
\newcommand{\boxx}{\ensuremath{\operatorname{box}}}
\newcommand{\clconv}{\ensuremath{\overline{\operatorname{conv}}\,}}
\newcommand{\cldom}{\ensuremath{\overline{\operatorname{dom}}\,}}
\newcommand{\clran}{\ensuremath{\overline{\operatorname{ran}}\,}}
\newcommand{\Nf}{\ensuremath{\nabla f}}
\newcommand{\NNf}{\ensuremath{\nabla^2f}}
\newcommand{\Fix}{\ensuremath{\operatorname{Fix}}}
\newcommand{\FFix}{\ensuremath{\overline{\operatorname{Fix}}\,}}
\newcommand{\aFix}{\ensuremath{\widetilde{\operatorname{Fix}\,}}}
\newcommand{\Id}{\ensuremath{\operatorname{Id}}}
\newcommand{\Max}{\ensuremath{\operatorname{max}}}
\newcommand{\Bb}{\ensuremath{\mathfrak{B}}}
\newcommand{\BB}{\ensuremath{\mathbb{B}}}
\newcommand{\Fb}{\ensuremath{\overrightarrow{\mathfrak{B}}}}
\newcommand{\Fprox}{\ensuremath{\overrightarrow{\operatorname{prox}}}}
\newcommand{\Bprox}{\ensuremath{\overleftarrow{\operatorname{prox}}}}
\newcommand{\Bproj}{\ensuremath{\overleftarrow{\operatorname{P}}}}
\newcommand{\Ri}{\ensuremath{\mathfrak{R}_i}}
\newcommand{\Dn}{\ensuremath{\,\overset{D}{\rightarrow}\,}}
\newcommand{\nDn}{\ensuremath{\,\overset{D}{\not\rightarrow}\,}}
\newcommand{\weakly}{\ensuremath{\,\rightharpoonup}\,}
\newcommand{\weaklys}{\ensuremath{\,\overset{*}{\rightharpoonup}}\,}
\newcommand{\gr}{\ensuremath{\operatorname{gra}}}
\newcommand{\g}{\ensuremath{\,\overset{g}{\rightarrow}}\,}
\newcommand{\p}{\ensuremath{\,\overset{p}{\rightarrow}}\,}
\newcommand{\e}{\ensuremath{\,\overset{e}{\rightarrow}}\,}
\newcommand{\Tbar}{\ensuremath{\overline{T}}}
\newcommand{\n}{\ensuremath{\,\overset{n}{\rightarrow}}\,}

\newcommand{\minf}{\ensuremath{-\infty}}
\newcommand{\pinf}{\ensuremath{+\infty}}
% \renewcommand{\iff}{\ensuremath{\Leftrightarrow}}
% \renewcommand{\phi}{\ensuremath{\varphi}} % The hell who added this? 
%\newcommand{\Real}{\ensuremath{\mathrm{Re}\,}}
\newcommand{\negent}{\ensuremath{\operatorname{negent}}}
\newcommand{\neglog}{\ensuremath{\operatorname{neglog}}}
\newcommand{\halb}{\ensuremath{\tfrac{1}{2}}}
\newcommand{\bT}{\ensuremath{\mathbf{T}}}
\newcommand{\bX}{\ensuremath{\mathbf{X}}}
\newcommand{\bL}{\ensuremath{\mathbf{L}}}
\newcommand{\bD}{\ensuremath{\boldsymbol{\Delta}}}
\newcommand{\bc}{\ensuremath{\mathbf{c}}}
\newcommand{\by}{\ensuremath{\mathbf{y}}}
\newcommand{\bx}{\ensuremath{\mathbf{x}}}
\newcommand{\bA}{{\bf A}}
\newcommand{\Other}{Indeterminate }
\newcommand{\other}{indeterminate }


%%% Raf's stuff  ===============================================================
\newcommand{\al}{\alpha}
\newcommand{\la}{\lambda}
\newcommand{\La}{\Lambda}
\newcommand{\pluss}{{\hskip1pt \raise1pt\vbox{\hrule width6pt \vskip1pt
\hrule width6pt}\kern-4pt{\lower1pt\hbox{\vrule height6pt \kern1pt\vrule
height6pt}}\hskip5pt}}
\newcommand{\timess}{\star}
\newcommand{\argmax}{\mathop{\rm argmax}\limits}
\newcommand{\argmin}{\mathop{\rm argmin}\limits}
\newcommand{\product}{\langle\cdot,\cdot\rangle}
\newcommand{\im}{\mathrm{Im}}
\newcommand{\multival}{\ensuremath{X\to 2^{X^*}}}
\newcommand{\SX}{\ensuremath{2^{X^*}}}


% Hongda's Stuff: 
\newcommand{\N}{\ensuremath{\mathbb N}}
% REQURES MATH TOOLS PACKAGE
\newcommand{\defeq}{\vcentcolon=}
\newcommand{\eqdef}{=\vcentcolon}
\newcommand{\dist}{\ensuremath{\operatorname{dist}}}
\newcommand{\todoinline}[1]{\todo[inline, caption={}]{#1}} % IMPORT WANG'S LATEX CUSTOM COMMANDS. 
\setbeamertemplate{theorems}[numbered] % ADD NUMBERING TO ALL AMS THEOREMS. 
\setbeamertemplate{footline}[frame number] % ADD PAGE NUMBERS ON BOTTOM. 
\setbeamertemplate{blocks}[rounded][shadow=false]
\setbeamertemplate{navigation symbols}{} 
% \setbeamercolor{block title}{bg=cyan,fg=black}  % CHANGE THE BLOCK STYLE IN BEAMER. 
% \setbeamercolor{block body}{bg=lime,fg=black} % CHANGE THE BLOCK STYLE IN BEAMER. 

% BIB STYLES SETTINGS ----------------------------------------------------------
\setbeamertemplate{bibliography item}{\insertbiblabel}
\setbeamerfont{bibliography item}{size=\footnotesize}
\setbeamerfont{bibliography entry author}{size=\footnotesize}
\setbeamerfont{bibliography entry title}{size=\footnotesize}
\setbeamerfont{bibliography entry location}{size=\footnotesize}
\setbeamerfont{bibliography entry note}{size=\footnotesize}
% \setbeamercovered{transparent}  % GREY OUT PAUSED FUTURE ITEMS IN SLID. 
\bibliographystyle{siam}

% --- Show a title page at the start of each section. 
% \AtBeginSection[]{
%   \begin{frame}
%   \vfill
%   \centering
%   \begin{beamercolorbox}[sep=8pt,center,shadow=true,rounded=true]{title}
%     \usebeamerfont{title}\insertsectionhead\par%
%   \end{beamercolorbox}
%   \vfill
%   \end{frame}
% }



% SLIDE INFORMATION ============================================================

\author[Hongda Li]{Hongda Li}

\title{
    Relaxed Weak Accelerated Proximal Gradient Method: a Unified Framework for Nesterov's Accelerations
}
% \newcommand{\email}{lalala@lala.la}
\institute[UBCO]{
    University of British Columbia Okanagan
}
\date[\today]{
    \today 
    \\
    2025 Annual Midwest Optimization Meeting
    \\ 
    \vspace{1cm} \tiny{Joint work with Shawn/Xianfu Wang}
}
\subject{Nesterov's acceleration and its applications}

% SLIDES ELEMENTS CUSTOMIZATIONS ===============================================
\theoremstyle{definition}
\newtheorem{remark}{Remark}[section]
\newtheorem{assumption}{Assumption}[section]
\newtheorem{proposition}{Proposition}[section]



% DOCUMENT STARTS ==============================================================
\begin{document}

\begin{frame}
    \titlepage
\end{frame}

\begin{frame}{ToC}
    \tableofcontents
\end{frame}

\section{Introduction and background}
    \subsection{Motivations}
        \begin{frame}{Nesterov's accelerated gradient in 1983}
            \begin{block}{Nesterov's accelerated gradient in 1983}
                Initialized $x_0 = y_0 \in \RR^n$. 
                The original formulation by Nesterov in 1983 \cite{nesterov_method_1983} has
                {\small
                \begin{align*}
                    & x_{k + 1} := y_k - L^{-1}\nabla F(y_k),
                    \\
                    & t_{k + 1} := 1/2\left(1 + \sqrt{1 + 4t_{k}^2}\right),
                    \\
                    & \theta_{k + 1} := (t_{k} - 1)/t_{k + 1}, \label{eqn:example-algorithm}
                    \\
                    & y_{k + 1} := x_{k + 1} + \theta_{k + 1}(x_{k + 1} - x_k).
                \end{align*}    
                }
            \end{block}
            When $F$ is convex and, $L$ Lipschitz smooth. 
            If minimizer $x^+$ exists, then: 
            \begin{itemize}
                \item a convergence rate of $\mathcal O(1/k^2)$ for $(F(x_k) - F(x^+))_{k \ge 1}$;
                \item this convergence rate is optimal in the sense as proposed by Nesterov \cite{nesterov_lectures_2018}.
            \end{itemize}
        \end{frame}
        \begin{frame}{Other choices for the momentum sequence part I}
            Major results from the literature regarding different choices for the momentum sequence $(t_k)_{k \ge 0}$. 
            \begin{block}{Beck, Teboulle 2009 {\cite[Theorem 4.4]{beck_fast_2009-1}}}
                Commonly known by the name FISTA, it allows $(t_k)_{k}$: 
                \begin{align*}
                    (\forall k \ge 1)\quad t_k (t_k - 1) &\le t_{k - 1}^2. 
                \end{align*}    
                When $t_k (t_k - 1) = t_{k - 1}^2$, it yields $\mathcal O(1/k^2)$ convergence rate of optimality gap
            \end{block}
            \begin{block}{Chambolle, Dossal 2015 \cite[Theorem 4.1]{chambolle_convergence_2015}}
                If $t_k = (k + a -1)/a$ in which case this gives $\theta_{k} = \frac{k - 2}{k + a - 1}$, then $(x_k)_{k \ge 0}$ has weak convergence in Hilbert space for all $a > 2$.    
            \end{block}
        \end{frame}
        \begin{frame}{Other choices for the momentum sequence part II}
            A finetuned constant momentum sequence gives optimal linear convergence rate for specific strongly convex objective. 
            \begin{block}{
                Nesterov 2018 {\cite[Theorem 2.2.3]{nesterov_lectures_2018}}, 
                Beck 2017 {\cite[Theorem 10.42]{beck_first-order_2017}}
            }
                If in addition whe $F:\RR^n \rightarrow \RR$ is $L$ smooth gradient, and $\mu > 0$ strongly convex, then for $(\theta_k)_{k\ge 0}$ such that: 
                \begin{align*}
                    \theta_k = \frac{\sqrt{L/\mu} - 1}{\sqrt{L/\mu} + 1}, 
                \end{align*}
                it gives $\mathcal O\left(\left(1 - \sqrt{\mu/L}\right)^k\right)$ convergence rate for $(F(x_k) - F(x^+))_{k \ge 1}$. 
            \end{block}
        \end{frame}
        \begin{frame}{Other choices for the momentum sequence part III}
            The cases when the inequality of Nesterov's rule is not satisfied had been investigated as well.
            \begin{block}{Apidopoulos et al. 2018 \cite{apidopoulos_convergence_2018}}
                If $\theta_k = \frac{k}{k + b}\;\forall k \ge 0$ where $b \in (0, 3)$ then, the optimality gap has a convergence rate of $\mathcal O(1/k^{2b/3})$. 
            \end{block}
        \end{frame}
    
    \subsection{Our discoveries and inspirations}
        \begin{frame}{Relaxed Weak Accelerated Proximal Gradient}
            We asked the following major question.  
            \begin{tcolorbox}
                How much can we relax the choice of the sequence $\theta_k$ and still get an upper bound for the convergence rate? 
            \end{tcolorbox}
            We addressed it by what we called: ``Relaxed Weak Accelerated proximal Gradient'' (R-WAPG). 
            \begin{enumerate}
                \item It's interpreted from the Nesterov's estimating sequence technique and based on the proximal gradient inequality. 
                \item It relaxes the assumption of the momentum sequence parameter while still retaining an upper bound on the optimality gap. 
            \end{enumerate}
        \end{frame}
        \begin{frame}{The assumptions we make throughout}
            \begin{assumption}[Convex smooth plus nonsmooth]
                Let the ambient space be $\RR^n$, equipped with Euclidean inner product and norm. 
                Define $F := f + g$.
                \begin{enumerate}
                    \item $f: \RR^n \rightarrow \RR$ is $L$ Lipschitz smooth and $\mu \ge 0$ strongly convex.
                    \item $g: \RR^n \rightarrow \overline \RR$ is proper, closed and convex. The extended real is defined as $\overline \RR := \RR \cup \{\infty\}$.
                    \item A minimizer $x^+$ exists for the optimization problem: $F^+ = \min_x \left\lbrace f(x) + g(x)\right\rbrace$.
                \end{enumerate}
            \end{assumption}
            \textbf{This assumption is the full scope of the theoretical and practical discussion.}
        \end{frame}
        \begin{frame}{Theoretical results summary}
            Let $(\alpha_k)_{k \ge0}, (\rho_k)_{k \ge 0}$ be two sequences. 
            Suppose that $\alpha_0 \in (0, 1]$ and for all $k \ge 1$, $\alpha_k \in (\mu/L, 1)$ and define $(\rho_k)_{k\ge0 }$ to be: 
            \begin{tcolorbox}\noindent\vspace{-1em}
                \begin{align*}
                    \rho_k &:= \frac{\alpha_{k + 1}^2 - (\mu/L)\alpha_{k + 1}}{(1 - \alpha_{k + 1})\alpha_k^2} \quad \forall (k \ge 0).
                \end{align*}
            \end{tcolorbox}
            We can show Nesterov's type accelerated proximal gradient algorithm (or equivalently, FISTA with $g \equiv 0$) has for all $k \ge 1$:
            \begin{tcolorbox}\noindent\vspace{-1em} 
                \begin{align*}
                    F(x_k) - F^+ 
                    \le
                    \mathcal O\left(
                        \left(
                            \prod_{i = 0}^{k - 1} \max(1, \rho_{i})
                        \right)
                        \prod_{i = 1}^{k} \left(1  - \alpha_i\right)
                    \right).
                \end{align*}
            \end{tcolorbox}
        \end{frame}
        \begin{frame}{Unifying variants of FISTA in the literatures}
            We tried it out on some variants in the literature. 
            Let's label 
            \begin{enumerate}
                \item[(i)] R-WAPG with $\mu \ge 0$, %\ref{def:wapg} 
                \item[(ii)] Chambolle, Dossal 2015 \cite{chambolle_convergence_2015} with $\mu \ge 0$,
                \item[(iii)] V-FISTA Beck (10.7.7) \cite{beck_first-order_2017}, with $\mu > 0$,
                \item[(iv)] R-WAPG with $\mu > 0$. %\ref{def:wapg}
            \end{enumerate}
            \begin{table}[H]
                \centering
                {\scriptsize
                \begin{tabular}{|l|l|l|l|l|}
                \hline
                    Algorithm 
                    & 
                    $\alpha_k$ 
                    & 
                    $\rho_k$ 
                    & 
                    $F(x_k) - F^+ \le \mathcal O(\cdot)$ 
                \\ \hline
                    (i) &
                    $\alpha_k \in(\mu/L, 1)$ &
                    $\rho_k > 0$ &
                    \begin{tabular}{l}
                        $\prod_{i = 0}^{k - 1}\max(1, \rho_i)(1 - \alpha_{i + 1})$
                        \\
                        % (Proposition \ref{prop:wapg-convergence})
                    \end{tabular}
                \\ \hline
                    (ii) &
                    $ 0 < \alpha_k^{-2} \le \alpha_{k + 1}^{-1} - \alpha_{k + 1}^{-2}$ &
                    $\rho_k \ge 1$ &
                    \begin{tabular}{l}
                        $\alpha_k^{2}$ \\ 
                        % (Theorem \ref{thm:r-wapg-on-cham-doss})
                    \end{tabular}
                \\ \hline
                    (iii) &
                    $\alpha_k = \sqrt{\mu/L}$ &
                    $\rho_k = 1$ &
                    \begin{tabular}{l}
                        $(1 - \sqrt{\mu/L})^k$,
                        \\
                        % (Theorem \ref{thm:fixed-momentum-fista}, remark)
                    \end{tabular}
                \\ \hline
                    (iv) &
                    $\alpha_k = \alpha \in (\mu/L, 1)$ &
                    $\rho_k = \rho > 0$ &
                    \begin{tabular}{l}
                        $\left(1 - \min\left(\mu/(\alpha L), \alpha\right)\right)^{k}$
                        \\
                        % (Theorem \ref{thm:fixed-momentum-fista})
                    \end{tabular}
                \\ \hline
                \end{tabular}
                }
            \end{table}
            Note that in Chambolle and Dossal \cite{chambolle_convergence_2015} and their sequence $t_k = \alpha_k^{-1}$ and $\alpha_k = a/(k + 1)$ satisfies $\alpha_k^{-1} \le \alpha_{k + 1}^{-1} - \alpha_{k + 1}^{-2}$. 
        \end{frame}
        \begin{frame}{The numerical experiment that had been inspired}
            Our theories suggest that the upper bound for $F(x_k) - F(x^+)$ in our R-WAPG framework is ultimately described by the sequence $(\alpha_k)_{k \ge 0}$.
            Since $\alpha_k \in (\mu/L, 1)$ and all convex function is strongly convex with strong convexity constant $\mu = 0$, this is what we did:
            \begin{tcolorbox}
                We made an algorithm which we called ``Free R-WAPG", it estimates the strong convexity constant $\mu$ using the Bregman Divergence at two successive iterates when the algorithm is running. 
                This information is used to determine the sequence $(\alpha_k)_{k \ge 0}$, and used to visualized our theoretical upper bound. 
            \end{tcolorbox}
        \end{frame}

\section{The R-WAPG framework and convergence rate}
    \subsection{Introducing the R-WAPG framework}
        \begin{frame}{Proximal gradient, gradient mapping}
            Starting in this section, we will highlight key theoretical results that will allow the unifying convergence rate for Euclidean variants of FISTA in the literature.
            \par
            We start off with the following definition. 
            \begin{definition}[The proximal gradient operator]
                Define for all $x\in \RR^n$: 
                \begin{align*}
                    T_L(y) 
                    &:= \argmin_{x \in \RR^n} \left\lbrace
                        g(x) + \langle \nabla f(y), x\rangle + L/2\Vert x - y\Vert^2
                    \right\rbrace 
                    \\
                    &= \left(I + L^{-1}\partial g\right)^{-1}
                    \left(I - L^{-1}\nabla f\right)(y),
                    \\
                    \mathcal G_L(y)
                    &:= L(y - T_L(y)).
                \end{align*}
            \end{definition}
            Note, the $I$ here is the identity operator in $\RR^n$. 
            $L > 0$ is the Lipschitz smoothness parameter for $f$. 
        \end{frame}
        \begin{frame}{The R-WAPG sequences}
            \begin{definition}[R-WAPG sequences]\label{def:rwapg-seq}
                Assume $0 \le \mu < L$. 
                Let $(\alpha_k)_{k \ge 0}, (\rho_k)_{k \ge 0}$ be such that: 
                \begin{enumerate}
                    \item $\alpha_0 \in (0, 1]$, 
                    \item $\alpha_k \in (\mu/L, 1) \quad (\forall k \ge 1)$, 
                    \item $\rho_k := \frac{\alpha_{k + 1}^2 - (\mu/L) \alpha_{k + 1}}{(1 - \alpha_{k + 1}) \alpha_k^2}$ for all $k \ge 0$. 
                \end{enumerate}  
                We call the sequences $(\alpha_k)_{k \ge 0}, (\rho_k)_{k \ge0}$ the R-WAPG sequences. 
            \end{definition}
        \end{frame}
        \begin{frame}{R-WAPG}
            \begin{definition}[R-WAPG]\label{def:wapg}
                Choose any $x_1 \in \RR^n, v_1 \in \RR^n$.
                Let $(\alpha_k)_{k \ge0}, (\rho_k)_{k \ge 0}$ be given by Definition \ref{def:rwapg-seq}.
                The algorithm generates the sequences of vectors $(y_k, x_{k + 1}, v_{k + 1})_{k \ge 1}$ for all $k\ge 1$ by such that:
                \begin{tcolorbox}\vspace{-1em}
                    \begin{align*}
                        \gamma_k &\defeq \rho_{k -1}L\alpha_{k - 1}^2,
                        \\
                        L\alpha_k^2 &= (1 - \alpha_k)\gamma_k + \mu \alpha_k, 
                        \\
                        \hat \gamma_{k + 1} & \defeq L\alpha_k^2,
                        \\
                        y_k &=
                        (\gamma_k + \alpha_k \mu)^{-1}(\alpha_k \gamma_k v_k + \hat\gamma_{k + 1} x_k),
                        \\
                        v_{k + 1} &=
                        \hat\gamma^{-1}_{k + 1}
                        \left(\gamma_k(1 - \alpha_k) v_k - \alpha_k \mathcal G_L (y_k) + \mu \alpha_k y_k\right),
                        \\
                        x_{k + 1} &= T_L (y_k).
                    \end{align*}
                \end{tcolorbox}
            \end{definition}
        \end{frame}
        \begin{frame}{Convergence of R-WAPG}
            After six pages of math in the paper (not necessarily dense), we deduced the following theorem: 
            \begin{proposition}[R-WAPG convergence]\label{prop:wapg-convergence}
                Fix any arbitrary $x^* \in \RR^n, N \in \mathbb N$.
                Let $(\alpha_k)_{k \ge 0}, (\rho_k)_{k \ge 0}$ be R-WAPG sequences.
                Let vector sequences $(y_k, v_{k}, x_{k})_{k \ge 1}$ given by Definition \ref{def:wapg}. 
                Then for all $k = 1, 2, \ldots, N$:
                {\small
                \begin{align*}
                    & F(x_{k + 1}) - F(x^*) + \frac{L \alpha_k^2}{2}\Vert v_{k + 1} - x^*\Vert^2
                    \\
                    &\le
                    \left(
                        \prod_{i = 0}^{k - 1} \max(1, \rho_{i})
                    \right)
                    \left(
                        \prod_{i = 1}^{k} \left(1  - \alpha_i\right)
                    \right)
                    \left(
                        F(x_1) - F(x^*) + \frac{L\alpha_0^2}{2}\Vert v_1 - x^*\Vert^2
                    \right).
                \end{align*}
                }
            \end{proposition}
            To use this theorem for the convergence of existing variants of Accelerated Proximal gradient method, we need alternative representations of R-WAPG to fit what commonly appears in the literature. 
        \end{frame}
    \subsection{Different representation of accelerated proximal gradient}
        \begin{frame}{Iterates by R-WAPG has other representations}
            \begin{proposition}[Alternative representations of the iterates]\label{prop:wapg-first-equivalent-repr}
                If the sequence $(y_k, v_k, x_k)_{k \ge 1}$ is produced by R-WAPG (Definition \ref{def:wapg}), and $\alpha_0 = 1, x_1 = v_1$. 
                Then they satisfy for all $k\ge 1$: 
                {\footnotesize
                \begin{align}
                    y_{k} &=
                    \left(
                        1 + \frac{L - L\alpha_{k}}{L\alpha_{k} - \mu}
                    \right)^{-1}
                    \left(
                        v_{k} +
                        \left(\frac{L - L\alpha_{k}}{L\alpha_{k} - \mu} \right) x_{k}
                    \right)
                    % \label{eqn:rwapg-first-equiv-form-eqn-1}
                    \\
                    x_{k + 1} 
                    &=y_k - L^{-1} \mathcal G_L (y_k),
                    \\
                    v_{k + 1} 
                    &=
                    \left(
                        1 + \frac{\mu}{L \alpha_k - \mu}
                    \right)^{-1}
                    \left(
                        v_k +
                        \left(\frac{\mu}{L \alpha_k - \mu}\right) y_k
                    \right) - \frac{1}{L\alpha_{k}}\mathcal G_L (y_k)
                    \\
                    &= 
                    x_{k + 1} + (\alpha_k^{-1} - 1)(x_{k + 1} - x_k), 
                    \\
                    y_{k + 1}
                    &= x_{k + 1} +
                    \frac{\rho_{k}\alpha_{k}(1 - \alpha_k)}
                    {\rho_k\alpha_k^2 + \alpha_k}(x_{k + 1} - x_k). 
                \end{align}
                }
            \end{proposition}
            When $\mu = 0$, then has:
            {\small
            \begin{align*}
                (\forall k \ge 1) \quad
                \frac{\rho_k\alpha_k(1 - \alpha_k)}{\rho_k\alpha_k^2 + \alpha_{k + 1}}
                & = \alpha_{k + 1}(\alpha_k^{-1} - 1).
            \end{align*}
            }
        \end{frame}
    \subsection{Unifying different variants of accelerated proximal gradient}
        \begin{frame}{Convergence of Chambolle and Dossal variant of FISTA}
            FISTA algorithm proposed in Chambolle and Dossal \cite{chambolle_convergence_2015} has a function value convergence fully described by Proposition \ref{prop:wapg-convergence}. 
            \begin{lemma}[R-WAPG sequence as inverted FISTA sequence]\label{lemma:inverted-fista-seq}
                Let R-WAPG sequence $(\rho_k)_{k \ge 0}, (\alpha_k)_{k \ge 0}$ be given by Definition \ref{def:rwapg-seq}.
                If it has $\mu = 0, \rho_k \ge 1\; \forall k \ge 0$, and $\alpha_0 = 1$, then:
                \begin{enumerate}
                    \item $\alpha_k^{-2} \ge \alpha_{k + 1}^{-2} - \alpha_{k + 1}^{-1}\; \forall k \ge 0$. 
                    \item Let $t_k := \alpha_k^{-1}$, then it would satisfy $0 < t_{k + 1} \le (1/2)\left(1 + \sqrt{1 + 4t_k^2}\right)\;\forall k\ge 0$, hence the name: ``inverted FISTA sequence''.
                    \item $\prod_{i = 1}^k\max(1, \rho_{k - 1})(1 - \alpha_k) = \alpha_k^2 \quad (\forall k \ge 1)$.
                \end{enumerate}
            \end{lemma}
            Let $\alpha_k = a/(k + 1)$, $a \ge 2$, Applying Proposition \ref{prop:wapg-convergence} yields $\mathcal O(1/k^2)$ convergence rate of $(F(x_k) - F^+)_{k \ge 1}$ after some algebra routine. 
        \end{frame}
        \begin{frame}{In addition, our theory made one new prediction}
            \begin{theorem}[fixed momentum APG]\label{thm:fixed-momentum-fista}
                {\small
                Assume $L > \mu > 0$, let a pair of constant R-WAPG sequence: $(\alpha_k)_{k \ge0}$ is a constant and $\alpha_k \in (\mu/L, 1)$ for all $k \ge 0$.
                Define $q := \mu/L$ and for any fixed $r \in \left(\sqrt{q}, \sqrt{q^{-1}}\right)$, represent $\alpha_k = \alpha = r \sqrt{q}$. 
                Consider the algorithm with a constant momentum specified by the following:
                \begin{tcolorbox}
                    Define $\theta = \left(1 - r^{-1}\sqrt{q}\right)(1 - r\sqrt{q})(1 - q)^{-1}$.
                    \\
                    Initialize $y_1 = x_1$; for $k = 1, 2, \ldots, N$, update:
                    \begin{align*}
                        &x_{k + 1} = y_k + L^{-1}\mathcal G_L (y_k)
                        ,
                        \\
                        & y_{k + 1} = x_{k + 1} + \theta(x_{k + 1} - x_k).
                    \end{align*}
                \end{tcolorbox}
                Then the algorithm generates $(x_k)_{k \ge 1}$ such that $(F(x_{k}) - F(x^*))_{k\geq 1}$ converges at a rate of $\mathcal O\left(\left(1 - \min\left(\mu/(\alpha L), \alpha\right)\right)^k\right)$.
                }
            \end{theorem}
        \end{frame}
\section{Adaptive momentum sequence and numerical experiments}
    \subsection{Free R-WAPG}
        \begin{frame}{Introducing numerical experiments}
            All convex functions are strongly convex with strong convexity constant $\mu = 0$. 
            With $\mu = 0$, it has from Definition \ref{def:rwapg-seq} that for all $k \ge 0$: 
            \begin{enumerate}
                \item It has $\alpha_k \in (0, 1)$. The sequence $(\alpha_k)_{k \ge 1}$ is as loose as possible. 
                \item It has $\rho_{k - 1} = \alpha_k/((1 - \alpha_k)\alpha_{k - 1}^2)$, hence $(1 - \alpha_k)\rho_{k - 1} = \alpha_k^2/\alpha_{k - 1}^2$. 
            \end{enumerate}
            This simplifies our convergence claim into: 
            \begin{align*}
                \prod_{i = 0}^{k - 1}
                \max(1, \rho_i)(1 - \alpha_{i + 1})
                &= 
                \prod_{i = 0}^{k - 1}
                \max(1 - \alpha_{i + 1}, \rho_i(1 - \alpha_{i + 1}))
                \\
                &= 
                \prod_{i = 0}^{k - 1}
                \max\left(
                1 - \alpha_{i + 1}, \frac{\alpha_{i + 1}^2}{\alpha_i^2}\right). 
            \end{align*}
            This inspired our numerical experiments. 
        \end{frame}
        \begin{frame}{Free R-WAPG}
            \begin{algorithm}[H]
                \begin{algorithmic}
                {\footnotesize
                \STATE{\textbf{Input: } $f, g, L > \mu \ge 0, x_0 \in \RR^n, N \in \N$}
                \STATE{\textbf{Initialize: }$y_0 := x_0;L_0 := 1; \mu_0 := 1/2; \alpha_0 = 1$;}
                \STATE{\textbf{Compute: } $f(y_k)$; }
                \FOR{$k = 0, 1, 2, \cdots, N$}
                    \STATE{\textbf{Compute: }$\nabla f(y_k); x^+:= [I + L_k^{-1}\partial g]^{-1}(y_k - L_k^{-1}\nabla f(y_k))$;}
                    \WHILE{$L_k/2\Vert x^+ - y_k\Vert^2 < D_f(x^+, y_k)$}
                        \STATE{$L_k:= 2L_k$;}
                        \STATE{$x^+ = [I + L_k^{-1}\partial g]^{-1}(y_k - L_k^{-1}\nabla f(y_k))$; }
                    \ENDWHILE
                    \STATE{$x_{k + 1} := x^+$;}
                    \STATE{
                        \textcolor{red}{
                            $\alpha_{k + 1} := (1/2)\left(\mu_k/L_k - \alpha_{k}^2 + \sqrt{(\mu_k/L_k - \alpha_{k}^2)^2 + 4\alpha_{k}^2}\right)$;
                        }
                    }
                    \STATE{$\theta_{k + 1} := \alpha_k(1 - \alpha_k)/(\alpha_k^2 + \alpha_{k + 1})$;}
                    \STATE{$y_{k + 1}:= x_{k + 1} + \theta_{k + 1}(x_{k + 1} - x_k)$; }
                    \STATE{\textbf{Compute: } $f(y_{k + 1})$}
                    \STATE{
                        \textcolor{red}{
                            $\mu_{k + 1} := D_f(y_{k + 1}, y_{k})/\Vert y_{k + 1} - y_k\Vert^2 + (1/2)\mu_k$;
                        }
                    }
                \ENDFOR
                }
                \end{algorithmic}
                \caption{Free R-WAPG}
                \label{alg:free-rwapg}
            \end{algorithm}
        \end{frame}
    \subsection{Numerical experiment, simple quadratic}
        \begin{frame}{A basic experiment on convex quadratic functions}
            Consider $\min_{x}\{F(x) := f(x) + 0\}$ with $f(x) = (1/2)\langle x, Ax\rangle$. 
            We are measuring: 
            \begin{align*}
                \delta_k := \log_2\left(
                    \frac{F(x) - F^+}{F(x_0) - F^+}\right).  
            \end{align*}
            Here are the parameters. 
            \begin{enumerate}
                \item $A \in \RR^{N \times N}$ square diagonal defined by 
                \begin{align*}
                    (\forall i = 1, \ldots, N)\; A_{i, i} = \begin{cases}
                        0 & i = 1
                        \\
                        \mu + \frac{(i - 1)(L - \mu)}{N - 1} & i \ge 2
                    \end{cases}
                \end{align*}
                \item $L = 1, \mu = 10^{-5}$ are known in prior. 
                \item All algorithm terminates after $\Vert \mathcal G_L(y_k) \Vert \le 10^{-10}$. 
                \item Initial conditions $x_0 \sim \mathcal N(I, \mathbf 0)$, i.i.d. 
            \end{enumerate}
        \end{frame}
        \begin{frame}{Experiment results 1}
            A realization of $x_0 \sim \mathcal N(I, \mathbf 0)$ is used for V-FISTA, M-FISTA (Monotone restarted FISTA) and Free R-WAPG. 
            Below are plots of the medium, minimum, and maximum of $\delta_k$: 
            \begin{figure}[H]
                \begin{subfigure}[b]{0.47\textwidth}
                    \centering
                    \includegraphics[width=\textwidth]{assets/simple_regression_batched-256.png}
                    \caption{$N = 256$, simple convex quadratic.}
                \end{subfigure}
                \hfill
                \begin{subfigure}[b]{0.47\textwidth}
                    \centering
                    \includegraphics[width=\textwidth]{assets/simple_regression_batched-1024.png}
                    \caption{$N = 1024$, simple convex quadratic. }
                \end{subfigure}
                \caption{
                    Statistics for experiments with simple convex quadratic for V-FISTA, M-FISTA, and R-WAPG.
                }
                \label{fig:simple-quadratic-NOG}
            \end{figure}
        \end{frame}
        \begin{frame}{$\mu$ estimations}
            Free R-WAPG estimates $\mu_k$ each iteration. 
            The value of $\mu_k$ had been recorded for one trial and this is a plot of the estimations: 
            \begin{figure}[H]
                \centering
                \begin{subfigure}[b]{0.47\textwidth}
                    \centering
                    \includegraphics[width=\textwidth]{assets/simple_regression_loss_sc_estimates_1024.png}
                \end{subfigure}
                \hfill
                \begin{subfigure}[b]{0.47\textwidth}
                    \centering
                    \includegraphics[width=\textwidth]{assets/simple_regression_loss_1024.png}
                \end{subfigure}
                \caption{
                    $N = 1024$, the $\mu$ estimates produced by Algorithm \ref{alg:free-rwapg} (R-WAPG) is recorded.
                }
                \label{fig:simple-quadratic-r-wapg-mu-estimates}
            \end{figure}
        \end{frame}
        \begin{frame}{Real time R-WAPG upper bound}
            By collecting $(\alpha_k)_{k \ge 0}$ while the algorithm is running we made the following plot for the simple quadratic experiment. 
            \begin{figure}[H]
                \centering
                \begin{subfigure}[b]{0.75\textwidth}
                    \centering
                    \includegraphics[width=\textwidth]{
                        assets/simple_regression_rwapg_upperbnd_1024.png
                    }
                \end{subfigure}
                \caption{
                    $N = 1024$, the upper bound estimates in real time from the collected $(\alpha_k)_{k \ge 0}$ sequence. 
                }
                \label{fig:simple-quadratic-r-wapg-rwapg-upperbnd}
            \end{figure}
        \end{frame} 
    \subsection{Numerical experiment, LASSO}
        \begin{frame}{The LASSO problem}
            The problem of LASSO from Tibshirani \cite{tibshirani_regression_1996} is the optimization problem $\min_{x \in \RR^n}\{(1/2)\Vert Ax - b\Vert^2 - \lambda\Vert x\Vert_1\}$. 
            \begin{enumerate}
                \item $A \in \RR^{N \times N}$, full of i.i.d random variable from a standard normal distribution. 
                \item Computed in prior, $L, \mu$ are parameters estimated by $\mu = 1/\Vert (A^TA)^{-1}\Vert$ and $L = \Vert A^TA\Vert$. The norm is the spectral norm. 
                \item The smallest value of $F(x_k)$, for all algorithms, over all tries are taken as an estimate of $F^+$. 
            \end{enumerate}
        \end{frame}
        \begin{frame}{Results plotted}
            With $A\in \RR^{N\times N}$ fixed, for each $x_0 \sim \mathcal N(I, \mathbf 0)$ realized, 30 experiments are preformed and the medium, minimum and maximum of $\delta_k$ are recorded for M-FISTA, Free R-WAPG, and V-FISTA. 
            \begin{figure}[H]
                \begin{subfigure}[b]{0.47\textwidth}
                    \centering
                    \includegraphics[width=\textwidth]{assets/lasso_batched_statistics_64-256.png}
                    \caption{LASSO experiment with $M = 64, N = 256$. Plots of minimum, maximum, and median $\delta_k$ with estimated $F^*$. }
                \end{subfigure}
                \begin{subfigure}[b]{0.47\textwidth}
                    \centering
                    \includegraphics[width=\textwidth]{assets/lasso_batched_statistics_64-128.png}
                    \caption{LASSO experiment with $M = 64, N = 128$. Plots of minimum, maximum, and median $\delta_k$ with estimated $F^*$. }
                \end{subfigure}
                \caption{LASSO experiments statistics for test algorithms. }
                \label{fig:batched-lasso}
            \end{figure}
        \end{frame}
        \begin{frame}{Estimates of strong convexity constant}
            The parameters in this specific trial are $\mu = 7.432363627613958\times 10^{-18}$ and $L = 2321.737206983643$. 
            \begin{figure}[H]
                \begin{subfigure}[b]{0.47\textwidth}
                    \includegraphics[width=\textwidth]{assets/lasso_loss_256.png}
                    \caption
                    {A single run of LASSO experiment displaying $F(x_k) - F^*$ for several test algorithms.
                    }
                \end{subfigure}
                \begin{subfigure}[b]{0.47\textwidth}
                    \includegraphics[width=\textwidth]{assets/lasso_sc_estimates_256.png}
                    \caption{The $\mu_k$ estimated by FR-WAPG for one LASSO experiment. }
                \end{subfigure}
                \caption{A single LASSO experiment results, with $M = 64, N=256$.}
                \label{fig:single-lass-mu-estimates}
            \end{figure}
        \end{frame}
        \begin{frame}{Real time R-WAPG upper bound}
            Below is the plot of the real time estimates of R-WAPG convergence bound using for the LASSO experiment. 
            \begin{figure}[H]
                \centering
                \begin{subfigure}[b]{0.75\textwidth}
                    \centering
                    \includegraphics[width=\textwidth]{
                        assets/lasso_rwapg_upperbnd_256.png
                    }
                \end{subfigure}
                \caption{
                    $N = 1024$, the upper bound estimates in real time from the collected $(\alpha_k)_{k \ge 0}$ sequence. 
                }
                \label{fig:single-lass-r-wapg-rwapg-upperbnd}
            \end{figure}
        \end{frame}

\section{References and draft on arxiv}
    \begin{frame}{Acknowledgement}
        The research of HL and XW was partially supported by the NSERC Discovery Grant of Canada.
        \par
        The travel of HL to this conference is supported by the NSF grant DMS-2526465.
    \end{frame}
    \begin{frame}{Thanks for the participations. Merci pour les participations.}
        Our paper can be found on arxiv using the following QR code: 
        \begin{figure}
            \centering
            \includegraphics[width=15em]{assets/paper-qrcode.png}
        \end{figure}
        Questions?
    \end{frame}
    \begin{frame}{References}        
        \bibliography{references/R-WAPG.bib}
    \end{frame}

\end{document}