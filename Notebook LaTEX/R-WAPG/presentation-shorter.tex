\documentclass[11pt]{beamer}
\usetheme{Madrid}
\usepackage[utf8]{inputenc}
\usepackage{hyperref}
\usepackage{amsmath}
\usepackage{amsfonts}
\usepackage{amssymb}
\usepackage{graphicx}
% \DeclareMathOperator{\argmin}{argmin}
\usepackage{algorithmic}
\usepackage{algorithm}
\usepackage{wrapfig}
\usepackage{subcaption}
\usepackage{tcolorbox}
\usepackage{mathtools}


% THEMES AND BEAMER SETTINGS ===================================================
% \usetheme{Madrid}
\graphicspath{{.}}


% These are Heniz's notations. 
\newcommand{\To}{\ensuremath{\rightrightarrows}}
\newcommand{\GX}{\ensuremath{\Gamma}}
\newcommand{\mal}{\ensuremath{\mathfrak{m}}}
\newcommand{\mumu}{\ensuremath{{\mu\mu}}}
\newcommand{\paver}{\ensuremath{\mathcal{P}}}
\newcommand{\ZZZ}{\ensuremath{{X \times X^*}}}
\newcommand{\RRR}{\ensuremath{{\RR \times \RR}}}
% \newcommand{\todo}{\hookrightarrow\textsf{TO DO:}}

\newcommand{\emp}{\ensuremath{\varnothing}}
%\newcommand{\la}{\ensuremath{\langle}}
%\newcommand{\ra}{\ensuremath{\rangle}}
\newcommand{\infconv}{\ensuremath{\mbox{\small$\,\square\,$}}}
\newcommand{\pscal}{\ensuremath{\scal{\cdot}{\cdot}}}
\newcommand{\Tt}{\ensuremath{\mathfrak{T}}}
\newcommand{\YY}{\ensuremath{\mathcal Y}}
\newcommand{\XX}{\ensuremath{\mathcal X}}
\newcommand{\HH}{\ensuremath{\mathcal H}}
\newcommand{\XP}{\ensuremath{\mathcal X}^*}
% \newcommand{\st}{\ensuremath{\;|\;}} CONFLICT WITH `soul' PACKAGE.
\newcommand{\zeroun}{\ensuremath{\left]0,1\right[}}

\newcommand{\lev}[1]{\ensuremath{\mathrm{lev}_{\leq #1}\:}}
\newcommand{\moyo}[2]{\ensuremath{\sideset{_{#2}}{}{\operatorname{}}\!#1}}
\newcommand{\pair}[2]{\left\langle{{#1},{#2}}\right\rangle}
%\newcommand{\scal}[2]{\left.\left\langle{#1}\:\right| {#2}  \right\rangle}
\newcommand{\scal}[2]{\langle{{#1},{#2}}\rangle}
\newcommand{\Scal}[2]{\left\langle{{#1},{#2}}\right\rangle}
%\newcommand{\scal}[2]{\braket{ {#1},{#2}}}

\newcommand{\yosida}{\ensuremath{ \; {}^}}
\newcommand{\exi}{\ensuremath{\exists\,}}
\newcommand{\GG}{\ensuremath{\mathcal G}}
\newcommand{\RR}{\ensuremath{\mathbb R}}
\newcommand{\SSS}{\ensuremath{\mathbb S}}
\newcommand{\CC}{\ensuremath{\mathbb C}}
\newcommand{\Real}{\ensuremath{\mathrm{Re}\,}}
\newcommand{\ii}{\ensuremath{\mathrm i}}
\newcommand{\RP}{\ensuremath{\left[0,+\infty\right[}}
\newcommand{\RPX}{\ensuremath{\left[0,+\infty\right]}}
\newcommand{\RPP}{\ensuremath{\,\left]0,+\infty\right[}}
\newcommand{\RX}{\ensuremath{\,\left]-\infty,+\infty\right]}}
\newcommand{\RXX}{\ensuremath{\,\left[-\infty,+\infty\right]}}
\newcommand{\KK}{\ensuremath{\mathbb K}}
\newcommand{\NN}{\ensuremath{\mathbb N}}
\newcommand{\nnn}{\ensuremath{{n \in \NN}}}
\newcommand{\thalb}{\ensuremath{\tfrac{1}{2}}}
\newcommand{\zo}{\ensuremath{{\left]0,1\right]}}}
\newcommand{\lzo}{\ensuremath{{\lambda \in \left]0,1\right]}}}
%\newcommand{\toppsepp}{\setlength{\partopsep}{-5pt}}
\newcommand{\menge}[2]{\big\{{#1} \mid {#2}\big\}}
\newcommand{\pfrac}[2]{\ensuremath{\mathlarger{\tfrac{#1}{#2}}}}


% MATH OPERATORS ===============================================================
% \newcommand{\monos}{\ensuremath{\mathcal M}}
\newcommand{\DD}{\operatorname{dom}f}
\newcommand{\IDD}{\ensuremath{\operatorname{int}\operatorname{dom}f}}
\newcommand{\CDD}{\ensuremath{\overline{\operatorname{dom}}\,f}}
\newcommand{\clspan}{\ensuremath{\overline{\operatorname{span}}}}
\newcommand{\cone}{\ensuremath{\operatorname{cone}}}
\newcommand{\dom}{\ensuremath{\operatorname{dom}}}
\newcommand{\closu}{\ensuremath{\operatorname{cl}}}
\newcommand{\cont}{\ensuremath{\operatorname{cont}}}
\newcommand{\mons}{\ensuremath{\mathcal{A}}}
\newcommand{\gra}{\ensuremath{\operatorname{gra}}}
\newcommand{\epi}{\ensuremath{\operatorname{epi}}}
\newcommand{\prox}{\ensuremath{\operatorname{Prox}_{\mu}}}
\newcommand{\hprox}{\ensuremath{\operatorname{prox}}}
\newcommand{\intdom}{\ensuremath{\operatorname{int}\operatorname{dom}}\,}
\newcommand{\inte}{\ensuremath{\operatorname{int}}}
\newcommand{\sri}{\ensuremath{\operatorname{sri}}}
\newcommand{\reli}{\ensuremath{\operatorname{ri}}}
\newcommand{\cart}{\ensuremath{\mbox{\LARGE{$\times$}}}}


\newcommand{\average}{\ensuremath{\mathcal{R}_{\mu}({\bf A},{\boldsymbol \lambda})}}
\newcommand{\averagebar}{\ensuremath{\mathcal{R}_{1}({\bf A},\bar{\lambda})}}
\newcommand{\averageonelambda}{\ensuremath{\mathcal{R}({\bf A},{\boldsymbol \lambda})}}
\newcommand{\averageonehalf}{\ensuremath{\mathcal{R}_{1}(A,1/2)}}
\newcommand{\averageinverse}{\ensuremath{\mathcal{R}_{\mu^{-1}}({\bf A}^{-1},{\boldsymbol \lambda})}}
\newcommand{\averageoneinverse}{\ensuremath{\mathcal{R}({\bf A}^{-1},{\boldsymbol \lambda})}}
\newcommand{\averagef}{\ensuremath{\mathcal{P}_{\mu}(f,\lambda)}}
\newcommand{\averagefone}{\ensuremath{\mathcal{P}_{1}(f,\lambda)}}
\newcommand{\averagefd}{\ensuremath{\mathcal{P}_{\mu}((f_{1},\ldots, f_{n}),(\lambda_{1},\ldots, \lambda_{n}))}}
\newcommand{\averagefik}{\ensuremath{\mathcal{P}_{\mu_{k}}((f_{1,k},\ldots,f_{n,k}),
(\lambda_{1,k},\ldots,\lambda_{n,k}))}}
\newcommand{\averagesub}{\ensuremath{\mathcal{R}_{\mu}(\partial f,\lambda)}}
\newcommand{\res}{\ensuremath{\mathcal{R}_{\mu}}}
\newcommand{\resmuk}{\ensuremath{\mathcal{R}_{\mu_{k}}}}
\newcommand{\newres}{\ensuremath{\mathcal{R}}}
\newcommand{\resmualpha}{\ensuremath{\mathcal{R}_{\alpha\mu}}}
\newcommand{\averageone}{\ensuremath{\mathcal{R}_{1}}}
\newcommand{\harm}{\ensuremath{\mathcal{H}(A,\lambda)}}
\newcommand{\arithmetic}{\ensuremath{\mathcal{A}(A,\lambda)}}

\newcommand{\WC}{\ensuremath{{\mathfrak W}}}
\newcommand{\SC}{\ensuremath{{\mathfrak S}}}
\newcommand{\card}{\ensuremath{\operatorname{card}}}
\newcommand{\bd}{\ensuremath{\operatorname{bdry}}}
\newcommand{\ran}{\ensuremath{\operatorname{ran}}}
\newcommand{\rec}{\ensuremath{\operatorname{rec}}}
\newcommand{\rank}{\ensuremath{\operatorname{rank}}}
\newcommand{\kernel}{\ensuremath{\operatorname{ker}}}
\newcommand{\conv}{\ensuremath{\operatorname{conv}}}
\newcommand{\segh}{\ensuremath{\operatorname{seg}}}
\newcommand{\boxx}{\ensuremath{\operatorname{box}}}
\newcommand{\clconv}{\ensuremath{\overline{\operatorname{conv}}\,}}
\newcommand{\cldom}{\ensuremath{\overline{\operatorname{dom}}\,}}
\newcommand{\clran}{\ensuremath{\overline{\operatorname{ran}}\,}}
\newcommand{\Nf}{\ensuremath{\nabla f}}
\newcommand{\NNf}{\ensuremath{\nabla^2f}}
\newcommand{\Fix}{\ensuremath{\operatorname{Fix}}}
\newcommand{\FFix}{\ensuremath{\overline{\operatorname{Fix}}\,}}
\newcommand{\aFix}{\ensuremath{\widetilde{\operatorname{Fix}\,}}}
\newcommand{\Id}{\ensuremath{\operatorname{Id}}}
\newcommand{\Max}{\ensuremath{\operatorname{max}}}
\newcommand{\Bb}{\ensuremath{\mathfrak{B}}}
\newcommand{\BB}{\ensuremath{\mathbb{B}}}
\newcommand{\Fb}{\ensuremath{\overrightarrow{\mathfrak{B}}}}
\newcommand{\Fprox}{\ensuremath{\overrightarrow{\operatorname{prox}}}}
\newcommand{\Bprox}{\ensuremath{\overleftarrow{\operatorname{prox}}}}
\newcommand{\Bproj}{\ensuremath{\overleftarrow{\operatorname{P}}}}
\newcommand{\Ri}{\ensuremath{\mathfrak{R}_i}}
\newcommand{\Dn}{\ensuremath{\,\overset{D}{\rightarrow}\,}}
\newcommand{\nDn}{\ensuremath{\,\overset{D}{\not\rightarrow}\,}}
\newcommand{\weakly}{\ensuremath{\,\rightharpoonup}\,}
\newcommand{\weaklys}{\ensuremath{\,\overset{*}{\rightharpoonup}}\,}
\newcommand{\gr}{\ensuremath{\operatorname{gra}}}
\newcommand{\g}{\ensuremath{\,\overset{g}{\rightarrow}}\,}
\newcommand{\p}{\ensuremath{\,\overset{p}{\rightarrow}}\,}
\newcommand{\e}{\ensuremath{\,\overset{e}{\rightarrow}}\,}
\newcommand{\Tbar}{\ensuremath{\overline{T}}}
\newcommand{\n}{\ensuremath{\,\overset{n}{\rightarrow}}\,}

\newcommand{\minf}{\ensuremath{-\infty}}
\newcommand{\pinf}{\ensuremath{+\infty}}
% \renewcommand{\iff}{\ensuremath{\Leftrightarrow}}
% \renewcommand{\phi}{\ensuremath{\varphi}} % The hell who added this? 
%\newcommand{\Real}{\ensuremath{\mathrm{Re}\,}}
\newcommand{\negent}{\ensuremath{\operatorname{negent}}}
\newcommand{\neglog}{\ensuremath{\operatorname{neglog}}}
\newcommand{\halb}{\ensuremath{\tfrac{1}{2}}}
\newcommand{\bT}{\ensuremath{\mathbf{T}}}
\newcommand{\bX}{\ensuremath{\mathbf{X}}}
\newcommand{\bL}{\ensuremath{\mathbf{L}}}
\newcommand{\bD}{\ensuremath{\boldsymbol{\Delta}}}
\newcommand{\bc}{\ensuremath{\mathbf{c}}}
\newcommand{\by}{\ensuremath{\mathbf{y}}}
\newcommand{\bx}{\ensuremath{\mathbf{x}}}
\newcommand{\bA}{{\bf A}}
\newcommand{\Other}{Indeterminate }
\newcommand{\other}{indeterminate }


%%% Raf's stuff  ===============================================================
\newcommand{\al}{\alpha}
\newcommand{\la}{\lambda}
\newcommand{\La}{\Lambda}
\newcommand{\pluss}{{\hskip1pt \raise1pt\vbox{\hrule width6pt \vskip1pt
\hrule width6pt}\kern-4pt{\lower1pt\hbox{\vrule height6pt \kern1pt\vrule
height6pt}}\hskip5pt}}
\newcommand{\timess}{\star}
\newcommand{\argmax}{\mathop{\rm argmax}\limits}
\newcommand{\argmin}{\mathop{\rm argmin}\limits}
\newcommand{\product}{\langle\cdot,\cdot\rangle}
\newcommand{\im}{\mathrm{Im}}
\newcommand{\multival}{\ensuremath{X\to 2^{X^*}}}
\newcommand{\SX}{\ensuremath{2^{X^*}}}


% Hongda's Stuff: 
\newcommand{\N}{\ensuremath{\mathbb N}}
% REQURES MATH TOOLS PACKAGE
\newcommand{\defeq}{\vcentcolon=}
\newcommand{\eqdef}{=\vcentcolon}
\newcommand{\dist}{\ensuremath{\operatorname{dist}}}
\newcommand{\todoinline}[1]{\todo[inline, caption={}]{#1}} % IMPORT WANG'S LATEX CUSTOM COMMANDS. 
\setbeamertemplate{theorems}[numbered] % ADD NUMBERING TO ALL AMS THEOREMS. 
\setbeamertemplate{footline}[frame number] % ADD PAGE NUMBERS ON BOTTOM. 
\setbeamertemplate{blocks}[rounded][shadow=false]
\setbeamertemplate{navigation symbols}{} 
\setbeamertemplate{itemize subitem}{$\circ$}
% \setbeamercolor{block title}{bg=cyan,fg=black}  % CHANGE THE BLOCK STYLE IN BEAMER. 
% \setbeamercolor{block body}{bg=lime,fg=black} % CHANGE THE BLOCK STYLE IN BEAMER. 

% BIB STYLES SETTINGS ----------------------------------------------------------
\setbeamertemplate{bibliography item}{\insertbiblabel}
\setbeamerfont{bibliography item}{size=\footnotesize}
\setbeamerfont{bibliography entry author}{size=\footnotesize}
\setbeamerfont{bibliography entry title}{size=\footnotesize}
\setbeamerfont{bibliography entry location}{size=\footnotesize}
\setbeamerfont{bibliography entry note}{size=\footnotesize}
% \setbeamercovered{transparent}  % GREY OUT PAUSED FUTURE ITEMS IN SLID. 
\bibliographystyle{siam}

% --- Show a title page at the start of each section. 
% \AtBeginSection[]{
%   \begin{frame}
%   \vfill
%   \centering
%   \begin{beamercolorbox}[sep=8pt,center,shadow=true,rounded=true]{title}
%     \usebeamerfont{title}\insertsectionhead\par%
%   \end{beamercolorbox}
%   \vfill
%   \end{frame}
% }



% SLIDE INFORMATION ============================================================

\author[Hongda Li]{Hongda Li}

\title{
    Relaxed Weak Accelerated Proximal Gradient Method: a Unified Framework for Nesterov's Accelerations
}
% \newcommand{\email}{lalala@lala.la}
\institute[UBCO]{
    University of British Columbia Okanagan
    \\
    WCOM 2025
}
\date[\today]{
    \today \\ \vspace{1cm} 
    \tiny{Joint work with Shawn/Xianfu Wang}\\ 
}


\subject{Nesterov's acceleration and its applications}

% SLIDES ELEMENTS CUSTOMIZATIONS ===============================================
\theoremstyle{definition}
\newtheorem{remark}{Remark}[section]
\newtheorem{assumption}{Assumption}[section]
\newtheorem{proposition}{Proposition}[section]



% DOCUMENT STARTS ==============================================================
\begin{document}

\begin{frame}
    \titlepage
\end{frame}

\begin{frame}{ToC}
    \tableofcontents
\end{frame}

\section{Introduction and background}
    \begin{frame}{Nesterov's accelerated gradient in 1983}
        \begin{block}{Nesterov's accelerated gradient in 1983}
            Initialized $x_0 = y_0 \in \RR^n$. 
            The original formulation by Nesterov in 1983 \cite{nesterov_method_1983} has
            {\small
            \begin{align*}
                & x_{k + 1} := y_k - L^{-1}\nabla F(y_k),
                \\
                & t_{k + 1} := 1/2\left(1 + \sqrt{1 + 4t_{k}^2}\right),
                \\
                & \theta_{k + 1} := (t_{k} - 1)/t_{k + 1}, \label{eqn:example-algorithm}
                \\
                & y_{k + 1} := x_{k + 1} + \theta_{k + 1}(x_{k + 1} - x_k).
            \end{align*}    
            }
        \end{block}
        When $F$ is convex and, $L$ Lipschitz smooth. 
        If minimizer $x^+$ exists, then: 
        \begin{itemize}
            \item a convergence rate of $\mathcal O(1/k^2)$ for $(F(x_k) - F(x^+))_{k \ge 1}$;
            \item this convergence rate is optimal in the sense as proposed by Nesterov \cite{nesterov_lectures_2018}.
        \end{itemize}
    \end{frame}
    \begin{frame}{Choices of momentum parameters $t_k, \theta_k$}
        \begin{block}{Here are prominent results in the literature}
            {\footnotesize
            \begin{itemize}
                \item Beck, Teboulle 2009 \cite[Theorem 4.4]{beck_fast_2009-1}, the FISTA paper: $t_k (t_k - 1) \le t_{k - 1}^2$. 
                    \begin{itemize}
                        {\footnotesize
                        \item $(F(x_k) - F(x^+))_{k \ge 0}$ converges at rate $\mathcal O(1/k^2)$ when $t_k(t_k - 1) = t_{k - 1}^2$. 
                        \item $\theta_k$ admits no closed form. 
                        }
                    \end{itemize}
                \item Chambolle and Dossal 2015 \cite[Theorem 4.1]{chambolle_convergence_2015}: $t_k = (k + a -1)/a$ with $a > 2$. 
                    \begin{itemize}
                        {\footnotesize
                        \item It etains convergence rate $\mathcal O(1/k^2)$ with $\theta_{k} = \frac{k - 2}{k + a - 1}$. 
                        \item Iterates converge weakly in Hilbert space. 
                        }
                    \end{itemize}
                \item Nesterov \cite[Theorem 2.2.3]{nesterov_lectures_2018} and, Beck \cite[Theorem 10.42]{beck_first-order_2017}
                : $\theta_k = \left(\sqrt{L/\mu} - 1\right)\left(\sqrt{L/\mu} + 1\right)^{-1}$.
                    \begin{itemize}
                        {\footnotesize
                        \item Converges at rate $\mathcal O\left(\left(1 - \sqrt{\mu/L}\right)^k\right)$, if $f$ is also $\mu$ strongly convex with $\mu > 0$. 
                        }
                    \end{itemize}
                \item Apidopoulos et al. 2018 \cite{apidopoulos_convergence_2018}: $\theta_k = \frac{k}{k + b}$ for $b \in (0, 3)$. 
                \begin{itemize}
                    {\footnotesize
                    \item It has convergence rate $\mathcal O(1/k^{2b/3})$. 
                    }
                \end{itemize}
            \end{itemize}    
            }
        \end{block}
    \end{frame}
    \begin{frame}{A major question}
        The following question lead our investigation: 
        \begin{tcolorbox}
            How much can we relax the choice of $\theta_k, t_k$ and still get an upper bound for the convergence rate? 
        \end{tcolorbox}
        We proposed a framework called Relaxed, Weak Accelerated Proximal Gradient (R-WAPG) and, these are what we did: 
        \begin{itemize}
            \item An upper bound of function value with the weakest assumption on momentum parameter.
            \item A unified analysis with strong convexity $\mu \ge 0$. 
            \item Various equivalent representation of R-WAPG are given and, convergence results verified for examples in the literature. 
            \item Numerical experiments are conducted to verify the theory. 
        \end{itemize}
    \end{frame}

\section{The R-WAPG framework}
    \begin{frame}{The assumptions we make throughout}
        \begin{assumption}[Convex smooth plus nonsmooth]
            Let the ambient space be $\RR^n$, equipped with Euclidean inner product and norm. 
            Define $F := f + g$.
            \begin{enumerate}
                \item $f: \RR^n \rightarrow \RR$ is $L$ Lipschitz smooth and $\mu \ge 0$ strongly convex.
                \item $g: \RR^n \rightarrow \overline \RR$ is proper, closed and convex. The extended real is defined as $\overline \RR := \RR \cup \{\infty\}$.
                \item A minimizer $x^+$ exists for the optimization problem: $F^+ = \min_x \left\lbrace f(x) + g(x)\right\rbrace$.
            \end{enumerate}
        \end{assumption}
        \textbf{This assumption is the full scope of the theoretical and practical discussion. }
    \end{frame}
    \begin{frame}{Proximal gradient, gradient mapping}
        \begin{definition}[The proximal gradient operator]
            Define for all $x\in \RR^n$: 
            \begin{align*}
                T_L(y) 
                &:= \argmin_{x \in \RR^n} \left\lbrace
                    g(x) + \langle \nabla f(y), x\rangle + L/2\Vert x - y\Vert^2
                \right\rbrace 
                \\
                &= \left(I + L^{-1}\partial g\right)^{-1}
                \left(I - L^{-1}\nabla f\right)(y),
                \\
                \mathcal G_L(y)
                &:= L(y - T_L(y)).
            \end{align*}
        \end{definition}
        Note, the $I$ here is the identity operator in $\RR^n$. 
        $L > 0$ is the Lipschitz smoothness parameter for $f$. 
    \end{frame}
    \begin{frame}{The R-WAPG sequences}
        \begin{definition}[R-WAPG sequences]\label{def:rwapg-seq}
            Assume $0 \le \mu < L$. 
            Let $(\alpha_k)_{k \ge 0}, (\rho_k)_{k \ge 0}$ be such that: 
            \begin{enumerate}
                \item $\alpha_0 \in (0, 1]$, 
                \item $\alpha_k \in (\mu/L, 1) \quad (\forall k \ge 1)$, 
                \item $\rho_k := \frac{\alpha_{k + 1}^2 - (\mu/L) \alpha_{k + 1}}{(1 - \alpha_{k + 1}) \alpha_k^2}$ for all $k \ge 0$. 
            \end{enumerate}  
            We call the sequences $(\alpha_k)_{k \ge 0}, (\rho_k)_{k \ge0}$ the R-WAPG sequences. 
        \end{definition}
    \end{frame}
    \begin{frame}{R-WAPG}
        \begin{definition}[R-WAPG]\label{def:wapg}
            Choose any $x_1 \in \RR^n, v_1 \in \RR^n$.
            Let $(\alpha_k)_{k \ge0}, (\rho_k)_{k \ge 0}$ be given by Definition \ref{def:rwapg-seq}.
            The algorithm generates a sequence of vector $(y_k, x_{k + 1}, v_{k + 1})_{k \ge 1}$ for all $k\ge 1$ by such that:
            \begin{tcolorbox}\vspace{-1em}
                \begin{align*}
                    \gamma_k &\defeq \rho_{k -1}L\alpha_{k - 1}^2,
                    \\
                    L\alpha_k^2 &= (1 - \alpha_k)\gamma_k + \mu \alpha_k, 
                    \\
                    \hat \gamma_{k + 1} & \defeq L\alpha_k^2,
                    \\
                    y_k &=
                    (\gamma_k + \alpha_k \mu)^{-1}(\alpha_k \gamma_k v_k + \hat\gamma_{k + 1} x_k),
                    \\
                    v_{k + 1} &=
                    \hat\gamma^{-1}_{k + 1}
                    \left(\gamma_k(1 - \alpha_k) v_k - \alpha_k \mathcal G_L (y_k) + \mu \alpha_k y_k\right),
                    \\
                    x_{k + 1} &= T_L (y_k).
                \end{align*}
            \end{tcolorbox}
        \end{definition}
    \end{frame}

\section{Our theoretical results}
    \begin{frame}{Iterates by R-WAPG has other representations}
        \begin{proposition}[Alternative representations of the iterates]\label{prop:wapg-first-equivalent-repr}
            If the sequence $(y_k, v_k, x_k)_{k \ge 1}$ is produced by R-WAPG (Definition \ref{def:wapg}), and $\alpha_0 = 1, x_1 = v_1$. 
            Then they satisfy for all $k\ge 1$: 
            {\footnotesize
            \begin{align}
                y_{k} &=
                \left(
                    1 + \frac{L - L\alpha_{k}}{L\alpha_{k} - \mu}
                \right)^{-1}
                \left(
                    v_{k} +
                    \left(\frac{L - L\alpha_{k}}{L\alpha_{k} - \mu} \right) x_{k}
                \right)
                % \label{eqn:rwapg-first-equiv-form-eqn-1}
                \\
                x_{k + 1} 
                &=y_k - L^{-1} \mathcal G_L (y_k),
                \\
                v_{k + 1} 
                &=
                \left(
                    1 + \frac{\mu}{L \alpha_k - \mu}
                \right)^{-1}
                \left(
                    v_k +
                    \left(\frac{\mu}{L \alpha_k - \mu}\right) y_k
                \right) - \frac{1}{L\alpha_{k}}\mathcal G_L (y_k)
                \\
                &= 
                x_{k + 1} + (\alpha_k^{-1} - 1)(x_{k + 1} - x_k), 
                \\
                y_{k + 1}
                &= x_{k + 1} +
                \frac{\rho_{k}\alpha_{k}(1 - \alpha_k)}
                {\rho_k\alpha_k^2 + \alpha_k}(x_{k + 1} - x_k). 
            \end{align}
            }
        \end{proposition}
        When $\mu = 0$, then has:
        {\small
        \begin{align*}
            (\forall k \ge 1) \quad
            \frac{\rho_k\alpha_k(1 - \alpha_k)}{\rho_k\alpha_k^2 + \alpha_{k + 1}}
            & = \alpha_{k + 1}(\alpha_k^{-1} - 1).
        \end{align*}
        }
    \end{frame}
    \begin{frame}{Convergence of R-WAPG}
        After six pages of math in the paper (not necessarily dense), we deduced the following theorem: 
        \begin{proposition}[R-WAPG convergence]\label{prop:wapg-convergence}
            Fix any arbitrary $x^* \in \RR^n$.
            Let $(\alpha_k)_{k \ge 0}, (\rho_k)_{k \ge 0}$ be R-WAPG sequences.
            Let vector sequences $(y_k, v_{k}, x_{k})_{k \ge 1}$ be given by Definition \ref{def:wapg}. 
            Then for all $k \ge 1$:
            {\small
            \begin{align*}
                & F(x_{k + 1}) - F(x^*) + \frac{L \alpha_k^2}{2}\Vert v_{k + 1} - x^*\Vert^2
                \\
                &\le
                \left(
                    \prod_{i = 0}^{k - 1} \max(1, \rho_{i})
                \right)
                \left(
                    \prod_{i = 1}^{k} \left(1  - \alpha_i\right)
                \right)
                \left(
                    F(x_1) - F(x^*) + \frac{L\alpha_0^2}{2}\Vert v_1 - x^*\Vert^2
                \right).
            \end{align*}
            }
        \end{proposition}
        To use this theorem for the convergence of existing variants of Accelerated Proximal gradient method, we need alternative representations of R-WAPG to fit what commonly appears in the literature. 
    \end{frame}
    \begin{frame}{Unifying variants of FISTA in the literatures}
        Let's label 
        \begin{enumerate}
            \item[(i)] R-WAPG with $\mu \ge 0$, %\ref{def:wapg} 
            \item[(ii)] Chambolle, Dossal 2015 \cite{chambolle_convergence_2015} with $\mu \ge 0$,
            \item[(iii)] V-FISTA Beck (10.7.7) \cite{beck_first-order_2017}, with $\mu > 0$,
            \item[(iv)] R-WAPG with $\mu > 0$. %\ref{def:wapg}
        \end{enumerate}
        \begin{table}[H]
            \centering
            {\scriptsize
            \begin{tabular}{|l|l|l|l|l|}
            \hline
                Algorithm 
                & 
                $\alpha_k$ 
                & 
                $\rho_k$ 
                & 
                $F(x_k) - F^+ \le \mathcal O(\cdot)$ 
            \\ \hline
                (i) &
                $\alpha_k \in(\mu/L, 1)$ &
                $\rho_k > 0$ &
                \begin{tabular}{l}
                    $\prod_{i = 0}^{k - 1}\max(1, \rho_i)(1 - \alpha_{i + 1})$
                    \\
                    % (Proposition \ref{prop:wapg-convergence})
                \end{tabular}
            \\ \hline
                (ii) &
                $ 0 < \alpha_k^{-2} \le \alpha_{k + 1}^{-1} - \alpha_{k + 1}^{-2}$ &
                $\rho_k \ge 1$ &
                \begin{tabular}{l}
                    $\alpha_k^{2}$ \\ 
                    % (Theorem \ref{thm:r-wapg-on-cham-doss})
                \end{tabular}
            \\ \hline
                (iii) &
                $\alpha_k = \sqrt{\mu/L}$ &
                $\rho_k = 1$ &
                \begin{tabular}{l}
                    $(1 - \sqrt{\mu/L})^k$,
                    \\
                    % (Theorem \ref{thm:fixed-momentum-fista}, remark)
                \end{tabular}
            \\ \hline
                (iv) &
                $\alpha_k = \alpha \in (\mu/L, 1)$ &
                $\rho_k = \rho > 0$ &
                \begin{tabular}{l}
                    $\left(1 - \min\left(\mu/(\alpha L), \alpha\right)\right)^{k}$
                    \\
                    % (Theorem \ref{thm:fixed-momentum-fista})
                \end{tabular}
            \\ \hline
            \end{tabular}
            }
        \end{table}
        Note that in Chambolle and Dossal \cite{chambolle_convergence_2015} and their sequence $t_k = \alpha_k^{-1}$ and $\alpha_k = a/(k + 1)$ satisfies $\alpha_k^{-1} \le \alpha_{k + 1}^{-1} - \alpha_{k + 1}^{-2}$. 
    \end{frame}

\section{Adaptive momentum sequence and numerical experiments}
    \begin{frame}{Introducing numerical experiments}
        All convex functions are strongly convex with strong convexity constant $\mu = 0$. 
        With $\mu = 0$, it has from Definition \ref{def:rwapg-seq} that for all $k \ge 0$: 
        \begin{enumerate}
            \item It allows $\alpha_k \in (0, 1)$. The sequence $(\alpha_k)_{k \ge 1}$ is as loose as possible. 
            \item It has $\rho_{k - 1} = \alpha_k/((1 - \alpha_k)\alpha_{k - 1}^2)$, hence $(1 - \alpha_k)\rho_{k - 1} = \alpha_k^2/\alpha_{k - 1}^2$. 
        \end{enumerate}
        This simplifies our convergence claim into: 
        \begin{align*}
            \prod_{i = 0}^{k - 1}
            \max(1, \rho_i)(1 - \alpha_{i + 1})
            &= 
            \prod_{i = 0}^{k - 1}
            \max(1 - \alpha_{i + 1}, \rho_i(1 - \alpha_{i + 1}))
            \\
            &= 
            \prod_{i = 0}^{k - 1}
            \max\left(
            1 - \alpha_{i + 1}, \frac{\alpha_{i + 1}^2}{\alpha_i^2}\right). 
        \end{align*}
        This insipired our numerical experiments. 
    \end{frame}
    \begin{frame}{Free R-WAPG}
        \begin{algorithm}[H]
            \begin{algorithmic}
            {\footnotesize
            \STATE{\textbf{Input: } $f, g, L > \mu \ge 0, x_0 \in \RR^n, N \in \N$}
            \STATE{\textbf{Initialize: }$y_0 := x_0;L_0 := 1; \mu_0 := 1/2; \alpha_0 = 1$;}
            \STATE{\textbf{Compute: } $f(y_k)$; }
            \FOR{$k = 0, 1, 2, \cdots, N$}
                \STATE{\textbf{Compute: }$\nabla f(y_k); x^+:= [I + L_k^{-1}\partial g]^{-1}(y_k - L_k^{-1}\nabla f(y_k))$;}
                \WHILE{$L_k/2\Vert x^+ - y_k\Vert^2 < D_f(x^+, y_k)$}
                    \STATE{$L_k:= 2L_k$;}
                    \STATE{$x^+ = [I + L_k^{-1}\partial g]^{-1}(y_k - L_k^{-1}\nabla f(y_k))$; }
                \ENDWHILE
                \STATE{$x_{k + 1} := x^+$;}
                \STATE{
                    \textcolor{red}{
                        $\alpha_{k + 1} := (1/2)\left(\mu_k/L_k - \alpha_{k}^2 + \sqrt{(\mu_k/L_k - \alpha_{k}^2)^2 + 4\alpha_{k}^2}\right)$;
                    }
                }
                \STATE{$\theta_{k + 1} := \alpha_k(1 - \alpha_k)/(\alpha_k^2 + \alpha_{k + 1})$;}
                \STATE{$y_{k + 1}:= x_{k + 1} + \theta_{k + 1}(x_{k + 1} - x_k)$; }
                \STATE{\textbf{Compute: } $f(y_{k + 1})$}
                \STATE{
                    \textcolor{red}{
                        $\mu_{k + 1} := D_f(y_{k + 1}, y_{k})/\Vert y_{k + 1} - y_k\Vert^2 + (1/2)\mu_k$;
                    }
                }
            \ENDFOR
            }
            \end{algorithmic}
            \caption{Free R-WAPG}
            \label{alg:free-rwapg}
        \end{algorithm}
    \end{frame}
    \begin{frame}{A basic experiment on convex quadratic functions}
        Consider $\min_{x}\{F(x) := f(x) + 0\}$ with $f(x) = (1/2)\langle x, Ax\rangle$. 
        We are measuring: 
        \begin{align*}
            \delta_k := \log_2\left(
                \frac{F(x_k) - F^+}{F(x_0) - F^+}\right).  
        \end{align*}
        Here are the parameters. 
        \begin{enumerate}
            \item $A \in \RR^{N \times N}$ square diagonal defined by 
            \begin{align*}
                (\forall i = 1, \ldots, N)\; A_{i, i} = \begin{cases}
                    0 & i = 1
                    \\
                    \mu + \frac{(i - 1)(L - \mu)}{N - 1} & i \ge 2
                \end{cases}
            \end{align*}
            \item $L = 1, \mu = 10^{-5}$ are known in prior. 
            \item All algorithm terminates after $\Vert \mathcal G_L(y_k) \Vert \le 10^{-10}$. 
            \item Initial conditions $x_0 \sim \mathcal N(I, \mathbf 0)$, i.i.d. 
        \end{enumerate}
    \end{frame}
    \begin{frame}{Experiment results 1}
        A realization of $x_0 \sim \mathcal N(I, \mathbf 0)$ is used for V-FISTA, M-FISTA (Monotone restarted FISTA) and Free R-WAPG. 
        Below are plots of the medium, minimum, and maximum of $\delta_k$: 
        \begin{figure}[H]
            \begin{subfigure}[b]{0.47\textwidth}
                \centering
                \includegraphics[width=\textwidth]{assets/simple_regression_batched-256.png}
                \caption{$N = 256$, simple convex quadratic.}
            \end{subfigure}
            \hfill
            \begin{subfigure}[b]{0.47\textwidth}
                \centering
                \includegraphics[width=\textwidth]{assets/simple_regression_batched-1024.png}
                \caption{$N = 1024$, simple convex quadratic. }
            \end{subfigure}
            \caption{
                Statistics for experiments with simple convex quadratic for V-FISTA, M-FISTA, and R-WAPG.
            }
            \label{fig:simple-quadratic-NOG}
        \end{figure}
    \end{frame}
    \begin{frame}{$\mu$ estimations}
        Free R-WAPG estimates $\mu_k$ each iteration. 
        The value of $\mu_k$ had been recorded for one trial and this is a plot of the estimations: 
        \begin{figure}[H]
            \centering
            \begin{subfigure}[b]{0.47\textwidth}
                \centering
                \includegraphics[width=\textwidth]{assets/simple_regression_loss_sc_estimates_1024.png}
            \end{subfigure}
            \hfill
            \begin{subfigure}[b]{0.47\textwidth}
                \centering
                \includegraphics[width=\textwidth]{assets/simple_regression_loss_1024.png}
            \end{subfigure}
            \caption{
                $N = 1024$, the $\mu$ estimates produced by Algorithm \ref{alg:free-rwapg} (R-WAPG) is recorded.
            }
            \label{fig:simple-quadratic-r-wapg-mu-estimates}
        \end{figure}
    \end{frame}
    \begin{frame}{Real time R-WAPG upper bound}
        By collecting $(\alpha_k)_{k \ge 0}$ while the algorithm is running we made the following plot for the simple quadratic experiment. 
        \begin{figure}[H]
            \centering
            \begin{subfigure}[b]{0.75\textwidth}
                \centering
                \includegraphics[width=\textwidth]{
                    assets/simple_regression_rwapg_upperbnd_1024.png
                }
            \end{subfigure}
            \caption{
                $N = 1024$, the upper bound estimates in real time from the collected $(\alpha_k)_{k \ge 0}$ sequence. 
            }
            \label{fig:simple-quadratic-r-wapg-rwapg-upperbnd}
        \end{figure}
    \end{frame} 

\section{References, acknowledgement and our draft on arxiv}
    \begin{frame}{Acknowledgements}
        The research of HL and XW was partially supported by the NSERC Discovery Grant of Canada.
        \par
        We respectfully acknowledge that the UBC Okanagan academic campus is situated on the traditional, ancestral, unceded territory of the Syilx Okanagan Nation. 
    \end{frame}
    \begin{frame}{Thanks for the participation. Merci pour les participations.}
        Our paper can be found on arxiv using the following QR code: 
        \begin{figure}
            \centering
            \includegraphics[width=15em]{assets/paper-qrcode.png}
        \end{figure}
        Questions?
    \end{frame}
    \begin{frame}{References}        
        \bibliography{references/R-WAPG.bib}
    \end{frame}

\end{document}