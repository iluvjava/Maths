\documentclass[11pt]{beamer}

\usetheme{Madrid}
\usepackage[utf8]{inputenc}
\usepackage{hyperref}
\usepackage{amsmath}
\usepackage{amsfonts}
\usepackage{amssymb}
\usepackage{graphicx}
\usepackage{algorithmic}
\usepackage{algorithm}
\usepackage{wrapfig}
\usepackage{subcaption}
\graphicspath{{.}}


% These are Heniz's notations. 
\newcommand{\To}{\ensuremath{\rightrightarrows}}
\newcommand{\GX}{\ensuremath{\Gamma}}
\newcommand{\mal}{\ensuremath{\mathfrak{m}}}
\newcommand{\mumu}{\ensuremath{{\mu\mu}}}
\newcommand{\paver}{\ensuremath{\mathcal{P}}}
\newcommand{\ZZZ}{\ensuremath{{X \times X^*}}}
\newcommand{\RRR}{\ensuremath{{\RR \times \RR}}}
\newcommand{\todo}{\hookrightarrow\textsf{TO DO:}}

\newcommand{\emp}{\ensuremath{\varnothing}}
%\newcommand{\la}{\ensuremath{\langle}}
%\newcommand{\ra}{\ensuremath{\rangle}}
\newcommand{\infconv}{\ensuremath{\mbox{\small$\,\square\,$}}}
\newcommand{\pscal}{\ensuremath{\scal{\cdot}{\cdot}}}
\newcommand{\Tt}{\ensuremath{\mathfrak{T}}}
\newcommand{\YY}{\ensuremath{\mathcal Y}}
\newcommand{\XX}{\ensuremath{\mathcal X}}
\newcommand{\HH}{\ensuremath{\mathcal H}}
\newcommand{\XP}{\ensuremath{\mathcal X}^*}
\newcommand{\st}{\ensuremath{\;|\;}}
\newcommand{\zeroun}{\ensuremath{\left]0,1\right[}}

\newcommand{\lev}[1]{\ensuremath{\mathrm{lev}_{\leq #1}\:}}
\newcommand{\moyo}[2]{\ensuremath{\sideset{_{#2}}{}{\operatorname{}}\!#1}}
\newcommand{\pair}[2]{\left\langle{{#1},{#2}}\right\rangle}
%\newcommand{\scal}[2]{\left.\left\langle{#1}\:\right| {#2}  \right\rangle}
\newcommand{\scal}[2]{\langle{{#1},{#2}}\rangle}
\newcommand{\Scal}[2]{\left\langle{{#1},{#2}}\right\rangle}
%\newcommand{\scal}[2]{\braket{ {#1},{#2}}}

\newcommand{\yosida}{\ensuremath{ \; {}^}}
\newcommand{\exi}{\ensuremath{\exists\,}}
\newcommand{\GG}{\ensuremath{\mathcal G}}
\newcommand{\RR}{\ensuremath{\mathbb R}}
\newcommand{\SSS}{\ensuremath{\mathbb S}}
\newcommand{\CC}{\ensuremath{\mathbb C}}
\newcommand{\Real}{\ensuremath{\mathrm{Re}\,}}
\newcommand{\ii}{\ensuremath{\mathrm i}}
\newcommand{\RP}{\ensuremath{\left[0,+\infty\right[}}
\newcommand{\RPX}{\ensuremath{\left[0,+\infty\right]}}
\newcommand{\RPP}{\ensuremath{\,\left]0,+\infty\right[}}
\newcommand{\RX}{\ensuremath{\,\left]-\infty,+\infty\right]}}
\newcommand{\RXX}{\ensuremath{\,\left[-\infty,+\infty\right]}}
\newcommand{\KK}{\ensuremath{\mathbb K}}
\newcommand{\NN}{\ensuremath{\mathbb N}}
\newcommand{\nnn}{\ensuremath{{n \in \NN}}}
\newcommand{\thalb}{\ensuremath{\tfrac{1}{2}}}
\newcommand{\zo}{\ensuremath{{\left]0,1\right]}}}
\newcommand{\lzo}{\ensuremath{{\lambda \in \left]0,1\right]}}}
%\newcommand{\toppsepp}{\setlength{\partopsep}{-5pt}}
\newcommand{\menge}[2]{\big\{{#1} \mid {#2}\big\}}
\newcommand{\pfrac}[2]{\ensuremath{\mathlarger{\tfrac{#1}{#2}}}}


% MATH OPERATORS ===============================================================
% \newcommand{\monos}{\ensuremath{\mathcal M}}
\newcommand{\DD}{\operatorname{dom}f}
\newcommand{\IDD}{\ensuremath{\operatorname{int}\operatorname{dom}f}}
\newcommand{\CDD}{\ensuremath{\overline{\operatorname{dom}}\,f}}
\newcommand{\clspan}{\ensuremath{\overline{\operatorname{span}}}}
\newcommand{\cone}{\ensuremath{\operatorname{cone}}}
\newcommand{\dom}{\ensuremath{\operatorname{dom}}}
\newcommand{\closu}{\ensuremath{\operatorname{cl}}}
\newcommand{\cont}{\ensuremath{\operatorname{cont}}}
\newcommand{\mons}{\ensuremath{\mathcal{A}}}
\newcommand{\gra}{\ensuremath{\operatorname{gra}}}
\newcommand{\epi}{\ensuremath{\operatorname{epi}}}
\newcommand{\prox}{\ensuremath{\operatorname{Prox}_{\mu}}}
\newcommand{\hprox}{\ensuremath{\operatorname{prox}}}
\newcommand{\intdom}{\ensuremath{\operatorname{int}\operatorname{dom}}\,}
\newcommand{\inte}{\ensuremath{\operatorname{int}}}
\newcommand{\sri}{\ensuremath{\operatorname{sri}}}
\newcommand{\reli}{\ensuremath{\operatorname{ri}}}
\newcommand{\cart}{\ensuremath{\mbox{\LARGE{$\times$}}}}


\newcommand{\average}{\ensuremath{\mathcal{R}_{\mu}({\bf A},{\boldsymbol \lambda})}}
\newcommand{\averagebar}{\ensuremath{\mathcal{R}_{1}({\bf A},\bar{\lambda})}}
\newcommand{\averageonelambda}{\ensuremath{\mathcal{R}({\bf A},{\boldsymbol \lambda})}}
\newcommand{\averageonehalf}{\ensuremath{\mathcal{R}_{1}(A,1/2)}}
\newcommand{\averageinverse}{\ensuremath{\mathcal{R}_{\mu^{-1}}({\bf A}^{-1},{\boldsymbol \lambda})}}
\newcommand{\averageoneinverse}{\ensuremath{\mathcal{R}({\bf A}^{-1},{\boldsymbol \lambda})}}
\newcommand{\averagef}{\ensuremath{\mathcal{P}_{\mu}(f,\lambda)}}
\newcommand{\averagefone}{\ensuremath{\mathcal{P}_{1}(f,\lambda)}}
\newcommand{\averagefd}{\ensuremath{\mathcal{P}_{\mu}((f_{1},\ldots, f_{n}),(\lambda_{1},\ldots, \lambda_{n}))}}
\newcommand{\averagefik}{\ensuremath{\mathcal{P}_{\mu_{k}}((f_{1,k},\ldots,f_{n,k}),
(\lambda_{1,k},\ldots,\lambda_{n,k}))}}
\newcommand{\averagesub}{\ensuremath{\mathcal{R}_{\mu}(\partial f,\lambda)}}
\newcommand{\res}{\ensuremath{\mathcal{R}_{\mu}}}
\newcommand{\resmuk}{\ensuremath{\mathcal{R}_{\mu_{k}}}}
\newcommand{\newres}{\ensuremath{\mathcal{R}}}
\newcommand{\resmualpha}{\ensuremath{\mathcal{R}_{\alpha\mu}}}
\newcommand{\averageone}{\ensuremath{\mathcal{R}_{1}}}
\newcommand{\harm}{\ensuremath{\mathcal{H}(A,\lambda)}}
\newcommand{\arithmetic}{\ensuremath{\mathcal{A}(A,\lambda)}}

\newcommand{\WC}{\ensuremath{{\mathfrak W}}}
\newcommand{\SC}{\ensuremath{{\mathfrak S}}}
\newcommand{\card}{\ensuremath{\operatorname{card}}}
\newcommand{\bd}{\ensuremath{\operatorname{bdry}}}
\newcommand{\ran}{\ensuremath{\operatorname{ran}}}
\newcommand{\rec}{\ensuremath{\operatorname{rec}}}
\newcommand{\rank}{\ensuremath{\operatorname{rank}}}
\newcommand{\kernel}{\ensuremath{\operatorname{ker}}}
\newcommand{\conv}{\ensuremath{\operatorname{conv}}}
\newcommand{\segh}{\ensuremath{\operatorname{seg}}}
\newcommand{\boxx}{\ensuremath{\operatorname{box}}}
\newcommand{\clconv}{\ensuremath{\overline{\operatorname{conv}}\,}}
\newcommand{\cldom}{\ensuremath{\overline{\operatorname{dom}}\,}}
\newcommand{\clran}{\ensuremath{\overline{\operatorname{ran}}\,}}
\newcommand{\Nf}{\ensuremath{\nabla f}}
\newcommand{\NNf}{\ensuremath{\nabla^2f}}
\newcommand{\Fix}{\ensuremath{\operatorname{Fix}}}
\newcommand{\FFix}{\ensuremath{\overline{\operatorname{Fix}}\,}}
\newcommand{\aFix}{\ensuremath{\widetilde{\operatorname{Fix}\,}}}
\newcommand{\Id}{\ensuremath{\operatorname{Id}}}
\newcommand{\Max}{\ensuremath{\operatorname{max}}}
\newcommand{\Bb}{\ensuremath{\mathfrak{B}}}
\newcommand{\BB}{\ensuremath{\mathbb{B}}}
\newcommand{\Fb}{\ensuremath{\overrightarrow{\mathfrak{B}}}}
\newcommand{\Fprox}{\ensuremath{\overrightarrow{\operatorname{prox}}}}
\newcommand{\Bprox}{\ensuremath{\overleftarrow{\operatorname{prox}}}}
\newcommand{\Bproj}{\ensuremath{\overleftarrow{\operatorname{P}}}}
\newcommand{\Ri}{\ensuremath{\mathfrak{R}_i}}
\newcommand{\Dn}{\ensuremath{\,\overset{D}{\rightarrow}\,}}
\newcommand{\nDn}{\ensuremath{\,\overset{D}{\not\rightarrow}\,}}
\newcommand{\weakly}{\ensuremath{\,\rightharpoonup}\,}
\newcommand{\weaklys}{\ensuremath{\,\overset{*}{\rightharpoonup}}\,}
\newcommand{\gr}{\ensuremath{\operatorname{gra}}}
\newcommand{\g}{\ensuremath{\,\overset{g}{\rightarrow}}\,}
\newcommand{\p}{\ensuremath{\,\overset{p}{\rightarrow}}\,}
\newcommand{\e}{\ensuremath{\,\overset{e}{\rightarrow}}\,}
\newcommand{\Tbar}{\ensuremath{\overline{T}}}
\newcommand{\n}{\ensuremath{\,\overset{n}{\rightarrow}}\,}

\newcommand{\minf}{\ensuremath{-\infty}}
\newcommand{\pinf}{\ensuremath{+\infty}}
\renewcommand{\iff}{\ensuremath{\Leftrightarrow}}
% \renewcommand{\phi}{\ensuremath{\varphi}}
%\newcommand{\Real}{\ensuremath{\mathrm{Re}\,}}
\newcommand{\negent}{\ensuremath{\operatorname{negent}}}
\newcommand{\neglog}{\ensuremath{\operatorname{neglog}}}
\newcommand{\halb}{\ensuremath{\tfrac{1}{2}}}
\newcommand{\bT}{\ensuremath{\mathbf{T}}}
\newcommand{\bX}{\ensuremath{\mathbf{X}}}
\newcommand{\bL}{\ensuremath{\mathbf{L}}}
\newcommand{\bD}{\ensuremath{\boldsymbol{\Delta}}}
\newcommand{\bc}{\ensuremath{\mathbf{c}}}
\newcommand{\by}{\ensuremath{\mathbf{y}}}
\newcommand{\bx}{\ensuremath{\mathbf{x}}}
\newcommand{\bA}{{\bf A}}
\newcommand{\Other}{Indeterminate }
\newcommand{\other}{indeterminate }


%%% Raf's stuff  ===============================================================
\newcommand{\al}{\alpha}
\newcommand{\la}{\lambda}
\newcommand{\La}{\Lambda}
\newcommand{\pluss}{{\hskip1pt \raise1pt\vbox{\hrule width6pt \vskip1pt
\hrule width6pt}\kern-4pt{\lower1pt\hbox{\vrule height6pt \kern1pt\vrule
height6pt}}\hskip5pt}}
\newcommand{\timess}{\star}
\newcommand{\argmax}{\mathop{\rm argmax}\limits}
\newcommand{\argmin}{\mathop{\rm argmin}\limits}
\newcommand{\product}{\langle\cdot,\cdot\rangle}
\newcommand{\im}{\mathrm{Im}}
\newcommand{\multival}{\ensuremath{X\to 2^{X^*}}}
\newcommand{\SX}{\ensuremath{2^{X^*}}}

\author{Hongda Li}
\title[Thesis Proposal Talk]{First Order Nonsmooth Optimization: Catalyst Acceleration and Unifying Nesterov's Acceleration}
\newcommand{\email}{lalala@lala.la}
\setbeamercovered{transparent}
\setbeamertemplate{navigation symbols}{}
%\logo{}
\institute[UBCO]{
    University of British Columbia Okanagan
}
\date{\today}
\subject{Nesterov's acceleration and its applications}

% ---------------------------------------------------------
% Selecione um estilo de referência
\bibliographystyle{IEEEtran}

%\bibliographystyle{abbrv}
%\setbeamertemplate{bibliography item}{\insertbiblabel}
% ---------------------------------------------------------

% ---------------------------------------------------------
\newtheorem{remark}{Remark}
\newtheorem{assumption}{Assumption}

\begin{document}

\begin{frame}
    \titlepage
\end{frame}
\begin{frame}{Overview}
    This talk will be based on the content of our draft paper and selected content of the Catalyst Meta Acceleration Framework. 
    Our preprint: 
    \begin{enumerate}
        \item X. Wang and H. Li, \textit{A Parameter Free Accelerated Proximal Gradient Method Without Restarting}, preprint, (2025).
    \end{enumerate}
    Catalyst Meta Acceleration:  
    \begin{enumerate}
        \item H. Lin, J. Mairal and Z. Harchaoui, \textit{A universal catalyst for first-order optimization}, in NISP, vol. 28, (2015). 
        \item \underline{\hspace{4em}}, \textit{Catalyst acceleration for first-order convex optimization: from theory to practice}, JMLR, 18 (2018), pp. 1–54.
    \end{enumerate}
\end{frame}
\begin{frame}{ToC}
    \tableofcontents
\end{frame}


\section{Introduction}
    \subsection{Notations and preliminaries}
        \begin{frame}{Notations and preliminaries}
            Throughout this talk, let $\RR^n$ be the ambient space equiped with Euclidean inner product and norm. 
            We consider 
            \begin{align}\label{eqn:additive-comp-obj}
                \min_{x \in \RR^n} \left\lbrace
                    F(x):= f(x) + g(x)
                \right\rbrace.
            \end{align}
            Unless specified, assume: 
            \begin{enumerate}
                \item $f:\RR^n \rightarrow \RR$ is $L$-Lipscthiz smooth $\mu \ge 0$ strongly convex, 
                \item $g:\RR^n \rightarrow \overline \RR$ is closed convex proper. 
            \end{enumerate}
        \end{frame}
        \begin{frame}{Notations and preliminaries}
            \begin{definition}[Proximal gradient operator]\label{def:proximal-gradient-operator}
                Define the proximal gradient operator $T_L$ on all $y \in \RR^n$: 
                \begin{align*}
                    T_L y := \argmin_{x \in \RR^n} \left\lbrace
                        g(x) + f(y) + \langle \nabla f(y), x - y\rangle 
                        + \frac{L}{2}\Vert x- y \Vert^2
                    \right\rbrace. 
                \end{align*}
            \end{definition}
            \begin{definition}[Gradient mapping operator]\label{def:gradient-mapping-operator}
                % Take $F := f + g$ as defined in this section. 
                Define the gradient mapping operator $\mathcal G_L$ on all $y \in \RR^n$: 
                \begin{align*}
                    \mathcal G_L (y):= L(y - T_L y). 
                \end{align*}
            \end{definition}
        \end{frame}
        \begin{frame}{Proximal gradient inequality}
            \begin{lemma}[The proximal gradient inequality]\label{thm:prox-grad-ineq}
                % Take $F:= f + g$ as defined in this section. 
                For all $y \in \RR^n$, $x \in \RR^n$, it has: 
                {\scriptsize
                \begin{align*}
                    (\forall x \in \RR^n)\quad 
                    F(x)  - F(T_Ly) - \langle L(y - T_Ly), x - y\rangle
                    - \frac{\mu}{2}\Vert x - y\Vert^2 - \frac{L}{2}\Vert y - T_Ly\Vert^2 
                    &\ge 0. 
                \end{align*}
                }
            \end{lemma}
            This lemma is crucial to developing results in our current draft paper. 
        \end{frame}
        \begin{frame}{Nesterov's estimating sequence example}
            \begin{definition}[Nesterov's estimating sequence]\label{def:nes-est-seq}
                For all $k \ge 0$, let $\phi_k : \RR^n \rightarrow\RR$ be a sequence of functions. 
                We call this sequence of functions a Nesterov's estimating sequence when it satisfies conditions: 
                \begin{enumerate}
                    \item There exists another sequence $(x_k)_{k \ge 0}$ such that for all $k \ge 0$ it has $F(x_k) \le \phi_k^*: =\min_{x}\phi_k(x)$. 
                    \item There exists a sequence of $(\alpha_k)_{k \ge 0}$ where $\alpha_k \in (0, 1)\; \forall k \ge0 $ such that for all $x \in \RR^n$ it has $\phi_{k + 1}(x) - \phi_k(x) \le - \alpha_k(\phi_k(x) - F(x))$. 
                \end{enumerate}
            \end{definition}
            The technique is widespread in the literatures and it's used to derive the convergence rate of acceleration on first order method, and the numerical algorithm itself. It is a two birds one stone technique. 
        \end{frame}
        \begin{frame}{Our works on R-WAPG}
            Here are contributions of our draft paper. 
            Recall the Nesterov's acceleration has momentum extrapolation updates on $y_{k + 1} = x_{k + 1} + \theta_{k + 1}(x_{k + 1} - x_k)$. 
            We proposed the idea of R-WAPG, a generic method that: 
            \begin{enumerate}
                \item Describe for momentum sequences that doesn't follow Nesterov's rules.
                \item Unifies the convergence rate analysis for several Euclidean variants of the FISTA method. 
                \item A parameter free numerical algorithm: ``Free R-WAPG'' method that has competitive numerical performance in practical settings without restarting. 
            \end{enumerate}
            Our work is inspired by considering Nesterov's estimating sequence where $F(x_k) + R_k = \phi_k^*$. 
        \end{frame}
        \begin{frame}{Introduction to Catalyst Acceleration}
            
        \end{frame}
\section{Content of the draft paper}

    \subsection{Direction of future works}

\section{Selected contents from Catalyst Meta Accelerations}
    \subsection{Direction of future works}


\section{References}
    \begin{frame}{Citation examples}
        Citation examples \cite{chambolle_convergence_2015}
    \end{frame}
    \begin{frame}[allowframebreaks]{References}
        % \bibliographystyle{apalike}
        \bibliography{references/proposal.bib}
    \end{frame}

\end{document}