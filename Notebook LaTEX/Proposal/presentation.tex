\documentclass[11pt]{beamer}
% PACKAGES =====================================================================
\usepackage[utf8]{inputenc}
\usepackage{hyperref}
\usepackage{amsmath}
\usepackage{amsfonts}
\usepackage{amssymb}
\usepackage{graphicx}
\usepackage{algorithmic}
\usepackage{algorithm}
\usepackage{wrapfig}
\usepackage{subcaption}
\usepackage{tcolorbox}

% THEMES AND BEAMER SETTINGS ===================================================
% \usetheme{Madrid}
\graphicspath{{.}}


% These are Heniz's notations. 
\newcommand{\To}{\ensuremath{\rightrightarrows}}
\newcommand{\GX}{\ensuremath{\Gamma}}
\newcommand{\mal}{\ensuremath{\mathfrak{m}}}
\newcommand{\mumu}{\ensuremath{{\mu\mu}}}
\newcommand{\paver}{\ensuremath{\mathcal{P}}}
\newcommand{\ZZZ}{\ensuremath{{X \times X^*}}}
\newcommand{\RRR}{\ensuremath{{\RR \times \RR}}}
\newcommand{\todo}{\hookrightarrow\textsf{TO DO:}}

\newcommand{\emp}{\ensuremath{\varnothing}}
%\newcommand{\la}{\ensuremath{\langle}}
%\newcommand{\ra}{\ensuremath{\rangle}}
\newcommand{\infconv}{\ensuremath{\mbox{\small$\,\square\,$}}}
\newcommand{\pscal}{\ensuremath{\scal{\cdot}{\cdot}}}
\newcommand{\Tt}{\ensuremath{\mathfrak{T}}}
\newcommand{\YY}{\ensuremath{\mathcal Y}}
\newcommand{\XX}{\ensuremath{\mathcal X}}
\newcommand{\HH}{\ensuremath{\mathcal H}}
\newcommand{\XP}{\ensuremath{\mathcal X}^*}
\newcommand{\st}{\ensuremath{\;|\;}}
\newcommand{\zeroun}{\ensuremath{\left]0,1\right[}}

\newcommand{\lev}[1]{\ensuremath{\mathrm{lev}_{\leq #1}\:}}
\newcommand{\moyo}[2]{\ensuremath{\sideset{_{#2}}{}{\operatorname{}}\!#1}}
\newcommand{\pair}[2]{\left\langle{{#1},{#2}}\right\rangle}
%\newcommand{\scal}[2]{\left.\left\langle{#1}\:\right| {#2}  \right\rangle}
\newcommand{\scal}[2]{\langle{{#1},{#2}}\rangle}
\newcommand{\Scal}[2]{\left\langle{{#1},{#2}}\right\rangle}
%\newcommand{\scal}[2]{\braket{ {#1},{#2}}}

\newcommand{\yosida}{\ensuremath{ \; {}^}}
\newcommand{\exi}{\ensuremath{\exists\,}}
\newcommand{\GG}{\ensuremath{\mathcal G}}
\newcommand{\RR}{\ensuremath{\mathbb R}}
\newcommand{\SSS}{\ensuremath{\mathbb S}}
\newcommand{\CC}{\ensuremath{\mathbb C}}
\newcommand{\Real}{\ensuremath{\mathrm{Re}\,}}
\newcommand{\ii}{\ensuremath{\mathrm i}}
\newcommand{\RP}{\ensuremath{\left[0,+\infty\right[}}
\newcommand{\RPX}{\ensuremath{\left[0,+\infty\right]}}
\newcommand{\RPP}{\ensuremath{\,\left]0,+\infty\right[}}
\newcommand{\RX}{\ensuremath{\,\left]-\infty,+\infty\right]}}
\newcommand{\RXX}{\ensuremath{\,\left[-\infty,+\infty\right]}}
\newcommand{\KK}{\ensuremath{\mathbb K}}
\newcommand{\NN}{\ensuremath{\mathbb N}}
\newcommand{\nnn}{\ensuremath{{n \in \NN}}}
\newcommand{\thalb}{\ensuremath{\tfrac{1}{2}}}
\newcommand{\zo}{\ensuremath{{\left]0,1\right]}}}
\newcommand{\lzo}{\ensuremath{{\lambda \in \left]0,1\right]}}}
%\newcommand{\toppsepp}{\setlength{\partopsep}{-5pt}}
\newcommand{\menge}[2]{\big\{{#1} \mid {#2}\big\}}
\newcommand{\pfrac}[2]{\ensuremath{\mathlarger{\tfrac{#1}{#2}}}}


% MATH OPERATORS ===============================================================
% \newcommand{\monos}{\ensuremath{\mathcal M}}
\newcommand{\DD}{\operatorname{dom}f}
\newcommand{\IDD}{\ensuremath{\operatorname{int}\operatorname{dom}f}}
\newcommand{\CDD}{\ensuremath{\overline{\operatorname{dom}}\,f}}
\newcommand{\clspan}{\ensuremath{\overline{\operatorname{span}}}}
\newcommand{\cone}{\ensuremath{\operatorname{cone}}}
\newcommand{\dom}{\ensuremath{\operatorname{dom}}}
\newcommand{\closu}{\ensuremath{\operatorname{cl}}}
\newcommand{\cont}{\ensuremath{\operatorname{cont}}}
\newcommand{\mons}{\ensuremath{\mathcal{A}}}
\newcommand{\gra}{\ensuremath{\operatorname{gra}}}
\newcommand{\epi}{\ensuremath{\operatorname{epi}}}
\newcommand{\prox}{\ensuremath{\operatorname{Prox}_{\mu}}}
\newcommand{\hprox}{\ensuremath{\operatorname{prox}}}
\newcommand{\intdom}{\ensuremath{\operatorname{int}\operatorname{dom}}\,}
\newcommand{\inte}{\ensuremath{\operatorname{int}}}
\newcommand{\sri}{\ensuremath{\operatorname{sri}}}
\newcommand{\reli}{\ensuremath{\operatorname{ri}}}
\newcommand{\cart}{\ensuremath{\mbox{\LARGE{$\times$}}}}


\newcommand{\average}{\ensuremath{\mathcal{R}_{\mu}({\bf A},{\boldsymbol \lambda})}}
\newcommand{\averagebar}{\ensuremath{\mathcal{R}_{1}({\bf A},\bar{\lambda})}}
\newcommand{\averageonelambda}{\ensuremath{\mathcal{R}({\bf A},{\boldsymbol \lambda})}}
\newcommand{\averageonehalf}{\ensuremath{\mathcal{R}_{1}(A,1/2)}}
\newcommand{\averageinverse}{\ensuremath{\mathcal{R}_{\mu^{-1}}({\bf A}^{-1},{\boldsymbol \lambda})}}
\newcommand{\averageoneinverse}{\ensuremath{\mathcal{R}({\bf A}^{-1},{\boldsymbol \lambda})}}
\newcommand{\averagef}{\ensuremath{\mathcal{P}_{\mu}(f,\lambda)}}
\newcommand{\averagefone}{\ensuremath{\mathcal{P}_{1}(f,\lambda)}}
\newcommand{\averagefd}{\ensuremath{\mathcal{P}_{\mu}((f_{1},\ldots, f_{n}),(\lambda_{1},\ldots, \lambda_{n}))}}
\newcommand{\averagefik}{\ensuremath{\mathcal{P}_{\mu_{k}}((f_{1,k},\ldots,f_{n,k}),
(\lambda_{1,k},\ldots,\lambda_{n,k}))}}
\newcommand{\averagesub}{\ensuremath{\mathcal{R}_{\mu}(\partial f,\lambda)}}
\newcommand{\res}{\ensuremath{\mathcal{R}_{\mu}}}
\newcommand{\resmuk}{\ensuremath{\mathcal{R}_{\mu_{k}}}}
\newcommand{\newres}{\ensuremath{\mathcal{R}}}
\newcommand{\resmualpha}{\ensuremath{\mathcal{R}_{\alpha\mu}}}
\newcommand{\averageone}{\ensuremath{\mathcal{R}_{1}}}
\newcommand{\harm}{\ensuremath{\mathcal{H}(A,\lambda)}}
\newcommand{\arithmetic}{\ensuremath{\mathcal{A}(A,\lambda)}}

\newcommand{\WC}{\ensuremath{{\mathfrak W}}}
\newcommand{\SC}{\ensuremath{{\mathfrak S}}}
\newcommand{\card}{\ensuremath{\operatorname{card}}}
\newcommand{\bd}{\ensuremath{\operatorname{bdry}}}
\newcommand{\ran}{\ensuremath{\operatorname{ran}}}
\newcommand{\rec}{\ensuremath{\operatorname{rec}}}
\newcommand{\rank}{\ensuremath{\operatorname{rank}}}
\newcommand{\kernel}{\ensuremath{\operatorname{ker}}}
\newcommand{\conv}{\ensuremath{\operatorname{conv}}}
\newcommand{\segh}{\ensuremath{\operatorname{seg}}}
\newcommand{\boxx}{\ensuremath{\operatorname{box}}}
\newcommand{\clconv}{\ensuremath{\overline{\operatorname{conv}}\,}}
\newcommand{\cldom}{\ensuremath{\overline{\operatorname{dom}}\,}}
\newcommand{\clran}{\ensuremath{\overline{\operatorname{ran}}\,}}
\newcommand{\Nf}{\ensuremath{\nabla f}}
\newcommand{\NNf}{\ensuremath{\nabla^2f}}
\newcommand{\Fix}{\ensuremath{\operatorname{Fix}}}
\newcommand{\FFix}{\ensuremath{\overline{\operatorname{Fix}}\,}}
\newcommand{\aFix}{\ensuremath{\widetilde{\operatorname{Fix}\,}}}
\newcommand{\Id}{\ensuremath{\operatorname{Id}}}
\newcommand{\Max}{\ensuremath{\operatorname{max}}}
\newcommand{\Bb}{\ensuremath{\mathfrak{B}}}
\newcommand{\BB}{\ensuremath{\mathbb{B}}}
\newcommand{\Fb}{\ensuremath{\overrightarrow{\mathfrak{B}}}}
\newcommand{\Fprox}{\ensuremath{\overrightarrow{\operatorname{prox}}}}
\newcommand{\Bprox}{\ensuremath{\overleftarrow{\operatorname{prox}}}}
\newcommand{\Bproj}{\ensuremath{\overleftarrow{\operatorname{P}}}}
\newcommand{\Ri}{\ensuremath{\mathfrak{R}_i}}
\newcommand{\Dn}{\ensuremath{\,\overset{D}{\rightarrow}\,}}
\newcommand{\nDn}{\ensuremath{\,\overset{D}{\not\rightarrow}\,}}
\newcommand{\weakly}{\ensuremath{\,\rightharpoonup}\,}
\newcommand{\weaklys}{\ensuremath{\,\overset{*}{\rightharpoonup}}\,}
\newcommand{\gr}{\ensuremath{\operatorname{gra}}}
\newcommand{\g}{\ensuremath{\,\overset{g}{\rightarrow}}\,}
\newcommand{\p}{\ensuremath{\,\overset{p}{\rightarrow}}\,}
\newcommand{\e}{\ensuremath{\,\overset{e}{\rightarrow}}\,}
\newcommand{\Tbar}{\ensuremath{\overline{T}}}
\newcommand{\n}{\ensuremath{\,\overset{n}{\rightarrow}}\,}

\newcommand{\minf}{\ensuremath{-\infty}}
\newcommand{\pinf}{\ensuremath{+\infty}}
\renewcommand{\iff}{\ensuremath{\Leftrightarrow}}
% \renewcommand{\phi}{\ensuremath{\varphi}}
%\newcommand{\Real}{\ensuremath{\mathrm{Re}\,}}
\newcommand{\negent}{\ensuremath{\operatorname{negent}}}
\newcommand{\neglog}{\ensuremath{\operatorname{neglog}}}
\newcommand{\halb}{\ensuremath{\tfrac{1}{2}}}
\newcommand{\bT}{\ensuremath{\mathbf{T}}}
\newcommand{\bX}{\ensuremath{\mathbf{X}}}
\newcommand{\bL}{\ensuremath{\mathbf{L}}}
\newcommand{\bD}{\ensuremath{\boldsymbol{\Delta}}}
\newcommand{\bc}{\ensuremath{\mathbf{c}}}
\newcommand{\by}{\ensuremath{\mathbf{y}}}
\newcommand{\bx}{\ensuremath{\mathbf{x}}}
\newcommand{\bA}{{\bf A}}
\newcommand{\Other}{Indeterminate }
\newcommand{\other}{indeterminate }


%%% Raf's stuff  ===============================================================
\newcommand{\al}{\alpha}
\newcommand{\la}{\lambda}
\newcommand{\La}{\Lambda}
\newcommand{\pluss}{{\hskip1pt \raise1pt\vbox{\hrule width6pt \vskip1pt
\hrule width6pt}\kern-4pt{\lower1pt\hbox{\vrule height6pt \kern1pt\vrule
height6pt}}\hskip5pt}}
\newcommand{\timess}{\star}
\newcommand{\argmax}{\mathop{\rm argmax}\limits}
\newcommand{\argmin}{\mathop{\rm argmin}\limits}
\newcommand{\product}{\langle\cdot,\cdot\rangle}
\newcommand{\im}{\mathrm{Im}}
\newcommand{\multival}{\ensuremath{X\to 2^{X^*}}}
\newcommand{\SX}{\ensuremath{2^{X^*}}} % IMPORT WANG'S LATEX CUSTOM COMMANDS. 
\setbeamertemplate{theorems}[numbered] % ADD NUMBERING TO ALL AMS THEOREMS. 
\setbeamertemplate{footline}[frame number] % ADD PAGE NUMBERS ON BOTTOM. 
% \setbeamercovered{transparent}  % GREY OUT PAUSED FUTURE ITEMS IN SLID. 
\setbeamertemplate{navigation symbols}{} 


% SLIDE INFORMATION ============================================================
\author{Hongda Li}
\title[Thesis Proposal Talk]{First Order Nonsmooth Optimization: Catalyst Acceleration and Unifying Nesterov's Acceleration}
% \newcommand{\email}{lalala@lala.la}
\institute[UBCO]{
    University of British Columbia Okanagan
}
\date{\today}
\subject{Nesterov's acceleration and its applications}


% SLIDES ELEMENTS CUSTOMIZATIONS ===============================================
\theoremstyle{definition}
\newtheorem{remark}{Remark}[section]
\newtheorem{assumption}{Assumption}[section]
\newtheorem{proposition}{Proposition}[section]
\bibliographystyle{siam}


% DOCUMENTS STARTS =============================================================
\begin{document}
\begin{frame}
    \titlepage
\end{frame}
\begin{frame}{Overview}
    This talk will be based on the content of our draft paper and selected content of the Catalyst Meta Acceleration Framework. 
    Our preprint: 
    \begin{enumerate}
        \item X. Wang and H. Li, \textit{A Parameter Free Accelerated Proximal Gradient Method Without Restarting}, preprint, (2025).
    \end{enumerate}
    Catalyst Meta Acceleration:  
    \begin{enumerate}
        \item H. Lin, J. Mairal and Z. Harchaoui, \textit{A universal catalyst for first-order optimization}, in NISP, vol. 28, (2015). 
        \item \underline{\hspace{4em}}, \textit{Catalyst acceleration for first-order convex optimization: from theory to practice}, JMLR, 18 (2018), pp. 1–54.
    \end{enumerate}
\end{frame}
\begin{frame}{ToC}
    \tableofcontents
\end{frame}


\section{Introduction}
    \subsection{Notations and preliminaries}
        \begin{frame}{Notations and preliminaries}
            Throughout this talk, let $\RR^n$ be the ambient space equiped with Euclidean inner product and norm. 
            We consider 
            \begin{align}\label{eqn:additive-comp-obj}
                \min_{x \in \RR^n} \left\lbrace
                    F(x):= f(x) + g(x)
                \right\rbrace.
            \end{align}
            Unless specified, assume: 
            \begin{enumerate}
                \item $f:\RR^n \rightarrow \RR$ is $L$-Lipscthiz smooth $\mu \ge 0$ strongly convex, 
                \item $g:\RR^n \rightarrow \overline \RR$ is closed convex proper. 
            \end{enumerate}
        \end{frame}
        \begin{frame}{Notations and preliminaries}
            \begin{definition}[Proximal gradient operator]\label{def:proximal-gradient-operator}
                Define the proximal gradient operator $T_L$ on all $y \in \RR^n$: 
                \begin{align*}
                    T_L y := \argmin_{x \in \RR^n} \left\lbrace
                        g(x) + f(y) + \langle \nabla f(y), x - y\rangle 
                        + \frac{L}{2}\Vert x- y \Vert^2
                    \right\rbrace. 
                \end{align*}
            \end{definition}
            \begin{definition}[Gradient mapping operator]\label{def:gradient-mapping-operator}
                % Take $F := f + g$ as defined in this section. 
                Define the gradient mapping operator $\mathcal G_L$ on all $y \in \RR^n$: 
                \begin{align*}
                    \mathcal G_L (y):= L(y - T_L y). 
                \end{align*}
            \end{definition}
        \end{frame}
        \begin{frame}{Proximal gradient inequality}
            \begin{lemma}[The proximal gradient inequality]\label{thm:prox-grad-ineq}
                % Take $F:= f + g$ as defined in this section. 
                For all $y \in \RR^n$, $x \in \RR^n$, it has: 
                {\scriptsize
                \begin{align*}
                    F(x)  - F(T_Ly) - \langle L(y - T_Ly), x - y\rangle
                    - \frac{\mu}{2}\Vert x - y\Vert^2 - \frac{L}{2}\Vert y - T_Ly\Vert^2 
                    &\ge 0. 
                \end{align*}
                }
            \end{lemma}
            This lemma is crucial to developing results in our current draft paper. 
        \end{frame}
        \begin{frame}{Nesterov's estimating sequence example}
            \begin{definition}[Nesterov's estimating sequence]\label{def:nes-est-seq}
                For all $k \ge 0$, let $\phi_k : \RR^n \rightarrow\RR$ be a sequence of functions. 
                We call this sequence of functions a Nesterov's estimating sequence when it satisfies conditions: 
                \begin{enumerate}
                    \item There exists another sequence $(x_k)_{k \ge 0}$ such that for all $k \ge 0$ it has $F(x_k) \le \phi_k^*: =\min_{x}\phi_k(x)$. 
                    \item There exists a sequence of $(\alpha_k)_{k \ge 0}$ where $\alpha_k \in (0, 1)\; \forall k \ge0 $ such that for all $x \in \RR^n$ it has $\phi_{k + 1}(x) - \phi_k(x) \le - \alpha_k(\phi_k(x) - F(x))$. 
                \end{enumerate}
            \end{definition}
            The technique is widespread in the literatures and it's used to derive the convergence rate of acceleration on first order method, and the numerical algorithm itself. It is a two birds one stone technique. 
        \end{frame}
        \begin{frame}{Our works on R-WAPG}
            Here are the contributions of our draft paper. 
            Recall the Nesterov's acceleration has momentum extrapolation updates on $y_{k + 1} = x_{k + 1} + \theta_{k + 1}(x_{k + 1} - x_k)$. 
            We proposed the idea of R-WAPG, a generic method that: 
            \begin{enumerate}
                \item Describe for momentum sequences that doesn't follow Nesterov's rules.
                \item Unifies the convergence rate analysis for several Euclidean variants of the FISTA method. 
                \item A parameter free numerical algorithm: ``Free R-WAPG'' method that has competitive numerical performance in practical settings without restarting. 
            \end{enumerate}
            Our work is inspired by considering Nesterov's estimating sequence where $F(x_k) + R_k = \phi_k^*$. 
        \end{frame}
        \begin{frame}{Introducing Catalyst Part I}
            \begin{block}{Introducing Catalyst}
                Let $F:\RR \rightarrow \overline \RR$ be $\mu \ge 0$ strongly convex and closed. 
                Let the initial estimate be $x_0 \in \RR^n$, fix parameters $\kappa > 0$ and $\alpha_0 \in (0, 1]$. 
                {\small 
                \begin{tcolorbox}
                    Initialize $x_0 = y_0$. Then the algorithm generates $(x_k, y_k)_{k\ge 0}$ for all $k \ge 1$ such that: 
                    \begin{align*}
                        & \text{find } x_k \approx \argmin_{x \in \RR^n} \left\lbrace F(x) + (\kappa/2)\Vert x - y_{k - 1}\Vert^2\right\rbrace, 
                        \\
                        & \text{find } \alpha_k \in (0, 1) \text{ such that } \alpha_k^2 = (1 - \alpha_k)\alpha_{k - 1}^2 + (\mu/(\mu + \kappa))\alpha_k,
                        \\
                        & 
                        y_{k} = x_k + \frac{\alpha_{k - 1}(1 - \alpha_{k - 1})}{\alpha_{k - 1}^2 + \alpha_k}(x_k - x_{k - 1}). 
                    \end{align*}
                \end{tcolorbox}
                }
            \end{block}
            We will return to this in the later slides. 
        \end{frame}
        \begin{frame}{Introducing Catalyst Part II}
            Catalyst by Lin, et al. \cite{lin_catalyst_2018, lin_universal_2015} has the theoretical and pratical importance: 
            \begin{enumerate}
                \item It's an early attempt at putting accelerated inexact proximal point method into a practical settings. 
                \item It finds application in machine learning and it accelerates the convergence of Varianced Reduced Method (A type of incremental method that is not slower than the exact counter part). 
                \item It demonstrates crucial ideas on how prove convergence rate where the evaluation of proximal point method is inexact in the convex settings. 
            \end{enumerate}
        \end{frame}

\section{Content of the draft paper}
    \subsection{The method of R-WAPG and its convergence}
        \begin{frame}{R-WAPG sequences}
            \begin{definition}[R-WAPG sequences]\label{def:rwapg-seq}
                Assume $0 \le \mu < L$. 
                The sequences $(\alpha_k)_{k \ge 0}, (\rho_k)_{k \ge 0}$ are valid for R-WAPG if all the following holds: 
                \begin{align*}
                    \alpha_0 &\in (0, 1], 
                    \\
                    \alpha_k &\in (\mu/L, 1) \quad (\forall k \ge 1), 
                    \\
                    \rho_k &:= \frac{\alpha_{k + 1}^2 - (\mu/L)\alpha_{k + 1}}{(1 - \alpha_{k + 1})\alpha_k^2} \quad \forall (k \ge 0). 
                \end{align*}
                We call $(\alpha_k)_{k \ge 0}, (\rho_k)_{k \ge 0}$ the \textbf{R-WAPG Sequences}. 
            \end{definition}
        \end{frame}
        \begin{frame}{The method of R-WAPG}
            \begin{definition}[Relaxed weak accelerated proximal gradient (R-WAPG)]\label{def:wapg}
                Choose any $x_1 \in \RR^n, v_1 \in \RR^n$. 
                Let $(\alpha_k)_{k \ge0}, (\rho_k)_{k \ge 0}$ be given by Definition \ref{def:rwapg-seq}. 
                The algorithm generates a sequence of vector $(y_k, x_{k + 1}, v_{k + 1})_{k \ge 1}$ for $k\ge 1$ by the procedures:  
                \begin{tcolorbox}
                    For $k=1, 2, 3, \ldots$
                    \begin{align*}
                        \gamma_k &:= \rho_{k -1}L\alpha_{k - 1}^2, 
                        \\
                        \hat \gamma_{k + 1} & := (1 - \alpha_k)\gamma_k + \mu \alpha_k = L\alpha_k^2, 
                        \\
                        y_k &= 
                        (\gamma_k + \alpha_k \mu)^{-1}(\alpha_k \gamma_k v_k + \hat\gamma_{k + 1} x_k), 
                        \\
                        g_k &= \mathcal G_L y_k, 
                        \\
                        v_{k + 1} &= 
                        \hat\gamma^{-1}_{k + 1}
                        (\gamma_k(1 - \alpha_k) v_k - \alpha_k g_k + \mu \alpha_k y_k), 
                        \\
                        x_{k + 1} &= T_L y_k. 
                    \end{align*}    
                \end{tcolorbox}
            \end{definition}
        \end{frame}
        \begin{frame}{Convergence of R-WAPG}
            The convergence claim of the method follows. 
            \begin{proposition}[R-WAPG convergence claim]\label{prop:wagp-convergence}
                Fix any arbitrary $x^* \in \RR^n, N \in \mathbb N$. 
                Let vector sequence $(y_k, v_{k}, x_{k})_{k \ge 1}$ and R-WAPG sequences $\alpha_k, \rho_k$ be given by Definition \ref{def:wapg}. 
                Define $R_1 = 0$ and suppose that for $k = 1, 2, \ldots, N$, we have $R_k$ recursively given by: 
                {\scriptsize
                \begin{align*}
                    R_{k + 1}
                    := 
                    \frac{1}{2}\left(
                        L^{-1} - \frac{\alpha_k^2}{\hat \gamma_{k + 1}}
                    \right)\Vert g_k\Vert^2
                    + 
                    (1 - \alpha_k)
                    \left(
                        \epsilon_k + R_k + 
                        \frac{\mu\alpha_k\gamma_k}{2\hat \gamma_{k + 1}}
                        \Vert v_k - y_k\Vert^2
                    \right). 
                \end{align*}
                }
                Then for all $k = 1, 2, \ldots, N$: 
                {\scriptsize
                \begin{align*}
                    & F(x_{k + 1}) - F(x^*) + \frac{L \alpha_k^2}{2}\Vert v_{k + 1} - x^*\Vert^2
                    \\
                    &\le 
                    \left(
                        \prod_{i = 0}^{k - 1} \max(1, \rho_{i})
                    \right)
                    \left(
                        \prod_{i = 1}^{k} \left(1  - \alpha_i\right)
                    \right)
                    \left(
                        F(x_1) - F(x^*) + \frac{L\alpha_0^2}{2}\Vert v_1 - x^*\Vert^2
                    \right). 
                \end{align*}
                }
            \end{proposition}
        \end{frame}
    \subsection{Equivalent forms of R-WAPG}
        \begin{frame}{Equivalent forms of R-WAPG}
            \begin{enumerate}
                \item Equivalent forms of R-WAPG exists and resembles variants of FISTA in the literatures
                \item We proved the equivalences in our draft papers and the convergence claim from previous applies to all the equivalent forms of R-WAPG which will follow. 
            \end{enumerate}    
        \end{frame}
        \begin{frame}{R-WAPG intermediate form}
            \begin{definition}[R-WAPG intermediate form]\label{def:r-wapg-intermediate}
                Assume $\mu < L$ and let $(\alpha_k)_{k \ge 0}, (\rho_k)_{k \ge 0}$ given by Definition \ref{def:rwapg-seq}. 
                Initialize any $x_1, v_1$ in $\RR^n$. 
                For $k \ge 1$, the algorithm generates sequence of vector iterates $(y_{k}, v_{k + 1}, x_{k + 1})_{k \ge 1}$ by the procedures: 
                {\small
                \begin{tcolorbox}
                    For $k = 1, 2, \ldots$
                    \begin{align*} 
                        & y_{k} = 
                        \left(
                            1 + \frac{L - L\alpha_{k}}{L\alpha_{k} - \mu}
                        \right)^{-1}
                        \left(
                            v_{k + 1} + 
                            \left(\frac{L - L\alpha_{k}}{L\alpha_{k} - \mu} \right) x_{k}
                        \right), 
                        \\
                        & x_{k + 1} = 
                        y_k - L^{-1} \mathcal G_L y_k, 
                        \\
                        & v_{k + 1} = 
                        \left(
                            1 + \frac{\mu}{L \alpha_k - \mu}
                        \right)^{-1}
                        \left(
                            v_k + 
                            \left(\frac{\mu}{L \alpha_k - \mu}\right) y_k
                        \right) - \frac{1}{L\alpha_{k}}\mathcal G_L y_k. 
                    \end{align*}
                \end{tcolorbox}
                }
            \end{definition}
            \pause
            \begin{enumerate}
                \item If, $\mu = 0$, this is Chapter 12 of in Ryu and Yin's Book \cite{ryu_large-scale_2022}, right after Theorem 17.
            \end{enumerate}
        \end{frame}
        \begin{frame}{R-WAPG similar triangle form}
            \begin{definition}[R-WAPG similar triangle form]\label{def:r-wapg-st-form}
            {\small
                Given any $(x_1, v_1)$ in $\RR^n$. 
                Assume $\mu < L$.
                Let the sequence $(\alpha_k)_{k \ge 0}, (\rho_k)_{k\ge 0}$ be given by Definition \ref{def:rwapg-seq}. 
                For $k \ge 1$, the algorithm generates sequences of vector iterates $(y_k, v_{k + 1}, x_{k + 1})_{k \ge 1}$ by the procedures: 
                \begin{tcolorbox}
                    For $k=1, 2, \ldots $
                    {\scriptsize
                    \begin{align*}
                        & y_k = 
                        \left(
                            1 + \frac{L - L\alpha_k}{L\alpha_k - \mu}
                        \right)^{-1}
                        \left(
                            v_k + 
                            \left(\frac{L - L\alpha_k}{L\alpha_k - \mu} \right) x_k
                        \right), 
                        \\
                        & x_{k + 1} = 
                        y_k - L^{-1} \mathcal G_L y_k, 
                        \\
                        & v_{k + 1} = 
                        x_{k + 1} + (\alpha_k^{-1} -1)(x_{k + 1} - x_k). 
                    \end{align*}   
                    } 
                \end{tcolorbox}
            }
            \end{definition}
            \pause
            {\small
            \begin{enumerate}
                \item Equation (2), (3), (4) in \cite{chambolle_convergence_2015} is a similar triangle formulation of FISTA with $\mu = 0$. 
                \item see (3.1, 4.1) in Lee et al. \cite{lee_geometric_2021} and Ahn and Sra \cite{ahn_understanding_2022} for graphical visualization of similar triangle form. 
            \end{enumerate}
            }
        \end{frame}

        \begin{frame}{R-WAPG momentum form}
            \begin{definition}[R-WAPG momentum form]\label{def:r-wapg-momentum-form}
                Given any $y_1 = x_1 \in \RR^n$, and sequences $(\rho_k)_{k \ge 0}, (\alpha_k)_{k\ge 0}$ Definition \ref{def:rwapg-seq}. 
                The algorithm generates iterates $x_{k + 1}, y_{k + 1}$ For $k = 1, 2, \cdots $ by the procedures: 
                {\small
                \begin{tcolorbox}
                    For $k=1, 2,\ldots $
                    \begin{align*}
                        & x_{k + 1} = y_k - L^{-1}\mathcal G_Ly_k, 
                        \\
                        & 
                        y_{k + 1} = 
                        x_{k + 1} + 
                        \frac{\rho_k\alpha_k(1 - \alpha_k)}{\rho_k\alpha_k^2 + \alpha_{k + 1}}(x_{k + 1} - x_k). 
                    \end{align*}    
                \end{tcolorbox}
                }
                In the special case where $\mu = 0$, the momentum term can be represented without parameter $\rho_k$: 
                $$
                    (\forall k \ge 1)\quad \frac{\rho_k\alpha_k(1 - \alpha_k)}{\rho_k\alpha_k^2 + \alpha_{k + 1}} 
                    = \alpha_{k + 1}(\alpha_k^{-1} - 1). 
                $$
            \end{definition}
        \end{frame}

    \subsection{Unified Convergence claim with relaxed Nesterov's sequence}
        \begin{frame}{Summary of our results}
            With the equivalent representations and the convergence claim for relaxed sequence $(\alpha_k)_{k \ge 0}$ of the R-WAPG, we are able to unifies: 
            \begin{enumerate}
                \item Several Euclidean variants of the FISTA algorithm. 
                \item Nontraditional choices of momentum sequences. 
            \end{enumerate}
            The table below summarizes our major results. \begin{table}[H]
            {\tiny
            \begin{tabular}{|l|l|l|l|}
            \hline
                Algorithm & $\mu$ & $\alpha_k, \rho_k$ & $F(x_k) - F^* \le \mathcal O(\cdot)$ 
            \\ \hline
                Definition \ref{def:wapg} & 
                $\mu \ge 0$ &
                $\alpha_k \in(\mu/L, 1)$, $\rho_k > 0$ & 
                \begin{tabular}{l}
                    $\prod_{i = 0}^{k-1} \max(1, \rho_i)(1 - \alpha_{i + 1})$
                    \\
                    (Proposition \ref{prop:wagp-convergence})
                \end{tabular}
            \\ \hline
                FISTA \cite{chambolle_convergence_2015} &  
                $\mu = 0$  &
                $ 0< \alpha_k^{-2} \le \alpha_{k + 1}^{-1} - \alpha_{k + 1}^{-2}$, $\rho_k \ge 1$ &
                \begin{tabular}{l}
                    $\alpha_k^{2}$ 
                    % \\ 
                    % (Theorem \ref{thm:r-wapg-on-cham-doss})    
                \end{tabular}
            \\ \hline
                \hspace{-1em}\begin{tabular}{l}
                    V-FISTA \\ (10.7.7) \cite{beck_first-order_2017} 
                \end{tabular} &
                $\mu > 0$ &
                $\alpha_k = \sqrt{\mu/L}$, $\rho_k = 1$ & 
                \begin{tabular}{l}
                    $(1 - \sqrt{\mu/L})^k$, 
                    % \\
                    % (Theorem \ref{thm:fixed-momentum-fista}, remark)
                \end{tabular}
            \\ \hline
                Definition \ref{def:wapg} &  
                $\mu > 0$ &
                $\alpha_k = \alpha \in (\mu/L, 1)$, $\rho_k = \rho > 0$ &  
                \begin{tabular}{l}
                    $\max(1 - \alpha, 1 - \mu/(\alpha L))^{k}$
                    % \\
                    % (Theorem \ref{thm:fixed-momentum-fista})
                \end{tabular}
            \\ \hline
            \end{tabular}
            }
            \end{table}
            These results are consistent of iteratures. 
            To the best of our knowledge, the last variant is a and we have the convergence claim for it using R-WAPG. 
        \end{frame}
    \subsection{A parameter free formulation of R-WAPG}
        \begin{frame}{Free R-WAPG}
            
        \end{frame}
    \subsection{Direction of future works}
        

\section{Selected contents from Catalyst Meta Accelerations}
    \subsection{Error sequences and convergence}
    \subsection{Direction of future works}


\section{References}
    \begin{frame}{Citation examples}
        Citation examples \cite{chambolle_convergence_2015}
    \end{frame}
    \begin{frame}[allowframebreaks]{References}
        % \bibliographystyle{apalike}
        \bibliography{references/proposal.bib}
    \end{frame}

\end{document}