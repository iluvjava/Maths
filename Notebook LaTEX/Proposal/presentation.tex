\documentclass[11pt]{beamer}
\usetheme{Madrid}
\usepackage[utf8]{inputenc}

\usepackage{hyperref}
\usepackage{amsmath}
\usepackage{amsfonts}
\usepackage{amssymb}
\usepackage{graphicx}
\DeclareMathOperator{\argmin}{argmin}
\usepackage{algorithmic}
\usepackage{algorithm}
\usepackage{wrapfig}
\usepackage{subcaption}
\graphicspath{{.}}

\author{Hongda Li}
\title{First Order Nonsmooth Optimization: Catalyst Acceleration and Unifying Nesterov's Acceleration}
% Informe o seu email de contato no comando a seguir
% Por exemplo, alcebiades.col@ufes.br
\newcommand{\email}{lalala@lala.la}
\setbeamercovered{transparent}
\setbeamertemplate{navigation symbols}{}
%\logo{}
\institute[]{
    University of British Columbia Okanagan
}
\date{\today}
\subject{Nesterov's acceleration and its applications}

% ---------------------------------------------------------
% Selecione um estilo de referência
\bibliographystyle{IEEEtran}

%\bibliographystyle{abbrv}
%\setbeamertemplate{bibliography item}{\insertbiblabel}
% ---------------------------------------------------------

% ---------------------------------------------------------
\newtheorem{remark}{Remark}
\newtheorem{assumption}{Assumption}

\begin{document}

\begin{frame}
    \titlepage
\end{frame}

\begin{frame}{ToC}
    \tableofcontents
\end{frame}

% \begin{frame}{Example Frame}
%     Cite something \cite{nesterov_accelerating_2008}. 
    
% \end{frame}

\section{Introduction}
    \begin{frame}{What this talk is based on}

        
    \end{frame}
    \subsection{Preliminaries}
        \begin{frame}{Foundations of convex analysis}
            
        \end{frame}
        \begin{frame}{Our works on R-WAPG}
            
        \end{frame}
        \begin{frame}{Catalyst Acceleration}
            
        \end{frame}


\section{References}
    \begin{frame}[allowframebreaks]{References}
        % \bibliographystyle{apalike}
        \bibliography{references/proposal.bib}
    \end{frame}

\end{document}