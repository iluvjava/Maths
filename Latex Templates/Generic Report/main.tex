\documentclass[]{article}
\usepackage{amsmath}
\usepackage{amsfonts} 
\usepackage[english]{babel}
\usepackage{amsthm}
\usepackage{mathtools}
\usepackage{subcaption}
\usepackage{hyperref}
\usepackage{algorithmic}
\usepackage{algorithm}
% \usepackage{minted}

% Basic Type Settings ----------------------------------------------------------
\usepackage[margin=1in,footskip=0.25in]{geometry}
\linespread{1}  % double spaced or single spaced
\usepackage[fontsize=12pt]{fontsize}
\usepackage{authblk}

\theoremstyle{definition}
\newtheorem{theorem}{Theorem}       % Theorem counter global 
\newtheorem{prop}{Proposition}[section]  % proposition counter is section
\newtheorem{lemma}{Lemma}[subsection]  % lemma counter is subsection
\newtheorem{definition}{Definition}
\newtheorem{remark}{Remark}[subsection]
{
    % \theoremstyle{plain}
    \newtheorem{assumption}{Assumption}
}
\numberwithin{equation}{subsection}

\hypersetup{
    colorlinks=true,
    linkcolor=blue,
    filecolor=magenta,
    urlcolor=cyan,
}
\usepackage[final]{graphicx}
\usepackage{listings}
\usepackage{courier}
\lstset{basicstyle=\footnotesize\ttfamily,breaklines=true}
\newcommand{\indep}{\perp \!\!\! \perp}
\usepackage{wrapfig}
\graphicspath{{.}}
\usepackage{fancyvrb}

%%
%% Julia definition (c) 2014 Jubobs
%%
\usepackage[T1]{fontenc}
\usepackage{beramono}
\usepackage[usenames,dvipsnames]{xcolor}
\lstdefinelanguage{Julia}%
  {morekeywords={abstract,break,case,catch,const,continue,do, else, elseif,%
      end, export, false, for, function, immutable, import, importall, if, in,%
      macro, module, otherwise, quote, return, switch, true, try, type, typealias,%
      using, while},%
   sensitive=true,%
   alsoother={$},%
   morecomment=[l]\#,%
   morecomment=[n]{\#=}{=\#},%
   morestring=[s]{"}{"},%
   morestring=[m]{'}{'},%
}[keywords,comments,strings]%
\lstset{%
    language         = Julia,
    basicstyle       = \ttfamily,
    keywordstyle     = \bfseries\color{blue},
    stringstyle      = \color{magenta},
    commentstyle     = \color{ForestGreen},
    showstringspaces = false,
}

\title{This is The Title}
\author{Name of the Author}

\begin{document}
\maketitle

\begin{abstract}
    This is the Abstract
\end{abstract}


\section{Introduction}
    This is an introduction. 

\section{Preliminaries}\label{sec:preliminaries}
    \hyperref[sec:preliminaries]{This} is the preliminary (hyperref without text labeling). 
    
\section{Blah Blah Bleeeh}
    This is the blah blah bleeeh section. 
    \subsection{Blah Blah Blah Bleeh Bleeh Bleeh}
    Check out this cool Bibtext ref \cite[this]{texbook}, woooooooooah, also it's in plain style. For some mind altering psychodelic, read \hyperref[alg:mhc]{algorithm \ref*{alg:mhc}} for the experience. For some brain expanding julia code, read \hyperref[code:brain_expand]{brain expanding julia code}. 
    \begin{algorithm}
        \begin{algorithmic}[H]
            \STATE{\textbf{Input: $X^{(t)}$}}
            \STATE{$Y^{(t)} \sim q (\cdot | X^{(t)})$}
            \STATE{
                $ 
                \rho(x, y) := 
                \min\left\lbrace
                    \frac{f(y)}{f(x)}\frac{q(x|y)}{q(y|x)}, 1
                \right\rbrace
                $ 
            }
            \STATE{
                $
                X^{(t + 1)} := 
                \begin{cases}
                    Y^{(t)} & \text{w.p}:  \rho(X^{(t)}, Y^{(t)})
                    \\
                    X^{(t)} &  \text{otherwise}
                \end{cases}$
            }
        \end{algorithmic}
        \caption{Metropolis Chain}
        \label{alg:mhc}
    \end{algorithm}
    \label{code:brain_expand}
    \lstinputlisting[language=julia, basicstyle=\ttfamily\tiny,numbers=left]{Code/juliacode.jl}
    
        

\appendix
\section{Bleeh Bleeh Bleeh I am not Listening}
    This is the Bleeh Bleeh Bleeh I am not Listening section. 

\section{This section is in another .tex file} 
    This is a new section. 
\subsection{Subsection}
This is a subsection. 

\subsection{Cute Subsection}
Check out this cute figure. In \hyperref[fig:cute_alto]{fig \ref*{fig:cute_alto}} is cute pink unicorn, and in \hyperref[fig:minty_and_alto]{fig \ref*{fig:minty_and_alto}}, the green earth pony is minty, a cookie pone. Theyare cute together. Read source to understand the use of ``subfloat'' and ``figure'' together with ``hyperref''. 

\begin{figure}
    \subfloat[Cute Alto]{
        \includegraphics[width=10em]{Assets/Alto_Legato.png}
        \label{fig:cute_alto}
    }
    \hfil
    \subfloat[Minty and Alto]{
        \includegraphics[width=10em]{Assets/Minty_and_Alto.png}
        \label{fig:minty_and_alto}
    }
\end{figure}



\bibliographystyle{plain}
\bibliography{refs.bib}


\end{document}