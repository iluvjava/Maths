\documentclass[12pt]{article}

% \input{presets/wang_full.tex}
\usepackage{ifthen}\newboolean{draftmode}\setboolean{draftmode}{true}


%\usepackage{showkeys}
%\usepackage{drftcite}
\usepackage{exscale,relsize}
\usepackage{amsmath}
\usepackage{amsfonts}
\usepackage[colorlinks=true, linkcolor=blue]{hyperref}
\usepackage{amssymb}
\usepackage{calc}
\usepackage{theorem}
\usepackage{pifont}      % needed by dingautolist
\usepackage{array}
\usepackage{color}
\usepackage{enumerate}
\usepackage{bbm}
\usepackage{graphicx} 
% \usepackage[draft]{graphicx}  % GRAPHICS DRAFT MODE ALTERNATIVE. 
\usepackage{subcaption}
\usepackage{caption}

% \usepackage{amsthm}



% Hongda's packages ============================================================
% \usepackage{booktabs}  % FOR TABLES WITH BOOKSTAB. 
\usepackage{algcompatible, algorithm}
\usepackage{mathtools}
\usepackage{tcolorbox}
\usepackage{soul}  % HILIGHTER

% MANAGE DRAFT MODE 
\ifthenelse{\boolean{draftmode}}{
    \usepackage[right]{showlabels}
    \usepackage[
        colorinlistoftodos,
        textsize=footnotesize, 
        color=yellow
    ]{todonotes} % see (https://tex.stackexchange.com/questions/9796/how-to-add-todo-notes) for more info. 
    \renewcommand{\showlabelfont}{\small\slshape\color{gray}}
}{
    \usepackage[disable]{todonotes}  % TODO NOTES DISABLE
    \renewcommand\hl[1]{#1}  % NULLIFY HILIGHT COMMAND.
}

% FORMATTING ===================================================================
\oddsidemargin -0.1cm
\textwidth  16.5cm
\topmargin  0.0cm
\headheight 0.0cm
\textheight 21.0cm
\parindent  4mm
\parskip    10pt
%\parskip    8pt
\tolerance  3000

% DRAFT FORMATTING =============================================================
% These are for todo notes, advise taken from 
% https://tex.stackexchange.com/questions/81666/extend-page-width-or-margin-for-todonotes-comments-or-other-package-comments
% \paperwidth=\dimexpr \paperwidth + 6cm\relax 
% \oddsidemargin=\dimexpr\oddsidemargin + 3cm\relax % DON'T USE
% \evensidemargin=\dimexpr\evensidemargin  + 3cm\relax 
% \marginparwidth=\dimexpr \marginparwidth  + 6cm\relax
% \paperheight=\dimexpr \paperheight + 6cm\relax
% THEOREM AND ENVIRONMENTS.  ===================================================

%\newenvironment{deflist}[1][\quad]%
%{\begin{list}{}{\renewcommand{\makelabel}[1]{\textrm{##1~}\hfil}%
%\settowidth{\labelwidth}{\textrm{#1~}}%
%\setlength{\leftmargin}{\labelwidth+\labelsep}}}%requires macro calc.sty
%{\end{list}}
%\newtheorem{theorem}{Theorem}%[section]
\newtheorem{theorem}{Theorem}[section]
\newtheorem{lemma}[theorem]{Lemma}
\newtheorem{fact}[theorem]{Fact}
\newtheorem{corollary}[theorem]{Corollary}
\newtheorem{proposition}[theorem]{Proposition}
\newtheorem{definition}[theorem]{Definition}
\newtheorem{conjecture}[theorem]{Conjecture}
\newtheorem{observation}[theorem]{Observation}
\newtheorem{openprob}[theorem]{Open Problem}
\theoremstyle{plain}{\theorembodyfont{\rmfamily}
\newtheorem{assumption}[theorem]{Assumption}}
\theoremstyle{plain}{\theorembodyfont{\rmfamily}
\newtheorem{condition}[theorem]{Condition}}
\theoremstyle{plain}{\theorembodyfont{\rmfamily}
% Removed due conflict with the algorithm environment. 
% \newtheorem{algorithm}[theorem]{Algorithm}}
\theoremstyle{plain}{\theorembodyfont{\rmfamily}
\newtheorem{example}[theorem]{Example}}
\theoremstyle{plain}{\theorembodyfont{\rmfamily}
\newtheorem{remark}[theorem]{Remark}}
\theoremstyle{plain}{\theorembodyfont{\rmfamily}
\newtheorem{application}[theorem]{Application}}
\def\proof{\noindent{\it Proof}. \ignorespaces}
%\def\endproof{\vbox{\hrule height0.6pt\hbox{\vrule height1.3ex%
%width0.6pt\hskip0.8ex\vrule width0.6pt}\hrule height0.6pt}}
%\numberwithin{equation}{section}
\def\endproof{\ensuremath{\quad \hfill \blacksquare}}

\renewcommand\theenumi{(\roman{enumi})}
\renewcommand\theenumii{(\alph{enumii})}
\renewcommand{\labelenumi}{\rm (\roman{enumi})}
\renewcommand{\labelenumii}{\rm (\alph{enumii})}



% These are Heniz's notations. 
\newcommand{\To}{\ensuremath{\rightrightarrows}}
\newcommand{\GX}{\ensuremath{\Gamma}}
\newcommand{\mal}{\ensuremath{\mathfrak{m}}}
\newcommand{\mumu}{\ensuremath{{\mu\mu}}}
\newcommand{\paver}{\ensuremath{\mathcal{P}}}
\newcommand{\ZZZ}{\ensuremath{{X \times X^*}}}
\newcommand{\RRR}{\ensuremath{{\RR \times \RR}}}
% \newcommand{\todo}{\hookrightarrow\textsf{TO DO:}}

\newcommand{\emp}{\ensuremath{\varnothing}}
%\newcommand{\la}{\ensuremath{\langle}}
%\newcommand{\ra}{\ensuremath{\rangle}}
\newcommand{\infconv}{\ensuremath{\mbox{\small$\,\square\,$}}}
\newcommand{\pscal}{\ensuremath{\scal{\cdot}{\cdot}}}
\newcommand{\Tt}{\ensuremath{\mathfrak{T}}}
\newcommand{\YY}{\ensuremath{\mathcal Y}}
\newcommand{\XX}{\ensuremath{\mathcal X}}
\newcommand{\HH}{\ensuremath{\mathcal H}}
\newcommand{\XP}{\ensuremath{\mathcal X}^*}
% \newcommand{\st}{\ensuremath{\;|\;}} CONFLICT WITH `soul' PACKAGE.
\newcommand{\zeroun}{\ensuremath{\left]0,1\right[}}

\newcommand{\lev}[1]{\ensuremath{\mathrm{lev}_{\leq #1}\:}}
\newcommand{\moyo}[2]{\ensuremath{\sideset{_{#2}}{}{\operatorname{}}\!#1}}
\newcommand{\pair}[2]{\left\langle{{#1},{#2}}\right\rangle}
%\newcommand{\scal}[2]{\left.\left\langle{#1}\:\right| {#2}  \right\rangle}
\newcommand{\scal}[2]{\langle{{#1},{#2}}\rangle}
\newcommand{\Scal}[2]{\left\langle{{#1},{#2}}\right\rangle}
%\newcommand{\scal}[2]{\braket{ {#1},{#2}}}

\newcommand{\yosida}{\ensuremath{ \; {}^}}
\newcommand{\exi}{\ensuremath{\exists\,}}
\newcommand{\GG}{\ensuremath{\mathcal G}}
\newcommand{\RR}{\ensuremath{\mathbb R}}
\newcommand{\SSS}{\ensuremath{\mathbb S}}
\newcommand{\CC}{\ensuremath{\mathbb C}}
\newcommand{\Real}{\ensuremath{\mathrm{Re}\,}}
\newcommand{\ii}{\ensuremath{\mathrm i}}
\newcommand{\RP}{\ensuremath{\left[0,+\infty\right[}}
\newcommand{\RPX}{\ensuremath{\left[0,+\infty\right]}}
\newcommand{\RPP}{\ensuremath{\,\left]0,+\infty\right[}}
\newcommand{\RX}{\ensuremath{\,\left]-\infty,+\infty\right]}}
\newcommand{\RXX}{\ensuremath{\,\left[-\infty,+\infty\right]}}
\newcommand{\KK}{\ensuremath{\mathbb K}}
\newcommand{\NN}{\ensuremath{\mathbb N}}
\newcommand{\nnn}{\ensuremath{{n \in \NN}}}
\newcommand{\thalb}{\ensuremath{\tfrac{1}{2}}}
\newcommand{\zo}{\ensuremath{{\left]0,1\right]}}}
\newcommand{\lzo}{\ensuremath{{\lambda \in \left]0,1\right]}}}
%\newcommand{\toppsepp}{\setlength{\partopsep}{-5pt}}
\newcommand{\menge}[2]{\big\{{#1} \mid {#2}\big\}}
\newcommand{\pfrac}[2]{\ensuremath{\mathlarger{\tfrac{#1}{#2}}}}


% MATH OPERATORS ===============================================================
% \newcommand{\monos}{\ensuremath{\mathcal M}}
\newcommand{\DD}{\operatorname{dom}f}
\newcommand{\IDD}{\ensuremath{\operatorname{int}\operatorname{dom}f}}
\newcommand{\CDD}{\ensuremath{\overline{\operatorname{dom}}\,f}}
\newcommand{\clspan}{\ensuremath{\overline{\operatorname{span}}}}
\newcommand{\cone}{\ensuremath{\operatorname{cone}}}
\newcommand{\dom}{\ensuremath{\operatorname{dom}}}
\newcommand{\closu}{\ensuremath{\operatorname{cl}}}
\newcommand{\cont}{\ensuremath{\operatorname{cont}}}
\newcommand{\mons}{\ensuremath{\mathcal{A}}}
\newcommand{\gra}{\ensuremath{\operatorname{gra}}}
\newcommand{\epi}{\ensuremath{\operatorname{epi}}}
\newcommand{\prox}{\ensuremath{\operatorname{Prox}_{\mu}}}
\newcommand{\hprox}{\ensuremath{\operatorname{prox}}}
\newcommand{\intdom}{\ensuremath{\operatorname{int}\operatorname{dom}}\,}
\newcommand{\inte}{\ensuremath{\operatorname{int}}}
\newcommand{\sri}{\ensuremath{\operatorname{sri}}}
\newcommand{\reli}{\ensuremath{\operatorname{ri}}}
\newcommand{\cart}{\ensuremath{\mbox{\LARGE{$\times$}}}}


\newcommand{\average}{\ensuremath{\mathcal{R}_{\mu}({\bf A},{\boldsymbol \lambda})}}
\newcommand{\averagebar}{\ensuremath{\mathcal{R}_{1}({\bf A},\bar{\lambda})}}
\newcommand{\averageonelambda}{\ensuremath{\mathcal{R}({\bf A},{\boldsymbol \lambda})}}
\newcommand{\averageonehalf}{\ensuremath{\mathcal{R}_{1}(A,1/2)}}
\newcommand{\averageinverse}{\ensuremath{\mathcal{R}_{\mu^{-1}}({\bf A}^{-1},{\boldsymbol \lambda})}}
\newcommand{\averageoneinverse}{\ensuremath{\mathcal{R}({\bf A}^{-1},{\boldsymbol \lambda})}}
\newcommand{\averagef}{\ensuremath{\mathcal{P}_{\mu}(f,\lambda)}}
\newcommand{\averagefone}{\ensuremath{\mathcal{P}_{1}(f,\lambda)}}
\newcommand{\averagefd}{\ensuremath{\mathcal{P}_{\mu}((f_{1},\ldots, f_{n}),(\lambda_{1},\ldots, \lambda_{n}))}}
\newcommand{\averagefik}{\ensuremath{\mathcal{P}_{\mu_{k}}((f_{1,k},\ldots,f_{n,k}),
(\lambda_{1,k},\ldots,\lambda_{n,k}))}}
\newcommand{\averagesub}{\ensuremath{\mathcal{R}_{\mu}(\partial f,\lambda)}}
\newcommand{\res}{\ensuremath{\mathcal{R}_{\mu}}}
\newcommand{\resmuk}{\ensuremath{\mathcal{R}_{\mu_{k}}}}
\newcommand{\newres}{\ensuremath{\mathcal{R}}}
\newcommand{\resmualpha}{\ensuremath{\mathcal{R}_{\alpha\mu}}}
\newcommand{\averageone}{\ensuremath{\mathcal{R}_{1}}}
\newcommand{\harm}{\ensuremath{\mathcal{H}(A,\lambda)}}
\newcommand{\arithmetic}{\ensuremath{\mathcal{A}(A,\lambda)}}

\newcommand{\WC}{\ensuremath{{\mathfrak W}}}
\newcommand{\SC}{\ensuremath{{\mathfrak S}}}
\newcommand{\card}{\ensuremath{\operatorname{card}}}
\newcommand{\bd}{\ensuremath{\operatorname{bdry}}}
\newcommand{\ran}{\ensuremath{\operatorname{ran}}}
\newcommand{\rec}{\ensuremath{\operatorname{rec}}}
\newcommand{\rank}{\ensuremath{\operatorname{rank}}}
\newcommand{\kernel}{\ensuremath{\operatorname{ker}}}
\newcommand{\conv}{\ensuremath{\operatorname{conv}}}
\newcommand{\segh}{\ensuremath{\operatorname{seg}}}
\newcommand{\boxx}{\ensuremath{\operatorname{box}}}
\newcommand{\clconv}{\ensuremath{\overline{\operatorname{conv}}\,}}
\newcommand{\cldom}{\ensuremath{\overline{\operatorname{dom}}\,}}
\newcommand{\clran}{\ensuremath{\overline{\operatorname{ran}}\,}}
\newcommand{\Nf}{\ensuremath{\nabla f}}
\newcommand{\NNf}{\ensuremath{\nabla^2f}}
\newcommand{\Fix}{\ensuremath{\operatorname{Fix}}}
\newcommand{\FFix}{\ensuremath{\overline{\operatorname{Fix}}\,}}
\newcommand{\aFix}{\ensuremath{\widetilde{\operatorname{Fix}\,}}}
\newcommand{\Id}{\ensuremath{\operatorname{Id}}}
\newcommand{\Max}{\ensuremath{\operatorname{max}}}
\newcommand{\Bb}{\ensuremath{\mathfrak{B}}}
\newcommand{\BB}{\ensuremath{\mathbb{B}}}
\newcommand{\Fb}{\ensuremath{\overrightarrow{\mathfrak{B}}}}
\newcommand{\Fprox}{\ensuremath{\overrightarrow{\operatorname{prox}}}}
\newcommand{\Bprox}{\ensuremath{\overleftarrow{\operatorname{prox}}}}
\newcommand{\Bproj}{\ensuremath{\overleftarrow{\operatorname{P}}}}
\newcommand{\Ri}{\ensuremath{\mathfrak{R}_i}}
\newcommand{\Dn}{\ensuremath{\,\overset{D}{\rightarrow}\,}}
\newcommand{\nDn}{\ensuremath{\,\overset{D}{\not\rightarrow}\,}}
\newcommand{\weakly}{\ensuremath{\,\rightharpoonup}\,}
\newcommand{\weaklys}{\ensuremath{\,\overset{*}{\rightharpoonup}}\,}
\newcommand{\gr}{\ensuremath{\operatorname{gra}}}
\newcommand{\g}{\ensuremath{\,\overset{g}{\rightarrow}}\,}
\newcommand{\p}{\ensuremath{\,\overset{p}{\rightarrow}}\,}
\newcommand{\e}{\ensuremath{\,\overset{e}{\rightarrow}}\,}
\newcommand{\Tbar}{\ensuremath{\overline{T}}}
\newcommand{\n}{\ensuremath{\,\overset{n}{\rightarrow}}\,}

\newcommand{\minf}{\ensuremath{-\infty}}
\newcommand{\pinf}{\ensuremath{+\infty}}
% \renewcommand{\iff}{\ensuremath{\Leftrightarrow}}
% \renewcommand{\phi}{\ensuremath{\varphi}} % The hell who added this? 
%\newcommand{\Real}{\ensuremath{\mathrm{Re}\,}}
\newcommand{\negent}{\ensuremath{\operatorname{negent}}}
\newcommand{\neglog}{\ensuremath{\operatorname{neglog}}}
\newcommand{\halb}{\ensuremath{\tfrac{1}{2}}}
\newcommand{\bT}{\ensuremath{\mathbf{T}}}
\newcommand{\bX}{\ensuremath{\mathbf{X}}}
\newcommand{\bL}{\ensuremath{\mathbf{L}}}
\newcommand{\bD}{\ensuremath{\boldsymbol{\Delta}}}
\newcommand{\bc}{\ensuremath{\mathbf{c}}}
\newcommand{\by}{\ensuremath{\mathbf{y}}}
\newcommand{\bx}{\ensuremath{\mathbf{x}}}
\newcommand{\bA}{{\bf A}}
\newcommand{\Other}{Indeterminate }
\newcommand{\other}{indeterminate }


%%% Raf's stuff  ===============================================================
\newcommand{\al}{\alpha}
\newcommand{\la}{\lambda}
\newcommand{\La}{\Lambda}
\newcommand{\pluss}{{\hskip1pt \raise1pt\vbox{\hrule width6pt \vskip1pt
\hrule width6pt}\kern-4pt{\lower1pt\hbox{\vrule height6pt \kern1pt\vrule
height6pt}}\hskip5pt}}
\newcommand{\timess}{\star}
\newcommand{\argmax}{\mathop{\rm argmax}\limits}
\newcommand{\argmin}{\mathop{\rm argmin}\limits}
\newcommand{\product}{\langle\cdot,\cdot\rangle}
\newcommand{\im}{\mathrm{Im}}
\newcommand{\multival}{\ensuremath{X\to 2^{X^*}}}
\newcommand{\SX}{\ensuremath{2^{X^*}}}


% Hongda's Stuff: 
\newcommand{\N}{\ensuremath{\mathbb N}}
% REQURES MATH TOOLS PACKAGE
\newcommand{\defeq}{\vcentcolon=}
\newcommand{\eqdef}{=\vcentcolon}
\newcommand{\dist}{\ensuremath{\operatorname{dist}}}
\newcommand{\todoinline}[1]{\todo[inline, caption={}]{#1}}

\newcommand{\inlinecode}[1]{\texttt{\footnotesize #1}}
\input{presets/julia_lstlisting.tex}

\begin{document}

\title{{\fontfamily{ptm}\selectfont Lorum ipsum}}

\author{
    Author 1 Name, Author 2 Name
    \thanks{
        Subject type, Some Department of Some University, Location of the University,
        Country. E-mail: \texttt{author.name@university.edu}.
    }
}

\date{March 2, 2020}

\maketitle
\todoinline{This paper is currently in draft mode. Check source to change options. }
% \vskip 8mm

\begin{abstract} 
    \noindent
    Lorem ipsum dolor sit amet, dicta iudicabit consequat ex vix, veniam legimus appetere has id, an pri graece epicuri detraxit. Ea aliquam expetendis posidonium eos, nam invenire corrumpit imperdiet ei. Et constituto dissentias usu, mel solum erant et. Mel dolorem menandri in.
    \todo{This is just psuedo text. }
\end{abstract}

\noindent{\bfseries 2010 Mathematics Subject Classification:}
Primary 47H05, 52A41, 90C25; Secondary 15A09, 26A51, 26B25, 26E60, 47H09, 47A63.
\noindent{\bfseries Keywords:}


% ==============================================================================

\section{Schur's Cone and Its Polar}
    \begin{example}
        Define
        \begin{equation}\label{eqn:cone-1}
            Q = \left\lbrace
                x\in \RR^n \left|
                    \sum_{i = 1}^{k} x_i
                    \ge 0, \;
                    k = 1 ,\cdots, n-1, \;
                    x_1 + \cdots + x_n = 0
                \right.
            \right\rbrace.
        \end{equation}
        Then
        \begin{align}
            Q^\circ &= \left\lbrace
               y\in\RR^n |
               \langle y, x\rangle \ge 0 \; \forall x \in Q
            \right\rbrace
            \\
            &=
            \left\lbrace
                y\in\RR^n | y_{1} \ge y_2\ge \cdots  \ge y_n
            \right\rbrace.
        \end{align}
    \end{example}
    \begin{proof}
        We have $Q = H(\RR^{n - 1}_+)$, where
        \begin{align}
            H =
            \begin{bmatrix}
                1 &  & &
                \\
                -1 & 1 & &
                \\
                 & -1 & &
                \\
                &  & \ddots&
                \\[-1em]
                &  & & 1
                \\
                &  && -1
            \end{bmatrix},
        \end{align}
        $H$ is a bi-diagonal matrix with zeros on the diagonal and -1 on its sub-diagonals. Indeed, for $x = (x_1, \cdots, x_{n - 1})\in \RR^{n - 1}_+$, we have
        \begin{align}
            y &= H x
            \\
            &=
            \begin{bmatrix}
                1 \\ -1 \\ 0 \\ \vdots \\ 0
            \end{bmatrix}x_1 +
            \begin{bmatrix}
                0 \\ 1 \\ -1 \\ \vdots \\ 0
            \end{bmatrix}x_2 +
            \cdots
            +
            \begin{bmatrix}
                0 \\ 0 \\ \vdots \\ 1 \\ -1
            \end{bmatrix} x_{n - 1}
            \\
            &=
            \begin{bmatrix}
                x_1 \\ x_2 - x_1 \\ x_3 - x_2
                \\ \vdots
                \\-x_{n - 2} + x_{n - 1}\\
                 - x_{n - 1}
            \end{bmatrix}
            = \begin{bmatrix}
                y_1 \\ y_2 \\ y_3 \\ \vdots \\ y_{n - 1} \\ y_n
            \end{bmatrix},
        \end{align}
        where $y_1 \ge 0, y_2 + y_1 \ge 0, \cdots, y_{n - 1}+ \cdots + y_1 \ge 0$, and $y_1 + \cdots + y_n = 0$. So $y\in Q$. Hence $H(\RR^{n - 1}_+)\subseteq Q$. Similarly, one can write for $y\in Q$, $\exists x \in \RR_+^{n- 1}$ with $y = Hx$. In fact, $x = y_1, x_2 = y_1 + y_2, \cdots, x_{n - 1} = y_1 + \cdots + y_{n - 1}, -x_{n - 1} = y_n$ because $y_1 + \cdots + y_n = 0$. In particular, $x\in \RR^{n - 1}_+$.
        \par
        Note that $y\in Q^\circ\iff y^{\intercal}a_i \ge 0$ for $i = 1,\cdots, n -1$. Where
        \begin{align}
            a_1 = \begin{bmatrix}
                1 \\ -1 \\ 0 \\ \vdots \\ 0,
            \end{bmatrix},
            a_2 = \begin{bmatrix}
                0 \\ 1 \\ -1 \\ \vdots \\ 0,
            \end{bmatrix},
            & \cdots,
            a_n =
            \begin{bmatrix}
                0 \\ \vdots \\ 0  \\ 1 \\ -1
            \end{bmatrix}.
        \end{align}
        Now $y^{\intercal} a_i \ge 0 \iff y_i - y_{i + 1}\ge 0$, for $i = 1,\cdots, n-1$. Hence $Q^\circ = \{ y\in \RR^n | y_1 \ge y_2 \ge \cdots \ge y_n\}$.

    \end{proof}

\section{Lorentz Cone and its Polar}
    \begin{example}
        \begin{equation}
            L = \left\lbrace
            (\xi, t) \in \RR^n \left| \;
                \Vert \xi\Vert \le t, \xi \in \RR^{n - 1}
            \right.
            \right\rbrace.
        \end{equation}
        Then the polar cone of $L$ is $L^\circ = \{(x, s) \;|\; \langle (x, s), (\xi, t)\rangle \ge 0\}\} = L$.
    \end{example}
    \begin{proof}
        \par
        Let $(x, s) \in L^\circ$. Then $\langle (x, s), (\xi, t)\rangle \ge 0$, $\forall (\xi, t) \in L$, i. e. $\langle x, \xi\rangle + st \ge 0$.
        \par
        For $t > 0$, we have $\langle x, \xi/t\rangle \ge -s$, i.e. $\langle x, - \xi/t\rangle \le s$. Using $\Vert \xi\Vert \le t$ and by taking supremum over $\Vert \xi/t\Vert\le 1$, we obtain  $\Vert x\Vert \le s$.
        \par
        Hence $L^\circ = L$.
    \end{proof}
    \begin{corollary}
        Let $A \in \RR^{n\times n}$ be positive definite. Define
        $$
            L_A =
            \left\lbrace
                (\xi, t)\in \RR^n \left|
                    \sqrt{\langle \xi, A\xi\rangle} \le t
                \right.
            \right\rbrace.
        $$
        Then $(L_A)^\circ = \{ (\xi, t) \;|\; \sqrt{\langle \xi, A^{-1}\xi\rangle}\le t\}$.
    \end{corollary}

\section{Testing the references}
    This book by Nesterov \cite{nesterov_lectures_2018}. 
    This book by Heinz \cite{bauschke_convex_2017}. 

\section{Testing figures and images}
    In Figure \ref{fig:two-test-figures}, it has Figure \ref{fig:two-test-fig-1} on the left and Figure \ref{fig:two-test-fig-2} on the right. 
    \begin{figure}[H]
        \begin{subfigure}[b]{0.47\textwidth}
            \centering
            \includegraphics[width=\textwidth]{assets/test-image-1.png}
            \caption{Alto the unicorn. }
            \label{fig:two-test-fig-1}
        \end{subfigure}
        \hfill
        \begin{subfigure}[b]{0.47\textwidth}
            \centering
            \includegraphics[width=\textwidth]{assets/test-image-2.png}
            \caption{Alto and Minty. }
            \label{fig:two-test-fig-2}
        \end{subfigure}
        \caption{The above 2 images are test images for the figure, subfigure latex commands. }
    \label{fig:two-test-figures}
    \end{figure}

\section*{Acknowledgments}
    XW were supported by NSERC Discovery grants.



\bibliographystyle{siam}

\bibliography{references/Books.bib}


\end{document}
